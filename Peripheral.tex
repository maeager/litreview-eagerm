

\subsection{Peripheral Auditory System}

\todo[inline]{Intro to section  on peripheral AN}


\todo[inline]{Small introduction to the outer and middle ear, and inner}

%\subsubsection{Inner Ear}

# \citep{RyugoParks:2003}
% Upon passing the Schwann-glia border (marking entrance
% into the central nervous system), individual auditory nerve
% fibres penetrate a variable distance into the nucleus, de-
% pending upon fibre CF, and bifurcate into an ascending
% branch and a descending branch. The ascending branch has a
% relatively straight trajectory into the AVCN and terminates
% as a large, axosomatic ending called the endbulb of Held.
% The descending branch likewise has a straight trajectory
% through the PVCN before entering the DCN. Along the
% way, these main branches give rise to short collaterals. The
% collaterals ramify further and exhibit en passant swellings
% and terminal boutons. Fibers of similar CFs disperse to
% form a 3-dimensional sheet running through the nucleus,
% and stacks of these sheets represent the isofrequency con-
% tours of the nucleus (Fig. 9, bottom). The sheets have a
% horizontal orientation within the ventral cochlear nucleus
% but twist caudally to form parasagittal sheets in the DCN.
% These projections underlie the tonotopic organization of
% the resident neurons of the cochlear nucleus [19,160,191].

% \citep{EvansNelson:1973,SpirouYoung:1991,YoungSpirouEtAl:1992,SpirouDavisEtAl:1999,YoungNelkenEtAl:1993,ArleKim:1991a}




% 5. Structure-function correlates
% 5.1. SR and peripheral correlates
% Morphologic specializations have been found in the in-
% nervation pattern of inner hair cells with respect to SR fibre
% groupings. High-SR fibres (>18 spikes/s) have thick periph-
% eral processes that tend to contact the “pillar” side of the
% inner hair cell, whereas low-SR fibres (<18 spikes/s) have
% thin peripheral processes that tend to contact the modiolar
% side of the hair cell [98,111]. Furthermore, there is SR
% segregation within the spiral ganglion. Low-SR neurons
% tend to be distributed on the side of the scala vestibuli,
% whereas high-SR fibres can be found throughout the gan-
% glion [82,100]. These peripheral differences are maintained
% by the pattern of central projections, and embedded within
% the tonotopic organization.
% 5.2. SR and central correlates
% There are morphologic correlates that correspond to
% groupings of fibres with respect to SR. Compared to fibres
% of high SR (>18 spikes/s), fibres of low SR (<18 spikes/s)
% exhibit different innervation characteristics with the IHCs
% [99,111], give rise to greater collateral branching in the
% AVCN [51], emit collaterals that preferentially innervate
% the small cell cap [100,177], and manifest striking special-
% izations in the large axosomatic endings, the endbulbs of
% Held [185] and their synapses [178].
% The typical high-SR fibre traverses the nucleus and gives
% rise to short collaterals that branch a few times before
% terminating (Fig. 10A). There was a suggestion that projec-
% tions of the different SR groups might be segregated along
% a medial-lateral axis within the core of the AVCN [94] but
% single-unit labelling studies do not unambiguously support or
% refute this proposal [51,82,100,208]. There are usually one
% or two terminal endbulbs at the anterior tip of the ascending
% branch, and the remaining terminals appear as en passant
% swellings or terminal boutons. It is presumed that these
% swellings are sites of synaptic interactions with other neu-
% ronal elements in the cochlear nucleus. Approximately 95%
% of all terminal endings were small and round, definable as
% “bouton-like” [163]. The remaining endings were modified
% endbulbs that tended to contact the somata of globular bushy
% cells and large endbulbs of Held that contacted the somata
% of spherical bushy cells. In contrast to birds, low-frequency
% myelinated auditory nerve fibres in mammals give rise
% to endbulbs. Furthermore, the endbulbs of low-frequency
% fibres tend to be the largest of the entire population of
% fibres.

% There is a clear SR-related difference in axonal branch-
% ing and the number of endings. Low-SR fibres give rise to
% greater collateral branching in the AVCN compared to that
% of high-SR fibres [51,100,101,208]. In cats, the ascending
% branch of low-SR fibres give rise to longer collaterals, twice
% as many branches (there are approximately 50 branches per
% low-SR fibre compared to 25 per high-SR fibre), and twice as
% many bouton endings (Fig. 10B). These endings, while more
% numerous, are also smaller compared to those of high-SR
% fibres [163]. The greater total collateral length is illustrated
% by low-SR fibres that have an average of 5 mm of collaterals
% per ascending branch compared to 2.8 mm of collaterals per
% high-SR fibre [51]. The inference from these observations is
% that low-SR fibres contact more neurons distributed over a
% wider region of the cochlear nucleus than do high-SR fibres.
% If the perception of loudness is proportional to the num-
% ber of active neurons [195], then this branching differential
% may provide the substrate. The activation of high-threshold,
% low-SR fibres by loud sounds would not only increase the
% pool of active auditory nerve fibres but also produce a spread
% of activity throughout the AVCN. This recruitment would
% be useful because the discharge rate of high-SR fibres is al-
% ready saturated at moderate sound levels.
% There is no systematic difference in the average number
% of terminals generated by the descending branch with re-
% spect to fibre SR. Low-SR fibres do, however, have a wider
% distribution across the frequency axis in the DCN as com-
% pared to high-SR fibres [171]. The endings lie within the
% deep layers of the DCN, below the pyramidal cell layer, and
% terminate primarily within the neuropil. The average termi-
% nal field width for low-SR fibres is 230.5 ± 73 ␮m, whereas
% that for high-SR fibres is 87.2 ± 41 ␮m. The significance of
% terminal arborization differences between high- and low-SR
% fibres might be involved in details of isofrequency laminae.
% The relatively short and narrow arborization of high-SR,
% low-threshold fibres could occupy the center of the lamina
% and endow those neurons with lower thresholds and sharper
% tuning. In contrast, the longer and broader terminal field of
% low-SR fibres could preferentially innervate the “edges” of
% the lamina. This kind of organization might establish a func-
% tional segregation of units having distinct physiological fea-
% tures within an isofrequency lamina, as has been proposed in
% the inferior colliculus [161] and auditory cortex [184,186].



\subsection{Auditory Nerve Fibres}


% 4.2. Spontaneous discharge rate (SR)
% Unlike birds, SR in the mammalian auditory nerve forms
% a bimodal distribution where 30–40% of the fibres have SR
% <10 spikes/s and 60–70% have SR >30 spikes/s. Threshold
% is correlated to the amount of spontaneous spike activity
% (SR) that occurs in the absence of experimenter-controlled
% stimulation. Low-SR fibres have relatively high thresholds,
% whereas high-SR fibres have low thresholds. Across the au-
% dible frequency range, fibres of similar CFs can vary in SR
% from near 0 to >100 spikes/s. The bimodal SR distribution
% is present across the entire audible frequency range for the
% animal (e.g. cat [50,88,97]; gerbil [183]; guinea pig [208]),
% and implies a general organizational principle for the
% mammalian auditory nerve. Fibers of the different SR
% groupings exhibit distinct physiologic features, especially
% in terms of their contribution to the dynamic range of
% hearing [50,59,179] and representation of speech sounds
% [110,180,212,223]. The collective evidence suggests that
% different SR groupings of auditory nerve fibres serve sep-
% arate roles in acoustic information processing. It might be
% that the high-SR fibres with their low thresholds prefer to
% function in quiet, whereas low-SR fibres with their high
% thresholds operate better in loud and noisy environments.




%%% Local Variables: 
%%% mode: latex
%%% mode: tex-fold
%%% TeX-master: "LiteratureReview"
%%% TeX-PDF-mode: nil
%%% End: 
