% This file was converted to LaTeX by Writer2LaTeX ver. 1.0.2 see
% http://writer2latex.sourceforge.net for more info
\documentclass[10pt,a4paper]{article} 
\usepackage[ascii]{inputenc}
\usepackage[T3,T1]{fontenc} 

\usepackage[noenc]{tipa}
\usepackage{tipx} 
\usepackage[geometry,weather,misc,clock]{ifsym}
%\usepackage{../hg/manuscript/style/uomthesis}
\input{../hg/manuscript/user-defined}

\usepackage{pifont} 
\usepackage{eurosym} 
\usepackage{amsmath}
\usepackage{wasysym} 
\usepackage{amssymb,amsfonts,textcomp} 
\usepackage{color}
\usepackage{array,calc} 
\usepackage{natbib} 
\usepackage{tabularx,supertabular}
\usepackage{rotating,calc}
\usepackage{booktabs,ltxtable,lscape}
\usepackage{hhline} 


% Text styles
%\newcommand\textstyleInternetlink[1]{\textcolor{blue}{#1}}
%\newcommand\textstylebibrecordhighlight[1]{#1}
%\newcommand\textstylePageNumber[1]{#1}
% Outline numbering
\setcounter{secnumdepth}{0} \makeatletter
\newcommand\arraybslash{\let\\\@arraycr} \makeatother
% Page layout (geometry)
\setlength\voffset{-1in} \setlength\hoffset{-1in} \setlength\topmargin{1.27cm}
\setlength\oddsidemargin{1.27cm} \setlength\textheight{20.518cm}
\setlength\textwidth{19.05cm} \setlength\footskip{2.441cm}
\setlength\headheight{1.27cm} \setlength\headsep{1.171cm}
% Footnote rule
\setlength{\skip\footins}{0.119cm}
\renewcommand\footnoterule{\vspace*{-0.018cm}\setlength\leftskip{0pt}\setlength\rightskip{0pt
    plus
    1fil}\noindent\textcolor{black}{\rule{0.25\columnwidth}{0.018cm}}\vspace*{0.101cm}}

\newcommand{\um}{$\mu$m}
\newcommand{\umsq}{$\mu$m$^2$}


\author{eagerm} 

\begin{document}

\tableofcontents%

\newpage
\section{Cochlear Nucleus Evidence}
Recent Reviews of characterised cells and projections
\citep{CantBenson:2003,RyugoParks:2003,SmithMassieEtAl:2005,YoungOertel:2004,OertelWrightEtAl:2010}

{\bfseries Table 1: Evidence of Connections}

% \begin{landscape}
%   \begin{tabularx}{\textwidth}{XXXXXXX}
%   \toprule
%  Connection & Transmitter/ Receptor & Synaptic Time Constant
%     (msec) & Bandwidth (octaves or Q10) & No. of Synapses /Strength &
%     Distribution & Delay\\ \midrule

%                       ANI\ensuremath{\rightarrow}TS                       & 
% Glutamate on fast AMPA receptors
% \citep{FerragamoGoldingEtAl:1998a,WentholdHunterEtAl:1993},Monosynaptic
% EPSPs that were blocked by DNQX , an antagonist of AMPA receptors
% \citep{FerragamoGoldingEtAl:1998a}; Slow depolarizations, the source of
% which has not been identified, that lasted between 0.2 and 1 s and were
% blocked APV , the (NMDA) receptor
% antagonist.\citep{FerragamoGoldingEtAl:1998a}; Glutamate and glutamine
% antibodies \citep{HackneyOsenEtAl:1990} whole cell patch recordings NMDA
% \ensuremath{\rightarrow} AMPA during development
% \citep{BellinghamLimEtAl:1998} GluR3, GluR4 main AMPA subunits electron
%              microscopic study \citep{WangWentholdEtAl:1998}              & Rise 0.15 Decay 0.36
% \citep{GardnerTrussellEtAl:2001,Gardner:2000,GardnerTrussellEtAl:1999,Oertel:1983}
%                                                                           & Narrow 0.23-0.39 oct \citep{PalmerJiangEtAl:1996} Q10 5.52{\textpm}1.4 in
% guinea pigs \citep{JiangPalmerEtAl:1996} compared to Q10 of 6.3 in AN;
% narrow and of input 75-100\um in mice \citep{OertelWuEtAl:1990} CS Q10\- =4,
% CT Q10=2 (low cf), 3.67 (high CF) guinea pig \citep{PalmerWallaceEtAl:2003}
%                      5.3 Q10 \citep{RhodeSmith:1986}                      & Small number
% \citep{Cant:1981,FayPopper:1994,ReddCahillEtAl:2002,RyugoWrigthEtAl:1993,Ryugo:1992,RyugoParks:2003}
% see table 1, cat \citep{SmithRhode:1989} estimate 5 to reach AP
% \citep{FerragamoGoldingEtAl:1998a} highly variable 0-6 per soma possibly two
%                   subtypes \citep{JosephsonMorest:1998}                   & Mainly dendritic; 30\% soma coverage
% \citep{Cant:1981,Cant:1982,RyugoWrightEtAl:1993,TolbertMorest:1982a} see
% Table 1 \citep{SmithRhode:1989} some cells had LS vesicles surrounding axon
% initial segment \citep{JosephsonMorest:1998} type II ANFs as well
%                         \citep{BensonBrown:2004}                          & 0.5msec
% \citep{Brown:1993,BrownLedwith:1990,WickesbergOertel:1993} 0.48-0.92 msec,
% mean 0.7 \citep{FerragamoGoldingEtAl:1998a} about 0.7 msec
% \citep{Oertel:1983} Chopper latencies 3.6 {\textpm}0.38 msec
% \citep{RhodeSmith:1986} latency 3.6{\textpm}1.2 msec anesthetised cat 3.5
% msec unanesthetised \citep{RhodeKettner:1987} depolarisation min SPL
% 31.7{\textpm}2.9, AP min SPL 41.8{\textpm}3.8 \citep{PaoliniClareyEtAl:2004}
%                  \\\hline ANI\ensuremath{\rightarrow}DS                   & Glutamate on AMPA receptors
%        \citep{FerragamoGoldingEtAl:1998a,WentholdHunterEtAl:1993}         & Rise 0.15 Decay
%             0.36 \citep{GardnerTrussellEtAl:1999,Oertel:1983}             & 2 oct below, 1 oct above
% \citep{PalmerJiangEtAl:1996} q10 3.56{\textpm}1.38
% \citep{JiangPalmerEtAl:1996} q10 3.1 at 10kHz dual component, stronger below
% CF \citep{PaoliniClark:1999} Q10 1.33-2.87 \citep{PalmerWallaceEtAl:2003} & 
% Many weak inputs \citep{FerragamoGoldingEtAl:1998a} 30 somatic inputs (30\%
%                round); see table 1 \citep{SmithRhode:1989}                & Dense somatic and dendritic
% \citep{Cant:1981,Cant:1982,RyugoWrightEtAl:1993} see Table
%                         1\citep{SmithRhode:1989}                          & As above, maybe less considering time to peak,
% see latencies in \citep{PaoliniClark:1999} Oc latency 2.8 {\textpm}0.09 msec
% \citep{RhodeSmith:1986} depolarisation min SPL 45.6{\textpm}2.7, AP min SPL
% 56.6{\textpm}2.2 \citep{PaoliniClareyEtAl:2004}\\\hline

%                       ANI\ensuremath{\rightarrow}TV                       & AMPA glutamate receptors
%                         \citep{ZhangOertel:1993}                          &    0.4 \citep{GardnerTrussellEtAl:1999}    & Similar to
% AN \citep{SpirouDavisEtAl:1999} mice dendrites lay in \~{}70\um of DCN
%                         \citep{ZhangOertel:1993}                          & \citep{SpirouDavisEtAl:1999,ZhangOertel:1993} ;
%      \~{}4 somatic contacts per cell, 10\% AN \citep{Alibardi:1999}       & 

% Mainly dendritic {\textless} 100\um
%  \citep{Alibardi:1999,Liberman:1993,RubioJuiz:2004,SpirouDavisEtAl:1999}  & As
% above, maybe more to DCN; shocks illicit EPSPs 1.0-1.5 msec
% \citep{ZhangOertel:1993}\\\hline
%                       TS\ensuremath{\rightarrow}TS                        & AMPA and/or
%        NMDA glutamate receptors \citep{FerragamoGoldingEtAl:1998a}        & Similar to
%           primary endings but measured.  Will be more for NMDA.           & Local axon
% collaterals have terminals only in dendritic band \citep{SmithRhode:1989} & 
% Based on rarity of small round vesicles on CS cells, influence is weak
% \citep{SmithRhode:1989} likely to come from 1 input
%                    \citep{FerragamoGoldingEtAl:1998a}                     & dendritic; \citep{JosephsonMorest:1998}
% in chinchilla SS ves \~{}5 per soma (high and low CF), more prevalent at the
%                               axon hillock                                & Min. synaptic delay \\\hline
%                       TS\ensuremath{\rightarrow}DS                        & 

%  AMPA and/or NMDA glutamate receptors \citep{FerragamoGoldingEtAl:1998a}  & As
%                                  above.                                   & Few local axonal collaterals lie in dendritic
%                      plane. \citep{SmithRhode:1989}                       & Based on rarity of small round vesicles on
% CS cells, influence is weak \citep{SmithRhode:1989} late EPSPs observed,
% likely one-2 inputs \citep{FerragamoGoldingEtAl:1998a, OertelWuEtAl:1990} & 
%                                     ?                                     & Min. synaptic delay \\ \hline
%                       TS\ensuremath{\rightarrow}TV                        & AMPA
% Glutamate
% \citep{DoucetRossEtAl:1999,FerragamoGoldingEtAl:1998a,ZhangOertel:1993} No
% TS terminals on TV cells in rats \citep{RubioJuiz:2004} In deep layer guinea
% pig \citep{PalmerWallaceEtAl:2003} small round vesicles
% \citep{Alibardi:1999} ;3 of 4 neurons had late EPSPs to AN shock, very young
%                       mice \citep{ZhangOertel:1993}                       &       Similar to AN AMPA receptors.        & Strong
% On-CF with weak off-CF See fig 13 \citep{OstapoffBensonEtAl:1999} ; mainly
% in iso-band but poor classification in rats
%         \citep{DoucetRossEtAl:1999,FriedlandPongstapornEtAl:2003}         & 
% \citep{OstapoffBensonEtAl:1999} 3 of 4 shocks elicited late EPSPs
%                         \citep{ZhangOertel:1993}                          & Mainly to fusiform cell layer, possibly on ends
% of TV cell dendrites \citep{OertelWuEtAl:1990} CS and CT collaterals in deep
% DCN possibly somatic \citep{PalmerWallaceEtAl:2003} somatic small round
%   vesicles presumably from T stellates guinea pig \citep{Alibardi:1999}   & 0.15
% sec min EPSP latency to VCN Glutamate puffs, main excitation at 0.3 sec Fig
% 11a, AN shock produces late EPSPs about 3msec \citep{ZhangOertel:1993}
% \\\hline
%                       DS\ensuremath{\rightarrow}TS                        & Glycine FerragamoGoldingEtAl:1998a
% could be mixed Gly/GABA \citep{AltschulerJuizEtAl:1993} pleomorphic vesicles
%    of Oc apposed to sustained chopper, fig 14 \citep{SmithRhode:1989}     & 

% 6-13 msec decay time
% \citep{AwatramaniTurecekEtAl:2005,HartyManis:1996,HartyManis:1998,LeaoOleskevichEtAl:2004}
%                                                                           & D stellate axon terminals cover 300 \um of VCN \citep{OertelWuEtAl:1990}
% AVCN collaterals centred on soma isofreq slab as dend, 1 octave above and
% below \citep{ArnottWallaceEtAl:2004} SBW =5.1kHz {\textpm}4.5 kHz all Ch, CS
% 4.66{\textpm}4.45kHz 88{\textpm}19\% suppression, CT 6.28{\textpm} 4.65kHz
%          96{\textpm}5\% suppression \citep{RhodeGreenberg:1994b}          & 
% \citep{FerragamoGoldingEtAl:1998a} See Table 1 \citep{SmithRhode:1989} PL
% \~{}70 (high) \~{}60 (low CF) per soma, \~{}1.7 per axon, FL \~{}20 (highCF)
%                \~{}10 (lowCF) \citep{JosephsonMorest:1998}                & 1 or 2 on soma; many gly and
% mixed gly/GABA on trunks; see Table 1\citep{SmithRhode:1989} more FL
%     vesicles on soma in high CF regions \citep{JosephsonMorest:1998}      & 

% \citep{FerragamoGoldingEtAl:1998a,NeedhamPaolini:2003} 1.2-3.5msec after
% shock to AN \citep{Oertel:1983}\\\hline
%                       DS\ensuremath{\rightarrow}DS                        & 
%                    \citep{FerragamoGoldingEtAl:1998a}?                    & 6-13 msec decay time
% \citep{AwatramaniTurecekEtAl:2005,HartyManis:1996,HartyManis:1998,LeaoOleskevichEtAl:2004}
%                                                                           &                                            & None observed \citep{SmithRhode:1989} IPSPs seen in slice without DCN
%                    \citep{FerragamoGoldingEtAl:1998a}                     &    See table 1 \citep{SmithRhode:1989}     & 
% Min. synaptic delay \\\hline
%                       DS\ensuremath{\rightarrow}TV                        & Glycinergic
% \citep{DoucetRyugoEtAl:1999,OertelWuEtAl:1990,OstapoffMorestEtAl1999,SpirouDavisEtAl:1999,ZhangOertel:1993}
%                                                                           & 

% 6-13 msec decay time
% \citep{AwatramaniTurecekEtAl:2005,FerragamoGoldingEtAl:1998a,HartyManis:1996,HartyManis:1998,LeaoOleskevichEtAl:2004}
%                                                                           & Lateral sidebands equivalent to DS bandwidth
% \citep{OstapoffMorestEtAl:1999,SpirouDavisEtAl:1999} in rats strong ventral
% to iso-frequency band (ie stronger high \ensuremath{\rightarrow}low cf)
% \citep{DoucetRyugoEtAl:1999,FriedlandPongstapornEtAl:2003} generally
% tonotopic arrangement of OnC axons in DCN \citep{ArnottWallaceEtAl:2004}
%     Notch Noise shows asymmetry in projection\citep{ReissYoung:2005}      & Few
% somatic glycine inputs
% \citep{OsenOttersenEtAl:1990,OstapoffMorestEtAl1999:1999,ZhangOertel:1993}
% \~{}4 somatic contacts most of which were Glycine (FP) \citep{Alibardi:1999}
%                                                                           & Few somatic glycine inputs
% \citep{OsenOttersenEtAl:1990,OstapoffMorestEtAl:1999,ZhangOertel:1993} few
%                  somatic contacts \citep{Alibardi:1999}                   & 

% 0.15 msec min EPSP latency to VCN shock , assume IPSP latency similar
% \citep{ZhangOertel:1993}\\\hline
%                       TV\ensuremath{\rightarrow}TS                        & 
% Glycine\citep{OertelWickesberg:1993,OstapoffMorestEtAl:1999,SaintBensonEtAl:1991,WickesbergOertel:1993}
%                 GABA mixed \citep{OsenOttersenEtAl:1990}                  & 
%    \citep{HartyManis:1998,OertelWickesberg:1993,WickesbergOertel:1993}    & 

% Slightly lateral, lighter on-CF \citep{OstapoffMorestEtAl:1999} concentrated
% on CF \citep{ZhangOertel:1993} FL vesicles more dense in high CF
%                       \citep{JosephsonMorest:1998}                        & Many \citep{OstapoffMorestEtAl:1999} see
% Table 1 \citep{SmithRhode:1989} PL \~{}70 (high) \~{}60 (low CF) per soma,
% \~{}1.7 per axon, FL \~{}20 (highCF) \~{}10 (lowCF)
%                       \citep{JosephsonMorest:1998}                        & Soma and mainly trunk
% \citep{AltschulerJuizEtAl:1993} see Table 1\citep{SmithRhode:1989} more FL
% vesicles on soma in high CF regions, some hillock contacts (75\% inhib)
%                       \citep{JosephsonMorest:1998}                        & 2.5 msec from AN shock to inhibition
% \citep{WickesbergOertel:1993} 600usec after AN excitation in choppers
% \citep{Wickesberg:1996} 0.1-0.3 msec glut or shock VCN
% \citep{ZhangOertel:1993}\\\hline
%                       TV\ensuremath{\rightarrow}DS                        & 

% Glycine \citep{OstapoffMorestEtAl:1999,SaintBensonEtAl:1991} Mixed
%                Glycine/GABA \citep{OsenOttersenEtAl:1990}                 & \citep{OstapoffMorestEtAl:1999}
%                                 as above                                  & Slightly lateral, lighter on-CF \citep{OstapoffMorestEtAl:1999}
% concentrated on CF \citep{ZhangOertel:1993} No inhibitory sidebands evident
%                                   in RA                                   &    See table 1 \citep{SmithRhode:1989}     & see Table 1
%                          \citep{SmithRhode:1989}                          & As above in TV\ensuremath{\rightarrow}TS\\\hline
%                      ANI,II\ensuremath{\rightarrow}G                      & 
% \citep{Cant:1992,FerragamoGoldingEtAl:1998a,RyugoWrightEtAl:1993,Ryugo:1992,RyugoParks:2003}
%             diffuse release sites \citep{HurdHutsonEtAl:1999}             & 
%    \citep{GardnerTrussellEtAl:1999} slower \citep{HurdHutsonEtAl:1999}    & Wide
% Dynamic range \citep{GhoshalKim:1997} dendritic field 50 and 100 $\mu $m
%                    \citep{FerragamoGoldingEtAl:1998a}                     &                                            & Dendritic; contact dend 0.5-1 \um
%                width) in GCD, SCC \citep{BensonBrown:2004}                & 1.3 msec minimum
% \citep{FerragamoGoldingEtAl:1998a} typeII up to 10msec estimated from axon
% width \citep{Brown:1993}\\\hline
%                     Granule\ensuremath{\rightarrow}G                      & NMDA
%                    \citep{FerragamoGoldingEtAl:1998a}                     &      \citep{GardnerTrussellEtAl:1999}      &   & 
%                    \citep{FerragamoGoldingEtAl:1998a}                     &                                            & \\\hline
% DS\ensuremath{\rightarrow}G,
%                                 granule ?                                 &                  Glycine                   & 

%                                                                           &                                            & D stellates have collaterals in GCD fig 4 \& 5 \citep{OertelWuEtAl:1990}
%    Gly IPSP observed in 1 of 5 cells \citep{FerragamoGoldingEtAl:1998}    &                                            & 
% Short, see Fig 3B in \citep{FerragamoGoldingEtAl:1998}\\\hline
% Golgi/GABA
%                       \ensuremath{\rightarrow}TS ?                        & Chinchilla
%          \citep{FerragamoGoldingEtAl:1998,JosephsonMorest:1998}           & 

%                    \citep{AwatramaniTurecekEtAl:2005}                     & High \ensuremath{\rightarrow}low CF
%                       \citep{JosephsonMorest:1998}                        &    See table 1 \citep{SmithRhode:1989}     & See
%                      table 1 \citep{SmithRhode:1989}                      & \\\hline
% Golgi/GABA
%                        \ensuremath{\rightarrow}DS                         & 
% \citep{EvansZhao:1998,FerragamoGoldingEtAl:1998a,Mugnaini:1985,MugnainiOsenEtAl:1980,
%                           SaintMorestEtAl:1989}                           &                                            & Wideband to wideband on CF
% \citep{EvansZhao:1998} hyperpolarising effects above high freq edge in mid
%                   CF neurons \citep{PaoliniClark:1999}                    & \~{}20 weak
%                    inputs\citep{SaintMorestEtAl:1989}                     & 

%    Dendrites reach into granule cell domain \citep{OertelWuEtAl:1990}     & 

% Hyperpolarisation occurs 10-15 msec after click
% \citep{PaoliniClark:1999}\\\hline
% Non-Auditory \ensuremath{\rightarrow} GCD
%                                                                           & Review see \citep{OhlroggeDoucetEtAl:2001} &   & & & & \\\hline
%                      ACh\ensuremath{\rightarrow} TS                       & 

%      Excitatory response to Acetylcholine \citep{FujinoOertel:2001}       & Nicotinic
% and muscarinic receptors.  Inhibit {\textquoteleft}leaky{\textquoteright}
% potassium currents (volt-insensitive)

%                         \citep{FujinoOertel:2001}                         &              Presumably on CF              & ? & SS vesicles (perhaps
%        ACh) contact the axon hillock \citep{JuizHelfertEtAl:1996a}        & \\\hline

%                      ACh\ensuremath{\rightarrow} DS                       & No response to ACh
% \citep{FujinoOertel:2001} Indirect excitation via granule cells
%          \citep{MuldersPaoliniEtAl:2003,MuldersWinterEtAl:2002}           &                                            &   & & & 


% \\ \bottomrule
% \end{tabularx}
% \end{landscape}



\section{Cell Morphology}

\begin{landscape}
%\afterpage
{\small\LTXtable{210mm}{CellMorphologyTable}}  
\end{landscape}

% \begin{flushleft}

%   \begin{tabularx}{\textwidth}{cXXXX}
% \toprule 
% Cells & Cell body xsection area & Soma diameter & Dendrite span &Other\\\midrule
%                                TS                                & soma area 262.3{\textpm}62 planar cell in rat
% \citep{DoucetRyugoEtAl:1999} area 249{\textpm}95\umsq \citep{DoucetRyugo:1997}
% CS 648 \umsq \citep{SmithRhode:1989} cat 366{\textpm}100\umsq
% \citep{ReddCahillEtAl:2002} soma xsectional area 200\umsq (CS), 240 (CT)
% \citep{PalmerWallaceEtAl:2003} type I stellate (IC label) chinchilla DAP
% (high CF) 192{\textpm}47 $\mu $m2,PV (low CF) 253{\textpm}63 $\mu $m2
%                   \citep{JosephsonMorest:1998}                   & 16-20\um diam \citep{DoucetRyugo:1997} type I
% stellate (IC label) chinchilla DAP (high CF) 14.5{\textpm}2.0 $\mu $m,
% PV(low CF) 17.6{\textpm}2.7 $\mu $m \citep{JosephsonMorest:1998} & 2-3 main
% dend, total dend length \~{}2000\um \citep{PalmerWallaceEtAl:2003} 75-100\um
% and 150-300\um parallel,3-4 primary dendrites \citep{SmithRhode:1989}
% dendrites emanated from soma in one direction (rarely bipolar) up to 200$\mu
% $m, 2-5 $\mu $m in diam chinchilla \citep{JosephsonMorest:1998}  & axon width
% .5-1.1\um \citep{OertelWuEtAl:1990} hillock initial segment averaged
% 2.5{\textpm}0.96 $\mu $m proximal diam and up to 13 $\mu $m long, distal
% initial segment 0.7{\textpm}0.2 $\mu $m,myelination began about 20 $\mu $m
% from the soma \citep{JosephsonMorest:1998}\\\hline
%                                TV                                & & 18.4{\textpm}3.2 \um diam guinea pig \citep{SaintBensonEtAl:1991} 13.34{\textpm}2.17 \um (11-16
%             range) guinea pig \citep{Alibardi:1999}              & Dend and axonal arbors restricted
%                      to isofrequency lamina                      & \\\hline
%                                DS                                & 963 \umsq xsection OC
% \citep{SmithRhode:1989} 501{\textpm}168 \umsq rat \citep{DoucetRyugoEtAl:1999}
% 466{\textpm}137\umsq \citep{DoucetRyugo:1997} cat xsection 571{\textpm} 228
% \umsq \citep{ReddCahillEtAl:2002} OC xsectional area 450\umsq
%                  \citep{PalmerWallaceEtAl:2003}                  & guinea pig 27\um diameter
% \citep{ArnottWallaceEtAl:2004} 22-26 \um diam \citep{DoucetRyugo:1997} 20-30
%                  \um \citep{PaoliniClark:1999}                   & 
% 250-350\um span of dend \citep{DoucetRyugo:1997} with aspinous dendrites, 4
% of 5 cells had 4 main dendrites, total dend length 6222 to 7351 \um
% (mean-6665 \um), dendrites extended widely in all directions,\~{}70 \um
% perpendicular to AN,3-6 primary dendrites at right angles to AN
%                     \citep{SmithRhode:1989}                      & axon width 0.7-1.2\um
% \citep{OertelWuEtAl:1990}\\\hline
%                              Golgi                               & 15 $\mu $m in diameter
%                    FerragamoGoldingEtAl:1998                     & & Smooth, tapering dendrites, between 50 and 100
% $\mu $m long, emanated in all directions \citep{FerragamoGoldingEtAl:1998}
%         Also see \citep{Cant:1993,MugnainiOsenEtAl:1980}         & A dense, axonal plexus,
% limited to the plane of the granule cell domain, extended about 250 $\mu $m
% from the soma in all directions \citep{FerragamoGoldingEtAl:1998} \\
% \bottomrule
% \end{tabularx}
% \end{flushleft} 


\section{ Intracellular  Physiology}

\begin{landscape}
%\afterpage
{\small\LTXtable{210mm}{IntracellularTable}}  
\end{landscape}

% \begin{flushleft}

%   \begin{tabularx}{\textwidth}{cXXXXXX}
%  \toprule                           
% Cells    & Current Clamp &        Membrane Currents         & Time Constant (msec) &   I-V    &    Input R    & Papers\\
% \midrule
%                                  TS                                   & Type 1, single exponential undershoot
%       \citep{FengKuwadaEtAl:1994,ManisMarx:1991,WuOertel:1984}        & No Low threshold
% K \citep{ManisMarx:1991} IA has a role in modulating the rate of repetitive
% firing.  Effect of Inhibition on T stellate cells could be to reset IA
%                      \citep{RothmanManis:2003c}                       & 6.5{\textpm}5.7 msec \citep{ManisMarx:1991}
% type I 9.1{\textpm}4.5 \citep{ManisMarx:1991} 6.2 to 18.0 msec
% \citep{FengKuwadaEtAl:1994} 6.9{\textpm}3msec, 10-90\% rise time was
%          1.05{\textpm}0.4msec \citep{IsaacsonWalmsley:1995}           & Linear
%                        \citep{ManisMarx:1991}                         & 

% 447{\textpm}265 Mohm isolated guinea pig stellate cell type 1 current clamp
% \citep{ManisMarx:1991} 44 to 151 M$\Omega $ (mean 89.4 {\textpm}24.4) mouse
% slice prep \citep{FerragamoGoldingEtAl:1998a}
% {\textquotedblright}stellate{\textquotedblright} 231{\textpm}113M$\Omega $,
% 14.9{\textpm}9pF primary membrane capacitance, room temp rat
% \citep{IsaacsonWalmsley:1995} dog {Bal, Baydas, Naziroglu 2009} 176+/- 35.9
%             Mohm membrane time constant 8.8 +/-1.4 (n=21)             & \\\hline
%                                  TV                                   & Linear,
%   regular spiking, double exp. Undershoots \citep{ZhangOertel:1993}   & Double
%               undershoots suggest ILT , Regular spiking               &               &     \citep{ZhangOertel:1993}     & 
% 100M{\textpm}20, but then state 85{\textpm}10 in table 1
%                       \citep{ZhangOertel:1993}                        & 
% \citep{EvansNelson:1973,WickesbergOertel:1990,WickesbergOertel:1993,WickesbergOertel:1988,WickesbergWhitlonEtAl:1991,Wickesberg:1996,YoungBrownell:1976,YoungVoigt:1981,ZhangOertel:1993}\\\hline
%                                  DS                                   & 

% {}-56{\textpm}3.2 mV RMP see fig 15 Double expon. Undershoot
%                \citep{PaoliniClark:1999,WuOertel:1984}                & Type I-i have high thresholds
% probably mediated by small ILT \citep{RothmanManis:2003c} Membrane
% properties of Oc cell have not bee adequately characterised, bu the
% information that is available (d stellate in mouse
% \citep{OertelWuEtAl:1990}) suggests that the low-threshold potassium channel
% that is important in extending the phase-locking range of bushy cells
% \citep{ManisMarx:1991,Oertel:1983} is not present in Oc neurons
%                      \citep{WhiteYoungEtAl:1994}                      &     Fast      & Linear \citep{PaoliniClark:1999} & 40M
% ohm \citep{OertelWuEtAl:1990} 96.2 {\textpm} 27.8 M$\Omega $ mouse slice
%                prep \citep{FerragamoGoldingEtAl:1998a}                & \\\hline
%                                 Golgi                                 & Regular spiking
% with overshooting action potentials and double exponential undershoot &               &                                  & 
%              Inward rectifying FerragamoGoldingEtAl:1998              &   130 Mohm    & 
% FerragamoGoldingEtAl:1998 \\
% \bottomrule
% \end{tabularx}
% \end{flushleft} 

\section{ Extracellular  Physiology}

\begin{landscape}
%\afterpage
{\small\LTXtable{210mm}{PhysiologyTable}}  
\end{landscape}


Table 1 \citep{SmithRhode:1989} Percentage Coverage in one CS and one OC neuron
in cat Electron Microsc.

\begin{flushleft}
  \begin{tabularx}{\textwidth}{XXXXX}
\toprule
                              &  \% area coverage   &           Round            & Pleomorphic (GABA) & Flat (GLy)   \\
               &                     & \multicolumn{3}{|c|}{Percent number (total percent coverage area)} \\ \midrule
             CS               &                     &  \\
            soma              &   21 (range 6-38)   &         36 (10.5)          &     47      & 17   \\
          Prox dend           &         46          &          43 (29)           &     34      & 23   \\
          Dist Dend           &         22          &          40 (12)           &     33      & 27   \\ \midrule
             OC               &                     &                            &             & \\
            soma              & 87 (never below 80) &          36 (43)           &     36      & 28   \\
          Prox dend           &         81          & 62                    (59) &     18      & 20   \\
          Dist Dend           &         22          &         25.5 (7.5)         &     41      & 33   \\
Axon                  hillock & Densely innervated  &                            &             & \\
\bottomrule
\end{tabularx}
\end{flushleft}

\subsection{\citep{JosephsonMorest:1998}}

\textsf{Inhibitory type endings composed most of the axonal endings contacting
  stellate cell somata and axons in the present sample from the AVCN. The
  inhibitory synapses are represented by pleomorphic-vesicle end- ings
  predominantly and by flattened-vesicle endings less commonly. Pleomorphic
  vesicles are usually asso- ciated with GABA, and flat vesicles with glycine,
  al- though co-localization has been reported (MatusDennison:1971; McLaughlin
  et al., 1974; Altschuler et al., 1993; Juiz et al., 1996). Feedback pathways
  from the superior olivary complex probably provide the greatest numbers of
  inhibitory endings, including both GABAergic and glycinergic neurons
  (BensonPotashner:1990; Ostapoff et al:1997), although the possibility that
  some of these may be cholinergic has been raised
  (McDonaldRasmussen:1971). Feed-forward projec- tions from the DCN may supply a
  number of glyciner- gic and GABAergic inputs (WickesburgOertel:1990; Saint
  Marie et al., 1991; Kolston et al., 1992). Glycinergic commissural connections
  with the opposite cochlear nucleus also exist (Wenthold, 1987; SchofieldCant,
  1996a).}







\citep{ReddCahillEtAl:2002} Tables 1 and 2 (means)

\begin{table}[h]
\caption{\citep[Table 1 in ][]{ReddCahillEtAl:2002}}

  \begin{tabularx}{\textwidth}{XXX}
\toprule
                                & Type I & Type II \\\midrule
Cell body xsectional area \umsq &  369   & 481     \\ 
     No. boutons (somatic)      &  13    & 28      \\
          \% coverage           &  27.8  & 75      \\  
       Bouton area \umsq        &  2.2   & 2.87    \\
SV{\textquoteright}s per \umsq  &  54.8  & 81.5    \\
   PostSynDensity area \umsq    & 0.107  & 0.138   \\
\bottomrule
\end{tabularx}
\end{table}

\begin{table}[h]
\caption{Bouton ultrastructure data ($\pm$ SD) \citep[Table 2 in ][]{ReddCahillEtAl:2002}}
\begin{tabularx}{\textwidth}{XXXX} \toprule
Multipolar cell type           &  Bouton area (lm2)         &    SVs per lm2                   &     PSD area (lm2)            \\ \midrule
Normal cats, type I MP cells   &  2.23  $\pm$     1.8       &    54.8  $\pm$   26              &     0.107  $\pm$     0.09     \\
(N = 6 cats, 10 cells)         &  (n    =     13 boutons)   &    (n     =       13  boutons)   &     (n      =        19 PSDs) \\
Normal cats, type II MP cells  &  2.87  $\pm$     1.5       &    81.5  $\pm$   40              &     0.138  $\pm$     0.09     \\
(N = 6 cats, 16 cells)         &  (n    =     16 boutons)   &    (n     =       16  boutons)   &     (n      =        28 PSDs) \\        
DWCs, type I MP cells          &  2.83  $\pm$     1.6       &    73.5  $\pm$   27              &     0.087  $\pm$     0.10     \\
(N = 4 cats, 14 cells)         &  (n    =     15 boutons)   &    (n     =       15  boutons)   &     (n      =        54 PSDs) \\        
DWCs, type II MP cells         &  1.94  $\pm$     1.1       &    45.9  $\pm$   23              &     0.184  $\pm$     0.30     \\           
(N = 4 cats, 15 cells)         &  (n   =      19 boutons)   &    (n     =       19 boutons)    &     (n      =        73 PSDs) \\
\bottomrule  
\end{tabularx}
\captionsize{ANOVA revealed that there were no signicant dierences between groups.}
\end{table}




\subsection{\citep{ArnottWallaceEtAl:2003}}
Oc( n=5)/OL(n=2) minimum thresholds that varied from 23 to 70 dB SPL and best frequencies varied from 0.98 kHz to 11.1 kHz
ovoid somata (diameter $\sim$27 $\mu$m) 

Four of the On-C cells had 4 primary dendrites, while one had 7. The total
dendritic length (sum of all dendrites) of the 4 re- constructed cells ranged
from 6222 to 7351 lm (mean-6665 lm). The dendrites extended widely in all
directions. The restricted width of the cochlear nucleus meant that, in the
mediolateral plane, the dendrites often spanned almost its entire extent. The
majority of the dendrites remained largely un- branched and tapered before
giving rise to a few fine, distal branches.





\citep{RhodeSmith:1986} descriptive statistics (note most values are affected by dependencies)

\begin{flushleft}

  \begin{tabularx}{\textwidth}{XXXX}
\toprule
                   &  AN  &  Onset Chopper  & Choppers    \\\midrule
Threshold (dB SPL) &  14  &  23  & 19   \\
SR   (spikes/sec)  &  41  &  6   & 29   \\
       Rmax        & 254  & 258  & 380  \\
     DR (dB)       &  22  &  57  & 29   \\
       Q10         & 4.8  & 3.3  & 5.3  \\
       Lat         & 2.4  & 2.8  & 3.6  \\
      SD lat       & 0.73 & 0.09 & 0.38 \\
        CV         & 0.75 & 0.67 & 0.32 \\
PTS (peak/steady)  & 5.2  & 31.3 & 5.0  \\
\bottomrule
\end{tabularx}
\end{flushleft}

\subsection{Envelope Modulation in Aud Brainstem}
\citep{CondonChangEtAl:1995,FrisinaKarcichEtAl:1996,FrisinaSmithEtAl:1990a,FrisinaSmithEtAl:1990,FrisinaWaltonEtAl:1993,KimSirianniEtAl:1990,Moller:1972,Moller:1976,ReesMoller:1983,ReesMoller:1987,ReesPalmer:1989,Rhode:1998,Rhode:1994,WangSachs:1992,ZhaoLiang:1995,ZhaoLiang:1997}

reviews \citep{Langner:1992,RhodeGreenberg:1994a}

Inhibitory Neurotransmitters
\citep{BackoffPalombiEtAl:1997,Caspary:1990,CasparyBackoffEtAl:1994,CasparyHaveyEtAl:1979,CasparyPazaraEtAl:1987,EbertOstwald:1995,EvansZhao:1993,MartinDicksonEtAl:1982,PalombiCaspary:1992,WickesbergOertel:1990,ZhaoEvans:1990}

In chinchilla (Backoff et al. 1999):


\begin{itemize}
\item
  \citep{CasparyBackoffEtAl:1994,CasparyHelfertEtAl:1997,CasparyPalombiEtAl:1993,EbertOstwald:1995,PalombiCaspary:1992}
  show that glycinergic and GABAergic inhibition can selectively shape the
  responses in the CN to CF tones bursts in both quiet and noise{\dots} and in
  the control of dynamic range, discharge rate, shaping temporal response
  patterns and frequency response areas.
\item \citep{WangSachs:1995,WangSachsEtAl:1992} in their modelling study they
  showed that the inclusion of somatic inhibitory inputs could account for the
  enhanced envelope coding observed in real neurons.  They speculated that
  inhibition raised the threshold for spike activation in VCN neurons and
  increased the degree of envelope modulation in the resulting period histogram.
\item Inhibitory effects at low to mid modulation (ie where energy from
  sidebands is closest to the carrier CF) demonstrate how this same focused
  inhibition could reduce the discharge rate and increase the synchronisation
  and modulation gain at selective frequencies,similar effect in rats
  \citep{Moller:1972}
\end{itemize}

\subsection{\citep{PalombiCaspary:1992}}
Possible sources of GABAergic inhibition to PVCN
neurons include fibres projecting from the SOC, the IC, the LL, the iDCN and
cDCN and/or other neurons in the PVCN.


\begin{itemize}
\item GABA affects discharge rate within the excitatory response area without  significantly compressing the RA.
\end{itemize}

GABA reduces tonic activity and increase
\subsection{(Kulesza et al. 2002) rat cell counts}

\begin{table}[h]
\caption{Cochlear nuclei (excluding granule cells) cell counts}
\begin{tabularx}{\textwidth}{XXXX}
\toprule
         Substructure           & Abbreviation & Number & S.E.M.a \\ \midrule
Anteroventral cochlear nucleus  &     AVCN     & 12800  & 460 \\
Posteroventral cochlear nucleus &     PVCN     &  8500  & 77 \\
    Dorsal cochlear nucleus     &     DCN      &  9000  & 251 \\
\bottomrule
\end{tabularx}
\footnotesize{SEM standard error of the mean}
\end{table}

\section{Auditory Nerve Innervation }

\citep{SmithSpirou:2002} 
Review in Integrative functions in the mammalian auditory pathway

The projection patterns of AN fibers to the cat DCN and VCN have been described in detail \citep{FeketeRouillerEtAl:1984,Liberman:1991,Liberman:1993,RoullierCronin-SchreiberEtAl:1986,Ryugo:1992,RyugoParks:2003}
and there are some spontaneous rate differences.  In the AVCN, low SRs have the
greatest number of terminals and contact the most cells, and high SR fibers the
least.  The small cell cap or rind region of the VCN is innervated mostly by low
and medium SRs.  Within the core region, two of the principle cell types, AVCN
multipolar cells and globular bushy cells, are contacted mostly by low-medium
and high SR fibers, respectively.

Review by \citep{Parks:2000}

\citep{PerneyKaczmarek:1997}

\citep{IsaacsonWalmsley:1995} rat AVCN

EPSCs evoked in stellate cells by stimulation of the auditory nerve were graded
with stimulus strength, indicating a high degree of convergence of input to
these cells. At depolarised membrane potentials, EPSCs evoked in stellate
neurons had dual-component potentials.  The slow component was blocked by the
NMDA antagonist APV, and the fast component was abolished by the non-NMDA
antagonist CNQX.

See Ryugo 2003 for review of innervation by AN

\subsection{Speech in AN}
B.J. May, A.Y. Huang, G.S. Le Prell, R.D. Heinz, Vowel formant
frequency discrimination in cats: comparison of auditory nerve
representations and psychophysical thresholds, Audiol. Neurosci. 3
(1996) 135--162.

M.B. Sachs, R.L. Winslow, C.C. Blackburn, Representation of
speech in the auditory periphery, in: W.M. Cowan (Ed.), Auditory
Function: Neurobiological Bases of Hearing, Wiley, New York,
1988, pp. 747--7

X. Wang, M.B. Sachs, Neural encoding of single-formant stimuli in
the cat. I. Responses of the auditory nerve fibers, J. Neurophysiol.
70 (1993) 1054--1075.

E.D. Young, M.B. Sachs, Representation of steady-state vowels
in the temporal aspects of the discharge patterns of populations
of auditory-nerve fibers, J. Acoust. Soc. Am. 66 (1979) 1381--
1403.

Sachs MB, Voigt HF, and Young ED. Auditory nerve representation of vowels in
background noise. J Neurophysiol 50: 27-45, 1983.

Sachs MB and Young ED. Encoding of Steady-state Vowels in the Auditory Nerve:
Representation in Terms of Discharge Rate. J Acoust Soc Am 66: 470 - 479, 1979.

Generally defined in terms of SR using either rate-place, temporal place or a
mixture of methods.

\section{Individual cells}
\subsection[T stellate cells]{T stellate cells}

T Stellate cells project to the contralateral inferior colliculus in mice
\citep{OertelWuEtAl:1990,RoullierRyugo:1984} Ryugo et al. 1981 and
cats \citep{Adams:1979,Adams:1983,Cant:1982,Oliver:1987,Osen:1972,RothAitkinEtAl:1978}.

In mice slices \citep{FerragamoGoldingEtAl:1998a}


\begin{itemize}
\item T stellate cells form one of the parallel ascending auditory pathways from
  the ventral cochlear nucleus to the inferior colliculi.  In considering the
  role of these neurons in the auditory pathway, the significance of the pattern
  of convergence of auditory nerve fiber inputs and the interaction of those
  synaptic inputs with the intrinsic electrical properties to generate the
  chopper responses to tones has been appreciated
  \citep{BanksSachs:1991,MolnarPfeiffer:1968;Oertel:1983,WangSachs:1995,WhiteYoungEtAl:1994,WuOertel:1984}
  . The present study indicates that neuronal circuits that provide long-lasting
  excitatory and inhibitory feed-forward interactions also contribute
  significantly to the responses of T stellate cells to activation of auditory
  nerve fibers.
\item The new results raise the question what is the source of the additional
  inputs. T stellate cells are known to receive glycinergic inhibition from
  tuberculoventral cells \citep{WickesbergOertel:1990} .  The present
  experiments show that, T stellate cells are a possible source of feed forward
  excitation and D stellate cells are a possible source of feed-forward
  inhibition. The finding that T stellate cells are influenced by GABAergic
  neurons is particularly intriguing. Golgi cells in the superficial granule
  cell domain are The only known source of GABA intrinsic to the VCN. They do
  not receive input from the large, myelinated type I auditory nerve fibers but
  may be innervated by the small, unmyelinated, type II auditory nerve fibers.
  These experiments thus raise the possibility that T stellate cells are
  influenced by neurons in the superficial graule layer and that they are
  influenced directly by acoustic input from the large, myelinated type I
  auditory nerve fibers and also indirectly by the small, unmyelinated, type II
  auditory nerve fibers through Golgi cells.
\item In mice, T stellate cells are contacted heavily on the cell body unlike in
  cats.  The short latency, sharply-timed responses to the onset of tones
  indicate chopper and onset-chopper units receive input from the large,
  myelinated ANFs.  In slices from mice, both D and T stellate cells respond to
  shocks of the AN with EPSPs.  As thresholds for EPSPs are low and latencies
  are {\textless} 1 msec, the input is probably form myelinated auditory nerve
  fibres.  Anatomic and electrophysiological evidence indicates that few ANFs
  innervate a T stellate cell.  The orientation of the dendrites of T stellate
  cells parallel to the path of the ANFs ans spanning a small proportion of the
  tonotopic axis indicates that T stellate cell dendrites are positioned to
  receive input from a limited group of fibres. The result that the amplitude of
  responses to shocks of the auditory nerve grow in three or four discrete jumps
  with shock strength indicates that the number of fibers innervating one T
  stellate cell in a mouse is small, perhaps as small as three or four (Fig. 1)
  . As any of the jumps in amplitude could have resulted from the recruitment of
  more than one fiber and as it is possible that inputs might have been cut or
  damaged, this estimate represents a minimum. No such Subthreshold jumps were
  found in octopus cells. This result also indicates that models of choppers,
  based on what is known in cats, that require the integration of many inputs
  might be oversimplified
  \citep{BanksSachs:1991,MolnarPfeiffer:1968,WangSachs:1995}.
\item NMDA excitation: NMDA-mediated slow depolarizations were generated with
  shock strengths greater than those required to produce apparently maximal
  monosynaptic EPSPs.  This finding suggests that different sources of Glut
  input may activate different populations of receptors.  It raises the
  possibility that type I auditory nerve fibres act primary through AMPA
  receptors, as they are known to do in other vertebrate cochlear nuclei
  \citep{RamanZhangEtAl:1994,ZhangTrussell:1994} whereas other sources of
  excitation, alone or in combination, are required to activate NMDA receptors.
  It is conceivable that type II ANFs contribute tpo the long slow
  depolaristation.  Small, unmyelinated fibres would be expected to have higher
  thresholds for shocks than large, myelinated fibres and their responses would
  be expected to be later (ie longer propagation time)
\item Recurrent excitation from T stellate cells: Late EPSPs are the source of
  interneurons in the VCN as cut axons are not spontaneously active.
  Application of glut to activates EPSPs indicating the dendrites of excitatory
  interneurons are accessible form the bath. T stellate cells terminate locally
  in the multipolar cells area of the PVCN
  \citep{FerragamoGoldingEtAl:1998a,OertelWuEtAl:1990}. This area is occupied by
  T stellate cells and occasional D stellate and bushy cells, some or all of
  which are therefore presumably their targets. The ultrastructure of T stellate
  cell terminals and functional studies of the inputs to the inferior colliculi
  is consistent with their being excitatory \citep{Oliver:1984,
    1987,SmithRhode:1989}. The present experiments provide functional evidence
  in support of the conclusion that T stellate cells mediate late EPSPs. If T
  stellate cells are excited by other T stellate cells, then disynaptic EPSPs
  that reflect the firing of other stellate cells should be observed under
  similar conditions as stellate cell firing. The present experiments reflect
  the parallel nature of T stellate cell firing and late EPSPs under five
  experimental conditions. 1) Stellate cells consistently are brought to
  threshold \~{}1 ms after shocks to the auditory nerve. Disynaptic EPSPs with
  latencies of \~{}1.6 ms are observed but in the presence of monosynaptic EPSPs
  and disynaptic IPSPs the early disynaptic EPSPs are sometimes difficult to
  resolve. 2) Strong shocks evoke a long, slow depolarization in T stellate
  cells that causes T stellate cells to fire hundreds of milliseconds after a
  strong shock to the auditory nerve.  Strong shocks also evoke very late EPSPs
  in T stellate cells. 3) APV reduces late firing and late EPSPs in T stellate
  cells. 4) The removal of extracellular Mg2/ enhances firing as well as late
  EPSPs. 5) Strychnine and bicuculline enhance firing as well as late EPSPs in T
  stellate cells. In summary, although the results of the present experiments
  are consistent with the conclusion that T stellate cells excite one another,
  it does not rule out the possibility that other, hitherto unknown, cells
  contribute to the excitation.
\end{itemize}

In barbiturate cats \citep{BlackburnSachs:1989} the chopper population
appears to be a homogeneous population with respect to first spike latency,
spontaneous rate, and phase-locking to
tones and a continuum with respect to discharge regularity, LTB P/S ratio, and
sustained discharge rate \citep{Bourk:1976,RhodeSmith:1986}. However,
examination of discharge rate and regularity as functions of time through the
response to STBs reveals several distinctive response patterns (Young et
al. 1988a). In one pattern, the unit undergoes essentially no rate adaptation
over the duration of a response to a STB (25 ms,Fig. 4A) these
{\textquotedblleft}sustained{\textquotedblright} choppers typically exhibit
little rate adaptation during the response to a LTB (400 ms; Fig. 4A, Inset) and
invariably have discharge regularities that are in the more regular one-half of
the chopper population (CVs {\textless} 0.35, and most have CVs {\textless}
0.25,Fig. 6B). Another distinctive pattern is one in which there is a very rapid
increase in mean IS1 (decrease in discharge rate) accompanied by a sharp
increase in discharge irregularity,both of these parameters appear to stabilize
(or at least to change much less rapidly) within 5- 10 ms of the response onset
(Fig. 4, B and C). Units with this {\textquotedblleft}transiently
adapting{\textquotedblright} response invariably
have spike discharge patterns that fall into the more irregular one-half of the
chopper population (CVs {\textgreater} 0.25 and most {\textgreater}
0.35,Fig. 6C). These two chopper patterns,
{\textquotedblleft}sustained{\textquotedblright} and
{\textquotedblleft}transiently adapting{\textquotedblright}
\citep{Bourk:1976,YoungRobertEtAl:1988, YoungShofnerEtAl:1988,
  BlackburnSachs:1989}, have IS1 mean and CV patterns that are clearly
distinguishable by visual inspection and do, in fact, generate two nearly
nonoverlapping populations with respect to discharge regularity. However, in
this study, these two easily distinguishable chopper patterns constitute only
onethird of the chopper units recorded. The remaining units all have firing
rates that decrease more or less continuously over the first 15 ms after
response onset.  Preliminary studies indicate that both the mean IS1 and
discharge regularity stabilize within 25-40 ms,these units thus undergo a rate
and regularity adaptation that is transient, but that occurs over a longer time
scale than that of the units referred to as {\textquotedblleft}transiently
adapting{\textquotedblright} choppers.  We have referred to this third pattern
as {\textquotedblleft}slowly adapting. {\textquotedblright}
{\textquotedblleft}Slowly adapting{\textquotedblright} choppers are found with
regularities that span virtually the entire range of the chopper population and
form a continuum with the {\textquotedblleft}sustained{\textquotedblright}
choppers at the more regular end of their spectrum. We have grouped the more
regular {\textquotedblleft}slowly adapting{\textquotedblright} choppers with the
{\textquotedblleft}sustained{\textquotedblright} choppers to form the ChS
population and have grouped the more irregular {\textquotedblleft}slowly
adapting{\textquotedblright} choppers with the {\textquotedblleft}transiently

adapting{\textquotedblright} choppers to form the ChT population.  Further
investigation of the slowly adapting choppers, with the use of 50-ms STBs, may
show a different measure or parameter to be preferable for partitioning this
population. In the companion paper, we show that this somewhat arbitrary
division of the chopper population seems to make some sense in terms of the
responses of these units to complex stimuli (Blackburn and Sachs, unpublished
observations). Finally, one caveat is worth emphasizing. To categorize low-BF
units with confidence, STBs must be presented asynchronously. Phase-locking can
obscure chopping peaks in the responses of chopper units with BFs {\textless}
1.0 kHz (Fig. 9). In addition, it can be very difficult to detect
{\textquotedblleft}notches{\textquotedblright} in the presence of phase-locking
in the responses of primary-like units with BFs up to -4.0 kHz because the width
of a typical notch, 0.25- 1.5 ms, is on the same time scale as the troughs in
phase-locked PSTHs.


\citep{RhodeGreenberg:1994b} pentobarbital cats

Interval histogram to tones is symmetric, narrow and unimodal.

Suppressive regions in MRA are extensive and deep adjacent to the
unit{\textquoteright}s excitatory area. (fig3)

suppression thresholds for upper and lower sidebands may differ (CS fig3.A)

Some units had wide suppression bandwidths (fig3B CT unit)

In mice \citep{OertelWuEtAl:1990} type I stellate cells have APs that did not
overshoot 0 mV with single exponential undershoot.  T stellates in mice are
distinguished by tufted endings of dendrites.



\citep{PaoliniClareyEtAl:2005,PaoliniClareyEtAl:2004} rat urethane

Inhibition plays a large part in T stellate response as ChS and two types of ChT
and in the latency.

Inhibition is thought to improve both rate-based and time-based frequency code

ChS\ \ Exc {\textgreater}{\textgreater} Inh

ChT1 E {\textgreater} I\ \

ChT2\ \ E=I

Given the precisely timed and prominent onset, D stellate cells presumably have
their greatest effect on the onset component of a T stellate
cell{\textquoteright}s tonic response to steady-state tonal stimulus.
{\textquotedblleft}The results led us to reassess the traditional view of the
role of wideband inhibition provided by D stellate cells and its contribution to
frequency coding in T stellate cells.{\textquotedblright}

Lateral inhibition observed outside CF similar to other studies
\citep{FerragamoGoldingEtAl:1998a,FujinoOertel:2001,RhodeGreenberg:1994b}. Hyperpolarisation
in T stellate cells is of short duration and follows response of D stellate
cells, at edge of response area fast inhibition occurred during initial
depolarisation and was associated with longer AP latencies to off-CF tones (not
related to shift in threshold).  Sustained hyperpolarisation was also evoked by
off-CF tones- sustained suppression of evoked activity or activity evoked by
noise.  The presence and effects of fast, short duration inhibition described
here indicate that the classic view of lateral inhibition in VCN T stellate
cells should include first spike latency as one of its outcomes.


\begin{itemize}
\item In response to CF tone, excitation in T stellates will arrive before
  inhibition from D stellates, which have higher thresholds and are delayed by
  propagation
\item Edge-CF tones will evoke EPSPs and IPSPs which will delay the first AP,
  thus differeing from the On-CF first spike latency
\end{itemize}
In guinea pigs \citep{RothmanManis:2003a} type I cells express IA in 65\% of
cases. Ih is also present but, they were unable to isolate the Ih due their
preparation of removing dendrites in cell isolation. This could suggest Ih is
mainly in dendrites however they claim it to be the result of enzymatic
treatment.

\citep{RhodeGreenberg:1994b}

\citep{YuYoung:2000}

Although responses of chopper and type IV neurons to notches are qualitatively
similar, the data in Figs.
\href{www.pnas.org/cgi/content/full/97/22/11780#F5}
and
\href{www.pnas.org/cgi/content/full/97/22/11780#F6}
show that the mechanisms that govern these responses are quite different. In
chopper neurons, the transformation of spectral level into average discharge
rate is predominantly a linear (first-order) process that involves a narrow band
of frequencies centered on BF. This property suggests that axonal projections
from a tonotopic array of chopper neurons transmit, to other auditory areas, a
set of rate responses that are homomorphic with spectral shape. In fact, chopper
neurons have been shown to produce stable tonotopic representations of vowel
spectral shape
(\href{www.pnas.org/cgi/content/full/97/22/11780#B8}).


Unlike choppers, type IV neurons do not encode spectral information through a
homomorphic spectral representation. This is apparent from (i) the complexity
and wide bandwidth of their first-order weighting functions, and (ii) the strong
nonlinearity in the stimulus-response function. The nature of these
nonlinearities has been discussed elsewhere
(\href{www.pnas.org/cgi/content/full/97/22/11780#B24},
\href{www.pnas.org/cgi/content/full/97/22/11780#B29})
and will not be further elaborated here. It is sufficient to point out that, as
Fig.
\href{www.pnas.org/cgi/content/full/97/22/11780#F3}
illustrates, type IV neurons give inhibitory responses to both a narrow peak of
energy (tone) and a narrow notch of energy located at BF. The rate dependence of
type IV neurons to spectral notch position (Figs.
\href{www.pnas.org/cgi/content/full/97/22/11780#F3}
and
\href{www.pnas.org/cgi/content/full/97/22/11780#F6}),
in addition to their wide bandwidth and nonlinear behavior, suggests that these
neurons are not simply detectors of spectral level. Type IV neurons provide a
second pathway of spectral information transmission
%\includegraphics[width=0.423cm,height=0.159cm]{Evidence20Tables-img1.eps}
a nonlinear one that signals the presence of a specific complex spectral
feature.


\citep{Trussell:2002}

In principle, differences among cells in the regularity of spiking could reflect
differences in the location of synapses,morphological data indicates that
stellate cells may indeed differ in the relative position of excitatory synapses
\citep{ SmithRhode:1989}. Similar conclusions were reached using more
sophisticated models based on actual morphometric data (White et al. 1994).  The
weakness of phase-locking then would be an outcome of the rigid response profile
of the cell to a stimulus and the filtering properties of the dendrites, which
together would blur the temporal information contained in the auditory nerve.

\subsection{\citep{JosephsonMorest:1998}}
In chinchilla \citep{JosephsonMorest:1998} terminal distribution on soma and
axon:


\begin{itemize}
\item Pleomorphic {\textgreater} Large spherical {\textgreater} flattened
  {\textgreater} small spherical
\item GABA or GABA/Gly {\textgreater} AN Glut {\textgreater} Gly {\textgreater}
  other excit
\item GABA more prominent in high frequencies
\item 8/20 had large AN inputs to axon hillock
\item Type I stellate cells are a heterogeneous group. Inhibitory synapses
  probably compose the majority of terminals. Some cells receive mostly
  inhibitory synapses near the presumed site of the spike generator, but others
  also have a prominent excitatory input. These findings call for a new look at
  the mechanisms for signal coding in stellate cells in the auditory system in
  particular and raise issues concerning the stochastic nature of information
  processing in sensory systems in general
\item Heterogeneous synaptic pattern in type I stellate cells:

  \begin{itemize}
  \item more flattened vesicle endings terminate on DAP type I stellate
    cells. The smallest type I stellate cells in DAP have more flattened-vesicle
    endings than the larger ones or than any of those in PV
  \item AN fiber input varies considerably among type I stellate cells
  \item Initial segments and axon hillocks display diverse synaptic profiles
  \item Some axo-somatic inputs may surround dendritic trunks this would allow a
    relatively small number of synapses to neutralize or override the activity
    of the large number of axodendritic synapses
  \end{itemize}
\item The potential effect on firing pattern of adding an excitatory synapse at
  the spike generator in the axon hillock/initial segment has not been
  considered previously. Would a stellate cell, receiving an excitatory
  axoaxonic input from the cochlea, fire less regularly than one receiving a
  similar synapse from an interneuron? How would that differ from a stellate
  cell receiving only inhibitory axoaxonic synapses or none at all?
\end{itemize}



\subsubsection{STOCHASTIC VERSUS STEREOTYPICAL PRINCIPLES OF
SYNAPTIC ORGANIZATION AND CELL TYPES}

Central to our understanding of information processing in the CNS has been the
principle that units of function are defined by specific morphological patterns
of synapses, usually found in association with particular morphological types of
neurons. This has been the guiding principle for study of signal processing in
the central auditory system, and in other sensory systems, for over three
decades (see, e.g., Morest et al., 1973, for an earlier discussion). This
principle currently extends to the concept that functional circuits are to be
defined in terms of the connections of specific types of neurons, so-called
circuits of cell types (see Morest, 1993, for a recent discussion). Thus,
parallel circuits of bushy cells, octopus cells, giant cells, fusiform cells,
and stellate cells, to mention a few stereotypes, could be construed in terms of
patterns of synaptic organization and connections that would be common to each
cell type.  The apparent success of neurobiologists in elucidating the details
of these circuits and cell types testifies to the wisdom of this approach.
However, the central assumption in this effort, namely, that the unit of
function may be safely treated as a stereotypical form, can be questioned. Is it
true that specific cell types and specific synaptic profiles are stereotypic by
nature? In other words, do we recognize these patterns because each type is
discretely determined, or do they represent modes in a continuum of
morphological variability? In the light of the present findings, we must now ask
to what extent the various subtypes of stellate cells recognized in this study
actually result from a stochastic mode of determination. By this principle, the
heterogeneity of the synaptic profiles of type I stellate cells may represent a
less hard-wired template, which permits a certain degree of flexibility in their
capacity to process information.


\begin{itemize}
\item Josephson concludes that Type I stellate cells participate in the
  processing of sound across all frequencies and their somatic and axonal
  synaptic profiles indicate that they are under significant central, inhibitory
  control as well as being recipients of excitatory VIIIth nerve endings in
  proximity to their spike generator.
\item Numbers of inhibitory and excitatory terminals vary throughout the AVCN
  and would shape the cells{\textquoteright} frequency response area and dynamic
  range.
\end{itemize}

\subsection{In cat (Saint Marie et al. 1989)}
 GAD+ terminals are extensive on bushy, octopus
and D stellate cells. They did not study T stellate cells.  (Munirathinam et
al. 2004) found 7 different synapses based on vesicle size and shape but did not
disclose what cells and where vesicles were distributed.

\subsection{In Cat \citep{Cant:1981}}
 large dendrites which taper from the body, soma
diameters up to 30 \um.  No. of contacts minimal on soma, increasing on dendrites
away from the soma

\subsection{In pentobarbital cat \citep{RhodeSmith:1986}}


\begin{itemize}
\item the varying distribution of auditory nerve inputs on the soma and
  dendrites of the various cell types within the nucleus,2) the intrinsic
  membrane characteristics of the various cell types causing different responses
  to the same input in different cell types,and 3) secondary excitatory and
  inhibitory inputs to different cell types.
\item Standard parameters: characteristic frequency (CF) threshold in decibel
  (dB) sound pressure level (SPL) (TH) spontaneous firing rate (SR) in spikes
  per second,maximum discharge rate (Rmax) during the final 60\% of the
  shorttone burst stimulus,dynamic range (DR) in dB,Q10 (equal to CF/BW where BW
  is the bandwidth at 10 dB above threshold) is a measure of tuning,first spike
  latency (Lat) in milliseconds, standard deviation of the first spike latency,
  peak-to-steadystate ratio (PTS), synchronization coefficient (R) is the
  maximum value obtained at CF,ratio of activity in response to an ascending
  vs. a descending swept frequency signal (SWR) and the first four central
  moments of the interspike-interval distribution (mean, standard deviation,
  skew, and kurtosis). Histograms, scatter plots, and regression line
  calculations were performed on combinations of all of the above parameters.
\end{itemize}

\subsection{In chlorolase cat \citep{Rhode:1994}}
 200\% AM and Noise AM


\begin{itemize}
\item Claim distribution of CV among chopper units is unimodal, hence CS/CT
  boundary of 0.32 is arbitrary
\item CS: tMTFs low pass at low intensity and bandpass at higher intensities

  \begin{itemize}
  \item Width of tMTF is narrower than filter function ( ie rate )
  \item One unit had a more robust synchrony code than rate code, non monotonic
    MRCs
  \end{itemize}
\item CT: tMTFs ar basically lowpass with a tendency to become bandpass at 70 dB
  SPL similar to 100\% AM \citep{RhodeGreenberg:1994a}
\end{itemize}

In pentobarbital cats (Rhode et al. 1983a) intracellular labelling and
physiological characterisation


\begin{itemize}
\item ChS tend to have smoother, less branched (simpler pattern) dendrites than
  ChT
\item Both types of units have local axonal collaterals in an area similar to
  their own dendritic field
\end{itemize}
Anesthetics alter the response of CN units especially in the DCN and PVCN

\subsection{\citep{DoucetRyugo:1997} RAT Planar Neurons}

The planar neurons of the rat correspond in many ways to stellate cells
classified as sustained chopper units

(Chop-S) in cats \citep{SmithRhode:1989} and stellate neurons with axons
projecting toward the trapezoid body

(T-stellate cells) found in mice (Oertel et al., 1990). Both Chop-S and
T-stellate neurons send a collateral towards the DCN in addition to sending
their major axon into the trapezoid body. The axonal arborization of one
T-stellate cell was distributed in a narrow
{\textquoteleft}{\textquoteleft}isofrequency{\textquoteright}{\textquoteright}
band in the DCN (Oertel et al., 1990), and resembles what we predict for planar
neurons on the basis of the pattern of their retrograde labeling.

An ultrastructural examination of the cell bodies of intracellularly labeled
Chop-S units in the PVCN revealed that they had few primary terminals contacting their cell bodies
\citep{SmithRhode:1989}. Multipolar cells in the AVCN also exhibiting few
somatic terminals were called type I stellate cells \citep{Cant:1981}. These
cells project to the contralateral inferior colliculus \citep{Cant:1982} and
send a collateral to the DCN as well \citep{Adams:1983}. Planar neurons project
to the DCN, and we observed many labeled fibers in the trapezoid body,
indicating that at least some of the labeled multipolar cells project their
axons through the trapezoid body. Chop-S units are narrowly tuned, both in terms
the shape of their tuning curves to pure tones \citep{RhodeSmith:1986} and their
inability to integrate acoustic energy when spread over wide frequency bands
(Palmer et al., 1996).  Chop-S units can be found throughout the VCN
\citep{Pfeiffer:1966,Bourk, 1976},Young et al., 1988. Planar neurons, like
T-stellate cells, have dendritic orientations consistent with narrow tuning,
being oriented

parallel to the path of auditory nerve fibers having similar CFs. Since planar
neurons, Chop-S neurons, and multipolar cells with few somatic terminals are all
found throughout both the AVCN and PVCN, it is arguable that these all represent
the same neuron class. An isofrequency lamina of the DCN receives two tonotopic
projections: an ``unprocessed''
primary projection from narrowly-tuned auditory nerve fibers and a
``processed'' projection from similarly-tuned planar neurons.
One unresolved issue is whether planar neurons and auditory nerve fibers contact
the same or different cell
types in the DCN. Since the terminals of Chop-S neurons contain round synaptic
vesicles \citep{SmithRhode:1989}, they are presumed to exert excitatory
postsynaptic effects. Consistent with this notion is the observation that focal
applications of glutamate in the VCN produced excitatory postsynaptic potentials
in DCN giant cells when the loci of application conformed to an isofrequency
band in the VCN \citep{ZhangOertel:1993a}. Which other DCN cell types are
contacted by Chop-S neurons is unknown, but a single reconstructed T-stellate
neuron was shown to terminate near the cell bodies of pyramidal cells (Oertel et
al., 1990). These observations are consistent with the idea that planar neurons
contact pyramidal and giant cells, cell types that have also been shown to
receive auditory nerve fiber input (Gonzalez et al., 1993).


\subsection{In chinchilla and gerbils (Feng et al. 1994)}
 show intracellularly, chop S and
chop T units have different ISI characteristics using tone and current stimuli.
Inhibitory events modulate the response of Choppers: IPSPs, hyperpolarizing
potentials at tones above best freq.


\begin{itemize}
\item Use average intracellular response to show hyperpolarisation in ChS and
  ChT.
\end{itemize}

\subsection{(Ostapoff et al. 1994)}


\begin{itemize}
\item compared the morphology to the electrophysiology reported by Feng.
  {\textquotedblleft}{\dots} response properties may be partially independent of
  neuronal structure. Morphologically distinct neurons can generate similar
  temporal patterns in response to acoustic stimuli.{\textquotedblright}
  {\textquotedblleft}Membrane properties and synaptic organization complement
  and interact with each other{\dots}. There is now considerable evidence that
  each morphological type of neuron has a distinct synaptic organization, which
  can be defined in terms of the cytology of the synaptic endings, the pattern
  of the distribution on the surface of the cell, the transmitters used, and the
  connections with other cell types in the auditory pathway.{\textquotedblright}
\item Extracellular recordings correlated with
  morphology:\citep{Bourk:1976,BourkMielcarzEtAl:1981,BrawerMorest:1975,BrawerMorestEtAl:1974,CantMorest:1984,GodfreyKiangEtAl:1975,Kane:1973,OstapoffMorestEtAl:1994,RitzBrownell:1982,TolbertMorest:1982a,
    TolbertMorest:1982b,WinterPalmer:1990,YoungRobertEtAl:1988}
\item Intracellular recordings correlated with morphology
  \citep{RhodeOertelEtAl1983,RhodeSmithEtAl:1983,RhodeGreenberg:1992,RoullierRyugo:1984,SmithRhode:1987,SmithRhode:1985,SmithRhode:1989}
\item Membrane properties from slices correlated with morphology
  \citep{HirschOertel:1988,HirschOertel:1988a,ManisMarx:1991,Oertel:1983,OertelWu:1989,OertelWuEtAl:1990,WuOertel:1984}
\item Correlation between response type and cell type: synaptic organisation in
  signal processing
\item Correlation between neuronal structure and membrane properties: analysis
  of mechanisms in signal processing
\end{itemize}

\subsection{\citep{SmithRhode:1989} report on 5 CS and 5 OC units, physiology + labelling}


\begin{itemize}
\item The multipolar cell group is a ubiquitous collection whose diverse
  features have made it difficult to parcellate its members into well defined
  subclasses.
\item OC and CS units both generally have low spon (0 spikes)
\item CS: narrow bandwidth (comparable to AN fibres at same CF), saturate about
  30dB above threshold, inhibitory sidebands shown in CS units with spon
\item OC: wide bandwidth, wide dynamic range normally over 60 dB (one over 90
  dB), show sharp reduction in spikes after 10-20 msec, at high SPL can resemble
  Chopper
\item Intracellular response:

  \begin{itemize}
  \item CS units difficult to record from, characterised by a graded increase in
    the level of depolarisation to a maximum as the CF region is approached,
    followed by a decrease as the swept tone moves away.  Regular firing is
    continuous during sweep tone.
  \item OC don{\textquoteright}t fire continuously during sweep, displays a
    large level of depolarisation but inconsistent firing
  \item Inhibition: CS hyperpolarising potentials seen either side of CF during
    sweep and at CF tone offset.  OC rarely if ever have hyperpolarising
    potentials, no offset inhibition.
  \end{itemize}
\end{itemize}

\subsection{(Smith et al. 1993) anesthetised cats}


\begin{itemize}
\item Type 1 multipolar

  \begin{itemize}
  \item Classified in AVCN cat as having sparse somatic coverage
    \citep{Cant:1981}
  \item AVCN type 1 send axon to IC \citep{Adams:1979,Cant:1982} via TB
    \citep{TolbertMorestEtAl:1982} and collateral to DCN \citep{Adams:1983b}
  \item (Rhode et al. 1983a) recorded from AVCN type 1 neurons that had CH
    responses and left via the TB
  \item Intracellular in vitro experiments showed type 1 cells had type 1
    current clamp (regular firing) response \citep{WuOertel:1984}
  \item PVCN/nerve root chopper responses recorded from type 1 cells whose axons
    entered the DCN and then the TB \citep{SmithRhode:1989}: EM analysis showed
    sparse somatic and distal dend innervation, half innervation from AN,
    proximal 100\um ( {\textonehalf} covered, 2/5 AN)
  \item Modelling studies show position of AN inputs can determine regularity
    \citep{BanksSachs:1991}
  \end{itemize}
\end{itemize}

\subsection{\citep{OertelFujino:2001}}


\begin{itemize}
\item T stellates are excited by ACh using both muscarinic and nicotinic
  receptors
\item The relatively high input resistance and low firing thresholds of T
  stellate cell cause small modulatory currents to produce large changes in
  firing rate.
\item The strong modulation of ACh is made possible by the relative paucity of
  ILT and Ih has interesting implications on the ability of these cells to
  encode spectral peaks in varying acoustic conditions.  Cholinergic efferents
  affect the dynamic range of T stellate and they might enhance the encoding of
  spectral peaks in noise and when animals are awake.
\item Dendritic alignment and number of synapses suggest T stellates are
  narrowly tuned
\item Inhibitory sidebands in RA are due in part to D stellate cells
\item VNTB \ensuremath{\rightarrow} VCN is cholinergic  \citep{SherriffHenderson:1994}
\item The branched tips of T stellate cell dendrites lie near granule cell
  domains in the vicinity of olivocochlear terminals
  \citep{OertelFujino:2001,OertelWuEtAl:1990}
\item VNTB receives input from T stellate cells
  \citep{SmithJorisEtAl:1993,ThompsonThompson:1991} and auditory cortex
  (Feliciano et al. 1995)
\item MOC efferents contribute to AN enhancement by inhibition but excite CN
  Tstellate cells
\end{itemize}
\subsection{In rat urethane \citep{EbertOstwald:1995}}


\begin{itemize}
\item Choppers in AVCN and PVCN have chopping rates 10-20dB above threshold of
  174-422Hz
\item GABA ionotophoresis reduced tone-evoked activity by 35\% in CS and 25\% in
  CT
\item CT: CV values increased by average of 46\% in first period, but most had
  similar CV values for the last 20msec.
\item {GABA}s physiological
  function is possibly to improve the contrast between transient acoustic
  signals and ongoing background activity. In order to test this hypothesis, the
  test tone was masked by continuous background noise. Indeed,
  {GABA} reduced the noise-evoked discharge more
  than the tone-evoked discharge, leaving the onset peak in the PSTHs almost
  unchanged. Thus, {GABA}ergic input improves the
  signal-to-noise ratio for acoustic transients in VCN neurons. Our data suggest
  that a functional role of {GABA} in the VCN is to
  act as a transmitter within a descending inhibitory feedback loop of the
  auditory brainstem which serves to improve the transmission of relevant
  acoustic signals in constant background noise.
\item GABAergic interneurons rarely found in VCN
  \citep{AdamsMugnaini:1987,Moore:1987} but CT units begin to adapt from about
  10-20msec post onset, hence longer feedback paths can be considered.
  GABAergic TB neurons project to the CN in guinea pigs and cats (VNTB
  bilaterally, LNTB ipsi) \citep{Adams:1983a,ShoreHelfertEtAl:1991,SpanglerCantEtAl:1987,WinterRobertsonEtAl:1989} direct evidence of GABAergic uptake and retro
  grade transport to LNTB and VNTB \citep{OstapoffMorestEtAl:1990}
\item GABAergic inhibitory feed back is tonotopic: (Spangler et al.  1987)
\item Selective GABAergic inhibition of VCN neurons depresses background and
  sustained activity more than onset activity in order to improve the
  representation of transient acoustic signals in background noise.  This could
  account for the loss of regular discharge in chopper neurons.
\end{itemize}

\subsection{\citep{CantBenson:2003}}
 Type I multipolar cells of the ventral cochlear nucleus
encode complex features of the stimulus important for the recognition of natural
sounds and are a major source of excitatory input to the inferior colliculus

{\textquotedblleft}One group of multipolar cells (to be referred to here as type
I multipolar cells) described by Smith and Rhode (1989) appear to be equivalent
to the T-stellate cells described in mouse [151], the planar cells described in
rat [55], and perhaps also to the type I stellate cells described in the
anterior part of the AVCN of cat [36].  They probably also include many of the
neurons studied by Feng and coworkers [58,165]. The dendrites of type I
multipolar cells are oriented such that they would be expected to receive
auditory nerve inputs from a restricted frequency range [54,55,151]. Compared to
other cell types in the VCN, these neurons receive relatively sparse synaptic
inputs to their cell bodies [12,102,220,237]. Input from auditory nerve fibers
and from abundant terminals containing glycine, GABA, or both are distributed on
the dendritic surface [103,113,200,256]. The few terminals that contact the
somatic surface usually appear to be inhibitory, but some cells receive
apparently excitatory contacts on the axonal hillock and initial segment
[102]. Most multipolar cells in the anterior AVCN also receive few somatic
inputs (type I stellate cells, [36,102]). These cells are on average smaller
than those in the posterior AVCN and anterior PVCN [44], and there may be some
subtle differences in their synaptic organization [102]. Whether the multipolar
cells in the anterior AVCN share all of the properties ascribed to the large
type I multipolar cells described by Smith and Rhode [220] remains to be
determined.

The type I multipolar cells are narrowly tuned and respond to tone bursts with
regular trains of action potentials, a response referred to as a
{\textquotedblleft}chopper{\textquotedblright} pattern (e.g.
[168,220]). Neurons that exhibit chopper responses can differ substantially in
their dendritic morphology ([58,179,194],cf. [30]) which suggests that a further
subdivision of this class of neurons may be possible. In mouse, the equivalent
cells (T-stellate cells) appear to integrate input from the auditory nerve with
that from other multipolar cells of both types

[61]. In general, the response properties of chopper units suggest that they
play an important role in encoding complex acoustic stimuli, perhaps including
speech sounds (e.g. [26,131,180]).

The projection pattern of type I multipolar cells is illustrated in Fig.
2F. The axons leave the cochlear nucleus via the trapezoid body
[55,151,220,245], where they make up the ventral thin fiber component
[31,215,245,248]. Possibly because they are thinner than the axons of the other
cell types, there have been few reports of successful intra-axonal injections of
these fibers so it is not entirely clear whether the different projections arise
from the same or different populations. Multipolar cells are a major source of
input from the cochlear nucleus to the contralateral inferior colliculus
[2,12,24,33,37,102,154,156,191,205]. It seems likely

that most, if not all, type I multipolar cells participate in this projection
[102]. The projection arises from neurons throughout the VCN, including all but
the most anterior part of the AVCN and the octopus cell area in the PVCN. The
same neurons that project to the inferior colliculus also send collateral
branches to the DCN ([4],also, [55,61,167,217]). In both targets, the synaptic
terminals contain round synaptic vesicles, compatible with an excitatory effect
(IC: [154],DCN: [220]). The projections from the cochlear nucleus have been
shown to directly contact neurons in the inferior colliculus that project to the
medial geniculate nucleus [156]. A smaller projection to the ipsilateral
inferior colliculus also arises from multipolar cells in the VCN
(e.g. [2,154]). The axons that make up this projection travel in the lateral
trapezoid body tract [245,248]. Multipolar cells in the VCN give rise to
projections to

the dorsomedial periolivary nucleus in cat [215] or superior paraolivary nucleus
in rat and guinea pig [64,201], to the ventral nucleus of the trapezoid body
[64,215] and to the ventral nucleus of the lateral lemniscus
[64,91,206,215]. The cells that give rise to these projections are probably the
type I multipolar cells [218]. Although it has not been established definitely,
it seems likely that these projections arise from the same cells that project to
the inferior colliculus. Multipolar cells of unknown type project to the
ipsilateral

lateral superior olivary nucleus and the lateral periolivary region in cats
[41,233,248]. In addition to their projection to the DCN, the type I multipolar
cells give rise to extensive collateral branches within the VCN
[4,61,151,220,238]. These appear to play an important role in shaping late
responses of cells in the VCN to auditory nerve stimulation
(e.g. [61]).{\textquotedblright}

\subsection{\citep{WangSachs:1992}}

A detailed description of ANF responses to narrowband stimuli is also relevant
to the analysis of signal transformations in the cochlear nucleus.  For example,
low spontaneous rate (SR) ANFs have higher thresholds and wider dynamic ranges
for rate responses to tones than do high SR ANFs \citep{Liberman:1978}{,Sachs et
  al. 1989}.  It has been suggested that spectral features of complex stimuli
are represented in the average discharge of low SR ANFs at high sound levels,
where rates of high SR ANFs are saturated
\citep{Delgutte:1982,WinslowBartaEtAl:1987}.  At low sound levels below the
thresholds of low SR ANFs, spectral features are represented in the discharge
rate of the high SR ANFs \citep{SachsYoung:1979}.  Chopper units in the AVCNmay
{\textquoteleft}listen selectively{\textquoteright} to high SR ANF inputs at low
sound levels and to low SR ANFs at high sound levels
\citep{BlackburnSachs:1990}.  However details of the processing are not known.
Evidence suggests that both low and high SR ANFs contact stellate cells
\citep{Liberman:1991,Ryugo:1992}, the source of chopper responses (Rhode et
al. 1983a), but we know little about how stellate cells integrate inputs form
different ANF SR groups.  One way to investigate such problems is to establish a
physiological marker that can distinguish low and high SR fibres at all sound
levels and then study this marker in chopper populations.  We will show that the
envelope fluctuations of temporal discharge patterns of responses to narrowband
sounds can be used as such a marker.


Single Formant Stimuli: Periodic impulse train\ensuremath{\rightarrow} 2ne order
RLC circuit \ensuremath{\rightarrow} Butterworth filter \ensuremath{\rightarrow}
{\textquoteleft}SFS{\textquoteright}

Modulation in chopper period histograms at high SPL could be due to inhibition
or from modulation in LSR ANFs.

\subsection{\citep{BlackburnSachs:1992} barbiturate-anesthetized cats}

The effects of off-BF input (either alone or presented simultaneously with a BF
tone in a two-tone stimulus) on the response patterns of choppers may include
not only rate inhibition but changes in the discharge regularity and the
temporal adaptation properties of the spike trains.

Majority of off-BF or two tone responses were comparable to on-BF responses in
terms of regularity. (119/149 frequencies in 45 units) Other 27 showed changes
in regularity. Do they mean average responses to the 149 different frequencies
used?

Changes to TAP in 32 of 171 frequencies examined in 53 units):


\begin{itemize}
\item Sustained to slowly adapting
\item Slowly adapting to transiently adapting
\item transiently adapting to slowly adapting
\item the inhibitory effect of off-BF input is not simply the result of the
  two-tone suppression at the level of the auditory nerve.
\end{itemize}
Majority show no change in TAP

These data are consistent with the hypothesis that regular and irregular
choppers are distinguished by the magnitude and/or types of inhibitory input
that they receive.

CV calculated over 30-40 msec of 50 msec tone

\subsection{Discussion of \citep{BanksSachs:1991}}
 They investigated in detail two potential
mechanisms for regularizing the spike trains and found that each are able to
effect regularization: I) the stellate cell model fires more regularly to distal
than to proximal inputs because of dendritic filtering of distal postsynaptic
potentials,2) the postsynaptic signals become more regular as more AN fiber
inputs converge on a single compartment because

of their shot-noise properties. The model regularity is affected in an analogous
manner when inhibitory inputs (also modeled as having the characteristics of AN
spike trains) are included: 1) the model fires less regularly when inhibitory
inputs are activated proximal to the excitatory inputs,and 2) the irregularizing
effect of the inhibitory input is greater the smaller the number of inputs that
generate a given magnitude of inhibitory input (again due to the shotnoise
properties of the input).

\subsection{(Clock et al. 1993)}
single unit thresholds\ensuremath{\rightarrow} temporal
integration of inputs : chinchilla

\subsection{(Pressnitzer et al. 2000) guinea pig urethane}


\begin{itemize}
\item Chopper units respond with more spikes to a ramped sinusoid rather than a
  damped sinusoid
\item Peak-to-total spike ratio is highest in Oc and Ch and less in PL, AN
\end{itemize}

\subsection{(Winter et al. 2003) guinea pigs}

IO vs BMF in choppers (Frisina found poor correlation using AM, Kim found some
correlation using noise)

This study supports Kim, correlation is strong in CS and OC units but not in CT.

At moderate to high sound levels, some chopper units show a band-pass modulation
transfer function (MTF: Frisina et al., 1990,Kim et al.,
1990,RhodeGreenberg:1994). The units most likely to show a band-pass MTF were
identified as CSs.

Shofner (1999) has presented evidence that multipolar cells, specifically
chopper units, were not well suited to encode the pitch of IRN. This conclusion
was based on the responses to IRN stimuli with negative gain. These stimuli

\label{tab:Winter2003}
elicit ambiguous pitches that were only signalled in the temporal discharge
patterns of PLunits (and hence bushy cells). The

responses of non-PLuni ts (mainly choppers) largely reflected the stimulus
envelope and not the fine timing in the waveform structure.  This result appears
to argue against chopper units playing a role in the encoding of the pitch of
complex sounds in their first-order ISIHs.  However, the examples given in
\citep{Shofner:1999} come from units with

relatively high BFs and a CT response pattern (see Fig. 2) it is possible that
low BF units, in particular those classified as CS, may show responses in their
temporal discharges related to the shift in time intervals associated with the
pitch of IRN

with negative gain. Until we have more data on this issue the role of CS units
in encoding the pitch of complex sounds remains unresolved.

the strongest temporal responses in units with low BFs (unpublished
observations) we also observed that some units, with BFs well removed from the
phase-locking region, and at relatively high stimulus presentation levels, could
encode the

delay of IRN in terms of their first-order interval statistics (Winter et al.,
2001). This result is in agreement with the vowel study presented here (see
Fig. 3), where both onset and chopper units showed a strong response to the f0,
despite their

BFs being above the phase-locking cut-off in the guinea pig (\_3.5 kHz,
PalmerRussell:1986). Stimulus presentation level may well explain the different
frequency ranges over which stimulus related features were observed between the
two

studies. It should, however, be made clear that the significance of high BF
units responding well to the f0 of complex sounds is, at best, uncertain.

Table 2

\begin{table}[h]
  \centering

  \caption{Mean and standard deviation of the peak in the first-order ISIH to WN and the peak in the interval enhancement plot}
  

\begin{tabularx}{\textwidth}{XXXX}
\toprule
                          &       CT       &     CS     & Onset \\ \midrule
    White noise (ms)      &   2.42$\pm$1.08   & 4.45$\pm$1.83 & 6.55$\pm$2.36 \\
Interval enhancement (ms) & 3.53$\pm$2.02\, *& 4.88$\pm$2.1  & 7.05$\pm$2.61\\
       Correlation        &      0.44      &    0.72    & 0.81 \\
\bottomrule
\end{tabularx}
\footnotesize{* Significant difference between peak in interval enhancement
and peak in WN (one-tailed student's t{}-test) at p {\textless} 0:05.}
\end{table}

The correlation between the WN and IRN peak responses for the three unit types
is shown in Table 2. There is a weak correlation for CT units and there are
strong correlations for the CS and OC unit types.

\subsection{\citep{RhodeKettner:1987} barbiturate and unanesthetised cats}

\begin{table}[htp]
      \begin{tabularx}{\textwidth}{XXXXXX}
\toprule
                    & Ch (unanesthetised) & Ch (anesthetized) & OC (unanesthetised) & OC (anesthetized) & DCN Chopper\\\midrule
 Threshold dB SPL   &         30          &        19         &         27          &        23         & \~{}18\\
        DR          &         31          &        39         &         35          &        57         & \\
Q$_{10}$ &         7.4         &        5.3        &          5          &        3.3        & 7.3\ensuremath{\rightarrow}5.2\\
   Latency msec     &         3.5         &        3.6        &    4.2{\textpm}3    &  2.8{\textpm}0.5  & 4.6 (unanesth) 4.8 (anesth)\\
         n          &         23          &        120        &          8          &        99         & 25\ensuremath{\rightarrow}148\\
\end{tabularx}
\end{table}


\subsection{TIMING PATHWAYS THAT LEAD TO THE MONAURAL
NUCLEI OF THE LATERAL LEMNISCUS \citep{Oertel:2005}}

Stellate Cells

Stellate cells form a major, direct pathway from the cochlear nuclei to the
contralateral
midbrain. In all higher vertebrates, stellate cells also contribute to an
indirect pathway to the midbrain through the periolivary and lateral lemniscal
nuclei. The fact that stellate cells form a major ascending pathway across
vertebrate classes suggests that the sharply tuned, tonic, consistent responses
to tones of stellate cells carry essential acoustic information. This pathway,
through {\textquotedblleft}choppers,{\textquotedblright} encodes the presence of
energy in a narrow acoustic band
with tonic firing,the firing of the population of stellate cells provides a
representation
of the spectral content of sounds. (The term chopping refers to the
modes that arise in histograms of responses to tones from the regularity of
firing, independent of periodicity in the sound stimulus.)

Midbrain-projecting stellate cells have been revealed in reptiles and birds. In
turtles, the nucleus magnocellularis contains stellate cells that resemble
stellate
cells in birds and mammals, both in their morphology and in their projection
to the midbrain (35). This class of cells has been studied more extensively in
birds where midbrain projecting stellate cells lie in the nucleus angularis
(162).
The nucleus angularis holds neurons with consistently timed, tonic chopping
responses to sound (163, 164). In barn owls, this nucleus has been shown to be
essential for encoding the spectral cues for localizing sound sources in
elevation
(87). Because the nucleus angularis of owls and other birds could hold more than
one group of neurons, the possibility remains that a subpopulation of neurons
performs this task.

In mammals, midbrain-projecting stellate cells also have chopping responses
(54, 165). Three features of these responses are noteworthy. The first is that
resolution of chopping in tonically firing neurons requires that the first
action
potential in a response to a tone has consistent timing (116, 166--171), with a
precision that is also reflected in synaptic responses from the auditory nerve
in
vitro (172--175). The second is that choppers encode the envelope of sound over
a wide dynamic range (167, 170, 171, 176). How the enhancement of amplitude
modulation over the auditory nerve inputs is produced is not entirely clear.
Stellate cells in mice receive GABAergic input that may modulate their firing
rates as a function of intensity from small neurons (Golgi cells) that surround
the VCN (175, 177), a region in which the responses of neurons to sound have
wide dynamic ranges (178). The third is that choppers effectively encode the
envelope of sound energy (167, 171), a characteristic that is important for
understanding
speech (179). As a population, choppers provide a rate-representation
of the spectrum of sound (170, 176).

How chopping responses are produced is not completely understood. It has
been suggested that stellate cells integrate input from large numbers of
auditory
nerve fibers. However, stellate cells in mice have been shown to receive input
from only a few (four to six) sharply timed auditory nerve fiber inputs (175).
Activation of these inputs with trains of shocks produces entrained responses
rather than chopping (172, 175), raising two questions: How are stellate cells
prevented from encoding the timing of auditory nerve inputs after the initial
action potential in response to sound, and how is their steady firing in
response
to tones produced from inputs that have strong onset transients?  Excitatory
interconnections between stellate cells produce prolonged excitation that
presumably
contributes to the shaping of responses to sound (175).

The axons of individual, chopping stellate cells have been labeled in cats
(180). They were followed to the contralateral VNTB and VNLL and to the
periolivary nuclei, but they faded before reaching the inferior
colliculus. Tracktracing
experiments support the conclusion that this group of stellate cells
projects to the VNTB, LNTB, VNLL, and periolivary nuclei (67, 117, 122, 143,
181--188) and show that they also project to the contralateral inferior
colliculus (101, 104--106, 188--193).

\subsection{Tuberculoventral cells} 

{\bfseries Table : Tuberculoventral Cells / Type II EIRA DCN
  units}

\begin{flushleft}

  \begin{tabularx}{\textwidth}{XXX}
    \toprule
    \citep{EvansNelson:1973,WickesbergOertel:1990,WickesbergOertel:1993,WickesbergOertel:1988,WickesbergWhitlonEtAl:1991,Wickesberg:1996,YoungBrownell:1976,YoungVoigt:1981,ZhangOertel:1993}
    &
    \citep{RhodeGreenberg:1992,ShofnerYoung:1985,SpirouDavisEtAl:1999} &
    \citep{DavisYoung:2000}\\\hline
 \citep{ReissYoung:2005} notch noise vs notch    edge.  &
    &
     BIC increases rate most
    Stry increases rate
    Thresholds for tones and noise stay the same\\\hline
  \end{tabularx}
\end{flushleft}

\subsection{\citep{OertelWu:1989} HRP and intracellular recordings in DCN}

The two tuberculoventral association cells in the deep DCN have axons that
terminate in both the deep DCN, within the same isofrequency lamina that contain
the cell body, and in the VCN.

Many of the cells in the deep DCN project to the VCN.  These tuberculoventral
association cells, the ``vertical'' cells of
Lorente de No (81), project topographically to the VCN through the lateral
ventral tubercular tract (Lorente de No{\textasciigrave} 1933, 1981)

Anatomy:

Cells \#16 and 17 Fig 7, have cell bodies in the deep DCN,their dendrites are
irregular in thickness, making them look beaded.  Their dendrites are confined
to the deep DCN layers.  Their axons have collaterals which terminate near the
cell body and also project to the VCN.  (Cell \#17) {\dots} had dendrites
aligned at the angle of isofrequency lamina in the deep DCN.  Most terminals in
the PVCN lay in a plane approximately parallel to the plane of section.

Physiology:

Fig 11 (Cells \#16,\#17) shocks to the AN illict EPSPs between 1 and 4 msec.  No
IPSPs observed however it is doubtful that the absence of IPSPs is meaningful
Current clamp response is Type I regular firing.

If tuberculoventral cells are silenced by barbiturate anaesthetics
\citep{EvansNelson:1973,RhodeOertelEtAl:1983a,RhodeOertelEtAl:1983,YoungBrownell:1976},
inhibition in VCN cells could have been missed.


\subsection{\citep{ReissYoung:2005}}


Type II responses to notch noise and noise bands only explain some features of
type IV responses Type II units are a prominent inhibitory input to DCN type IV
units \citep{VoigtYoung:1990}. Although they are not considered to be a source
of inhibition for noise notches centered at BF \citep{NelkenYoung:1994}, their
responses for notches centered away from BF have not been studied extensively.
We found that seven of seven type II neurons studied showed very similar,
strongly modulated responses to notch noise and noise band sweeps (Fig. 9). This
response modulation is consistent with previous type II data
\citep{NelkenYoung:1994}.  These responses differed from the type IV responses
described above in that peak responses were not seen at the rising band edge
(Fig. 9C,D) but rather at the falling band edge (Fig. 9E,F). However, note that
type II units responded to both edges, unlike rising edge-sensitive type IV
units, which were inhibited by the falling edge.  These high-rate responses to
notch noise seem inconsistent with previously described weak rate responses to
BBN in type II units \citep{YoungVoigt:1982,SpirouDavisEtAl:1999}. However, the
data are consistent, as is shown by the observation that the excitatory
responses decreased as notch width decreased, converging to the BBN response at
very narrow notch widths.  Figure 10 shows a comparison of average type II and
type IV responses to notches and noise bands. All of the type II responses
studied at octave notch width are superimposed on a normalized frequency scale
(Fig. 10A), and these are averaged and compared with type IV responses in Figure
10B. The type II frequency axes are shifted to lower frequencies relative to the
type IV axes to reflect the empirically measured 0.1 octave frequency difference
between CIA CFs and type IV BFs (Fig. 8C). This frequency difference is
consistent with the difference in BFs between type II and type IV units showing
cross-correlation evidence of an inhibitory synaptic connection
\citep{VoigtYoung:1990}.

Note that type II activity is qualitatively the inverse of the responses of the
high-rate type IV group (Fig. 10B, thin black lines). For example, the strong
inhibition of type II units when the notch is centered near BF corresponds to an
increase in rate in the type IV units, especially noticeable for the widest
notch


\subsection{\citep{SpirouDavisEtAl:1999}}
used 201 TV cells from cats, due to difficulties
with holding type II units only 41 have two or more stimuli.  Tone RL, Noise RL,
tone RA, Two tone RA etc.


\begin{itemize}
\item 0 spon rate
\item Narrow V shaped tuning
\item Strong response to tones, Non-monotonic RL
\item Strychnine: slight increase in rate, increase in Low freq side band
\item BIC: greater increase in rate, mainly on CF, deep upper side band
  inhibition
\end{itemize}

\subsection{\citep{Rhode:1999} pentobarbital anesthetised cats}


\begin{itemize}
\item Recorded from vertical cells that do not project to VCN (layer III)
\item Type II responses to EIRA, On-graded to PSTHs, some Ch-t, CV average
  04-0.5
\item Nearly all vertical cells in the deep DCN project to the AVCN (Saint Marie
  et al. 1991)
\item FTCs: Closed contours 0-shaped
\item Threshold 8-51 dB average 27
\item Max rates {\textless}100 sp/s
\end{itemize}

\subsection{\citep{YoungVoigt:1982} 49 type II units in deep DCN of cat}


\begin{itemize}
\item Spon rate {\textless}2.5, 80\% {\textless}1.0 sp/s
\item Non-monotonic rate level function, -3 to 1 slope after peak average \~{}-1
  \%/dB
\item Relative Noise Rate (RNR=maxNR/maxToneRate) {\textless}0.3
\item PSTH: On-L (5), P/B (4)
\item FTC: open contours \~{}ANFs, Q10 \~{}ANFs, Q40 {\textgreater}ANFs
\item Threshold 5-20 dB
\item Max rates {\textless}300 sp/s
\end{itemize}
\subsection{\citep{JorisSmith:1998} chloralose anesthetised cats}


\begin{itemize}
\item Latency : clicks --poor response, tones - \~{}5msec, AM - \~{}4.5msec
\item Very fast onset inhibition must be restricting type II units from
  responding to clicks
\item Latencies for type II compare well with (Godfrey et al. 1975a, b) On-S
  which had non-monotonic RL, zero spon, noise not tested
\item Amplitude Modulation: AM RL is excitatory but still has negative slop at
  high SPL, synch at high level non-linear

  \begin{itemize}
  \item SI -Level slopes from \~{}0.8 at low SPL down to 0 at 60 dB where phase
    transition occurs, synchronisation to 2fm is greater than synch to fm at
    this point
  \item Period histograms at high SPL are consistent with a tightly phase-locked
    inhibitory input at roughly the same phase as an excitatory input that is
    temporally more dispersed
  \item Other AM studies of type II are
    {\textquotedblleft}scant{\textquotedblright}
    \citep{KimSirianniEtAl:1990,ZhaoLiang:1995}
  \item Median maxSI type II = 0.79
  \end{itemize}
\item MTF: only at low SPL suprathreshold

  \begin{itemize}
  \item Low pass generally, one with notch \~{}250Hz, narrower than Oc but
    better than IV-III
  \item 3dB cutoff \~{}400Hz
  \end{itemize}
\end{itemize}
\subsection{\citep{RhodeGreenberg:1994b} pentobarbital cats}


\begin{itemize}
\item Recorded from one OG unit (Fig 6) that had a nonlinear RL, narrow BW, and
  poor response to BBN
\item MRA was highly suppressive due to zero noise rate. Max \~{}15 sp/s on CF
\end{itemize}

\subsection{Hancock and Voigt ARO 2000: pentobarbital gerbils (barbiturate recordings in DCN
not reliable)}

Anatomical features
conical dendritic field ending in thick tuft
tuft far removed from cell body, confined to frequency lamina
Physiological features
axon forms a dense band parallel to frequency lamina
monotonic rate-level curves featuring large firing rates
CT+ response patterns
above-BF tones produce hyperpolarizations

Anatomical properties of 2 TV cells in Hancock and Voigt

\begin{flushleft}

  \begin{tabularx}{\textwidth}{XXXXXXXXX}
\toprule
       & Resting Potential & Action Potential  &  BF  & Threshold Tone & Threshold Noise
       &       Spon        & Dend Surface Area & Total dend length\\\midrule
Cell 1 &       {}-68       &       69(1)       & 4.75 &       15       & 30 & 21 & 37734 & 4252\\
Cell 2 &       {}-67       &     34 (-33)      & 2.34 &       20       & 35 & 0  & 33919 & 3754\\
\bottomrule
\end{tabularx}
\end{flushleft}


\subsection{\citep{YuYoung:2000}}

Although responses of chopper and type IV neurons to notches are qualitatively
similar, the data in Figs.
\href{www.pnas.org/cgi/content/full/97/22/11780#F5}
and
\href{www.pnas.org/cgi/content/full/97/22/11780#F6}
show that the mechanisms that govern these responses are quite different. In
chopper neurons, the transformation of spectral level into average discharge
rate is predominantly a linear (first-order) process that involves a narrow band
of frequencies centered on BF. This property suggests that axonal projections
from a tonotopic array of chopper neurons transmit, to other auditory areas, a
set of rate responses that are homomorphic with spectral shape. In fact, chopper
neurons have been shown to produce stable tonotopic representations of vowel
spectral shape
(\href{www.pnas.org/cgi/content/full/97/22/11780#B8}).


Unlike choppers, type IV neurons do not encode spectral information through a
homomorphic spectral representation. This is apparent from (i) the complexity
and wide bandwidth of their first-order weighting functions, and (ii) the strong
nonlinearity in the stimulus-response function. The nature of these
nonlinearities has been discussed elsewhere
(\href{www.pnas.org/cgi/content/full/97/22/11780#B24},
\href{www.pnas.org/cgi/content/full/97/22/11780#B29})
and will not be further elaborated here. It is sufficient to point out that, as
Fig.
\href{www.pnas.org/cgi/content/full/97/22/11780#F3}
illustrates, type IV neurons give inhibitory responses to both a narrow peak of
energy (tone) and a narrow notch of energy located at BF. The rate dependence of
type IV neurons to spectral notch position (Figs.
\href{www.pnas.org/cgi/content/full/97/22/11780#F3}
and
\href{www.pnas.org/cgi/content/full/97/22/11780#F6}),
in addition to their wide bandwidth and nonlinear behavior, suggests that these
neurons are not simply detectors of spectral level. Type IV neurons provide a
second pathway of spectral information transmission
%\includegraphics[width=0.423cm,height=0.159cm]{Evidence20Tables-img1.eps}
a nonlinear one that signals the presence of a specific complex spectral
feature.


\section{D stellate cells} 

{D  stellate Cells / Onset Chopper Units}
 
\citep{PalmerJiangEtAl:1996,PalmerWallaceEtAl:2003,PalmerWinter:1996,PaoliniClareyEtAl:2005,PaoliniClareyEtAl:2004,PaoliniClark:1999,PaoliniClarkEtAl:1997,PaoliniCotterillEtAl:1998,PaoliniCotterillEtAl:1998a}


Physiological characterisation plus reconstruction of cell axons and dendrites
 \citep{ArnottWallaceEtAl:2004} See Fig 3 of \citep{DoucetRyugo:1997} for good D
 stellate reconstruction

\citep{JorisSmith:1998} OC cells recorded from the DAS along with type II, II
and IV units in DCN, DAS of cats


\begin{itemize}
\item AM RL {\textless} tone RL {\textless} noise RL
\item AM SI-Level maintains above 0.8 up to 60dB then slopes down to 0.6 at 80
  dB, phase is linear over SPL
\item AM experiments are recorded using long AM stimuli, rate responses should
  be match to long tone responses as well (more significant for type IV)
\item Median maxSI = 0.93 (n=12), 3dB cutoff CFs{\textgreater}10kHz comparable
  to ANFs \~{}1000Hz
\end{itemize}

\citep{YoungBrownell:1976} decerebrate cat


\begin{itemize}
\item Found predominantly inhibitory contralateral effects in DCN of cat and
  reported that effects were weaker and more variable for contralateral tones
  than for BBN.  They reported that the response to binaural noise was
  intermediate between the ipsilateral (excitatory response and the
  contralateral (inhibitory) response, which suggested a functional relevance
  for these binaural interactions in natural listening conditions
\end{itemize} 

In mice \citep{OertelWuEtAl:1990} type II stellate cells have APs that did
overshoot 0mV with double exponential undershoot. IPSPs and late EPSPs
observed. Generally found near the lateral surface. Whether the difference with
cats represent a species difference or a difference in sampling is not clear.

In cats type II multipolar cells \citep{Cant:1981} have dense somatic covering
including Pleo (ie GABA ) terminals see Fig 13.  Generally found near nerve
root.

\subsection{\citep{DoucetRyugo:1997} Radiate Neurons}

Radiate neurons in the rat share many morphological features with stellate
neurons classified as Onset-chopper

units (On-C) in the cat \citep{SmithRhode:1989} and to stellate neurons in the
mouse that send their major axon

dorsally beneath the DCN (D-stellate,Oertel et al., 1990). Both On-C units and
D-stellate neurons project a collat- eral axon into the DCN, but their major
axons could not be followed to their termination points outside the cochlear
nucleus. Radiate neurons project to the DCN, but at this time it is not known if
they also send a major axon elsewhere because the axonal labeling in the dorsal
output tracts that we observed could have been due to pyramidal and giant
neurons labeled by the DCN injections. The DCN terminal field of one D-stellate
neuron was found to extend over a broad region of the DCN (Oertel et al., 1990).

Our observation that labeled radiate neurons could be observed in any region of
the VCN after a restricted DCN injection is consistent with the idea that single radiate neurons innervate a
broad region of the DCN.

Radiate, D-stellate, and On-C units all exhibit relatively large cell bodies and
share two atypical dendritic features that tend to unify them as a group. First,
the dendrites of these cells are prominent, extending hundreds of micrometers
across the path of auditory nerve fibers arising from broad regions of the
cochlea. Second, the distal tips of the dendrites frequently penetrate into the
granule cell domain. This morphology is also strikingly similar to what has been
described for commissural neurons of the VCN \citep{SchofieldCant:1996}. The
radial trajectory of these dendrites is consistent with their being
physiologically responsive to a broad range of frequencies, and in fact, On-C
units appear to integrate the activity of auditory nerve fibers having a wide
range of CFs
\citep{RhodeSmith:1986,WinterPalmer:1995,JiangPalmerEtAl:1996,PalmerWallaceEtAl:1996}.

At the ultrastructural level, On-C neurons of the PVCN have cell bodies that are
contacted by many terminals

\citep{SmithRhode:1989}. In the AVCN, the type II stellate cell was identified
on the basis of its numerous somatic terminals \citep{Cant:1981}, and while it
is not known where these cells project, it is known that they do not project to
the contralateral inferior colliculus \citep{Cant:1982}. The terminals of On-C
neurons contain pleomorphic synaptic vesicles, indicating that they probably
have an inhibitory effect on their target neurons. No direct evidence exists
about the types of DCN neurons contacted by On-C neurons, but D-stellate neurons
appear to contact giant and vertical cells \citep{ZhangOertel:1993a,b}. These
observations are consistent with the idea that radiate cells give rise to an
inhibitory projection to the DCN. In order to account for the response
properties of type II and type IV units of the DCN, both modeling and
physiological studies have proposed that each neuron type is innervated by a
broadly-tuned source of inhibition (modeling studies:
ArleKim:1991,PontDamper:1991,Blum, et al., 1995,physiological studies:
WinterPalmer:1993,Nelken andYoung, 1995).  For example, Type II neurons are
excited by narrowband stimuli within their response areas, but respond weakly or
not at all to broadband noise \citep{YoungBrownell:1976}. This property of Type
II units could be explained if they receive a broadly-tuned inhibitory input
from On-C neurons \citep{WinterPalmer:1993}. On-C neurons have also been
postulated to be the source of wideband inhibition necessary to explain the
sensitivity of type IV units to spectral notches \citep{NelkenYoung:1995}. We
have shown that each DCN locus is innervated by a complement of radiate neurons
whose dendrites collectively span the entire tonotopic axis of the VCN. Because
radiate neurons share many of the same cytological features characteristic of
On-C neurons, they are strong candidates for being the source of wideband
inhibition to the DCN.

\subsection{\citep{SmithMassieEtAl:2005}}
The combined evidence from several
different studies (Cant and Gaston 1982; Wenthold 1987; Kolston et al. 1992;
Shore et al. 1992; Schofield and Cant 1996b; Alibardi 1998; Needham and Paolini
2003; Palmer et al. 2003; Arnott et al. 2004) suggests that the large VCN
multipolar cells whose response to short tones have been labeled onset-chopper
\citep[OC][]{RhodeOertelEtAl:1983a,SmithRhode:1989} closely resemble a population
of large glycinergic multipolar cells that project to the opposite
CN. Single-cell labeling studies in the cat (Smith and Rhode 1989) and guinea
pig (Palmer et al. 2003; Arnott et al. 2004) have reported that the axons of
these cells provide collateral innervation to both the ipsilateral dorsal and
ventral cochlear nuclei before heading dorsally out of the CN. The only direct
evidence that the glycinergic multipolar cells projecting to the opposite
cochlear nucleus and the cells with OC response features are the same population
is one juxtacellularly labeled cell in the guinea pig (Arnott et al. 2004) and
one intra-axonally labeled cell in the cat that we described in a preliminary
report (Joris et al. 1992).

We examined three of these cells at the EM level to compare them
with a previous report (Smith and Rhode 989).  Both the cell body and proximal
dendritic tree showed a dense synaptic coverage (Figs.
\href{www3.interscience.wiley.com.mate.lib.unimelb.edu.au/cgi-bin/fulltext/109873297/main.html,ftx_abs#FIG3}A,B,
\href{www3.interscience.wiley.com.mate.lib.unimelb.edu.au/cgi-bin/fulltext/109873297/main.html,ftx_abs#FIG4}A,D),
a common feature of this cell type \citep{SmithRhode:1989}.  The axons of all
the cells located in the VCN had collaterals in both the VCN and DCN. Most of
the collateral field in the DCN innervated both fusiform and deep layers in and
around the frequency region of the DCN that, based on reported frequency maps
(Spirou et al. 1993), corresponds to the CF of the OC cell (Fig
\href{www3.interscience.wiley.com.mate.lib.unimelb.edu.au/cgi-bin/fulltext/109873297/main.html,ftx_abs#FIG3}D).
In the DCN the synaptic terminals could be seen on large dendrites as well as
cell bodies
(Fig. \href{www3.interscience.wiley.com.mate.lib.unimelb.edu.au/cgi-bin/fulltext/109873297/main.html,ftx_abs#FIG3}E)
and contained nonround vesicles. The OC cell located in the deep DCN also had
collateral branches that innervated the deep DCN and fusiform cell layer in and
around the location of the parent cell body
(Fig. \href{www3.interscience.wiley.com.mate.lib.unimelb.edu.au/cgi-bin/fulltext/109873297/main.html,ftx_abs#FIG4}C,E).
We could not find any axon collaterals of this cell that headed for the VCN.


\subsection{\citep{EvansZhao:1993}}


\begin{itemize}
\item Good Synch to Cosine at high and low freq, Poor synch to Random at high
  freq
\item BIC changes

  \begin{itemize}
  \item OnC/L \ensuremath{\rightarrow} PL, reduced thresholds by 12dB average
  \item increases rate in sustained PSTH
  \item BW (10 times AN) and RA spread to accommodate reduced thresholds
  \end{itemize}
\item Strychnine : no changes, little or no effect
\item Facilitation outside conventional receptive field could be masked by GABAA
  \citep{JiangPalmerEtAl:1996,WinterPalmer:1995}
\item Phase incoherence in echoic situations is a potential problem for the
  temporal encoding of the fundamental frequency of harmonics.  GABAA{}-ergic
  inhibition in the case of OnC/L cells may be a mechanism for minimizing these
  effects.
\item Tone-evoked IPSPs responsible rather than tonic inhibition
  responsible. Increased bandwidth of integration and or increased coincidences
\item OnC have no lateral inhibition (S Greenberg): question of anesthesia or
  something else
\end{itemize}

\subsection{\citep{CantBenson:2003}}
Type II multipolar cells of the ventral cochlear nucleus
are the source of widespread inhibitory inputs to the ipsilateral dorsal
cochlear nucleus and the contralateral dorsal and ventral cochlear nuclei.

{\textquotedblleft}A second, less plentiful, group of large multipolar cells
described by Smith and Rhode in the cat ([220],referred to here as type II
multipolar cells) probably corresponds to the D-stellate cells in mouse [151]
and to the radiate cells in rat [55] and also probably to the commissural cells
described in a number of species [13,42,113,204,213,254]. These neurons can be
distinguished from the type I multipolar cells in a number of ways. On average,
they are larger, and their dendrites radiate across the isofrequency laminae of
the VCN [54,61,151,169]. Both the dendrites and the somatic surface are covered
with synaptic terminals, many of which appear to arise from the auditory nerve
[13,220]. (The so-called type II stellate cells in the anterior AVCN [36] are
also characterized by extensive somatic innervation, but as noted above, it has
not been established whether the cells in the anterior AVCN should be considered

to be a different population from the multipolar cells in the rest of the AVCN
and in the PVCN.) Few of the synaptic terminals on the type II multipolar cells
are glycinergic [113], but many of the cells themselves are glycinergic
[23,54,113,254]. Type II multipolar cells give rise to terminal dendritic tufts
that characteristically extend into the granule cell layer bordering the VCN
[55,151,179,204,220],cf.  [142]). The type II multipolar cells are broadly tuned
and correspond to a group of units characterized physiologically as
{\textquotedblleft}onset chopper{\textquotedblright} (OnC) units, which show a
well-timed onset response followed by a few regularly timed spikes that

are not sustained throughout the stimulus as are those of
{\textquotedblleft}chopper{\textquotedblright} units [168,169,220]. The OnC
units are widely scattered throughout the VCN [168,169] matching the
distribution of the commissural cells [113,204,254]. Compatible with their
wide-reaching dendritic arbor, the type II multipolar cells appear to integrate
inputs over a wide frequency range [168]. The known targets of the type II
ultipolar cells are illustrated in Fig. 2G. The axons leave the cochlear nucleus
via the intermediate acoustic stria [3,61,113,204,220]. The distribution of
multipolar neurons in the VCN whose axons leave via the IAS is very similar to
that of the commissural cells (compare [3,42]), and the only known targets of
these axons are the contralateral dorsal and ventral cochlear nuclei
[13,113,204]. The axons also give rise to extensive collateral branches within
the parent VCN and send widespread terminations to the ipsilateral DCN
[4,55,61,151,164,220]. In both the ipsilateral and contralateral cochlear
nuclei, the axonal terminals contain pleomorphic or flattened synaptic vesicles,
consistent with an inhibitory effect on the target cells [13,220]. They would be
expected to form a major source of wideband inhibition to the DCN [54]. The type
II multipolar cells do not appear to project to the inferior colliculus
[12,102]. It is not known if they give rise to any of the branches making up
other projections attributed to multipolar cells.{\textquotedblright}


\citep{BrewForsythe:1995}

\subsection{In rat urethane \citep{PaoliniClark:1999}}


\begin{itemize}
\item Increasing intensity below CF had a greater depolarizing effect than
  similar increases above CF
\item FS latency and variability decrease as intensity is increased for any freq
  in RA, dependent on rise time of initial depolarization
\item On CF: offset hyperpolarisation common
\item At High freq edge: onset and occasionally an offset AP see fig 10
\item Above High Freq edge long-lasting hyperpolarisation (fig10a,11) seen in
  neurons when tone presented above high freq edge or in combination with CF
  tone
\item Clicks evoke rise in depolarization followed by hyperpolarisation after
  10-15msec
\item PSTH may also be analogous CT Rhode and Smith (1986) when the late
  activity is vigorous. Smith and Rhode (1989) refer to the intracellular
  studies of Romand (1978, 1979) that relate to Ct units. They state that the
  descriptions of these units are similar to those observed in their OC
  neurons. In this investigation we classified our units as OC in accordance
  with Rhode and Smith (1986). Although others established various
  classification methods to distinguish between OC and CT units
  \citep{BlackburnSachs:1989,WinterPalmer:1995}, absolute criteria for this
  distinction are absent. In the absence of absolute criteria it is possible
  that some of our neurons may also be classified as Ct neurons by
  others. However, as the neurons recorded in this investigation show similar
  intracellular response features to tones and share similar morphology, it is
  likely that they represent a uniform cell class.
\end{itemize}

\begin{itemize}
\item The initial component appears to be intensity dependent.  intensity is
  coded by the initial component only when frequencies are presented below or
  close to CF.
\item Sustained component depends on both the frequency and intensity.  The
  change in depolarization amplitude was greatest at the low-frequency edge of a
  neuron{\textquoteright}s response area and decreased with increasing
  frequency.
\end{itemize}

\begin{itemize}
\item Polysynaptic inhibition is evident in D stellate cells
  \citep{FerragamoGoldingEtAl:1998a,PaoliniClark:1999, PaoliniClark:1998}
\item Suppression of GABA in DCN resulted in a decrease in response threshold
  for units in the VCN (Paolini et al. 1998)
\item Shunting inhibition suggested by \citep{SmithRhode:1989} Vrest close to
  Chloride reversal potential
\item \citep{WickesbergOertel:1988} found that cells were labeled dorsally
  (higher CF) when HRP was injected to a localized portion of the VCN. Hence
  high CF \ensuremath{\rightarrow} low CF inhibitory source.
\end{itemize}

\subsection{In mice slice \citep{FerragamoGoldingEtAl:1998a}}


\begin{itemize}
\item GABAergic inhibition plays a prominent role in the synaptic responses of D
  stellate cells. Both the frequency and duration of firing were augmented in
  the presence of picrotoxin, a GABAA blocker.
\end{itemize}
\citep{RhodeSmith:1986}

{\bfseries Fig7F \citep{RhodeSmith:1986} shows isorate curves for five Oc
  neurons in cat}

\subsection{On-C neurons \citep{ArnottWallaceEtAl:2004} in guinea pigs}


\begin{itemize}
\item Physiological responses. Five cells, labeled in the AVCN, were of the type
  defined as On-C by Winter and Palmer (1995).
\item fully reconstructing four of the cells using NeurolucidaTM (Neurolucida,
  Microbrightfield,Colchester, VT).
\item tonotopic gradient within AVCN starts with high frequencies at the
  dorsocaudal edge and ends with low frequencies at the ventroanterior edge;
\item Despite the length of the axon, for all four of the reconstructed On-C
  cells, local terminations appeared to be arranged largely within a volume of
  tissue representing BFs that were within 1 octave of the injected
  cell{\textquoteright}s BF. They appeared to be centered on the same
  isofrequency slab as the dendrites (Fig. 9A,B).
\end{itemize}






\section{Golgi cells}

\citep{Cant:1993,MugnainiOsenEtAl:1980}

The Golgi cells are the only cell type in the small cell cap that receives
extensive somatic input.

Small round terminals are most common, then Flattened/pleomorhpphic and rarely
are Large round terminal sfound.

Granule cells

For response characteristics using cFos expression see Yang, Neurosci 136 (2005)



\section{Modelling Stellate Cells}
\subsection{Auditory model used in Modelling Stellate cells}


\begin{table}[htp]
  \centering
      \begin{tabularx}{\textwidth}{XXXXXX}
        \toprule
 CN Study & Range & Spon Rate & Per channel & Saturation & AN        Model\\\midrule
 \citep{ErikssonRobert:1999} & 120 channels (200-20000Hz) &        65, 40 and 5 (5, 20 and 30 Thr;30, 60 and {\textgreater}90 dB satur) & &        & \citep{RobertEriksson:1999}\\\hline
 \citep{ReissYoung:2005} & 1/100        octave over 4 octaves centred on 2.5kHz & Only high spon (sound        restricted to pass-band levels of 0dB) & 10 unique spike times randomly        selected each BF & & \citep{BruceSachsEtAl:2003} based on        \citep{ZhangHeinzEtAl:2001} \\\hline
 \citep{HancockVoigt:1999} & 0.2 oct        separation over audible range in cat, 5kHz centre \citep{Liberman:1978}        & Uses SR and TH experi-mental data from cat \citep{Liberman:1978} & 800        total, 40 per channel & & \citep{Carney:1993}\\\hline  
      \citep{ErikssonRobert:1999} &        Phase-locking AN below 1kHz or AN at 6kHz & High & Up to 50 independent        inputs & Steady-state responses for Phase locking &        \citep{RobertEriksson:1999}\\\hline
 \citep{PressnitzerMeddisEtAl:2001}& 10        channels (two octaves below to one octave above) & & 20 fibres per        channel & & \citep{MeddisHewittEtAl:1990}\\\hline

\citep{BahmerLangner:2006,BahmerLangner:2006a} &&&&&\\ \hline

\citep{WiegrebeMeddis:2004} &&&&&\\ 
\bottomrule
      \end{tabularx}
\end{table}


\begin{table}[h]
  \tablehead{}
  \begin{tabularx}{\textwidth}{XXXXX}
    \hline Study & Description & Specialisation & Cells & \\\hline
    \citep{ErikssonRobert:1999} & Point neuron model based on
    \citep{ArleKim:1991a} plus dendritic filter & an after-spike
    hyperpolarization potassium(A-type) conductance, dendritic filter & Stellate
    (with extra K+), D stellate and TV cells (simple point model) & \\\hline
    \citep{HancockVoigt:1999,ReissYoung:2005} & Point Neuron Models of DCN &
    Simple, all syn time constants are 10 msec & TV, DS and Fusiform cells &
    \\\hline
 \citep{BanksSachs:1991} & Active somatic and axonal compartments
    with passive dendrites & Modified versions of HH equations (INa IHT) & Just
    stellate & \\\hline
 \citep{WangSachs:1995} & Same as above & Shfted (INa
    IHT) 10mV more positive from Banks and Sachs & Just Stellate & \\\hline
    \citep{ArleKim:1991b} & MacGregor Point neuron model & IHT treated as on/off
    & & \\\hline
 (Hewitt et al. 1992) & MacGregor Point neuron model & & &
    \\\hline
 (Pressnitzer et al. 2001) & MacGregor Point neuron model & & 50
    Narrow band excitatory cells (T stellate cell) all with 1.1kHz BF &

    Vrest -60mV,$\tau $m 2msec,Ri 33M$\Omega $,Ek-10,gk20nS, $\tau$Gk2.5msec, Th 5.3 Th boost 0mV, $\tau $Th 20\\
    & & & 50 Wide band inhibitor (D stellate cells) all with 1.1kHz BF & Vrest
    -60mV,$\tau $m 1msec,Ri 33M$\Omega $,Ek-10,gk40nS, $\tau$Gk1msec, Th 10, Th
    boost 10mV, $\tau $Th 11\\\hline
 \citep{RothmanManis:2003c} & Conductance
    Model & Accurate kinetic models of (INa IHT ILT IA and Ih) at
    22{\textordmasculine}C & Type 1 --classic &
    7.0 msec tm 473 Mohm input R \\
    & & & Type 1 --transient with IA &
    4.0 msec tm  453 Mohm input R\\
    & & & Type 1-2 with ILT & 3.7 msec 312 Mohm\\\hline
  \end{tabularx}
\end{table} 


\begin{table}[h]
\caption{ Rothman and
  Manis VCN model cell types}

  \begin{tabularx}{\textwidth}{XXXXXX}
    \toprule & \multicolumn{5}{m{13.01cm}}{Model Type}\\
 & I-c &    I-t & I-II & II-I & II\\\midrule
 gNa, nS & 1000 & 1000 & 1000 & 1000 &    1000\\
    gHT, nS & 150 & 80 & 150 & 150 &    150\\
    gLT, nS & 0 & 0 & 20 & 35 &    200\\
    gA, nS & 0 & 65 & 0 & 0 &    0\\
    gh, nS & 0.5 & 0.5 & 2 & 3.5 &    20\\
    glk, nS & 2 & 2 & 2 & 2 &    2\\
    Vrest, mV & {}-63.9 & {}-64.2 & {}-64.1 & {}-63.8 &    {}-63.6\\
    Rrest, M$\Omega $ & 473 & 453 & 312 & 244 &    71\\
    $\tau $m, ms & 7.0 & 4.0 & 3.7 & 2.9 &    0.9\\
    Vth, mV & {}-38.3 & {}-34.9 & {}-51.2 & {}-58.0 &    {}-62.2\\
    S-50/-70, nS & 0.3 & 0.3 & 5.0 & 12.6 &    49.5\\
    gE$\Phi $ at 22{\textordmasculine}C, nS & 2.0 & 2.2 & 2.8 & 3.2 &    8.6\\
    gE$\Phi $ at 38{\textordmasculine}C, nS & 11 & 12 & 15 & 17 & 34\\
\bottomrule
  \end{tabularx}
\end{table} 

\begin{landscape}
%\afterpage
{\small\LTXtable{210mm}{ModellingCNTable}}  
\end{landscape}


% \begin{table}[h]
% \caption{ Connections  used in modeling studies}

%   \begin{tabularx}{\textwidth}{XXXXXXX}
%   \toprule
%  Connection &    Study &    Synaptic Time Constant (msec) &    Bandwidth (octaves) &    No. of Synapses /Strength &    Distribution &    Delay\\\midrule
%     ANI\ensuremath{\rightarrow}TS &    \citep{ErikssonRobert:1999} &    1 msec &    &    N=6 or 15 (L,M,H even) G=8.5/n single or G=2.0/n per compartment &    1) Dendrites only 2) Dendrites and Soma &    1.7ms \\\hline
%  & \citep{RothmanManis:2003c} & {0.4 msec time to peak alpha at 22C (7.1msec half      width in type 1, 1.6msec type 2) 0.07 msec alpha at 38C} & &     {gE$\Phi $ at 22{\textordmasculine}C=2 or 2.2 for      type 1, subthreshold gE=0.5*gE$\Phi $, n=1,3,10 or 50 } &     {On soma of single compartment } &     none\\\hline 
%  &    \citep{PressnitzerMeddisEtAl:2001} &     {5msec} &     {One channel (20 inputs)} &     {3nA current pulse (then filtered)} &     {PSPs filtered using IIR low pass filter} &     \\\hline 
% {ANI\ensuremath{\rightarrow}DS} &    {\citep{ErikssonRobert:1999}} &    {1 msec } &    {One octave above and below } &    {1/20 threshold for activation } &     &    1.7ms\\
%     {\citep{ReissYoung:2005}} &    {10msec} &    {3 oct (Gaussian effectively makes 2 oct)} &    {N*g=10.8} &    {} &    \\\hline 
% &    {(Pressnitzer et al. 2001)} &    {1msec} &    {10 channels (-2/+1 oct) even weighting (10*20      inputs)} &    {0.3nA current pulse (then filtered)} &    {PSPs filtered using IIR low pass filter} &    \\\hline 
% ANI\ensuremath{\rightarrow}TV &     \citep{ErikssonRobert:1999} &    1 msec &    {\textpm}0.1 &    G+=2.3/n, n=21 &   &    1.7ms\\\hline 
% &    \citep{ReissYoung:2005} &    10 msec &    0.2 integration window &    N*g=23 &    &    \\\hline 
% TS\ensuremath{\rightarrow}TS &  \citep{WiegrebeMeddis:2004}  &    &    &    &    &    \\\hline 
% &\citep{BahmerLangner:2006,BahmerLangner:2006a}&&&&&0.4 msec\\\hline
% TS\ensuremath{\rightarrow}DS &
%     &
%     &
%     &
%     &
%     &
%     \\\hline 
% TS\ensuremath{\rightarrow}TV &    &   &    &    &    &    \\\hline 
% DS\ensuremath{\rightarrow}TS &     \citep{ErikssonRobert:1999} &     6 msec &     &     N=3 or 8, Gweak=0.5/n Gmod=1/n Gstr=1.5/n &    &    1.2{\textpm}0.5 msec \\\hline 
% &    (Pressnitzer et al. 2001) &     5 msec (?)  &     All neurons share same BF 1.1kHz &    {}-1nA current pulse, N=50 &    &    2 msec delay introduced\\\hline 
% DS\ensuremath{\rightarrow}DS &    &     &     &     &    &    \\\hline 
% DS\ensuremath{\rightarrow}TV &    \citep{ErikssonRobert:1999} &    6msec &    {\textpm}0.4 &    G-=1.5/n, n=8 &    &    \\\hline 
% &    \citep{ReissYoung:2005} &    10msec &    0.05 oct with +0.3 oct offset (input BF rel to target BF) &    N*g=7 &    &    \\\hline 
% & \citep{YoungDavis:2002,HancockDavisEtAl:2001,SpirouDavisEtAl:1999,HancockDavisEtAl:1997,DavisVoigt:1996,DavisVoigt:1994,DavisVoigt:1991}    &   &    &    &    &    \\\hline 
% &\citep{ReissYoung:2005}    &   &    &    &    &    \\\hline 
% TV\ensuremath{\rightarrow}TS &    \citep{ErikssonRobert:1999} &    6 msec &    &    N=3 or 8, Gweak=0.5/n Gmod=1/n Gstr=1.5/n &    &    1.2{\textpm}0.5 msec \\\hline 

% TV\ensuremath{\rightarrow}DS &     \citep{ErikssonRobert:1999} &    6 msec &    {\textpm}0.4 &    G-=2.3/n, n=12 &    &    \\\hline 
% ANII\ensuremath{\rightarrow}G &    &  &     &     &    &    \\\hline 
% Granule\ensuremath{\rightarrow}G &    &    &    &    &    &    \\\hline 
% Golgi\ensuremath{\rightarrow}TS &    &     &   &    &    &    \\\hline 
% Golgi\ensuremath{\rightarrow}DS &    &    &    &    &    &    \\\hline
%   \end{tabularx}
% \end{flushleft}


\citep{RothmanManis:2003a, b, c} Present a robust model of VCN neurons based on
previous experimental studies.


\begin{itemize}
\item Replicates current clamp responses: type I of stellate cells and type II
  of bushy cells
\item Replicates simple PSTH responses solely based on sub- or supra-threshold
  excitation at the soma
\item Replicates Phase locking capabilities of neurons (simulated steady state
  input)
\item Derived from a complete characterisation of K+ currents rather than ad hoc
  assumptions: hence more accurate.  The greatest difference between the
  previous models is voltage dependent IHT and ILT currents.
\item IA has a role in modulating the rate of repetitive firing.  Increasing gA
  counteracted depolarising effects of EPSPs, thereby increasing threshold for
  AP.  EFFECTS of INHIBITION on T stellate cells could be to reset IA
\item ILT plays a role in type II by reducing input R hence reducing the
  membrane time constant.  In intermediate type I-i cells small amount of ILT
  had a greater affect on rate of firing than IA. Small ILT could also benefit
  neurons by reducing EPSPs near the axon hillock and reducing AP back
  propagation in proximal dendrites.
\item Modulation of ILT: Coincedence detection neurons would benefit from ILT
  upregulation to reduce membrane time constant, reduce EPSP height and
  width. \ensuremath{\rightarrow} enhanced temporal acuity at onset, reduced
  firing during sustained period, reduced spontaneous activity, reduction in
  refractory period that leads to faster firing rates.  Increasing ILT drops
  Vrest, hence Ih must be used to counterbalance ILT.
\end{itemize}

\section{Speech processing}

\subsection{\citep{PalmerShamma:2003} }
Good review of speech processing focusing on
physiological mechanisms underlying the processing of pitch and timbre


\begin{itemize}
\item Includes discussion on encoding of the shape of acoustic spectrum

  \begin{itemize}
  \item Place, temporal and mixed place-temporal representations
  \item Spectral dynamics : freq sweeps, VOT, consonant formant transitions
  \item Contextual effects: suppression effects in AN and CN
  \end{itemize}
\item Encoding of Pitch:

  \begin{itemize}
  \item Spectral pitch hypothesis
  \item Temporal pitch hypothesis
  \item Correlates of Pitch: AM coding in AN and CN
  \item Sensitivity to FM
  \end{itemize}
\end{itemize}
\citep{Rhode:1998} Looks at single formant stimuli in chinchilla for F0
(100-200Hz) and F1 (256-782Hz) in 95 combinations.


\begin{itemize}
\item Use autocorrelation, ISI and synch for each f0
\item On, Ch better than Primary-like
\item All units showed gain in phase locking to f0
\item Preservation of all-order ISI representation of f0 and F1 in all
  populations
\end{itemize}


\subsection{\citep{ClareyPaoliniEtAl:2004}}

\begin{itemize}
\item Presented stop CVC syllables to guinea pigs (urethane) and found
  statistically significant differences in all cell types, best responses were
  from those with BFs above the first formant
\item Prominence of response to stop release, voice onset and noise in between
  offset and onset are infl;uenced by

  \begin{itemize}
  \item Intensity
  \item Spectrum of onset (stop release)
  \item BF
  \item Cell type
  \end{itemize}
\item Pressnitzer et al 2000 demonstrated better coding of damped and ramped
  sinusoids in CH and ON compared to PL -- temporal asymmetry similar to natural
  glottal pulses and stop releases
\item CH response encoded VOT through clear response peaks

  \begin{itemize}
  \item Reduced activity between Onset and voice onset -- inhibiting spon
    activity
  \item Peaks well defined (1 msec bins) narrower than PL
  \item First glottal pulse excites wide range of BFs (\~{}7-8 kHz) depending on
    syllable
  \item Low BF CH showed good response to pre-first glottal pulse in /tot/
  \end{itemize}
\item ON responses were accurate at 75 dB (all ON neurons (6.9-10kHz BFs)

  \begin{itemize}
  \item Very precise response to stop release and voice onset
  \item Some cells responded to peaks in vowel consistently
  \end{itemize}
\item Ensemble PSTHs

  \begin{itemize}
  \item At moderate and moderately to high speech SPLs (65,75 and 85 dB) the
    VOT{\textquoteright}s of the three voiced stop syllables (fig 8) are
    accurately reflected in the interval between sharp and statistically
    significant increases in discharge rate associated with stop release and
    voice onset
  \item AS SPL decreases the magnitude of the average stop release peak
    decreases in all syllables
  \item Similar behaviour for whispered syllables: reduction in stop release
    spikes with decreasing SPL
  \item Saturation effects at high SPL in some units
  \end{itemize}
\item Activity during VOT

  \begin{itemize}
  \item Affected by noise: can be estimated from energy in auditory filter
  \item CH units with high SR showed suppression during VOT when formant peak
    lie adjacent to the units BF
  \end{itemize}
\end{itemize}

\subsection{Coding of F0 in voiced sounds in onset units \citep{PalmerWinter:1993}}


\begin{itemize}
\item F0 varies over an octave (80-150 Hz in males, 160 to 300 Hz in females and
  200-400 in children)
\item Studied extensively psychophysically \citep{Evans:1978, Greenberg:1980}
\item Harmonic series (including voiced speech) produce modulation of the
  discharge at the F0 of the fibres at frequencies where their response area is
  wide enough to allow more than one component to interact, provided a single
  intense component is not dominating the output.
\item Phase-locking to F0 occurs across all AN fibres, and to harmonics of the
  F0 \citep{MillerSachs:1984,Palmer:1990}, fine structure of inharmonic series
  evident in AN and allows AN to convey perceived pitch \citep{Evans:1978,
    DelgutteKiang:1991}
\item CN: Kim and collegues have shown improvement in ability to sychronise
  \citep{KimLeonard:1988}(,Kim et al. 1986) onset units described as having
  {\textquotedblleft}pitch period following response{\textquotedblright}
\end{itemize}

\subsection{\citep{RecioRhode:2000}}

\begin{itemize}
\item CT found to respond better to steady state of vowel pitch pulses than at
  onset
\item Calculated period histograms and FFT
\item responses of CT neurons to /$\varepsilon $/, mean values of the
  synchronization indices (evaluated at f0) at the onset and during the
  steady-state part of the response are 0.52 (63 neurons $\sigma $ = 0.24) and
  0.65 (63 neurons, $\sigma $ =0.23), respectively
\item PL responses more random than CS and CT: CS CT consistent over several
  stimuli, test spike count per stimulus with Fano factor \citep{Teich:1989} F =
  Var(Ni)/Mean(Ni) , for Poisson F=1, with refractory Var(Ni)
  {\textless}Mean(Ni)
\item Few CT neurons exhibited rate suppression ({\textless}5\%), less than
  Spon, may be due to low spon already in Choppers \citep{RhodeGreenberg:1994b}
\item FFTs of period histograms

  \begin{itemize}
  \item PL phase-lock to formant closest to their BF, BFs {\textless}900 Hz
    phase lock to F1 and second harmonic of F1
  \item CT at 40dB doesn{\textquoteright}t indicate presense of F2, CT with BFs
    {\textgreater}1kHz only phase lock to f0
  \item At 60dB PL didn{\textquoteright}t deteriorate, CT units
    {\textgreater}800Hz improved F1 locking
  \end{itemize}
\item Dominant Frequency component Analysis \citep{SinexGeisler:1983}

  \begin{itemize}
  \item Highest peak in FFT(PST) then plot against BF
  \end{itemize}
\item Normalised rate better in CT than PL

  \begin{itemize}
  \item Quality of F1-T1-F2 trough uses quality measure (r(F1)- r(T1)
    +r(F2)-r(T1))/0.5
  \end{itemize}
\item Rate Analysis using SMP

  \begin{itemize}
  \item Best representation found in chopper neurons
  \end{itemize}
\item PL encode spectrum of vowel (position of F1 and F2) in time of APs even at
  60-80dB (strong synch to F1 and F2)
\item Chopper neurons rarely encode F2, choppers with BFs {\textgreater}2kHz are
  dominated by f0 phase-locking
\item Temporal responses (is SI values) as expected from other studies
\item Rate-based CS and CT better than PL, worst responses from /o/ and /a/
  where F1 and F2 are close together
\item CT not better than CS as was the case in \citep{BlackburnSachs:1990} for
  both temporal and rate based measures (different species used, PVCN vs AVCN?)
\end{itemize}


\subsection{\citep{BlackburnSachs:1990} barbiturate cat}


\begin{itemize}
\item Choppers have degraded temporal representations compared to AN
\item ChS and ChT have better rate-place representations than whole AN,
  resembles low to medium ANF
\item They suggest: {\textquotedblleft}that a functional partition of the AVCN
  chopper population could yield two distinct rate representations in response
  to a complex stimulus: one that is graded with stimulus (over 30-40 dB range)
  and that, even at rate saturation, maintains a {\textquoteleft}low
  contrast{\textquoteright} stimulus representation,and a second that maintains
  a robust, {\textquoteleft}high contrast{\textquoteright} stimulus
  representation at all levels but that confers less information about stimulus
  level.{\textquotedblright}
\item Use 35, 55, 75 dB stim levels
\item Calc FFT of period histogram, SI at each harmonic and plot against BF,
  Normalised Rate-place using moving-window 0.25oct wide \citep{YoundSachs:1981}
\item ChS show peaks in rate-place representation at 35 dB at 1st, 2nd and 3rd
  formant frequencies, definition of peaks deteriorates with increasing level,
  rates at peak saturate while rates in troughs increase.  Graded response to
  vowel over wider dynamic range with somewhat lower contrast.
\item ChT normailsed rate profiles show similar peaks throughout the range of
  stimulus levels used (25-75), at high levels definition of peaks better than
  ChS population at any level.  Shape of profile does not change below the 3rd
  formant for stimulus levels 35-75.  {\textquotedblleft}HIGH
  Contrast{\textquotedblright} representation of the stimulus at least up to the
  region of the 3rd formant, above 3rd formant energy spreading is typical of
  noise spreading
\end{itemize}

\subsection{\citep{MayPrellEtAl:1998}}

SMP in awake cats exhibited same general response properties of anesthetized
cats but larger between subject differences in vowel driven rates.


\begin{itemize}
\item May goes into detail about how to use the SMP to centre the formant or
  trough onto the BF of the unit
\item Signal detection methods based on driven rate to a vowel
\end{itemize}

\subsection{\citep{SachsWinslowEtAl:1988} representation of speech in the auditory periphery}

{\textquotedblleft}Viability of an average rate code in some circumstances
requires assumptions that can neither be refuted of supported by
data.{\textquotedblright}

A code based on phase-locked features of unit responses provides and extremely
precise representation of all but very high-frequency speech features
(i.e. fricative consonants).  However, there is little evidence that the central
nervous system is capable of using such fine temporal details of the auditory
nerve discharge patterns.

Overall good summary of speech representation in the auditory nerve as of 1988--
it deals with rate place and temporal-place representations of stimulus spectra,
their dependence on level, the role of Low SR ANFs, differences in envelopes of
rate profiles across vowels, the role of the efferent system, temporal responses
distributed across CF (ALSR), temporal responses in AVCN.

\citep{Geisler:1988}

\citep{RhodeGreenberg:1994b}


\begin{itemize}
\item the rate/place representation for vocalic stimuli in the AN, which
  resolves the spectral peaks at low sound pressure levels (SPLs) and is
  preserved to a degree among the low and medium spontaneous rate (SR) fibers at
  higher intensities, is severely compromised in the presence of broadband noise
  \citep{GeislerGamble:1989,SachsVoigtEtAl:1983} even among fibers of low SR
  activity \citep{SachsVoigtEtAl:1983,SilkesGeisler:1991}.
\end{itemize}

\section{Inhibitory Influence on RA}
\citep{CasparyBackoffEtAl:1994}


\begin{itemize}
\item Pentobarbital chinchillas
\item Blockade of GABA and/ or glycine inputs was found to increase discharge
  rate primarily within the excitatory response area of neurons displaying
  chopper and primary-like temporal responses with little or no change in
  bandwidth or in off-characteristic frequency (CF) discharge rate.
\end{itemize}

\begin{itemize}
\item GABA is involved in circuits that adjust relative gain to enable the
  detection of signals in noise by enhancing signals relative to background
  noise.
\item ON-CF inhibition in auditory pathway
  \citep{CasparyPalombi:1993,CasparyPalombiEtAl:1993,EvansZhao:1993,PalombiCaspary:1992}
\end{itemize}

Octave bandwidth analysis (table 3)
\begin{flushleft}
  \tablehead{}
  \begin{tabularx}{\textwidth}{XXXXXX}
    \hline 
AVCN choppers & n & N effected (effect {\textgreater}10\%) & control    & drug & \% increase\\\hline 
BIC & 27 & 8 & 1.55 & 1.74 & 8.8\\\hline
    Strychnine & 21 & 9 & 1.73 & 2.0 & 15.7\\\hline
  \end{tabularx}
\end{flushleft}

Inhibitory RA (table 2)

\begin{flushleft}
  \tablehead{}
  \begin{tabularx}{\textwidth}{XXX}
    \toprule
& BIC & Stychnine \\\midrule 
n & 27 & 21\\        
Near CF & 20 & 12\\ 
Broad & 4 & 3\\     
Low/high & 2 & 4\\  
Off-CF & 1 &    2\\ 
\bottomrule
  \end{tabularx}
\end{flushleft}

\begin{itemize}
\item Intracellular recordings from presumed bushy and stellate neurons in vivo
  have shown evidence of inhibition \citep{RhodeSmith:1986,SmithRhode:1989},
  which has been supported by in vitro physiological and pharmacological studies
  with GABA or glycine application \citep{WuOertel:1986} . These findings are
  supported by immunocytochemical studies using antibodies for GABA and glycine
  showing labeled puncta present in large numbers surrounding somata and
  dendrites of bushy cells and stellate cells with abundant somatic contacts
  and, to a lesser extent, onto stellate cells with sparse somatic contacts
  \citep{AltschulerBetzEtAl:,SaintOstapoffEtAl:1993,SaintBensonEtAl:1991,SaintMorestEtAl:1989,WentholdHuieEtAl:1987,FexAltschulerEtAl:1986,WentholdZempelEtAl:1986,1986}.
\item AVCN choppers have greater dynamic range than ANFs for tones in noise
  \citep{MaySachs:1998,MaySachs:1992,PalmerEvans:1982,GeislerSilkes:1991,RhodeGeislerEtAl:1978,RhodeSmith:1986,YoungCostalupesEtAl:1983,YoungShofnerEtAl:1988,YoungRobertEtAl:1988}
\end{itemize}

\subsection{\citep{RhodeGreenberg:1994b} pentobarbital anesthetized cats}

Presented Noise and tones simultaneously to AN and CN units


\begin{itemize}
\item A primary consequence of lateral suppression is to preserve sharp
  frequency selectivity of CN neurons at moderate to high sound pressure levels,
  particularly in background noise. In this fashion lateral suppressive
  mechanisms potentially enhance the representation of spectral information on
  the basis of place/rate information relative to that in the auditory nerve
  under noisy background conditions.
\item Lateral suppressive mechanisms probably underlie the dynamic range shift
  seen in the presence of a simultaneously presented noise. This mechanism may
  be crucial for preserving the ability to perceive signals in a noisy
  background.
\item the rate/place representation for vocalic stimuli in the AN, which
  resolves the spectral peaks at low sound pressure levels (SPLs) and is
  preserved to a degree among the low and medium spontaneous rate (SR) fibers at
  higher intensities, is severely compromised in the presence of broadband noise
  \citep{GeislerGamble:1989} ,Sachs et al. 1984 even among fibers of low SR
  activity \citep{SachsVoigtEtAl:1983,MillerSachs:1984,SilkesGeisler:1991}.
\end{itemize}

\subsection{Lateral suppression}


\begin{itemize}
\item One possibility is that inhibitory mechanisms beyond the AN
  {\textquotedblleft}sharpen{\textquotedblright} the place/rate representation
  of complex signals in a manner analogous to that postulated by von Bekesy
  (1967) for the cochlea and by Hartline and Ratliff (1957) for the retina. In
  the CN such lateral suppression would presumably operate through a reduction
  of discharge rate at the edges of the tonotopically mapped excitation evoked
  by spectral peaks, thereby increasing the effective neural contrast between
  spectral peaks and valleys.
\item \end{itemize}

\section{Forward masking / short-term adaptation}

(Boettcher et al. 1990)

Difference between psychophysical and single unit data


\begin{itemize}
\item Single unit data dynamic range is typically less than 50 dB, whereas
  psychophysical DR is 80-100dB
\item Effect of masker duration extends over longer duration (up to 1000ms)
  compared to physiological duration (less than 150msec)
\item Recovery is linear in log time psychophysically but exponential in
  physiology \citep{Carlyon:1988}
\end{itemize}

IHC depletion effect ?  \citep{WestermanSmith:1987}

Electrical stimulation studies suggest central component. Fredrickson and Gerken
(1978) in behaving cats and Shannon (1983) in CI patients : time needed for
complete recovery of sensitivity exceeded the values reported for acoustic
stimulation

\section{Gain Modulation in Birds}

\citep{Lu:2007}

GABAergic transmission in NM is unusual in that it is depolarizing but potently
inhibitory %\citep{Hyson et al. 1995,Lachica et al. 1994,LuTrussell:2001,MonsivaisRubel:2001}.
It is depolarizing because of an unusually high intracellular Cl\_ concentration
(35--60 mM) in NM neurons even after neuronal maturation, measured with
perforatedpatch recording techniques
\citep{LuTrussell:2001,MonsivaisRubel:2001}. It is potently inhibitory because
GABAinduced membrane depolarization not only produces shunting inhibition, and
inactivates Na\_ channels, but also activates a low-threshold K\_ conductance
\citep{MonsivaisRubel:2001}.  The GABAergic inputs enhance phase-locking
fidelity of NM neurons in response to auditory inputs and thus improve the
ability of NM neurons to code timing information of sound stimuli (Monsivais et
al. 2000).  The GABAergic inputs are also proposed to function as a gain control
for the excitatory inputs to NM (Burger et al. 2005,Dasika et al. 2005).


\bibliographystyle{abbrvnat} 
\bibliography{../hg/manuscript/bib/MyBib}
\end{document}
