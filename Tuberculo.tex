%\section{Tuberculoventral cells}

%\subsection{Background}

\subsubsection{Morphology and Cellular Mechanisms of Tuberculoventral cells}

Tuberculoventral neurons in the deep layer of the DCN provide a delayed, frequency-specific
glycinergic inhibition to TS and DS cells in the VCN
\citep{ZhangOertel:1993,WickesbergOertel:1988}.  

Planar multipolar or vertical cells are the most populated in the deep layers of the \DCN
and correspond to neurons with a Type II \EIRA \citep{Rhode:1999} and are immunolabelled with glycine.

Not all vertical cells send axona collaterals to the \VCN \citep{Rhode:1999} and not all Type II units can be antidromically activated by shocks to the VCN

lateral tuberculoventral tract


The dendrites of TV cells are
aligned with ANFs and indicating narrow frequency tuning. TV cells have low
spontaneous rates and variable PSTHs; “pauser,” “chopper,” or “onset/sustained”
have been recorded \citep{ShofnerYoung:1985,SpirouDavisEtAl:1999}. They have
little or no response to wide band noise and firing rates to CF tones that are
non-monotonic functions of intensity.

Anterograde labelling in the DCN suggests glycinergic tuberculoventral cells
project tonotopically to the VCN not just on-CF, but also to the low and high
frequency side bands in the AVCN
\citep{OstapoffFengEtAl:1994,MunirathinamOstapoffEtAl:2004}.  Ultra-structural
labelling of synapses in the rat DCN suggest TV cells are inhibited by DS cells
and from sources in the DCN but excitatory inputs were not found from TS cells
\citep{RubioJuiz:2004}.  Intracellular responses from labeled TV cells in the mouse
show clear excitatory input from TS cells and diffuse inhibitory input from DS
cells \citep{ZhangOertel:1993}.


Tuberculoventral cells have a classic type I regular-spiking response to current clamp 
\citep{ZhangOertel:1993}. , with double exponential action potentials 

% Double
%               undershoots suggest ILT , Regular spiking               &               &     \citep{ZhangOertel:1993}     & 
% 100M{\textpm}20, but then state 85{\textpm}10 in table 1
%                       \citep{ZhangOertel:1993}                        & 
% \citep{EvansNelson:1973,WickesbergOertel:1990,WickesbergOertel:1993,WickesbergOertel:1988,WickesbergWhitlonEtAl:1991,Wickesberg:1996,YoungBrownell:1976,YoungVoigt:1981,ZhangOertel:1993}\\\hline


% \subsubsection{Synaptophysiology of Tuberculoventral cells}

% \subsubsubsection{Afferent input}

% In rats, the sole excitatory afferent input to Tuberculoventral cells comes from
% the auditory nerve \citep{RubioJuiz:2004}.


% Glutamate AMPA receptors [\citenum{ZhangOertel:1993}]                          
% The IPSC dynamics of the AMPA synapse in TV cells is significantly slower in TV cells (time constant 0.4 ms, rise time 0.15 ms) than the VCN stellate and DCN fusiform AMPA synapses (0.36 ms)  \citep{GardnerTrussellEtAl:1999}.


% $\sim$70$\mu$m (mice [\citenum{SpirouDavisEtAl:1999}])
%                               % RF Post
% $\sim$70$\mu$m (mice [\citenum{ZhangOertel:1993,SpirouDavisEtAl:1999}])
%                                 % Number
% $\sim$4 somatic contacts per cell, 10\% AN (guinea pig [\citenum{Alibardi:1999}]), 
% 17.7\% (ANF terminals, 32.5\% GLUT) of total synaptic terminals in deep layer (rat [\citenum{RubioJuiz:2004}])
% % [\citenum{SpirouDavisEtAl:1999,ZhangOertel:1993}] ;       
% %                                & % Position  
% Mainly dendritic {$<{}100\mu$}m [\citenum{Alibardi:1999,Liberman:1993,SpirouDavisEtAl:1999}]
%   %
% 1.0--1.5 ms [\citenum{ZhangOertel:1993,Rhode:1999,SpirouDavisEtAl:1999}]


% \citep{RubioJuiz:2004}
% In the DL 517 synaptic profiles were sampled. Synap-
% tic endings immunoreactive for glutamate were ϳ30% (n ϭ
% 168) of the total computed in this layer (Fig. 7B). The four
% subtypes of GLU-synaptic endings were observed and the
% GLU1 (n ϭ 81) was 3-fold more abundant than GLU4 (n ϭ
% 51), GLU2 (n ϭ 16), and GLU3 (n ϭ 20). Synaptic profiles
% immunoreactive for glycine and GABA represented 17% (n ϭ
% 88) of the total (Fig. 7B), and the GLY/GABA1 subtype (n ϭ
% 69) was three times more abundant than GLY/GABA2 (n ϭ
% 19). Synaptic profiles immunoreactive for glycine were the
% most heavily represented in this layer (48% of the total
% 7B), and the GLY2 subtype (n ϭ 175) was almost 2.5-fold
% more abundant than GLY1 (n ϭ 71). GABA-synaptic end-
% ings were less numerous (n ϭ 15; Fig. 7B).


% Synaptic endings on vertical cells (VCs). VCs are
% the most abundant inhibitory interneuron in the DL of the
% DCN. Cell bodies were located in the DL and had an approx-
% imate size of 15 ␮m in diameter. The nucleus was indented
% and centered located in the cell body and was surrounded by
% rough endoplasmic reticulum. Usually, a dendritic tree was
% observed attached to the cell body towards the superficial or
% the deepest part of the DL. Dendrites analyzed had an ap-
% proximate diameter of 1.2–2.5 ␮m. Cell bodies and dendrites
% were observed only immunoreactive for glycine (data not
% shown). These inhibitory interneurons received six of the
% nine subtypes of synaptic endings described above (Fig. 8).
% GLU1 was the only subtype found immunoreactive for glu-
% tamate on vertical cells and was preferentially on dendrites.
% Both types of synaptic endings immunoreactive for glycine
% and GABA (GLY/GABA1 and GLY/GABA2) were observed
% and both preferentially distributed on the cell body. How-
% ever, GLY/GABA1 was almost 3-fold more abundant than
% GLY/GABA2. The two types of synaptic endings immunore-
% active for glycine were observed distributed on vertical cells.
% GLY1 was observed preferentially on the cell body. The
% GLY2 was almost 2-fold more abundant and was found both
% on the cell body and on dendrites. GABA endings, although
% rare, were observed on both cell bodies and dendrites.


% The observation of inhibitory syn-
% apses on DCN interneurons suggests the presence of dis-
% inhibitory circuits. Consistent with this idea, inhibitory
% postsynaptic potentials can be recorded from VCs and can
% be blocked with both strychnine and bicuculline (Davis
% and Young, 2000). Indirect evidence of inhibition onto VCs
% also comes from their lack of spontaneous activity despite
% the presence of short latency excitation following auditory
% nerve stimulation (Zhang and Oertel, 1994). The blockade
% of glycine and or GABA does not increase spontaneous
% rates in the cell (Davis and Young, 2000).

% source of inhibitory afferents onto VCs is unknown and
% may come from intrinsic and/or extrinsic sources (Fig. 11).
% It has been suggested that glycinergic D-stellate cells from
% the ventral cochlear nucleus contact VCs (Davis and
% Young, 2000). Connections among VCs as well as those
% between CWCs and VCs may also be present (Manis et al.,
% 1994; Smith and Rhode, 1999). Consistent with this idea,
% GLY/GABA terminals were observed on both VC and
% CWCs.

% \subsubsubsection{DS to TV}

% OnC projections to the DCN have also been implicated in the wide-band glycin-
% ergic inhibition demonstrated physiologically in DCN principal cells, as well as
% in the cells designated type II or vertical (Caspary et al., 1987; Young et al.,
% 1992; Nelken and Young, 1994; Backoff et al., 1997; Joris and Smith, 1998;
% Spirou et al., 1999; Davis and Young, 2000; Ander- son and Young, 2004). Our
% electron microscopy of OnC terminals in the deep DCN shows that they can synapse
% on cell bodies or dendrites in this region, which is consis- tent with a
% potential influence of the OnCs on these cell types.

% \citep{DoucetRossEtAl:1999,DoucetRyugo:1997,FriedlandPongstapornEtAl:2003,DoucetRyugo:2006}
% Retrogradely labelled radiate neurons exhibited intense glycine immunoreactivity

% \subsubsubsection{TS to TV}
% %                       TS\ensuremath{\rightarrow}TV                        & AMPA
% % Glutamate
% % \citep{DoucetRossEtAl:1999,FerragamoGoldingEtAl:1998a,ZhangOertel:1993} No
% % TS terminals on TV cells in rats \citep{RubioJuiz:2004} In deep layer guinea
% % pig \citep{PalmerWallaceEtAl:2003} small round vesicles
% % \citep{Alibardi:1999} ;3 of 4 neurons had late EPSPs to AN shock, very young
% %                       mice \citep{ZhangOertel:1993}                       &       Similar to AN AMPA receptors.        & Strong
% % On-CF with weak off-CF See fig 13 \citep{OstapoffBensonEtAl:1999} ; mainly
% % in iso-band but poor classification in rats
% %         \citep{DoucetRossEtAl:1999,FriedlandPongstapornEtAl:2003}         & 
% % \citep{OstapoffBensonEtAl:1999} 3 of 4 shocks elicited late EPSPs
% %                         \citep{ZhangOertel:1993}                          & Mainly to fusiform cell layer, possibly on ends
% % of TV cell dendrites \citep{OertelWuEtAl:1990} CS and CT collaterals in deep
% % DCN possibly somatic \citep{PalmerWallaceEtAl:2003} somatic small round
% %   vesicles presumably from T stellates guinea pig \citep{Alibardi:1999}   & 0.15
% % sec min EPSP latency to VCN Glutamate puffs, main excitation at 0.3 sec Fig
% % 11a, AN shock produces late EPSPs about 3 msec \citep{ZhangOertel:1993}



% TS\ensuremath{\rightarrow}TV                        

% Glutamatergic AMPA receptor  [\citenum{DoucetRossEtAl:1999,FerragamoGoldingEtAl:1998a,ZhangOertel:1993}].
% %Small round vesicles present in deep layer of DCN (guinea pig [\citenum{Alibardi:1999}])

% Similar to ANF AMPA receptors.        
%         % RF Pre

%  % RF Post
% Strong On-CF with weak off-CF  (See fig 13 [\citenum{OstapoffBensonEtAl:1999}]). 
% Mainly in iso-band but poor classification in (rats: [\citenum{DoucetRossEtAl:1999,FriedlandPongstapornEtAl:2003}])         
%  %Number
% 3 of 4 neurons had late EPSPs to AN shock (very young mice [\citenum{ZhangOertel:1993}]).
% [\citenum{OstapoffMorestEtAl:1999}] 
% %                                & %Position
% No TS terminals on TV cells in rats (rats: [\citenum{RubioJuiz:2004}]).
% Mainly to fusiform cell layer, possibly on ends of TV cell dendrites (mice [\citenum{OertelWuEtAl:1990}]). 
% CS and CT collaterals in deep DCN possibly somatic (guinea pigs [\citenum{PalmerWallaceEtAl:2003}]).
% Somatic small round vesicles presumably from T stellates  (guinea pigs [\citenum{Alibardi:1999}])   
%  %Delay 
% 2 ms (Shock to VCN, AN shock produces late EPSPs about 3 msec, mice Fig.~10 \textit{cell 1} [\citenum{ZhangOertel:1993}])
% 150--250 ms (VCN Glutamate puffs, mice:  [\citenum{ZhangOertel:1993}]).


\subsubsection{Acoustic Response of Tuberculoventral cells}


% Responses of tuberculoventral neurons to sound
 Recordings \textit{in vivo} indicate that tuberculoventral cells probably have type II characteristics
 and respond with “onset” or “chopper”
 temporal response patterns \citep{ZhangOertel:1993b}. Units with type II responses are sharply tuned, they
 have thresholds - 10 dB higher than other units with which
 they are intermingled, and they do not respond to broad-
 band noise \citep{SpirouDavisEtAl:1999,YoungBrownell:1976,Young:1980,SachsYoung:1980,YoungVoigt:1982,ShofnerYoung:1985,VoigtYoung:1990,YoungSpirouEtAl:1992,Rhode:1999}. Young and his colleagues have
 shown that most neurons in the deep DCN respond to
 sound with either of two major types of response maps, type
 II or type IV 
\citep{EvansNelson:1973,ShofnerYoung:1985,VoigtYoung:1980,VoigtYoung:1990,Young:1980,YoungBrownell:1976}. 

%Type III units may represent a sepa-
% rate population of cells, although their distinction
% from type II units is not absolutely clear (Young and Voigt
% 1982). Units with type IV responses can be subdivided into
% those that do and those that do not have well-defined
% borders of a central inhibitory
% area (Spirou and Young 1991). The question how response maps and anatomic
% types are correlated has not been completely resolved.
 Many neurons in the DCN that could be driven electrically
 from the VCN had response maps of type II \citep{Young:1980}.
% Of 29 type II units tested, 6, or - l/5, could be driven antidro-
% mically from the VCN; of 7 type II units tested 1, or l/7,
% could be driven from the dorsal acoustic stria. 
Therefore
 some, but not necessarily all, type II units are tuberculoven-
 tral cells. Type II units have variable temporal firing pat-
 terns, “pauser,” “chopper,” or “onset/sustained,”
 and have firing rates that are nonmonotonic
 functions of intensity \citep{ShofnerYoung:1985}. Rhode and his colleagues,
 too, find variable temporal firing patterns in the deep layers
 of the dorsal cochlear nucleus. They suggest that “onset/
 graded” responses come from type II units (Rhode and
 Greenberg 1992). Onset / sustained \citep{ShofnerYoung:1985} and onset / graded peristimulus
 time histograms (Rhode and Greenberg 1992) are similar in that these units
 have a high probability of firing at the beginning of a tone at
 the best frequency and that the probability of firing drops
 off after the beginning of the tone.



\subsubsection{TODO Notch detection in DCN and echo suppression in VCN}



%\subsubsection{Output of Tuberculoventral cells}

%\subsubsection{Neuromodulatory effects in Tuberculoventral cells}



