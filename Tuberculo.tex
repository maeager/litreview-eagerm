\section{Tuberculoventral cells}

%\subsection{Background}

\subsection{Cellular mechanisms of Tuberculoventral cells}

In rats, the sole excitatory afferent input to tuberculoventral cells comes from the auditory nerve \citep{RubioJuiz:2004}.

\citep{RubioJuiz:2004}
% In the DL 517 synaptic profiles were sampled. Synap-
% tic endings immunoreactive for glutamate were ϳ30% (n ϭ
% 168) of the total computed in this layer (Fig. 7B). The four
% subtypes of GLU-synaptic endings were observed and the
% GLU1 (n ϭ 81) was 3-fold more abundant than GLU4 (n ϭ
% 51), GLU2 (n ϭ 16), and GLU3 (n ϭ 20). Synaptic profiles
% immunoreactive for glycine and GABA represented 17% (n ϭ
% 88) of the total (Fig. 7B), and the GLY/GABA1 subtype (n ϭ
% 69) was three times more abundant than GLY/GABA2 (n ϭ
% 19). Synaptic profiles immunoreactive for glycine were the
% most heavily represented in this layer (48% of the total
% 7B), and the GLY2 subtype (n ϭ 175) was almost 2.5-fold
% more abundant than GLY1 (n ϭ 71). GABA-synaptic end-
% ings were less numerous (n ϭ 15; Fig. 7B).


% Synaptic endings on vertical cells (VCs). VCs are
% the most abundant inhibitory interneuron in the DL of the
% DCN. Cell bodies were located in the DL and had an approx-
% imate size of 15 ␮m in diameter. The nucleus was indented
% and centered located in the cell body and was surrounded by
% rough endoplasmic reticulum. Usually, a dendritic tree was
% observed attached to the cell body towards the superficial or
% the deepest part of the DL. Dendrites analyzed had an ap-
% proximate diameter of 1.2–2.5 ␮m. Cell bodies and dendrites
% were observed only immunoreactive for glycine (data not
% shown). These inhibitory interneurons received six of the
% nine subtypes of synaptic endings described above (Fig. 8).
% GLU1 was the only subtype found immunoreactive for glu-
% tamate on vertical cells and was preferentially on dendrites.
% Both types of synaptic endings immunoreactive for glycine
% and GABA (GLY/GABA1 and GLY/GABA2) were observed
% and both preferentially distributed on the cell body. How-
% ever, GLY/GABA1 was almost 3-fold more abundant than
% GLY/GABA2. The two types of synaptic endings immunore-
% active for glycine were observed distributed on vertical cells.
% GLY1 was observed preferentially on the cell body. The
% GLY2 was almost 2-fold more abundant and was found both
% on the cell body and on dendrites. GABA endings, although
% rare, were observed on both cell bodies and dendrites.


% The observation of inhibitory syn-
% apses on DCN interneurons suggests the presence of dis-
% inhibitory circuits. Consistent with this idea, inhibitory
% postsynaptic potentials can be recorded from VCs and can
% be blocked with both strychnine and bicuculline (Davis
% and Young, 2000). Indirect evidence of inhibition onto VCs
% also comes from their lack of spontaneous activity despite
% the presence of short latency excitation following auditory
% nerve stimulation (Zhang and Oertel, 1994). The blockade
% of glycine and or GABA does not increase spontaneous
% rates in the cell (Davis and Young, 2000).

% source of inhibitory afferents onto VCs is unknown and
% may come from intrinsic and/or extrinsic sources (Fig. 11).
% It has been suggested that glycinergic D-stellate cells from
% the ventral cochlear nucleus contact VCs (Davis and
% Young, 2000). Connections among VCs as well as those
% between CWCs and VCs may also be present (Manis et al.,
% 1994; Smith and Rhode, 1999). Consistent with this idea,
% GLY/GABA terminals were observed on both VC and
% CWCs.


\subsection{Acoustic Response of Tuberculoventral cells}

\subsection{Notch detection in DCN and echo suppression in VCN}

%\subsection{Output of Tuberculoventral cells}

%\subsection{Neuromodulatory effects in Tuberculoventral cells}



