\begin{longtable}{cXXXXXX}%
%
\caption{Intracellular experiments in the VCN}\label{tab:Connections} \\
\toprule  
Cells    & Current Clamp &        Membrane Currents         & Time Constant (msec) &   I-V    &    Input R    & Papers\\ \midrule 
\endfirsthead

\multicolumn{7}{c}{{\bfseries \tablename\ \thetable{} -- continued from previous page}} \\
\midrule Cells    & Current Clamp &        Membrane Currents         & Time Constant (msec) &   I-V    &    Input R    & Papers \\ \midrule 
\endhead

\midrule \multicolumn{7}{c}{{Continued on next page}} \\ %\midrule
\endfoot
\bottomrule
\endlastfoot

   TS     & Type 1, single exponential undershoot
      \citep{FengKuwadaEtAl:1994,ManisMarx:1991,WuOertel:1984}        & No Low threshold
K \citep{ManisMarx:1991} IA has a role in modulating the rate of repetitive
firing.  Effect of Inhibition on T stellate cells could be to reset IA
                     \citep{RothmanManis:2003c}                       & 6.5{\textpm}5.7 msec \citep{ManisMarx:1991}
type I 9.1{\textpm}4.5 \citep{ManisMarx:1991} 6.2 to 18.0 msec
\citep{FengKuwadaEtAl:1994} 6.9{\textpm}3msec, 10-90\% rise time was
         1.05{\textpm}0.4msec \citep{IsaacsonWalmsley:1995}           & Linear
                       \citep{ManisMarx:1991}                         & 

447{\textpm}265 Mohm isolated guinea pig stellate cell type 1 current clamp
\citep{ManisMarx:1991} 44 to 151 M$\Omega $ (mean 89.4 {\textpm}24.4) mouse
slice prep \citep{FerragamoGoldingEtAl:1998a}
{\textquotedblright}stellate{\textquotedblright} 231{\textpm}113M$\Omega $,
14.9{\textpm}9pF primary membrane capacitance, room temp rat
\citep{IsaacsonWalmsley:1995} dog {Bal, Baydas, Naziroglu 2009} 176+/- 35.9
            Mohm membrane time constant 8.8 +/-1.4 (n=21)             & \\\hline
                                 TV                                   & Linear,
  regular spiking, double exp. Undershoots \citep{ZhangOertel:1993}   & Double
              undershoots suggest ILT , Regular spiking               &               &     \citep{ZhangOertel:1993}     & 
100M{\textpm}20, but then state 85{\textpm}10 in table 1
                      \citep{ZhangOertel:1993}                        & 
\citep{EvansNelson:1973,WickesbergOertel:1990,WickesbergOertel:1993,WickesbergOertel:1988,WickesbergWhitlonEtAl:1991,Wickesberg:1996,YoungBrownell:1976,YoungVoigt:1981,ZhangOertel:1993}\\\hline
                                 DS                                   & 
{}-56{\textpm}3.2 mV RMP see fig 15 Double expon. Undershoot
               \citep{PaoliniClark:1999,WuOertel:1984}                & Type I-i have high thresholds
probably mediated by small ILT \citep{RothmanManis:2003c} Membrane
properties of Oc cell have not bee adequately characterised, bu the
information that is available (d stellate in mouse
\citep{OertelWuEtAl:1990}) suggests that the low-threshold potassium channel
that is important in extending the phase-locking range of bushy cells
\citep{ManisMarx:1991,Oertel:1983} is not present in Oc neurons
                     \citep{WhiteYoungEtAl:1994}                      &     Fast      & Linear \citep{PaoliniClark:1999} & 40M
ohm \citep{OertelWuEtAl:1990} 96.2 {\textpm} 27.8 M$\Omega $ mouse slice
               prep \citep{FerragamoGoldingEtAl:1998a}                & \\\hline
                                Golgi                                 & Regular spiking
with overshooting action potentials and double exponential undershoot &               &                                  & 
             Inward rectifying FerragamoGoldingEtAl:1998              &   130 Mohm    & 
FerragamoGoldingEtAl:1998 \\

\end{longtable}

