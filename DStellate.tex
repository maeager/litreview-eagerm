

%\section{D~stellate cells}

%\subsubsection{Background}




% The glycine-positive, presumptive commissural neu-
% rones differed clearly from most of the surrounding cells
% by exhibiting only very few perisomatic and peridendritic
% (single-glycine- and double-labelled) puncta (Figs. 4C,
% 5 C). In fact the coverage of these cells appeared to be
% even less than that of the octopus cells, although the
% individual puncta appeared larger. Isolated dendritic
% segments were distinguishable because of their glycine-
% immunoreactivity.
% \citep{KolstonOsenEtAl:1992}


% \citep{PaoliniClark:1999} Long-lasting hyperpolarization was also seen in
% neurons during the presentation of frequency tones just above the high-frequency
% edge of their response area. With increasing click intensity both the
% depolarization and hyperpolarization increased in amplitude
 

% Evidence for inhibition on OnC neurons can also be seen with
% click stimuli. Hyperpolarization was observed in this investi-
% gation in the presence or absence of action potentials evoked
% by a single click stimulus. Click stimulation would excite not
% only a large proportion of excitatory projections to OnC neurons
% but would also evoke polysynaptic inhibitory and excitatory
% drive from intrinsic or extrinsic sources. The resultant hyper-
% polarization may be due to interplay between such excitatory
% and inhibitory influences. On presentation of successive click
% stimuli, hyperpolarization was not evident, although action
% potential generation decreased, suggesting the presence of in-
% hibitory influences.
% However, in past in vivo intracellular investigations, inhibi-
% tion was not consistently seen. Smith and Rhode (1989) rarely
% observed dips in the level of depolarization or distinct hyper-
% polarization below the cell’s resting potential that might indi-
% cate inhibition, whereas Feng et al. (1994) did observe hyper-
% polarizing influences in a Ct neuron akin to those observed in
% this investigation. This inconsistency may in part be related to
% the frequency of presentation as hyperpolarizing responses
% below resting potential were only seen in this investigation
% when tones were presented at or above the high-frequency edge
% of the neurons’ response area. The use of a nonbarbiturate
% anesthetic may also account for the presence of hyperpolariza-
% tion seen in this investigation. The ability of barbiturates to
% influence inhibitory mechanisms was well documented by
% Evans and Nelson (1973) in the cochlear nucleus. In this
% investigation, we used urethan, which when compared with
% barbiturates appears to have only minor effects on synaptic
% transmission (Crawford 1970; Maggi and Meli 1986).




% Further evidence for the involvement of inhibitory mecha-
% nisms in shaping the OnC response is provided by Ferragamo et
% al. (1998). They demonstrated with an in vitro preparation that
% D-stellate cells receive both glycinergic and GABAergic inhi-
% bition. D-stellate cells responded to auditory nerve stimulation
% with a fast followed by a slow depolarization. Although there
% was an absence of hyperpolarization, application of picrotoxin,
% a GABAA receptor blocker, to the cochlear nucleus slice en-
% hanced the firing of D-stellate cells. Ferragamo et al. (1998)
% propose that this GABAergic inhibition may originate from
% Golgi cells in the superficial granule cell domain. This re-
% sponse to auditory nerve stimulation is also seen in a stellate
% cell response type to intracochlear electrical stimulation re-
% corded in vivo (Paolini and Clark 1998). This response type
% also showed little visible evidence of inhibition, although a




% spike could not be evoked on the second excitatory postsyn-
% aptic potential response. Results obtained in this investigation
% and those of previous investigations (Ferragamo et al. 1998;
% Paolini and Clark 1998) suggest that polysynaptic inhibitory
% drive may play an important role in regulating the influence of
% excitation on these OnC neurons.
% In addition to GABAergic inhibition, glycinergic inhibitory
% input may also result from tuberculoventral cells in the dorsal
% cochlear nucleus (DCN) (Saint Marie et al. 1991; Wickesberg
% and Oertel 1990). Our recent study has shown that the DCN is
% an important contributor to on-frequency inhibition of VCN
% activity (Paolini et al. 1998). Suppression of DCN by applica-
% tion of muscimol, a GABA agonist, resulted in a decrease in
% response threshold for units in the VCN (Paolini et al. 1998).
% Extrinsic inhibitory input may also originate from the superior
% olivary complex, which has been shown to be an important
% source of inhibitory input to the cochlear nucleus (Ostapoff et
% al. 1997; Saint Marie et al. 1993).
% Inhibitory projections may contribute to the OnC response
% through shunting inhibition, as proposed by Smith and Rhode
% (1989). They suggest that the output of the cell may be reduced
% when the membrane potential is close to the reversal potential
% of chloride ions. At this potential, the amplitude of any depo-
% larization will be substantially reduced when the chloride chan-
% nel is activated, leading to increased conductance. Shunting
% inhibition may account for the drop in the level of depolariza-
% tion to sustained levels seen in this investigation. The extent of
% depolarization seen during the sustained component changes
% with frequency of presentation. In response to tones presented
% 30 – 40 dB above threshold at the low-frequency edge of the
% neurons’ response area, the amplitude of the sustained depo-
% larization approached that of the initial component. This in-
% crease in the level of depolarization observed during the sus-
% tained component may result from a decrease of inhibitory
% drive or of shunting inhibition.
% Alternatively, the increase in the level of sustained depolar-
% ization seen on presentation of tones on the low-frequency side
% of CF may also be explained by an increase in auditory nerve
% fibre convergence. As the frequency is decreased and at high
% intensities more auditory nerve fibres are recruited as low-
% frequency tails are commonly seen in the tuning characteristics
% of these fibres. However, an increase in the level of depolar-
% ization and spike activity was not seen during the initial com-
% ponent of the response when tones were presented at these
% frequencies. This suggests that, although more auditory nerve
% fibres are excited by low-frequency stimuli at high intensities,
% it is unlikely that they all contribute to the summation of inputs
% to OnC neurons.




%\citep{SmithMassieEtAl:2005}

% Onset choppers. 
The combined evidence from several
 different studies (Cant and Gaston, 1982; Wenthold, 1987;
 Kolston et al., 1992; Shore et al., 1992; Schofield and Cant,
 1996b; Alibardi, 1998; Needham and Paolini, 2003;
 Palmer et al., 2003; Arnott et al., 2004) suggests that the
 large VCN multipolar cells whose response to short tones
 have been labeled onset-chopper (OnC; Rhode et al., 1983a;
 Smith and Rhode, 1989) closely resemble a population of
 large glycinergic multipolar cells that project to the opposite CN. Single-cell labelling studies in the cat (Smith and
 Rhode, 1989) and guinea pig (Palmer et al., 2003; Arnott
 et al., 2004) have reported that the axons of these cells
 provide collateral innervation to both the ipsilateral dorsal and ventral cochlear nuclei before heading dorsally out
 of the CN. 

% The only direct evidence that the glycinergic
%  multipolar cells projecting to the opposite cochlear nucleus
%  and the cells with OnC response features are the same
%  population is one juxtacellularly labeled cell in the guinea
%  pig (Arnott et al., 2004) and one intra-axonally labeled cell
%  in the cat that we described in a preliminary report \citep{SmithMassieEtAl:2005}




Type II multipolar cells of the ventral cochlear
 nucleus are the source of widespread inhibitory inputs
 to the ipsilateral dorsal cochlear nucleus and the
 contralateral dorsal and ventral cochlear nuclei \citep{CantBenson:2003}.
 A second, less plentiful, group of large multipolar cells
 described by Smith and Rhode in the cat (\citealt{SmithRhode:1989}; referred
 to here as type II multipolar cells) probably corresponds
 to the D-stellate cells in mouse \citep{Oertel:1983} and to the radiate
 cells in rat \citep{DoucetRyugoEtAl:1999} and also probably to the commissural cells
 described in a number of species.% [13,42,113,204,213,254].
 These neurons can be distinguished from the type I multipolar cells (TS cells) in a number of ways. On average, they are
 larger, and their dendrites radiate across the isofrequency
 laminae of the VCN. %[54,61,151,169]. 
Both the dendrites
 and the somatic surface are covered with synaptic terminals, many of which appear to arise from the auditory nerve.% [13,220]. (The so-called type II stellate cells in the anterior
% AVCN [36] are also characterized by extensive somatic in-
% nervation, but as noted above, it has not been established
% whether the cells in the anterior AVCN should be consid-
% ered to be a different population from the multipolar cells in
% the rest of the AVCN and in the PVCN.) 
%Few of the synap-
% tic terminals on the type II multipolar cells are glyciner-
% gic [113], but many of the cells themselves are glycinergic
% [23,54,113,254]. Type II multipolar cells give rise to termi-
% nal dendritic tufts that characteristically extend into the gran-
% ule cell layer bordering the VCN ([55,151,179,204,220];
% cf. [142]).
% The type II multipolar cells are broadly tuned and corre-
% spond to a group of units characterized physiologically as
% “onset chopper” (OnC ) units, which show a well-timed on-
% set response followed by a few regularly timed spikes that
% are not sustained throughout the stimulus as are those of
% “chopper” units [168,169,220]. 


\subsubsection{TODO Cellular Mechanisms of D~stellate Cells}

DS cells have the largest cell body in the \VCN  (27 \um~guinea pig: \citealt{ArnottWallaceEtAl:2004},
22--26 \um rat: \citealt{DoucetRyugo:1997}, 20--30 \um~rat: \citealt{PaoliniClark:1999}). 

% The multipolar and cross-sectional area  (cat:
% \citealt{ReddCahillEtAl:2002,SmithRhode:1989}, rat:
% \citealt{DoucetRyugoEtAl:1999DoucetRyugo:1997,DoucetRyugo:2006}, guinea pig:
% \citealt{PalmerWallaceEtAl:2003}).


% Cross-sectional area
% 963          (cat [\citenum{SmithRhode:1989}]) 
% 501{$\pm$}168  (rat [\citenum{DoucetRyugoEtAl:1999}])
% 466{$\pm$}137 (rat [\citenum{DoucetRyugo:1997}]) 
% 418{$\pm$}140 (rat [\citenum{DoucetRyugo:2006}])
% 571{$\pm$}228  (cat [\citenum{ReddCahillEtAl:2002}]) 
% 450           (guinea pig [\citenum{PalmerWallaceEtAl:2003}])                  


Four to 5 apinous dendrites eminate from D stellate cells in rats and span 250--350 \um in all directions \citep{DoucetRyugo:1997}.
%]) with aspinous dendrites, 
%4 of 5 cells had 4 main dendrites, total dend length 6222 to 7351 \um
%(mean-6665 \um), 
% Dendrites extended widely in all directions. ,
%$\sim$70 \um perpendicular to AN 3-6 primary dendrites at right angles to AN        [\citenum{SmithRhode:1989}]
%& 
%axon width 0.7-1.2 \um [\citenum{OertelWuEtAl:1990}]\\\hline

Intracellular recordings in DS cells shows mixed behaviour between phasic and
regular firing, and generally have low membrane resistance.  DS cell action
potentials have double exponentials with undershoot
\citep{PaoliniClark:1999,WuOertel:1984}.  Membrane properties of \OnC cell have
not been adequately characterised, but the information that is available (D
stellate in mice \citealt{OertelWuEtAl:1990}) that suggests the low-threshold
potassium channel that is important in extending the phase-locking range of
bushy cells \citep{ManisMarx:1991,Oertel:1983} is not present in OnC neurons
\citep{WhiteYoungEtAl:1994}.  The intermediate property was classified in mice
as Type I-i have high thresholds mediated by small \IKLT
\citep{RothmanManis:2003b}.
%&     Fast      & Linear \citep{PaoliniClark:1999} & 
Small amounts of \IKLT contribute to the low membrane resistance (40 M$\Omega$ in mice \citealt{OertelWuEtAl:1990},  $96.2 \pm 27.8$ M$\Omega$ mice: \citealt{FerragamoGoldingEtAl:1998a}.



% \subsubsection{Synaptophysiology of D~stellate Cells}

% \citep{FerragamoGoldingEtAl:1998a}
% % D stellate cells received glycinergic inhibition. Figs. 13A
% % and 14 show that 1 mM strychnine eliminated the occasional
% % IPSPs but that glycinergic inhibition in D stellate cells did
% % not prominently affect synaptic responses.
% % To test whether NMDA receptors mediate the long-lasting
% % firing of D stellate cells in responses to shocks, 100 mM
% % APV was applied to the bath. Figure 13A shows that APV
% % reversibly eliminated the long-lasting depolarization. The
% % remaining early excitation was blocked reversibly by DNQX
% % (n Å 2/2). To determine whether NMDA receptors were
% % intrinsic to the recorded cell or on excitatory interneurons,
% % the voltage dependence of the late response was examined
% % (Fig. 13B). The long, late depolarization that caused the D
% % stellate cell to fire for Ç200 ms was shortened to Ç100 ms
% % when the cell was hyperpolarized, this is consistent with
% % synaptic excitation mediated by NMDA receptors that were
% % intrinsic to the D stellate cell.
% % GABAergic inhibition plays a prominent role in the synap-
% % tic responses of D stellate cells. Even in the absence of
% % visible IPSPs, picrotoxin, a blocker of GABAA receptors,
% % enhanced the firing of the cell (Fig. 14; n Å 1/1). Both
% % the frequency and duration of firing were augmented in the
% % presence of picrotoxin.


\subsubsection{Acoustic Response of D~stellate  Cells}



The \OnC units are widely scattered throughout the VCN %[168,169]
matching the distri- bution of the commissural cells
\citep{PaoliniClareyEtAl:2005,NeedhamPaolini:2007}. %[113,204,254].
Compatible with their wide-reaching dendritic arbor, the type II multi- polar
cells appear to integrate inputs over a wide frequency range
\citep{PalmerWallaceEtAl:2003}.  The known targets of the type II multipolar
cells are illustrated in Figure~\ref{fig:microcircuit}. The axons leave the
cochlear nucleus via the intermediate acoustic stria. %[3,61,113,204,220].
The distribution of multipolar neurons in the VCN whose axons leave via the IAS
is very similar to that of the commissural cells %(compare [3,42])
, and the only known targets of these axons are the contralateral dorsal and
ventral cochlear nuclei.% [13,113,204].
The axons also give rise to extensive collateral branches within the parent VCN
and send widespread terminations to the ipsilateral DCN
(References). %[4,55,61,151,164,220].
In both the ipsilateral and contralateral cochlear nuclei, the axonal terminals
contain pleomorphic or flattened synaptic vesicles, consistent with an
inhibitory effect on the target cells. %[13,220].

The type II multipolar cells do not appear to project to the inferior colliculus, and their role would be expected to form a major source of wideband inhibition to the VCN and DCN. % [54]. [12,102]. 
%It is not known if they give rise to any of the branches making up other projections attributed to multipolar cells.








%\subsubsection{Periodicity and Temporal Coding in the Ventral Cochlear Nucleus}


%\subsubsection{Commissural and dorsal output of D~stellate cells}




\subsubsection{Neuromodulatory Effects on D~stellate Cells}


% \citep{DoucetRyugo:1997}
% Depending on somatic location, dendritic processes were
% extended into the nearest and most proximal region of the
% surrounding granule cell domain. Caudally-distributed
% cells projected their dendrites into the lamina between the
% PVCN and the DCN, laterally located cells extended into
% the superficial layer of granule cells along the lateral edge
% of the AVCN, dorsally-located cells reached the subpedun-
% cular region in the dorsomedial part of VCN, and medially
% positioned cells contacted the medial sheet of granule cells
% or sometimes even intermingled with fibres associated
% with the vestibular nerve root.



\begin{itemize}
\item sensitive to neuromodulatory currents \citep{FujinoOertel:2001}
% \begin{itemize}
% \item high input resistance $\rightarrow$ amplify small current inputs \citep{FujinoOertel:2001}
% \item no LKT in TS,  LKT makes bushy and octopus insensitive to steady currents \citep{OertelFujino:2001,McGinleyOertel:2006}
% \item Ih higher in TS \& activated more at lower potentials than in bushy and octopus, so that it is less active at rest
% \item high resistance $\rightarrow$ greater voltage changes in small modulating current $\rightarrow$ Ih can be modulated by G-protein coupled receptors, hence making TS more excitable when Ih activated \citep{RodriguesOertel:2006}
% \end{itemize}
\end{itemize}


\begin{itemize}
\item ANF provide glutamatergic excitation for D stellates  \citep{Cant:1981,FerragamoGoldingEtAl:1998a,Alibardi:1998a}
\item Excitation from other T stellate cells \citep{FerragamoGoldingEtAl:1998a}
\item Glycine from other DS cells \citep{FerragamoGoldingEtAl:1998a}
\item Glycine from TV cells \citep{WickesbergOertel:1990,ZhangOertel:1993b}
\end{itemize}

     
\subsubsection{Golgi cells (GABA)}

Small, single-GABA-labelled terminals (presumably Golgi cells) apposed to
dendrites of \DS cells (glycine-positive) in the granule cell domain have been
confirmed histologically in different animals (cat: \citealt{SmithRhode:1989},
guinea pig: \citealt{KolstonOsenEtAl:1992}).


\begin{itemize}
\item no IPSPs or IPSCs but presence of GABAa receptors and response changes to bicuculine \citep{WuOertel:1986,OertelWickesberg:1993,FerragamoGoldingEtAl:1998a}
\item dend filter obscures PSPs
\item Golgi cells are GABAergic and lie within the granule cell domains around the VCN and terminate near the fine distal dendrites of T stellate cells
\end{itemize}

b. Periolivary cells (GABA + GAD - glutamic acid decarboxylase) 

\begin{itemize}
\item observed in PVCN close to TS \citep{AdamsMugnaini:1987}
\item GAD effectively Glycine \citep{GoldingOertel:1997}
\item can also arise from GABAergic neurons in ipsi LNTB and DM Periolivary
\end{itemize}

\subsubsection{VNTB cells (ACh)}

\begin{itemize}
\item collateral branches of OCB go to GCD \citep{MellottMottsEtAl:2011,SherriffHenderson:1994,OsenRoth:1969}
\item DS have nicotinic and muscarinic ACh receptors \citep{FujinoOertel:2001}
\item ACh input to DS, together with OC-cochlea, enhances spectral peaks in noise  \citep{FujinoOertel:2001}
\end{itemize}

\subsubsection{NE and 5HT}

\begin{itemize}
\item Raphe nuclei (5HT)
\item Locus coeruleus Peribrachial cells (NE)
\item both terminate in PVCN \citep{KlepperHerbert:1991,Thompson:2003,ThompsonLauder:2005,Thompson:2003a,ThompsonWiechmann:2002,BehrensSchofieldEtAl:2002,ThompsonThompson:2001,ThompsonThompson:2001a,ThompsonMooreEtAl:1995,ThompsonThompsonEtAl:1994}
\item both increase firing in T stellates \citep{OertelWrightEtAl:2011} in presence of glut and gly blockers -> hence act on post synapse (TS cells)
\item both G-protein coupled, both act on either pre or post synaptic cells
\item NE affects probability of release at calyx of Held
\item NE increases firing rate of choppers \citep{KosslVater:1989,Ebert:1996}
\item 5HT excites or inhibits choppers \emph{in vivo} \citep{EbertOstwald:1992}
\end{itemize}





% \citep{JamalZhangEtAl:2011,
% LuRubioEtAl:2008,
% WuJen:2006,
% AwatramaniTurecekEtAl:2005,
% Ben-Ari:2005,
% HuaWangEtAl:2005,
% IrfanZhangEtAl:2005,
% JenWu:2005,
% LuBurgerEtAl:2005,
% MartyLlano:2005,
% Rubio:2005,
% MahendrasingamWallamEtAl:2004,
% RubioJuiz:2004,
% Alibardi:2003,
% CasparyPalombiEtAl:2002,
% CamposCaboEtAl:2001,
% LiaoVanEtAl:2000,
% LimAlvarezEtAl:2000,
% MahendrasingamWallamEtAl:2000,
% MarianowskiLiaoEtAl:2000,
% PirkerSchwarzerEtAl:2000,
% WangCasparyEtAl:2000,
% KemmerVater:1997,
% OstapoffBensonEtAl:1997,
% JuizHelfertEtAl:1996,
% LeReesEtAl:1996,
% EbertOstwald:1995,
% EbertOstwald:1995a,
% GleichBielenbergEtAl:1995,
% SunejaBensonEtAl:1995a,
% WinerLarueEtAl:1995,
% JuizAlbinEtAl:1994,
% VareckaWuEtAl:1994,
% CasparyPalombi:1993,
% CasparyPalombiEtAl:1993,
% PotashnerBensonEtAl:1993,
% KolstonOsenEtAl:1992,
% PalombiCaspary:1992,
% OsenLopezEtAl:1991,
% OsenOttersenEtAl:1990,
% CarrFujitaEtAl:1989,
% JuizHelfertEtAl:1989,
% SaintMorestEtAl:1989,
% OberdorferParakkalEtAl:1988,
% FexAltschulerEtAl:1986,
% MelanderHokfeltEtAl:1986,
% PeyretGeffardEtAl:1986,
% WentholdZempelEtAl:1986,
% Mugnaini:1985,
% ThompsonCortezEtAl:1985,
% CasparyHaveyEtAl:1979,
% Wenthold:1979,
% Davies:1975,
% FisherDavies:1976}


% ChandaXu-Friedman:2010
%      Chanda, Xu-Friedman            2010 J Neurophysiol 104, 2063--74
%      Neuromodulation by GABA converts a relay into a coincidence detector.

% FukuiBurgerEtAl:2010
%      Fukui, Burger, Ohmori, Rubel   2010 J Neurosci 30, 12075--83
%      GABAergic inhibition sharpens the frequency tuning and enhances phase locking in chicken nucleus magnocellularis neurons.

% IrieOhmori:2008
%      Irie, Ohmori                   2008 Biochem Biophys Res Commun 367, 503--08
%      Presynaptic {GABAB} receptors modulate synaptic facilitation and depression at distinct synapses in fusiform cells of mouse dorsal cochlear nucleus

% LuRubioEtAl:2008
%      Lu, Rubio, Trussell            2008 Neuron 57, 524--35
%      Glycinergic {Transmission} {Shaped} by the {Corelease} of {GABA} in a {Mammalian} {Auditory} {Synapse

% KuleszaKadnerEtAl:2007
%      Kulesza, Kadner, Berrebi       2007 J Neurophysiol 97, 1610--20
%      Distinct {Roles} for {Glycine} and {GABA} in {Shaping} the {Response} {Properties} of {Neurons} in the {Superior} {Paraolivary} {Nucleus} of the {Rat

% Lu:2007
%      Lu                             2007 J Neurophysiol 97, 1018--29
%      Endogenous {mGluR} {Activity} {Suppresses} {GABAergic} {Transmission} in {Avian} {Cochlear} {Nucleus} {Magnocellularis} {Neurons

% WuJen:2006
%      Wu, Jen                        2006 Hear Res 215, 56--66
%      The role of {GABAergic} inhibition in shaping duration selectivity of bat inferior collicular neurons determined with temporally patterned sound trains

% AwatramaniTurecekEtAl:2005
%      Awatramani, Turecek, Trussell  2005 J Neurophysiol 93, 819--28
%      Staggered {Development} of {GABAergic} and {Glycinergic} {Transmission} in the {MNTB

% Ben-Ari:2005
%      Ben-Ari                        2005 Trends Neurosci 28, 277
%      The multiple facets of {GABA

% BurgerPfeifferEtAl:2005
%      Burger, Pfeiffer, Westrum, ... 2005 J Comp Neurol 489, 11--22
%      Expression of {GABA(B)} receptor in the avian auditory brainstem: {Ontogeny,} afferent deprivation, and ultrastructure

% Gutierrez:2005
%      Gutierrez                      2005 Trends Neurosci 28, 297--303
%      The dual {glutamatergic-GABAergic} phenotype of hippocampal granule cells

% HestrinGalarreta:2005
%      Hestrin, Galarreta             2005 Trends Neurosci 28, 304--09
%      Electrical synapses define networks of neocortical {GABAergic} neurons

% HuaWangEtAl:2005
%      Hua, Wang, Xiao                2005 Lin Chuang Er Bi Yan Hou Ke Za Zhi 19, 315--7
%      [Distribution of gama-aminobutyric acid {(GABA)ergic} neurons in rats cochlear nuclei after unilateral cochlea ablation]

% IrfanZhangEtAl:2005
%      Irfan, Zhang, Wu               2005 Hear Res 203, 159--71
%      Synaptic transmission mediated by ionotropic glutamate, glycine and {GABA} receptors in the rat's ventral nucleus of the lateral lemniscus

% JenWu:2005
%      Jen, Wu                        2005 Hear Res 202, 222--34
%      The role of {GABAergic} inhibition in shaping the response size and duration selectivity of bat inferior collicular neurons to sound pulses in rapid sequences

% LlinasUrbanoEtAl:2005
%      Llinas, Urbano, Leznik, Ram... 2005 Trends Neurosci 28, 325--33
%      Rhythmic and dysrhythmic thalamocortical dynamics: {GABA} systems and the edge effect

% LuBurgerEtAl:2005
%      Lu, Burger, Rubel              2005 J Neurophysiol 93, 1429--38
%      GABA(B)} receptor activation modulates {GABA(A)} receptor-mediated inhibition in chicken nucleus magnocellularis neurons

% MartyLlano:2005
%      Marty, Llano                   2005 Trends Neurosci 28, 284--89
%      Excitatory effects of {GABA} in established brain networks

% MerchanAguilarEtAl:2005
%      Merchan, Aguilar, Lopez-Pov... 2005 Neurosci 136, 907--25
%      The inferior colliculus of the rat: {Quantitative} immunocytochemical study of {GABA} and glycine

% OnoYanagawaEtAl:2005
%      Ono, Yanagawa, Koyano          2005 Neurosci Res 51, 475--92
%      GABAergic} neurons in inferior colliculus of the {GAD67-GFP} knock-in mouse: {Electrophysiological} and morphological properties

% RepresaBen-Ari:2005
%      Represa, Ben-Ari               2005 Trends Neurosci 28, 278--83
%      Trophic actions of {GABA} on neuronal development

% Rubio:2005
%      Rubio                          2005 J Comp Neurol 485, 266--66
%      Differential distribution of synaptic endings containing glutamate, glycine, and {GABA} in the rat dorsal cochlear nucleus (vol 477, pg 253, 2004)

% LujanShigemotoEtAl:2004
%      Lujan, Shigemoto, Kulik, Juiz  2004 J Comp Neurol 475, 36--46
%      Localization of the {GABAB} receptor 1a/b subunit relative to glutamatergic synapses in the dorsal cochlear nucleus of the rat

% MahendrasingamWallamEtAl:2004
%      Mahendrasingam, Wallam, Pol... 2004 Eur J Neurosci 19, 993--1004
%      An immunogold investigation of the distribution of {GABA} and glycine in nerve terminals on the somata of spherical bushy cells in the anteroventral cochlear nucleus of guinea pig

% RubioJuiz:2004
%      Rubio, Juiz                    2004 J Comp Neurol 477, 253--72
%      Differential distribution of synaptic endings containing glutamate, glycine, and {GABA} in the rat dorsal cochlear nucleus

% SivaramakrishnanSterbing-DAngeloEtAl:2004
%      Sivaramakrishnan, Sterbing-... 2004 J Neurosci 24, 5031--43
%      GABA(A)} synapses shape neuronal responses to sound intensity in the inferior colliculus

% ZhangSunejaEtAl:2004
%      Zhang, Suneja, Potashner       2004 J Neurosci Res 75, 361--70
%      Protein kinase {A} and calcium/calmodulin-dependent protein kinase {II} regulate glycine and {GABA} release in auditory brain stem nuclei




% WisdenCopeEtAl:2002
%      Wisden, Cope, Klausberger, ... 2002 Neuropharmacology 43, 530--49
%      Ectopic expression of the {GABAA} receptor [alpha]6 subunit in hippocampal pyramidal neurons produces extrasynaptic receptors and an increased tonic inhibition


















Roles in the IC
% ZhangKelly:2003
%      Zhang, Kelly                   2003 J Neurophysiol 90, 477--90
%      Glutamatergic and {GABAergic} {Regulation} of {Neural} {Responses} in {Inferior} {Colliculus} to {Amplitude-Modulated} {Sounds
% CasparyPalombiEtAl:2002
%      Caspary, Palombi, Hughes       2002 Hear Res 168, 163--73
%      GABAergic} inputs shape responses to amplitude modulated stimuli in the inferior colliculus

% CasparyHelfertEtAl:1997
%      Caspary, Helfert, Palombi      1997 in: Acoustic Signal Processing in the Central Auditory System
%      The role of {GABA} in shaping frequency response properties in the chinchilla inferior colliculus

% Pollak:1997
%      Pollak                         1997 Ann Otol Rhinol Laryn 168, 44--54
%      Roles of {GABAergic} inhibition for the binaural processing of multiple sound sources in the inferior colliculus















Birds
% LuTrussell:2001
%      Lu, Trussell                   2001 J Physiol 535, 125--31
%      Mixed excitatory and inhibitory {GABA-mediated} transmission in chick cochlear nucleus

% BartheldCodeEtAl:1989
%      von Bartheld, Code, Rubel      1989 J Comp Neurol 287, 470--83
%      GABAergic} neurons in brainstem auditory nuclei of the chick: distribution, morphology, and connectivity

% CarrFujitaEtAl:1989
%      Carr, Fujita, Konishi          1989 J Comp Neurol 286, 190--207
%      Distribution of {GABAergic} neurons and terminals in the auditory system of the barn owl



RNA expression
% JamalZhangEtAl:2011
%      Jamal, Zhang, Finlayson, Po... 2011 Neuroscience , 
%      The level and distribution of the GABA(B)R2 receptor subunit in the rat's central auditory system.

% CamposCaboEtAl:2001
%      Campos, de Cabo, Wisden, Ju... 2001 Neurosci 102, 625--38
%      Expression of {GABA(A)} receptor subunits in rat brainstem auditory pathways: cochlear nuclei, superior olivary complex and nucleus of the lateral lemniscus

% LiaoVanEtAl:2000
%      Liao, Van Den Abbeele, Herm... 2000 Hear Res 150, 12--26
%      Expression of {NMDA,} {AMPA} and {GABA(A)} receptor subunit {mRNAs} in the rat auditory brainstem. {II.} {Influence} of intracochlear electrical stimulation

% MarianowskiLiaoEtAl:2000
%      Marianowski, Liao, Van Den ... 2000 Hear Res 150, 1--11
%      Expression of {NMDA,} {AMPA} and {GABA(A)} receptor subunit {mRNAs} in the rat auditory brainstem. {I.} {Influence} of early auditory deprivation

% PirkerSchwarzerEtAl:2000
%      Pirker, Schwarzer, Wieselth... 2000 Neurosci 101, 815--50
%      GABAA} receptors: immunocytochemical distribution of 13 subunits in the adult rat brain

% HysonSadler:1997
%      Hyson, Sadler                  1997 J Mol Neurosci 8, 193--205
%      Differences in expression of {GABAA} receptor subunits, but not benzodiazepine binding, in the chick brainstem auditory system


% JuizAlbinEtAl:1994
%      Juiz, Albin, Helfert, Altsc... 1994 Brain Res 639, 193--201
%      Distribution of {GABAA} and {GABAB} binding sites in the cochlear nucleus of the guinea pig


% VareckaWuEtAl:1994
%      Varecka, Wu, Rotter, Frostholm 1994 J Comp Neurol 339, 341--52
%      GABAA/benzodiazepine} receptor alpha 6 subunit {mRNA} in granule cells of the cerebellar cortex and cochlear nuclei: expression in developing and mutant mice



Pharmacological/Physiological effects
% LimAlvarezEtAl:2000
%      Lim, Alvarez, Walmsley         2000 J Physiol 525 Pt 2, 447--59
%      GABA} mediates presynaptic inhibition at glycinergic synapses in a rat auditory brainstem nucleus

% WangCasparyEtAl:2000
%      Wang, Caspary, Salvi           2000 Neuroreport 11, 1137--40
%      GABA-A} antagonist causes dramatic expansion of tuning in primary auditory cortex


% BrenowitzDavidEtAl:1998
%      Brenowitz, David, Trussell     1998 Neuron 20, 135--41
%      Enhancement of synaptic efficacy by presynaptic {GABA(B)} receptors


% BackoffPalombiEtAl:1997
%      Backoff, Palombi, Caspary      1997 Hear Res 110, 155--63
%      Glycinergic and {GABAergic} inputs affect short-term suppression in the cochlear nucleus

% ShoreBledsoe:1997
%      Shore, Bledsoe                 1997 Assoc Res Otolaryngol Abstr , 
%      Effects of {GABA} and {VNTB} stimulation on forward masking functions in ventral cochlear nucleus

% GoldingOertel:1996
%      Golding, Oertel                1996 J Neurosci 16, 2208--19
%      Context-dependent synaptic action of glycinergic and {GABAergic} inputs in the dorsal cochlear nucleus


% LeReesEtAl:1996
%      Le Beau, Rees, Malmierca       1996 J Neurophysiol 75, 902--19
%      Contribution of {GABA-} and glycine-mediated inhibition to the monaural temporal response properties of neurons in the inferior colliculus

% EbertOstwald:1995
%      Ebert, Ostwald                 1995 Hear Res 91, 160--6
%      GABA} alters the discharge pattern of chopper neurons in the rat ventral cochlear nucleus

% EbertOstwald:1995a
%      Ebert, Ostwald                 1995 Exp Brain Res 104, 310--22
%      GABA} can improve acoustic contrast in the rat ventral cochlear nucleus

% RazaMilbrandtEtAl:1994
%      Raza, Milbrandt, Arneric, C... 1994 Hear Res 77, 221--30
%      Age-related changes in brainstem auditory neurotransmitters: measures of {GABA} and acetylcholine function

% YangPollak:1994
%      Yang, Pollak                   1994 J Neurophysiol 71, 2014--24
%      GABA} and glycine have different effects on monaural response properties in the dorsal nucleus of the lateral lemniscus of the mustache bat

% CasparyPalombi:1993
%      Caspary, Palombi               1993 Assoc Res Otolaryngol Abstr 109, 
%      GABA} inputs control discharge rate within the excitatory response area of chinchilla inferior colliculus neurons

% CasparyPalombiEtAl:1993
%      Caspary, Palombi, Backoff, ... 1993 in: The Mammalian Cochlear Nuclei: Organisation and Function
%      GABA} and glycine inputs control discharge rate within the excitatory response area of primary-like and phase-locked {AVCN} neurons

% OtisDeEtAl:1993
%      Otis, De Koninck, Mody         1993 J Physiol (Lond) 463, 391--407
%      Characterization of synaptically elicited {GABAB} responses using patch- clamp recordings in rat hippocampal slices

% PalombiCaspary:1992
%      Palombi, Caspary               1992 J Neurophysiol 67, 738--46
%      GABAA} receptor antagonist bicuculline alters response properties of posteroventral cochlear nucleus neurons


% WalshMcGeeEtAl:1990
%      Walsh, McGee, Fitzakerley      1990 J Neurophysiol 64, 961--77
%      GABA} actions within the caudal cochlear nucleus of developing kittens

% CasparyHaveyEtAl:1979
%      Caspary, Havey, Faingold       1979 Brain Res 172, 179--85
%      Effects of microiontophoretically applied glycine and {GABA} on neuronal response patterns in the cochlear nuclei

% FaingoldGehlbachEtAl:1989
%      Faingold, Gehlbach, Caspary    1989 Brain Res 500, 
%      On the role of {GABA} as an inhibitory neurotransmitter in inferior colliculus neurons: iontophoretic studies

\citep{RamanZhangEtAl:1994}
\citep{FrisinaWaltonEtAl:1993,FrisinaSmithEtAl:1990}

\citep{JosephsonMorest:2003,BellinghamLimEtAl:1998,BilakMorest:1998,HunterPetraliaEtAl:1993}

Reviews of AMPA subtype in the cochlear nucleus and auditory brainstem nuclei
\citep{GardnerTrussellEtAl:2001,Parks:2000}.

BellinghamLimEtAl:1998
     Bellingham, Lim, Walmsley      1998 J Physiol (Lond) 511, 861--69
     Developmental changes in {EPSC} quantal size and quantal content at a central glutamatergic synapse in the rat

BilakMorest:1998
     Bilak, Morest                  1998 Synapse 28, 251--70
     Differential expression of the metabotropic glutamate receptor {mGluR1alpha} by neurons and axons in the cochlear nucleus: in situ hybridization and immunohistochemistry

OtisWuEtAl:1996
     Otis, Wu, Trussell             1996 J Neurosci 16, 1634--44
     Delayed clearance of transmitter and the role of glutamate transporters at synapses with multiple release sites


Ultrastructural Analysis

The cytological composition and ultrastructure of the
DCN are relatively well known. Apart from large neurons
that project ouside the nucleus (mainly pyramidal cells),
other smaller interneurons, called cartwheel, Golgi,
32 Glycinergic and GABAergic neurons, L. Alibardi
stellate, unipolar brush cells, commissural and vertical
(or tuberculo-ventral) neurons, have been character-
ized (Kane, 1974; Mugnaini et al. 1980, 1997; Fiori
 Mugnaini, 1981; Kane et al. 1981; Wouterlood 
Mugnaini, 1984; Mugnaini, 1985; Berrebi  Mugnaini,
1991; Floris et al. 1994; Ryugo  Willard, 1985; Alibardi,
1999a,b, 2000a,b; Kemmer  Vater, 2001)



% Alibardi:2003
%      Alibardi                       2003 J Anat 203, 31--56
%      Ultrastructural distribution of glycinergic and {{GABAergic}} neurons and axon terminals in the rat dorsal cochlear nucleus, with emphasis on granule cell areas


% KemmerVater:2001a
%      Kemmer, Vater                  2001 Anat Embryol 203, 429--47
%      Functional organization of the dorsal cochlear nucleus of the horseshoe bat {(Rhinolophus} rouxi) studied by {GABA} and glycine immunocytochemistry and electron microscopy

% AcsadyKamondiEtAl:1998
%      Acsady, Kamondi, Sik, Freun... 1998 J Neurosci 18, 3386--403
%      GABAergic} cells are the major postsynaptic targets of mossy fibers in the rat hippocampus


% JuizCamposEtAl:1996
%      Juiz, Campos, Helfert, Alts... 1996 J Hirnforsch 37, 51--6
%      Silver intensification of immunocolloidal gold on ultrathin plastic sections applied to the study of the neuronal distribution of {GABA} and glycine

% JuizHelfertEtAl:1996
%      Juiz, Helfert, Bonneau, Cam... 1996 J Hirnforsch 37, 561--74
%      Distribution of glycine and {GABA} immunoreactivities in the cochlear nucleus: quantitative patterns of putative inhibitory inputs on three cell types

% JuizAlbinEtAl:1994
%      Juiz, Albin, Helfert, Altsc... 1994 Brain Res 639, 193--201
%      Distribution of {GABAA} and {GABAB} binding sites in the cochlear nucleus of the guinea pig

% PotashnerBensonEtAl:1993
%      Potashner, Benson, Ostapoff... 1993 in: The Mammalian Cochlear Nuclei: Organisation and Function
%      Glycine and {GABA:} transmitter candidates of projections descending to the cochlear nucleus

% KolstonOsenEtAl:1992
%      Kolston, Osen, Hackney, Ott... 1992 Anat Embryol 186, 443--65
%      An atlas of glycine- and {GABA-like} immunoreactivity and colocalization in the cochlear nuclear complex of the guinea pig



% OberdorferParakkalEtAl:1988
%      Oberdorfer, Parakkal, Altsc... 1988 Hear Res 33, 229--38
%      Ultrastructural localization of {GABA-immunoreactive} terminals in the anteroventral cochlear nucleus of the guinea pig

% Mugnaini:1985
%      Mugnaini                       1985 J Comp Neurol 235, 61--81
%      GABA} neurons in the superficial layers of the rat dorsal cochlear nucleus: light and electron microscopic immunocytochemistry.

MugnainiOsenEtAl:1980

% SaintMorestEtAl:1989
%      Saint Marie, Morest, Brandon   1989 Hear Res 42, 97--112
%      The form and distribution of {GABAergic} synapses on the principal cell types of the ventral cochlear nucleus of the cat
% SunejaPotashnerEtAl:1998
%      Suneja, Potashner, Benson      1998 Exp Neurol 151, 273--88
%      Plastic changes in glycine and {GABA} release and uptake in adult brain stem auditory nuclei after unilateral middle ear ossicle removal and cochlear ablation



Histology and Microscopy, immuno-reactive labeling
% KemmerVater:2001a
%      Kemmer, Vater                  2001 Anat Embryol 203, 429--47
%      Functional organization of the dorsal cochlear nucleus of the horseshoe bat {(Rhinolophus} rouxi) studied by {GABA} and glycine immunocytochemistry and electron microscopy

% RiquelmeSaldanaEtAl:2001
%      Riquelme, Saldana, Osen, Ot... 2001 J Comp Neurol 432, 409--24
%      Colocalization of {GABA} and glycine in the ventral nucleus of the lateral lemniscus in rat: an in situ hybridization and semiquantitative immunocytochemical study


% ChaudhryReimerEtAl:1998
%      Chaudhry, Reimer, Bellocchi... 1998 J Neurosci 18, 9733--50
%      The vesicular {GABA} transporter, {VGAT,} localizes to synaptic vesicles in sets of glycinergic as well as {GABAergic} neurons

% Gil-LoyzagaBartolomeEtAl:1998
%      Gil-Loyzaga, Bartolome, Ibanez 1998 Histol Histopathol 13, 415--24
%      Synaptophysin immunoreactivity in the cat cochlear nuclei

% GleichVater:1998
%      Gleich, Vater                  1998 Cell Tissue Res 293, 207--25
%      Postnatal development of {GABA-} and glycine-like immunoreactivity in the cochlear nucleus of the {Mongolian} gerbil {(Meriones} unguiculatus)

% LachicaKatoEtAl:1998
%      Lachica, Kato, Lippe, Rubel    1998 J Neurobiol 37, 321--37
%      Glutamatergic and {GABAergic} agonists increase {[Ca2+]i} in avian cochlear nucleus neurons


% YangWuEtAl:1998
%      Yang, Wu, Fang                 1998 Zhonghua Yi Xue Za Zhi 78, 30--2
%      [Changes of {GABA} immunoreactivity in aged rat cochlear nucleus]


% KemmerVater:1997
%      Kemmer, Vater                  1997 Cell Tissue Res 287, 487--506
%      The distribution of {GABA} and glycine immunostaining in the cochlear nucleus of the mustached bat {(Pteronotus} parnellii)

% OstapoffBensonEtAl:1997
%      Ostapoff, Benson, Saint Marie  1997 J Comp Neurol 381, 500--12
%      GABA}- and glycine-immunoreactive projections from the superior olivary complex to the cochlear nucleus in guinea pig


% JuizCamposEtAl:1996
%      Juiz, Campos, Helfert, Alts... 1996 J Hirnforsch 37, 51--6
%      Silver intensification of immunocolloidal gold on ultrathin plastic sections applied to the study of the neuronal distribution of {GABA} and glycine

% JuizHelfertEtAl:1996
%      Juiz, Helfert, Bonneau, Cam... 1996 J Hirnforsch 37, 561--74
%      Distribution of glycine and {GABA} immunoreactivities in the cochlear nucleus: quantitative patterns of putative inhibitory inputs on three cell types


% GleichBielenbergEtAl:1995
%      Gleich, Bielenberg, Strutz     1995 Neuroreport 7, 29--32
%      Sound induced expression of {c-Fos} in {GABA} positive neurones of the gerbil cochlear nucleus

% SunejaBensonEtAl:1995a
%      Suneja, Benson, Gross, Pota... 1995 J Neurochem 64, 147--60
%      Uptake and release of {D-aspartate,} {GABA,} and glycine in guinea pig brainstem auditory nuclei


% WinerLarueEtAl:1995
%      Winer, Larue, Pollak           1995 J Comp Neurol 355, 317--53
%      GABA} and glycine in the central auditory system of the mustache bat: structural substrates for inhibitory neuronal organization


% PotashnerBensonEtAl:1993
%      Potashner, Benson, Ostapoff... 1993 in: The Mammalian Cochlear Nuclei: Organisation and Function
%      Glycine and {GABA:} transmitter candidates of projections descending to the cochlear nucleus

% KolstonOsenEtAl:1992
%      Kolston, Osen, Hackney, Ott... 1992 Anat Embryol 186, 443--65
%      An atlas of glycine- and {GABA-like} immunoreactivity and colocalization in the cochlear nuclear complex of the guinea pig


% VaterKosslEtAl:1992
%      Vater, Kossl, Horn             1992 J Comp Neurol 325, 183--206
%      GAD-} and {GABA-immunoreactivity} in the ascending auditory pathway of horseshoe and mustached bats

% OsenLopezEtAl:1991
%      Osen, Lopez, Slyngstad, Ott... 1991 J Neurocytol 20, 17--25
%      GABA-like} and glycine-like immunoreactivities of the cochlear root nucleus in rat

% OsenOttersenEtAl:1990
%      Osen, Ottersen, Storm-Mathisen 1990 in: Glycine Neurotransmission
%      Colocalization of glycine-like and {GABA-like} immunoreactivities: a semiquantitative study of individual neurons in the dorsal cochlear nucleus of cat.

% OstapoffMorestEtAl:1990
%      Ostapoff, Morest, Potashner    1990 J Chem Neuroanat 3, 285--95
%      Uptake and retrograde transport of {[3H]GABA} from the cochlear nucleus to the superior olive in the guinea pig


% SaintMorestEtAl:1989
%      Saint Marie, Morest, Brandon   1989 Hear Res 42, 97--112
%      The form and distribution of {GABAergic} synapses on the principal cell types of the ventral cochlear nucleus of the cat


% JuizHelfertEtAl:1989
%      Juiz, Helfert, Wenthold, De... 1989 Brain Res 504, 173--79
%      Immunocytochemical localization of the {GABAA/benzodiazepine} receptor in the guinea pig cochlear nucleus: evidence for receptor localization heterogeneity

% FexAltschulerEtAl:1986
%      Fex, Altschuler, Kachar, We... 1986 Brain Res 366, 106--17
%      GABA} visualized by immunocytochemistry in the guinea pig cochlea in axons and endings of efferent neurons

% MelanderHokfeltEtAl:1986
%      Melander, Hokfelt, Rokaeus,... 1986 J Neurosci 6, 3640--54
%      Coexistence of galanin-like immunoreactivity with catecholamines, 5- hydroxytryptamine, {GABA} and neuropeptides in the rat {CNS

% OberdorferParakkalEtAl:1988
%      Oberdorfer, Parakkal, Altsc... 1988 Hear Res 33, 229--38
%      Ultrastructural localization of {GABA-immunoreactive} terminals in the anteroventral cochlear nucleus of the guinea pig


% PeyretGeffardEtAl:1986
%      Peyret, Geffard, Aran          1986 Hear Res 23, 115--21
%      GABA} immunoreactivity in the primary nuclei of the auditory central nervous system

% WentholdZempelEtAl:1986
%      Wenthold, Zempel, Parakkal,... 1986 Brain Res 380, 7--18
%      Immunocytochemical localization of {GABA} in the cochlear nucleus of the guinea pig


% Mugnaini:1985
%      Mugnaini                       1985 J Comp Neurol 235, 61--81
%      GABA} neurons in the superficial layers of the rat dorsal cochlear nucleus: light and electron microscopic immunocytochemistry.


% ThompsonCortezEtAl:1985
%      Thompson, Cortez, Lam          1985 Brain Res 339, 119--22
%      Localization of {GABA} immunoreactivity in the auditory brainstem of guinea pigs


% Wenthold:1979
%      Wenthold                       1979 Brain Res 162, 338--43
%      Release of endogenous glutamic acid, aspartic acid and {GABA} from cochlear nucleus slices

% FisherDavies:1976
%      Fisher, Davies                 1976 J Neurochem 27, 1145--55
%      GABA} and its related enzymes in the lower auditory system of the guinea pig

% Davies:1975
%      Davies                         1975 Brain Res 83, 27--33
%      The distribution of {GABA} transaminase-containing neurones in the cat cochlear nucleus

