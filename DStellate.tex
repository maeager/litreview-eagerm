\section{D~stellate cells}

%\subsection{Background}



\citep{SmithMassieEtAl:2005}

% Onset choppers. The combined evidence from several
% different studies (Cant and Gaston, 1982; Wenthold, 1987;
% Kolston et al., 1992; Shore et al., 1992; Schofield and Cant,
% 1996b; Alibardi, 1998; Needham and Paolini, 2003;
% Palmer et al., 2003; Arnott et al., 2004) suggests that the
% large VCN multipolar cells whose response to short tones
% have been labeled onset-chopper (OC; Rhode et al., 1983a;
% Smith and Rhode, 1989) closely resemble a population of
% large glycinergic multipolar cells that project to the oppo-
% site CN. Single-cell labeling studies in the cat (Smith and
% Rhode, 1989) and guinea pig (Palmer et al., 2003; Arnott
% et al., 2004) have reported that the axons of these cells
% provide collateral innervation to both the ipsilateral dor-
% sal and ventral cochlear nuclei before heading dorsally out
% of the CN. The only direct evidence that the glycinergic
% multipolar cells projecting to the opposite cochlear nucleus
% and the cells with OC response features are the same
% population is one juxtacellularly labeled cell in the guinea
% pig (Arnott et al., 2004) and one intra-axonally labeled cell
% in the cat that we described in a preliminary report (Joris


\subsection{Cellular Mechanisms of D~stellate Cells}


\subsection{Synaptophysiology of D~stellate Cells}

\citep{FerragamoGoldingEtAl:1998a}
% D stellate cells received glycinergic inhibition. Figs. 13A
% and 14 show that 1 mM strychnine eliminated the occasional
% IPSPs but that glycinergic inhibition in D stellate cells did
% not prominently affect synaptic responses.
% To test whether NMDA receptors mediate the long-lasting
% firing of D stellate cells in responses to shocks, 100 mM
% APV was applied to the bath. Figure 13A shows that APV
% reversibly eliminated the long-lasting depolarization. The
% remaining early excitation was blocked reversibly by DNQX
% (n Å 2/2). To determine whether NMDA receptors were
% intrinsic to the recorded cell or on excitatory interneurons,
% the voltage dependence of the late response was examined
% (Fig. 13B). The long, late depolarization that caused the D
% stellate cell to fire for Ç200 ms was shortened to Ç100 ms
% when the cell was hyperpolarized, this is consistent with
% synaptic excitation mediated by NMDA receptors that were
% intrinsic to the D stellate cell.
% GABAergic inhibition plays a prominent role in the synap-
% tic responses of D stellate cells. Even in the absence of
% visible IPSPs, picrotoxin, a blocker of GABAA receptors,
% enhanced the firing of the cell (Fig. 14; n Å 1/1). Both
% the frequency and duration of firing were augmented in the
% presence of picrotoxin.


\subsection{Acoustic Response of D~stellate  Cells}







\subsection{Periodicity and Temporal Coding in the Ventral Cochlear Nucleus}


%\subsection{Commissural and dorsal output of D~stellate cells}




\subsection{Neuromodulatory Effects on D~stellate Cells}


% \citep{DoucetRyugo:1997}
% Depending on somatic location, dendritic processes were
% extended into the nearest and most proximal region of the
% surrounding granule cell domain. Caudally-distributed
% cells projected their dendrites into the lamina between the
% PVCN and the DCN, laterally located cells extended into
% the superficial layer of granule cells along the lateral edge
% of the AVCN, dorsally-located cells reached the subpedun-
% cular region in the dorsomedial part of VCN, and medially
% positioned cells contacted the medial sheet of granule cells
% or sometimes even intermingled with fibers associated
% with the vestibular nerve root.


\begin{itemize}
\item sensitive to neuromodulatory currents \citep{FujinoOertel:2001}
\begin{itemize}
\item high input resistance $\rightarrow$ amplify small current inputs \citep{FujinoOertel:2001}
\item no LKT in TS,  LKT makes bushy and optopus insensitive to steady currents \citep{OertelFujino:2001,McGinleyOertel:2006}
\item Ih higher in TS \& activated more at lower potentials than in bushy and octopus, so that it is less active at rest
\item high resistance $\rightarrow$ greater voltage changes in small modulating current $\rightarrow$ Ih can be modulated by G-protein coupled receptors, hence making TS more excitable when Ih activated \citep{RodriguesOertel:2006}
\end{itemize}
\end{itemize}

\begin{enumerate}
\item Driving inputs
\end{enumerate}
Proximal dendrites and at the soma:

\begin{itemize}
\item ANF provide glutamatergic excitation for T stellates  \citep{Cant:1981,FerragamoGoldingEtAl:1998a,Alibardi:1998a}
\begin{itemize}
\item only 5 or 6 in mice \citep{FerragamoGoldingEtAl:1998a,CaoOertel:2010}
\end{itemize}
\item Recurrent excitation from other T stellate cells \citep{FerragamoGoldingEtAl:1998a}
\item Glycine from DS cells \citep{FerragamoGoldingEtAl:1998a}
\item Glycine from TV cells \citep{WickesbergOertel:1990,ZhangOertel:1993b}
\item Neuromodulatory
\end{itemize}
     No signs of PSP or PSCs hence distal or G-protein coupled, effects on time-course minimal
     
a. Golgi cells (GABA)

\begin{itemize}
\item no IPSPs or IPSCs but presence of GABAa receptors and response changes to bicuculine \citep{WuOertel:1986,OertelWickesberg:1993,FerragamoGoldingEtAl:1998a}
\item dend filter obscures PSPs
\item Golgi cells are GABAergic and lie within the granule cell domains around the VCN and terminate near the fine distal dendrites of T stellate cells
\end{itemize}
b. Periolivary cells (GABA + GAD - glutamic acid decarboxylase) 

\begin{itemize}
\item observed in PVCN close to TS \citep{AdamsMugnaini:1987}
\item GAD effectively Glycine \citep{GoldingOertel:1997}
\item can also arise from GABAergic neurons in ipsi LNTB and DM Periolivary
\end{itemize}
c. VNTB cells (ACh)

\begin{itemize}
\item collateral branches of OC go to GCD \citep{MellottMottsEtAl:2011,SherriffHenderson:1994,OsenRoth:1969}
\item TS have nicotinic and muscarinic ACh receptors \citep{FujinoOertel:2001}
\item ACh input to TS, together with OC-cochlea, enhances spectral peaks in noise  \citep{FujinoOertel:2001}
\end{itemize}
d. NE and 5HT

\begin{itemize}
\item Raphe nuclei (5HT)
\item Locus coeruleus Peribrachial cells (NE)
\item both terminate in PVCN \citep{KlepperHerbert:1991,Thompson:2003,ThompsonLauder:2005,Thompson:2003a,ThompsonWiechmann:2002,BehrensSchofieldEtAl:2002,ThompsonThompson:2001,ThompsonThompson:2001a,ThompsonMooreEtAl:1995,ThompsonThompsonEtAl:1994}
\item both increase firing in T stellates \citep{OertelWrightEtAl:2010} in presence of glut and gly blockers -> hence act on post synapse (TS cells)
\item both G-protein coupled, both act on either pre or post synaptic cells
\item NE affects probabilty of release at calyx of Held
\item NE increases firing rate of choppers \citep{KosslVater:1989,Ebert:1996}
\item 5HT excites or inhibits choppers \emph{in vivo} \citep{EbertOstwald:1992}
\end{itemize}
