
\section{D~stellate cells}

%\subsection{Background}


% \citep{PaoliniClark:1999} Long-lasting hyperpolarization was also seen in
% neurons during the presentation of frequency tones just above the high-frequency
% edge of their response area. With increasing click intensity both the
% depolarization and hyperpolarization increased in amplitude
 

% Evidence for inhibition on OC neurons can also be seen with
% click stimuli. Hyperpolarization was observed in this investi-
% gation in the presence or absence of action potentials evoked
% by a single click stimulus. Click stimulation would excite not
% only a large proportion of excitatory projections to OC neurons
% but would also evoke polysynaptic inhibitory and excitatory
% drive from intrinsic or extrinsic sources. The resultant hyper-
% polarization may be due to interplay between such excitatory
% and inhibitory influences. On presentation of successive click
% stimuli, hyperpolarization was not evident, although action
% potential generation decreased, suggesting the presence of in-
% hibitory influences.
% However, in past in vivo intracellular investigations, inhibi-
% tion was not consistently seen. Smith and Rhode (1989) rarely
% observed dips in the level of depolarization or distinct hyper-
% polarization below the cell’s resting potential that might indi-
% cate inhibition, whereas Feng et al. (1994) did observe hyper-
% polarizing influences in a Ct neuron akin to those observed in
% this investigation. This inconsistency may in part be related to
% the frequency of presentation as hyperpolarizing responses
% below resting potential were only seen in this investigation
% when tones were presented at or above the high-frequency edge
% of the neurons’ response area. The use of a nonbarbiturate
% anesthetic may also account for the presence of hyperpolariza-
% tion seen in this investigation. The ability of barbiturates to
% influence inhibitory mechanisms was well documented by
% Evans and Nelson (1973) in the cochlear nucleus. In this
% investigation, we used urethan, which when compared with
% barbiturates appears to have only minor effects on synaptic
% transmission (Crawford 1970; Maggi and Meli 1986).




% Further evidence for the involvement of inhibitory mecha-
% nisms in shaping the OC response is provided by Ferragamo et
% al. (1998). They demonstrated with an in vitro preparation that
% D-stellate cells receive both glycinergic and GABAergic inhi-
% bition. D-stellate cells responded to auditory nerve stimulation
% with a fast followed by a slow depolarization. Although there
% was an absence of hyperpolarization, application of picrotoxin,
% a GABAA receptor blocker, to the cochlear nucleus slice en-
% hanced the firing of D-stellate cells. Ferragamo et al. (1998)
% propose that this GABAergic inhibition may originate from
% Golgi cells in the superficial granule cell domain. This re-
% sponse to auditory nerve stimulation is also seen in a stellate
% cell response type to intracochlear electrical stimulation re-
% corded in vivo (Paolini and Clark 1998). This response type
% also showed little visible evidence of inhibition, although a




% spike could not be evoked on the second excitatory postsyn-
% aptic potential response. Results obtained in this investigation
% and those of previous investigations (Ferragamo et al. 1998;
% Paolini and Clark 1998) suggest that polysynaptic inhibitory
% drive may play an important role in regulating the influence of
% excitation on these OC neurons.
% In addition to GABAergic inhibition, glycinergic inhibitory
% input may also result from tuberculoventral cells in the dorsal
% cochlear nucleus (DCN) (Saint Marie et al. 1991; Wickesberg
% and Oertel 1990). Our recent study has shown that the DCN is
% an important contributor to on-frequency inhibition of VCN
% activity (Paolini et al. 1998). Suppression of DCN by applica-
% tion of muscimol, a GABA agonist, resulted in a decrease in
% response threshold for units in the VCN (Paolini et al. 1998).
% Extrinsic inhibitory input may also originate from the superior
% olivary complex, which has been shown to be an important
% source of inhibitory input to the cochlear nucleus (Ostapoff et
% al. 1997; Saint Marie et al. 1993).
% Inhibitory projections may contribute to the OC response
% through shunting inhibition, as proposed by Smith and Rhode
% (1989). They suggest that the output of the cell may be reduced
% when the membrane potential is close to the reversal potential
% of chloride ions. At this potential, the amplitude of any depo-
% larization will be substantially reduced when the chloride chan-
% nel is activated, leading to increased conductance. Shunting
% inhibition may account for the drop in the level of depolariza-
% tion to sustained levels seen in this investigation. The extent of
% depolarization seen during the sustained component changes
% with frequency of presentation. In response to tones presented
% 30 – 40 dB above threshold at the low-frequency edge of the
neurons’ response area, the amplitude of the sustained depo-
larization approached that of the initial component. This in-
crease in the level of depolarization observed during the sus-
tained component may result from a decrease of inhibitory
drive or of shunting inhibition.
Alternatively, the increase in the level of sustained depolar-
ization seen on presentation of tones on the low-frequency side
of CF may also be explained by an increase in auditory nerve
fiber convergence. As the frequency is decreased and at high
intensities more auditory nerve fibers are recruited as low-
frequency tails are commonly seen in the tuning characteristics
of these fibers. However, an increase in the level of depolar-
ization and spike activity was not seen during the initial com-
ponent of the response when tones were presented at these
frequencies. This suggests that, although more auditory nerve
fibers are excited by low-frequency stimuli at high intensities,
it is unlikely that they all contribute to the summation of inputs
to OC neurons.




\citep{SmithMassieEtAl:2005}

% Onset choppers. The combined evidence from several
% different studies (Cant and Gaston, 1982; Wenthold, 1987;
% Kolston et al., 1992; Shore et al., 1992; Schofield and Cant,
% 1996b; Alibardi, 1998; Needham and Paolini, 2003;
% Palmer et al., 2003; Arnott et al., 2004) suggests that the
% large VCN multipolar cells whose response to short tones
% have been labeled onset-chopper (OC; Rhode et al., 1983a;
% Smith and Rhode, 1989) closely resemble a population of
% large glycinergic multipolar cells that project to the oppo-
% site CN. Single-cell labeling studies in the cat (Smith and
% Rhode, 1989) and guinea pig (Palmer et al., 2003; Arnott
% et al., 2004) have reported that the axons of these cells
% provide collateral innervation to both the ipsilateral dor-
% sal and ventral cochlear nuclei before heading dorsally out
% of the CN. The only direct evidence that the glycinergic
% multipolar cells projecting to the opposite cochlear nucleus
% and the cells with OC response features are the same
% population is one juxtacellularly labeled cell in the guinea
% pig (Arnott et al., 2004) and one intra-axonally labeled cell
% in the cat that we described in a preliminary report (Joris



\citep{CantBenson:2003}
% Type II multipolar cells of the ventral cochlear
% nucleus are the source of widespread inhibitory inputs
% to the ipsilateral dorsal cochlear nucleus and the
% contralateral dorsal and ventral cochlear nuclei
% A second, less plentiful, group of large multipolar cells
% described by Smith and Rhode in the cat ([220]; referred
% to here as type II multipolar cells) probably corresponds
% to the D-stellate cells in mouse [151] and to the radiate
% cells in rat [55] and also probably to the commissural cells
% described in a number of species [13,42,113,204,213,254].
% These neurons can be distinguished from the type I mul-
% tipolar cells in a number of ways. On average, they are
% larger, and their dendrites radiate across the isofrequency
% laminae of the VCN [54,61,151,169]. Both the dendrites
% and the somatic surface are covered with synaptic termi-
% nals, many of which appear to arise from the auditory nerve
% [13,220]. (The so-called type II stellate cells in the anterior
% AVCN [36] are also characterized by extensive somatic in-
% nervation, but as noted above, it has not been established
% whether the cells in the anterior AVCN should be consid-
% ered to be a different population from the multipolar cells in
% the rest of the AVCN and in the PVCN.) Few of the synap-
% tic terminals on the type II multipolar cells are glyciner-
% gic [113], but many of the cells themselves are glycinergic
% [23,54,113,254]. Type II multipolar cells give rise to termi-
% nal dendritic tufts that characteristically extend into the gran-
% ule cell layer bordering the VCN ([55,151,179,204,220];
% cf. [142]).
% The type II multipolar cells are broadly tuned and corre-
% spond to a group of units characterized physiologically as
% “onset chopper” (OnC ) units, which show a well-timed on-
% set response followed by a few regularly timed spikes that
% are not sustained throughout the stimulus as are those of
% “chopper” units [168,169,220]. The OnC units are widely
% scattered throughout the VCN [168,169] matching the distri-
% bution of the commissural cells [113,204,254]. Compatible
% with their wide-reaching dendritic arbor, the type II multi-
% polar cells appear to integrate inputs over a wide frequency
% range [168].
% The known targets of the type II multipolar cells are il-
% lustrated in Fig. 2G. The axons leave the cochlear nucleus
% via the intermediate acoustic stria [3,61,113,204,220]. The
% distribution of multipolar neurons in the VCN whose axons
% leave via the IAS is very similar to that of the commissural
% cells (compare [3,42]), and the only known targets of these
% axons are the contralateral dorsal and ventral cochlear nuclei
% [13,113,204]. The axons also give rise to extensive collat-
% eral branches within the parent VCN and send widespread
% terminations to the ipsilateral DCN [4,55,61,151,164,220].
% In both the ipsilateral and contralateral cochlear nuclei, the
% axonal terminals contain pleomorphic or flattened synaptic
% vesicles, consistent with an inhibitory effect on the target
% cells [13,220]. They would be expected to form a major
% source of wideband inhibition to the DCN [54]. The type II
% multipolar cells do not appear to project to the inferior col-
% liculus [12,102]. It is not known if they give rise to any of
% the branches making up other projections attributed to mul-
% tipolar cells.



\subsection{Cellular Mechanisms of D~stellate Cells}


\subsection{Synaptophysiology of D~stellate Cells}

\citep{FerragamoGoldingEtAl:1998a}
% D stellate cells received glycinergic inhibition. Figs. 13A
% and 14 show that 1 mM strychnine eliminated the occasional
% IPSPs but that glycinergic inhibition in D stellate cells did
% not prominently affect synaptic responses.
% To test whether NMDA receptors mediate the long-lasting
% firing of D stellate cells in responses to shocks, 100 mM
% APV was applied to the bath. Figure 13A shows that APV
% reversibly eliminated the long-lasting depolarization. The
% remaining early excitation was blocked reversibly by DNQX
% (n Å 2/2). To determine whether NMDA receptors were
% intrinsic to the recorded cell or on excitatory interneurons,
% the voltage dependence of the late response was examined
% (Fig. 13B). The long, late depolarization that caused the D
% stellate cell to fire for Ç200 ms was shortened to Ç100 ms
% when the cell was hyperpolarized, this is consistent with
% synaptic excitation mediated by NMDA receptors that were
% intrinsic to the D stellate cell.
% GABAergic inhibition plays a prominent role in the synap-
% tic responses of D stellate cells. Even in the absence of
% visible IPSPs, picrotoxin, a blocker of GABAA receptors,
% enhanced the firing of the cell (Fig. 14; n Å 1/1). Both
% the frequency and duration of firing were augmented in the
% presence of picrotoxin.


\subsection{Acoustic Response of D~stellate  Cells}







\subsection{Periodicity and Temporal Coding in the Ventral Cochlear Nucleus}


%\subsection{Commissural and dorsal output of D~stellate cells}




\subsection{Neuromodulatory Effects on D~stellate Cells}


% \citep{DoucetRyugo:1997}
% Depending on somatic location, dendritic processes were
% extended into the nearest and most proximal region of the
% surrounding granule cell domain. Caudally-distributed
% cells projected their dendrites into the lamina between the
% PVCN and the DCN, laterally located cells extended into
% the superficial layer of granule cells along the lateral edge
% of the AVCN, dorsally-located cells reached the subpedun-
% cular region in the dorsomedial part of VCN, and medially
% positioned cells contacted the medial sheet of granule cells
% or sometimes even intermingled with fibers associated
% with the vestibular nerve root.


\begin{itemize}
\item sensitive to neuromodulatory currents \citep{FujinoOertel:2001}
\begin{itemize}
\item high input resistance $\rightarrow$ amplify small current inputs \citep{FujinoOertel:2001}
\item no LKT in TS,  LKT makes bushy and octopus insensitive to steady currents \citep{OertelFujino:2001,McGinleyOertel:2006}
\item Ih higher in TS \& activated more at lower potentials than in bushy and octopus, so that it is less active at rest
\item high resistance $\rightarrow$ greater voltage changes in small modulating current $\rightarrow$ Ih can be modulated by G-protein coupled receptors, hence making TS more excitable when Ih activated \citep{RodriguesOertel:2006}
\end{itemize}
\end{itemize}

\begin{enumerate}
\item Driving inputs
\end{enumerate}
Proximal dendrites and at the soma:

\begin{itemize}
\item ANF provide glutamatergic excitation for T stellates  \citep{Cant:1981,FerragamoGoldingEtAl:1998a,Alibardi:1998a}
\begin{itemize}
\item only 5 or 6 in mice \citep{FerragamoGoldingEtAl:1998a,CaoOertel:2010}
\end{itemize}
\item Recurrent excitation from other T stellate cells \citep{FerragamoGoldingEtAl:1998a}
\item Glycine from DS cells \citep{FerragamoGoldingEtAl:1998a}
\item Glycine from TV cells \citep{WickesbergOertel:1990,ZhangOertel:1993b}
\item Neuromodulatory
\end{itemize}
     No signs of PSP or PSCs hence distal or G-protein coupled, effects on time-course minimal
     
a. Golgi cells (GABA)

\begin{itemize}
\item no IPSPs or IPSCs but presence of GABAa receptors and response changes to bicuculine \citep{WuOertel:1986,OertelWickesberg:1993,FerragamoGoldingEtAl:1998a}
\item dend filter obscures PSPs
\item Golgi cells are GABAergic and lie within the granule cell domains around the VCN and terminate near the fine distal dendrites of T stellate cells
\end{itemize}
b. Periolivary cells (GABA + GAD - glutamic acid decarboxylase) 

\begin{itemize}
\item observed in PVCN close to TS \citep{AdamsMugnaini:1987}
\item GAD effectively Glycine \citep{GoldingOertel:1997}
\item can also arise from GABAergic neurons in ipsi LNTB and DM Periolivary
\end{itemize}
c. VNTB cells (ACh)

\begin{itemize}
\item collateral branches of OC go to GCD \citep{MellottMottsEtAl:2011,SherriffHenderson:1994,OsenRoth:1969}
\item TS have nicotinic and muscarinic ACh receptors \citep{FujinoOertel:2001}
\item ACh input to TS, together with OC-cochlea, enhances spectral peaks in noise  \citep{FujinoOertel:2001}
\end{itemize}
d. NE and 5HT

\begin{itemize}
\item Raphe nuclei (5HT)
\item Locus coeruleus Peribrachial cells (NE)
\item both terminate in PVCN \citep{KlepperHerbert:1991,Thompson:2003,ThompsonLauder:2005,Thompson:2003a,ThompsonWiechmann:2002,BehrensSchofieldEtAl:2002,ThompsonThompson:2001,ThompsonThompson:2001a,ThompsonMooreEtAl:1995,ThompsonThompsonEtAl:1994}
\item both increase firing in T stellates \citep{OertelWrightEtAl:2010} in presence of glut and gly blockers -> hence act on post synapse (TS cells)
\item both G-protein coupled, both act on either pre or post synaptic cells
\item NE affects probability of release at calyx of Held
\item NE increases firing rate of choppers \citep{KosslVater:1989,Ebert:1996}
\item 5HT excites or inhibits choppers \emph{in vivo} \citep{EbertOstwald:1992}
\end{itemize}


