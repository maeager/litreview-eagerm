%% LaTeX2e file 
%% generated by the `filecontents' environment
%% from source `ltxtable' on 
%%
%\begin{table}[tp]
%  \centering
%\begin{center}
%\renewcommand{\tabularxcolumn}[1]{>{\arraybackslash}m{#1}} 
%vertically centered column cells
\newcolumntype{Z}{>{\centering\arraybackslash}X}  % centered cell
\newcolumntype{Y}{>{\raggedright\arraybackslash}X} %left aligned
\begin{longtable}{cYYYYYYY}%
%
\caption{Evidence of Connections in the VCN}\label{tab:Connections} \\
\toprule  {\textbf{Connection}} & {\textbf{Sypnapse Type}} & {\textbf{Synaptic Time Constant (ms)}} & \multicolumn{2}{c}{\textbf{Receptive Field}} & {\textbf{Number of Synapses}} & {\textbf{Placement}} & {\textbf{Delay (ms)}}\\
&&& \small{Pre-cell axon coverage} & \small{Post-cell dendrite coverage} &&\small{Somatic and dendritic}& \small{Pre-cell activation via shock illiciting PSP} \\ \midrule 
\endfirsthead

\multicolumn{8}{c}{{\bfseries \tablename\ \thetable{} -- continued from previous page}} \\
\midrule {\textbf{Connection}} & {\textbf{Syn. Type}} & {\textbf{Time Constant}} & {\textbf{Axon Field}}& {\textbf{Dend. Field}} & {\textbf{No. Synapses}} & {\textbf{Placement}} & {\textbf{Delay}} \\ \midrule 
\endhead

\midrule \multicolumn{8}{c}{{Continued on next page}} \\ %\midrule
\endfoot
\bottomrule
\endlastfoot

AN\ensuremath{\rightarrow}TS                       
& % Synapse Type 
1. Glutamatergic AMPA receptors (possible NMDA) \citep{FerragamoGoldingEtAl:1998a}. 
%2. Glutamatergic NMDA receptors \citep{FerragamoGoldingEtAl:1998a}.
% Glutamate and glutamine antibodies \citep{HackneyOsenEtAl:1990}. 
% Whole cell patch recordings NMDA \ensuremath{\rightarrow} AMPA during development \citep{BellinghamLimEtAl:1998}. 
% GluR3, GluR4 main AMPA subunits electron microscopic study \citep{WangWentholdEtAl:1998}.              
& % Time Constant 
Rise 0.15 ms,  Decay 0.36 ms \citep{GardnerTrussellEtAl:1999}
%\citep{Oertel:1983}
& % RF Pre
70$\mu$m (HSR), 100$\mu$m (LSR)  \citep{CantBenson:2003,RyugoParks:2003}
%\citep[Single fibres][]{OertelWuEtAl:1990,Ryugo:2008,MeltzerRyugo:2006,RyugoParks:2003,Ryugo:1992,BrownBerglundEtAl:1988,RoullierCronin-SchreiberEtAl:1986,FeketeRouillerEtAl:1984} 
& % RF Post
75-100$\mu$m \citep[Mouse]{OertelWuEtAl:1990} 
% 0.23-0.39 oct \citep[anesthetized guinea pig][]{PalmerJiangEtAl:1996} 
% Q\textsubscript{10} 5.52$pm$1.4  \citep[guinea pig]{JiangPalmerEtAl:1996} compared to Q10 of 6.3 in AN;
% CS Q\textsubscript{10} 4,CT Q\textsubscript{10} 2 (low cf), 3.67 (high CF)  \citep[guinea pig]{PalmerWallaceEtAl:2003}
% Q\textsubscript{10} 5.3  \citep{RhodeSmith:1986}.
% Q\textsubscript{10} 7.4 (unanesthetised) 5.3 (barbiturate) \citep[cat][]{RhodeKettner:1987}.                      
& % Number 
Small number on soma, highly variable, 0-6 per soma depending on subtypes. Dendrites more coverage of ANF contacts \citep{Cant:1981,FayPopper:1994,ReddCahillEtAl:2002,RyugoWrigthEtAl:1993,Ryugo:1992,RyugoParks:2003,FerragamoGoldingEtAl:1998a,SmithRhode:1989,JosephsonMorest:1998}.
& % Position
Mainly dendritic, $\sim$30\% soma coverage \citep[Cat][]{Cant:1981,Cant:1982,RyugoWrightEtAl:1993,TolbertMorest:1982a}. 
Soma: 36$\pm$10.5 \%  of 21 (range 6-38) \% area coverage, Prox: 43$\pm$29 \%  of 46 \% area cov., Distal: 40$\pm$12 \%  of 22 \% area cov. \citep[cat][]{SmithRhode:1989}
Some cells had synapses surrounding the axon initial segment \citep[chinchilla][]{JosephsonMorest:1998}. 
Type II ANFs diffuse synapses on distal dendrites \citep[mouse][]{BensonBrown:2004}.
& % Delay 
0.7 (range 0.48-0.92) \citep[Mouse][]{FerragamoGoldingEtAl:1998a,Oertel:1983}. 
0.5  \citep[theoretical][]{Brown:1993,BrownLedwith:1990}.
0.5  \citep[Chinchilla][]{WickesbergOertel:1993}. 

%  3.6$\pm$0.38 ms \citep{RhodeSmith:1986} 
%  3.6$\pm$1.2 ms \citep[anesthetised cat][]{RhodeKettner:1987} 
% 3.5 ms \citep[unanesthetised cat][]{RhodeKettner:1987} 
% Acoustic threshold
%  depolarisation min SPL 31.7$\pm$2.9, 
%  AP min SPL 41.8$\pm$3.8  \citep{PaoliniClareyEtAl:2004}
\\ \midrule
ANF\ensuremath{\rightarrow}DS                   
& % Synapse Type 
Glutamate AMPA receptors \citep{FerragamoGoldingEtAl:1998a,WentholdHunterEtAl:1993}.
& % Time Constant 
Rise 0.15, Decay 0.36 \citep{GardnerTrussellEtAl:1999,Oertel:1983}.  
& % RF Pre
As above
& % RF Post
% Q\textsubscript{10} 3.56$\pm$1.38 \citep{JiangPalmerEtAl:1996}. 
% Q\textsubscript{10} 3.1  \citep[Rat, at 10kHz dual component, stronger below CF][]{PaoliniClark:1999}. 
% Q\textsubscript{10} 1.33-2.87 \citep{PalmerWallaceEtAl:2003}. 
2 oct below, 1 oct above \citep{PalmerJiangEtAl:1996}. 
& % Number
Many weak inputs \citep{FerragamoGoldingEtAl:1998a} 
30 somatic inputs (30\% round); see table 1 \citep{SmithRhode:1989}.
& % Position
Dense somatic and dendritic \citep{Cant:1981,Cant:1982,RyugoWrightEtAl:1993}.
Round vesicles \% of total area coverage \citep[cat][]{SmithRhode:1989}:
Soma: 36$\pm$43\%  of 87\% TAC (range $\geq$80), 
Proximal dendrite: 62$\pm$59\%  of 81\% TAC, 
Distal dendrite: 25.5$\pm$7.5\%  of 22\% TAC. 
& % Delay
As above, maybe less considering time to peak,
see latencies in \citep{PaoliniClark:1999} 
2.8 $\pm$0.09 msec \citep[Oc FSL latency][]{RhodeSmith:1986} 
% depolarisation min SPL 45.6$\pm$2.7, AP min SPL 56.6$\pm$2.2 \citep{PaoliniClareyEtAl:2004}
\\ \midrule
ANF \ensuremath{\rightarrow}TV          
& % Type
Glutamate AMPA receptors \citep{ZhangOertel:1993}                          
& % Time Constant   
Rise 0.15, Decay 0.4 \citep{GardnerTrussellEtAl:1999}.
& % RF Pre
\~{}70$\mu$m \citep[mouse][]{SpirouDavisEtAl:1999}
& % RF Post
$\sim$70$\mu$m \citep[mouse][]{ZhangOertel:1993,SpirouDavisEtAl:1999}
& % Number
$\sim$4 somatic contacts per cell, 10\% AN \citep[guinea pig][]{Alibardi:1999}, 
17.7\% (ANF terminals, 32.5\% GLUT) of total synaptic terminals in deep layer \citep[rat][]{RubioJuiz:2004}
% \citep{SpirouDavisEtAl:1999,ZhangOertel:1993} ;       
& % Position
Proximal dendrite, none on soma \citep[rat][]{RubioJuiz:2004}  
Mainly dendritic {\textless} 100$\mu$m \citep{Alibardi:1999,Liberman:1993,SpirouDavisEtAl:1999}
& %
1.0--1.5 ms \citep{ZhangOertel:1993,Rhode:1999,SpirouDavisEtAl:1999}
\\ \midrule
ANF\ensuremath{\rightarrow}G                      
& % Type 
AMPA (LSR), NMDA (type II ANF, granule cells)
\citep{Cant:1992,FerragamoGoldingEtAl:1998a,RyugoWrightEtAl:1993,Ryugo:1992,RyugoParks:2003}.
Diffuse release sites \citep{HurdHutsonEtAl:1999}             
& %Time Constant 
0.4 ms (AMPA)   \citep{GardnerTrussellEtAl:1999}. 
slower (NMDA, diffuse) \citep{HurdHutsonEtAl:1999}.    
& % Pre
LSR 175~$\mu$m (1 mm low CF)\citep{Ryugo:2008}, type II ANF less organised in GCD \citep{WeedmanPongstapornEtAl:1996,RyugoWrigthEtAl:1993}
& % Post 
%Wide Dynamic range \citep{GhoshalKim:1997} 
250--100 $\mu$m \citep{FerragamoGoldingEtAl:1998a}                     
&                                            
& % Position
Mainly contacts on thin dendrites (0.5-1 $\mu$m width) in GCD, SCC 
\citep{BensonBrown:2004,FerragamoGoldingEtAl:1998}                
& % Delay
1.3 msec minimum \citep{FerragamoGoldingEtAl:1998a}, 
type II up to 10msec \citep[theoretical][]{Brown:1993}
\\ \midrule
TS\ensuremath{\rightarrow}TS                        
& % Type 
AMPA and/or NMDA glutamate receptors \citep{FerragamoGoldingEtAl:1998a}        
& % 
Similar to primary endings but not measured.  
NMDA not measured.           
&
$\sim$70$\mu$m \citep[cat][]{SmithRhode:1989} 
& % RF
$\sim$70$\mu$m \citep[cat][]{SmithRhode:1989} 
& % Number
Based on rarity of small round vesicles on CS cells, influence is weak \citep{SmithRhode:1989} 
likely to come from 1 input \citep{FerragamoGoldingEtAl:1998a}                     
\~{}5 per soma  \citep[chinchilla][]{JosephsonMorest:1998}
& %Position
Somatic (high and low CF), more prevalent at the axon hillock   \citep[chinchilla][]{JosephsonMorest:1998}
& %delay
Min.\ synaptic delay 
\\ \midrule
TS\ensuremath{\rightarrow}DS                        
& %Type 
AMPA and/or NMDA glutamate receptors \citep{FerragamoGoldingEtAl:1998a}  
& As per TS\ensuremath{\rightarrow}TS.
& % RF Pre
& % RF Post
Few local axonal collaterals lie in dendritic plane \citep{SmithRhode:1989}, but DS dendrites cover 1/3 of VCN
& 
Based on rarity of small round vesicles on CS cells, influence is weak \citep{SmithRhode:1989} late EPSPs observed,
likely one-2 inputs \citep{FerragamoGoldingEtAl:1998a, OertelWuEtAl:1990} 
& ---                                     
& Min. synaptic delay \\ \midrule
TS\ensuremath{\rightarrow}TV                        
& %Type
AMPA Glutamate \citep{DoucetRossEtAl:1999,FerragamoGoldingEtAl:1998a,ZhangOertel:1993}.
No TS terminals on TV cells in rats \citep[rat][]{RubioJuiz:2004}.
Small round vesicles present in deep layer of DCN \citep[guinea pig][]{Alibardi:1999}
& 
Similar to AN AMPA receptors.        
& % RF Pre
& % RF Post

Strong On-CF with weak off-CF  \citep[See fig 13][]{OstapoffBensonEtAl:1999}. 
Mainly in iso-band but poor classification in rats \citep{DoucetRossEtAl:1999,FriedlandPongstapornEtAl:2003}         
& %Number
3 of 4 neurons had late EPSPs to AN shock \citep[very young mice][]{ZhangOertel:1993}.
\citep{OstapoffBensonEtAl:1999} 

& %Position
Mainly to fusiform cell layer, possibly on ends of TV cell dendrites \citep[mouse][]{OertelWuEtAl:1990}. 
CS and CT collaterals in deep DCN possibly somatic \citep[guinea pig][]{PalmerWallaceEtAl:2003}.
Somatic small round vesicles presumably from T stellates  \citep[guinea pig][]{Alibardi:1999}   
& %Delay 
0.15 ms \citep[min EPSP latency to VCN Glutamate puffs, main excitation at 0.3 ms, AN shock produces late EPSPs about 3 msec][]{ZhangOertel:1993}.
\\ \midrule
DS\ensuremath{\rightarrow}TS                        
& % Type 
Glycine GlyR receptors \citep[mouse][]{FerragamoGoldingEtAl:1998a}.
Flat vesicles (DS) apposed to TS units \citep[cat][]{SmithRhode:1989}     
Could be mixed Gly/GABA \citep{AltschulerJuizEtAl:1993} 
& % Time Constant
Rise 0.4 ms, Decay 2.5 ms \citep[spontaneous IPSCs in rat MNTB neurons,][]{AwatramaniTurecekEtAl:2005}
Rise 0.46$\pm$0.05 ms \citep[spontaneous IPSCs in AVCN bushy cells, mouse][]{LimOleskevichEtAl:2003}

%Decay  5.47 $\pm$0.19 \citep[very young MNTB rat][]{AwatramaniTurecekEtAl:2005}
%Decay 6--13 ms \citep[Slice prep 30 C degrees; VCN guinea pig][]{HartyManis:1998}.
%Activation to 1mM Gly 2.0$\pm$1.2 ms (range 0.8 to 4.6 ms), deactivation to 1s Gly $\tau_{\textrm{fast}}$ 15.5 ms and $\tau_{\textrm{fast}}$ 73.4 ms \citep[MNTB mouse][]{LeaoOleskevichEtAl:2004}.

Decay 1.6 ms \citep[mouse VCN,]{Oertel:1983}
Decay 5.4 ms \citep{OertelWickesberg:1993,WickesbergOertel:1993}    
Activation 2.0$\pm$1.2 ms Decay 5.3 ms \citep[Gly puffs at 22$^\circ$C (Q$_{10}$ 2.1) in  guinea pig VCN,][]{HartyManis:1998}
& % RF Pre
DS axon terminals cover 300 $\mu$m of VCN \citep[mouse][]{OertelWuEtAl:1990}.
AVCN collaterals centred on soma isofreq. as dend, 1 octave above and 2 oct below \citep[gerbil][]{ArnottWallaceEtAl:2004} 
SBW=5.1kHz$\pm$4.5 kHz all Ch, CS 4.66$\pm$4.45kHz 88$\pm$19\% suppression, CT 6.28$\pm$ 4.65kHz    96$\pm$5\% suppression \citep{RhodeGreenberg:1994b}
& % RF Post
 
& %Position
\citep[mice][]{FerragamoGoldingEtAl:1998a} 
See Table 1 \citep[cat][]{SmithRhode:1989} 
 \~{}70 (high) \~{}60 (low CF) per soma, \~{}1.7 per axon, FL \~{}20 (highCF)
               \~{}10 (lowCF) \citep[chinchilla][]{JosephsonMorest:1998}                

& %Number
1 or 2 on soma; many gly and mixed gly/GABA on trunks; see Table 1\citep{SmithRhode:1989} more FL
    vesicles on soma in high CF regions \citep{JosephsonMorest:1998}      
& 
1.2-3.5msec shock to AN \citep{FerragamoGoldingEtAl:1998a,NeedhamPaolini:2003} \citep{Oertel:1983}
\\ \midrule
DS\ensuremath{\rightarrow}DS                        
& %type 
Glycinergic                   \citep{FerragamoGoldingEtAl:1998a}?                    
& % time constant 
6-13 msec decay time \citep{AwatramaniTurecekEtAl:2005,HartyManis:1996,HartyManis:1998,LeaoOleskevichEtAl:2004}
& % RF Pre
Axon collaterals: 1/2 of CN
& % RF Post
Dendrites: 1/3 rd cover of CN
& % number
None observed but large Gly terminals apposed to DS \citep[cat][]{SmithRhode:1989} 
IPSPs seen in slice without DCN \citep{FerragamoGoldingEtAl:1998a}                     
&    
Flat Vesicles on D stellate cells \citep[cat][]{SmithRhode:1989}:
Soma: 28 \%  of 87 (range never below 80) \% area coverage. 
Proximal dendrite: 20 \%  of 81 \% area coverage.  
Distal dendrite: 33 \%  of 22 \% area coverage.   

See table 1 \citep{SmithRhode:1989}     
& 
Min. synaptic delay 
\\ \midrule
DS\ensuremath{\rightarrow}TV                        
& %Type
Glycinergic \citep{DoucetRyugoEtAl:1999,OertelWuEtAl:1990,OstapoffMorestEtAl1999,SpirouDavisEtAl:1999,ZhangOertel:1993}. 
& 
Rise 0.4 ms Decay 2.3 ms \citep{AwatramaniTurecekEtAl:2005}
6-13 msec decay time \citep{FerragamoGoldingEtAl:1998a,HartyManis:1996,HartyManis:1998,LeaoOleskevichEtAl:2004}
& 
Lateral sidebands equivalent to DS bandwidth
\citep{OstapoffMorestEtAl:1999,SpirouDavisEtAl:1999} 
in rats strong ventral to iso-frequency band (ie stronger high \ensuremath{\rightarrow}low cf)
\citep{DoucetRyugoEtAl:1999,FriedlandPongstapornEtAl:2003} 
Generally tonotopic arrangement of OnC axons in DCN \citep{ArnottWallaceEtAl:2004}
Notch Noise possibly shows asymmetry in projection \citep{ReissYoung:2005}.      
& 
Few somatic glycine inputs 
\citep{OsenOttersenEtAl:1990,OstapoffMorestEtAl1999:1999,ZhangOertel:1993}
\~{}4 somatic contacts most of which were Glycine (FP) \citep{Alibardi:1999}
33.85 \% total synaptic contact in deep layer \citep[rat][]{RubioJuiz:2004}
& 
Few somatic glycine inputs \citep{OsenOttersenEtAl:1990,OstapoffMorestEtAl:1999,ZhangOertel:1993} 
few  somatic contacts \citep{Alibardi:1999}                   
Half somatic, half dendritic \citep[rat][]{RubioJuiz:2004}
& 
0.15 msec min EPSP latency to VCN shock , assume IPSP latency similar
\citep{ZhangOertel:1993}\\ \midrule
TV\ensuremath{\rightarrow}TS                        
& %type 
Glycine \citep{OertelWickesberg:1993,OstapoffMorestEtAl:1999,SaintBensonEtAl:1991,WickesbergOertel:1993}
Gly\slash GABA mixed \citep[Cat,][]{OsenOttersenEtAl:1990}                  
& % time constant 
Rise 0.4 ms, Decay 2.5 ms \citep[spontaneous IPSCs in rat MNTB neurons,][]{AwatramaniTurecekEtAl:2005}
Decay 1.6 ms \citep[mouse VCN,]{Oertel:1983}
Decay 5.4 ms \citep{OertelWickesberg:1993,WickesbergOertel:1993}    
Rise 2.0$\pm$1.2 ms (0.8--4.6 ms), Decay 5.3 ms \citep[Gly puffs at 22$^\circ$C (Q$_{10}$ 2.1) in  guinea pig VCN,][]{HartyManis:1998}
& %Pre RF
Slightly lateral, lighter on-CF \citep{OstapoffMorestEtAl:1999} 
concentrated on CF \citep{ZhangOertel:1993} 
FL vesicles more dense in high CF \citep{JosephsonMorest:1998}                        
& % Post RF
& % number
Many \citep{OstapoffMorestEtAl:1999} see
Table 1 \citep{SmithRhode:1989} 
PL \~{}70 (high) \~{}60 (low CF) per soma,
\~{}1.7 per axon, 
FL \~{}20 (highCF) \~{}10 (lowCF)  \citep{JosephsonMorest:1998}                        
& %position 
Soma and mainly trunk \citep{AltschulerJuizEtAl:1993} 
see Table 1 \citep{SmithRhode:1989} 
more FL vesicles on soma in high CF regions, some hillock contacts (75\% inhib) \citep{JosephsonMorest:1998}
& %
0.1--0.3 msec glut or shock VCN \citep{ZhangOertel:1993}
0.6 ms after AN excitation in choppers \citep{Wickesberg:1996}. 
2.5 msec from AN shock to inhibition \citep{WickesbergOertel:1993}. 
\\ \midrule
TV\ensuremath{\rightarrow}DS                        
&  % Type 
Glycine \citep{OstapoffMorestEtAl:1999,SaintBensonEtAl:1991}. 
Mixed  Glycine/GABA \citep{OsenOttersenEtAl:1990}                 
& %Time Constant
\citep{OstapoffMorestEtAl:1999}  as above                                  
& %pre RF
Slightly lateral, lighter on-CF \citep{OstapoffMorestEtAl:1999}
concentrated on CF \citep{ZhangOertel:1993} 
No inhibitory sidebands evident in RA                                   
& %post RF
Dendrites cover 1/3 CN \citep[guinea pig][]{PalmerJiangEtAl:1996}.
& See table 1 \citep{SmithRhode:1989}     
& See Table 1 \citep{SmithRhode:1989}                          
& 
As above in TV\ensuremath{\rightarrow}TS
\\ \midrule
%Granule\ensuremath{\rightarrow}GLG                     
%& NMDA  Glutamatergic receptor                 \citep{FerragamoGoldingEtAl:1998a}                     
%& % Time constant
%     \citep{GardnerTrussellEtAl:1999}      
%& % Pre
%& % post   
%& % number
%                   \citep{FerragamoGoldingEtAl:1998a}                     
%& % Position                                           
%& % Delay 
%\\ \midrule
DS\ensuremath{\rightarrow}GLG                                 
&                  
Glycinergic GlyR receptor                  
&  % Time constant
& 
& 
D stellates have collaterals in GCD \citep[mouse][]{OertelWuEtAl:1990}
Gly IPSP observed in 1 of 5 cells \citep{FerragamoGoldingEtAl:1998}    
&                                            
& % Delay 
Short, see Fig 3B in \citep{FerragamoGoldingEtAl:1998}
\\ \midrule
GLG\ensuremath{\rightarrow}TS                         
&% Type 
GABA$_{\textrm{A}}$ receptor  \citep[bicuculine-sensitive VCN T stellate cell, mouse slice preparation][]{FerragamoGoldingEtAl:1998} \citep[Chinchilla][]{JosephsonMorest:1998}
%Ferragamo et al. 1998 found no GABAergic IPSPs but the cells were still sensitive to bicuculine
& % Time constant
Rise 0.7 ms, Decay 9 ms  \citep[probably a combination of two decay time constants, fast 5-10 slow 20-60][]{AwatramaniTurecekEtAl:2005}                     
& % RF Pre
100 $\mu$m in GCD \citep{FerragamoGoldingEtAl:1998}
& % RF Post
Distal dendrites.
High CF\ensuremath{\rightarrow}low CF \citep[chinchilla][]{JosephsonMorest:1998}                        
& %Number    
5-15 \citep[Estimate, cat][]{SmithRhode:1989}     
& % Position
Pleomorphic vesicle coverage in cat \citep{SmithRhode:1989}:
Soma: 17\% of 21\% total (range 6-38 total). 
Prox: 23\% of 46\% total. 
Dist: 27\% of 22\% total. 
& %Delay
Min.\ synaptic delay plus cable delay from distal dendrites.
Somatic GABA terminals most likely from superior olive, delay $\sim$1 ms
\\ \midrule
GLG\ensuremath{\rightarrow}DS                         
& % Type
{GABAergic} {GABA$_{\textrm{A}}$} receptor \citep{EvansZhao:1998,FerragamoGoldingEtAl:1998a,Mugnaini:1985,MugnainiOsenEtAl:1980,SaintMorestEtAl:1989}                           
& % Time Constant                                            
& % RF Pre
200 $\mu$m axon collateral spread in GCD \citep{FerragamoGoldingEtAl:1998}
& % RF Post
Dendrites cover 1/3 CN (effect AN input 2 oct below, 1 oct above) \citep[guinea pig][]{PalmerJiangEtAl:1996}.
Wideband to wideband on CF \citep{EvansZhao:1998} 
Hyperpolarising effects above high freq edge in mid CF neurons \citep[rat][]{PaoliniClark:1999}.                    
& % Number 
\~{}20 weak inputs\citep{SaintMorestEtAl:1989}                     
& % Position
Pleomorphic vesicles coverage in cat \citep{SmithRhode:1989}:
Soma: 36 \%  of 87 (range never below 80) \% area coverage. 
Proximal dendrite: 18 \%  of 81 \% area coverage.  
Distal dendrite: 41 \%  of 22 \% area coverage.   
Dendrites reach into granule cell domain, within reach of Golgi axons \citep{OertelWuEtAl:1990,ArnottWallaceEtAl:2004}     
& %Delay
Hyperpolarisation occurs 10-15 msec after click \citep{PaoliniClark:1999}
\\ \midrule
\end{longtable}





%%% Local Variables: 
%%% mode: latex
%%% mode: tex-fold
%%% TeX-master: "LiteratureReview"
%%% TeX-PDF-mode: nil
%%% End: 
