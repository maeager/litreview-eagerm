%% LaTeX2e file 
%% generated by the `filecontents' environment
%% from source `ltxtable' on 
%%
%\begin{table}[tp]
%  \centering
%\begin{center}
%\renewcommand{\tabularxcolumn}[1]{>{\arraybackslash}m{#1}}  %vertically centered column cells
\newcolumntype{Z}{>{\centering\arraybackslash}X}  % centered cell
\newcolumntype{Y}{>{\raggedright\arraybackslash}X} %left aligned
\begin{longtable}{cYYYYYY}%
%
\caption{Evidence of Connections in the VCN}\label{tab:Connections} \\
\toprule  {\textbf{Connection}} & {\textbf{Synapse Type}} & {\textbf{Synaptic Time Constant (ms)}} &             \multicolumn{2}{c}{\textbf{Receptive Field}}             & {\textbf{Number of Synapses}} %&     {\textbf{Placement}}     
& {\textbf{Delay (ms)}}\\
                                &                          &                                        & \small{Pre-cell axon coverage} & \small{Post-cell dendrite coverage} &                              % & \small{Somatic and dendritic}
& \small{Pre-cell activation via shock eliciting PSP} \\ \midrule 
\endfirsthead

\multicolumn{7}{c}{{\bfseries \tablename\ \thetable{} -- continued from previous page}} \\
\midrule {\textbf{Connection}}  &   {\textbf{Syn. Type}}   &        {\textbf{Time Constant}}        &      {\textbf{Axon Field}}     &       {\textbf{Dend. Field}}        &    {\textbf{No. Synapses}}   % &     {\textbf{Placement}}     
& {\textbf{Delay}} \\ \midrule 
\endhead

\midrule \multicolumn{7}{c}{{Continued on next page}} \\ %\midrule
\endfoot
\bottomrule
\endlastfoot

ANF\ensuremath{\rightarrow}TS                       
& % Synapse Type 
Glutamatergic AMPA receptors (possible NMDA) [\citenum{FerragamoGoldingEtAl:1998a}]. 
%2. Glutamatergic NMDA receptors [\citenum{FerragamoGoldingEtAl:1998a}].
% Glutamate and glutamine antibodies [\citenum{HackneyOsenEtAl:1990}]. 
% Whole cell patch recordings NMDA \ensuremath{\rightarrow} AMPA during development [\citenum{BellinghamLimEtAl:1998}]. 
% GluR3, GluR4 main AMPA subunits electron microscopic study [\citenum{WangWentholdEtAl:1998}].              
& % Time Constant 
0.36 ms (Rise time 0.15 ms [\citenum{GardnerTrussellEtAl:1999}])
%[\citenum{Oertel:1983}]
& % RF Pre
70 \um (HSR), 100$\mu$m (LSR)  [\citenum{CantBenson:2003}]
87.2$\pm$41 \um (HSR), 230.5$\pm$73 \um (LSR) (cat [\citenum{RyugoParks:2003}])
%(Single fibres [\citenum{OertelWuEtAl:1990,Ryugo:2008,MeltzerRyugo:2006,RyugoParks:2003,Ryugo:1992,BrownBerglundEtAl:1988,RoullierCronin-SchreiberEtAl:1986,FeketeRouillerEtAl:1984}]) 
& % RF Post
75--100$\mu$m (mice \citenum{OertelWuEtAl:1990}) 
% 0.23-0.39 oct (anesthetized guinea pig [\citenum{PalmerJiangEtAl:1996}]) 
% Q\textsubscript{10} 5.52$pm$1.4  (guinea pig [\citenum{JiangPalmerEtAl:1996}]) compared to Q10 of 6.3 in AN;
% CS Q\textsubscript{10} 4,CT Q\textsubscript{10} 2 (low cf), 3.67 (high CF)  ( guinea pig [\citenum{PalmerWallaceEtAl:2003}])
% Q\textsubscript{10} 5.3  [\citenum{RhodeSmith:1986}].
% Q\textsubscript{10} 7.4 (unanesthetised) 5.3 (barbiturate) (cat [\citenum{RhodeKettner:1987}]).                      
& % Number 
%only 5 or 6 (mice \citep{FerragamoGoldingEtAl:1998a,CaoOertel:2010})
Small number on soma, highly variable, 0-6 per soma depending on subtypes.
%Dendrites more coverage of ANF contacts [\citenum{Cant:1981,FayPopper:1994,ReddCahillEtAl:2002,RyugoWrigthEtAl:1993,Ryugo:1992,RyugoParks:2003,FerragamoGoldingEtAl:1998a,SmithRhode:1989,JosephsonMorest:1998}].
% & % Position
%Mainly dendritic, $\sim$30\% soma coverage (Cat [\citenum{Cant:1981,Cant:1982,RyugoWrightEtAl:1993,TolbertMorest:1982a}]). 
% (cat RL vesicles [\citenum{SmithRhode:1989}]
% Soma: 36\% of terminals (10.5\% actual area coverage), 
% Prox: 43\% of terminals (29\% actual area cov.), 
% Dist: 40\% of terminals (12\%  actual area cov.))
(cat: [\citenum{SmithRhode:1989,Cant:1981,FayPopper:1994,ReddCahillEtAl:2002,RyugoWrigthEtAl:1993,Ryugo:1992,RyugoParks:2003}], mice [\citenum{FerragamoGoldingEtAl:1998a,CaoOertel:2010}], chinchilla: [\citenum{JosephsonMorest:1998}])
%Some cells had synapses surrounding the axon initial segment (chinchilla [\citenum{JosephsonMorest:1998}]). 
%Type II ANFs diffuse synapses on distal dendrites (mice [\citenum{BensonBrown:2004}]).
                                & % Delay 
0.7 ms (range 0.48-0.92 ms) (mice: [\citenum{FerragamoGoldingEtAl:1998a,Oertel:1983}]). 
0.5 ms (mice theoretical: [\citenum{Brown:1993,BrownLedwith:1990}]).
0.5 ms (chinchilla: [\citenum{WickesbergOertel:1993}]). 

%  3.6$\pm$0.38 ms [\citenum{RhodeSmith:1986}] 
%  3.6$\pm$1.2 ms (anesthetised cat [\citenum{RhodeKettner:1987}]) 
% 3.5 ms (unanesthetised cat [\citenum{RhodeKettner:1987}]) 
% Acoustic threshold
%  depolarisation min SPL 31.7$\pm$2.9, 
%  AP min SPL 41.8$\pm$3.8  [\citenum{PaoliniClareyEtAl:2004}]
\\ \midrule
%*****************************************
%*****************************************
ANF\ensuremath{\rightarrow}DS                   
                                & % Synapse Type 
Glutamate AMPA receptors [\citenum{FerragamoGoldingEtAl:1998a,WentholdHunterEtAl:1993}].
                                & % Time Constant 
0.36 ms (Rise time 0.15 ms [\citenum{GardnerTrussellEtAl:1999}])
%Rise 0.15, Decay 0.36 [\citenum{GardnerTrussellEtAl:1999,Oertel:1983}].  
                                & % RF Pre
As above
                                & % RF Post
250--350 \um (rat: [\citenum{DoucetRyugo:1997}])
% Q\textsubscript{10} 3.56$\pm$1.38 [\citenum{JiangPalmerEtAl:1996}]. 
% Q\textsubscript{10} 3.1  (Rat, at 10kHz dual component, stronger below CF [\citenum{PaoliniClark:1999}]). 
% Q\textsubscript{10} 1.33-2.87 [\citenum{PalmerWallaceEtAl:2003}]. 
% 2 oct below, 1 oct above (guinea pig [\citenum{PalmerJiangEtAl:1996}, rat [\citenum{PaoliniClark:1999}]). 
                                & % Number
%Many weak inputs [\citenum{FerragamoGoldingEtAl:1998a}] 
%30 somatic inputs (30\% round); see table 1 [\citenum{SmithRhode:1989}].
%                                & % Position
Dense somatic and dendritic coverage (mice: [\citenum{Cant:1981,Cant:1982,RyugoWrightEtAl:1993,FerragamoGoldingEtAl:1998a}], cat [\citenum{SmithRhode:1989}])
%Soma: 36\% of terminals (43\%  of TAC), 
%Prox: 62\% of terminals (59\% of TAC, 73\% actual coverage), 
%Dist: 25.5\% of terminals (7.5\% of TAC). 
                                & % Delay
As above, maybe less considering time to peak,
see latencies in [\citenum{PaoliniClark:1999}] 
2.8 $\pm$0.09 msec (Oc FSL latency [\citenum{RhodeSmith:1986}]) 
% depolarisation min SPL 45.6$\pm$2.7, AP min SPL 56.6$\pm$2.2 [\citenum{PaoliniClareyEtAl:2004}]
\\ \midrule
%*****************************************
%*****************************************
ANF \ensuremath{\rightarrow}TV          
                                & % Type
Glutamate AMPA receptors [\citenum{ZhangOertel:1993}]                          
                                & % Time Constant   
0.4 ms (Rise time 0.15 ms [\citenum{GardnerTrussellEtAl:1999}])
%Rise 0.15, Decay 0.4 [\citenum{GardnerTrussellEtAl:1999}].
                                & % RF Pre
$\sim$70$\mu$m (mice: [\citenum{SpirouDavisEtAl:1999}])
                                & % RF Post
$\sim$70$\mu$m (mice: [\citenum{ZhangOertel:1993,SpirouDavisEtAl:1999}])
                                & % Number
Mainly dendritic (guinea pig: [\citenum{Alibardi:1999}], cats: [\citenum{Liberman:1993,SpirouDavisEtAl:1999}], rat: [\citenum{RubioJuiz:2004}])
%, guinea pig: $\sim$4 somatic contacts per cell, 10\% AN, rest dendritic [\citenum{Alibardi:1999}] 
%17.7\% (ANF terminals, 32.5\% GLUT) of total synaptic terminals in deep layer (rat [\citenum{RubioJuiz:2004}])
% [\citenum{SpirouDavisEtAl:1999,ZhangOertel:1993}] ;       
%                                & % Position  
                                & %
1.0--1.5 ms [\citenum{ZhangOertel:1993,Rhode:1999,SpirouDavisEtAl:1999}]
\\ \midrule
%*****************************************
%*****************************************
ANF\ensuremath{\rightarrow}G                      
                                & % Type 
AMPA (LSR), NMDA (type II ANF, granule cells)
[\citenum{Cant:1992,FerragamoGoldingEtAl:1998a,RyugoWrightEtAl:1993,Ryugo:1992,RyugoParks:2003}].
Diffuse release sites [\citenum{HurdHutsonEtAl:1999}]             
                                & %Time Constant 
0.4 ms (AMPA)   [\citenum{GardnerTrussellEtAl:1999}]. 
-- (NMDA, diffuse) [\citenum{HurdHutsonEtAl:1999}].    
                                & % Pre
LSR 175~$\mu$m (1 mm low CF)[\citenum{Ryugo:2008}], type II ANF less organised in GCD \citenum{WeedmanPongstapornEtAl:1996,RyugoWrigthEtAl:1993}]
                                & % Post 
%Wide Dynamic range [\citenum{GhoshalKim:1997}] 
250--100 $\mu$m [\citenum{FerragamoGoldingEtAl:1998a}]                     
                                & 
%                                & % Position
Mainly contacts on thin dendrites (0.5--1 $\mu$m width)  [\citenum{BensonBrown:2004,FerragamoGoldingEtAl:1998}]                
                                & % Delay
1.3 [\citenum{FerragamoGoldingEtAl:1998a}], 
(type II up to 10ms, theoretical [\citenum{Brown:1993}])
\\ \midrule
%*****************************************
%*****************************************
TS\ensuremath{\rightarrow}TS                        
                                & % Type 
AMPA and/or NMDA glutamate receptors [\citenum{FerragamoGoldingEtAl:1998a}]        
                                & % 
Similar to primary endings but not measured.  
NMDA not measured.           
                                & 
$\sim$70$\mu$m (cat [\citenum{SmithRhode:1989}]) 
                                & % RF
$\sim$70$\mu$m (cat [\citenum{SmithRhode:1989}]) 
                                & % Number
Rare (cat: [\citenum{SmithRhode:1989}]) 
1 input (mice:[\citenum{FerragamoGoldingEtAl:1998a}])                     
$\sim$5 (chinchilla: [\citenum{JosephsonMorest:1998}])
%                                & %Position
Somatic (high and low CF), more prevalent at the axon hillock   
(chinchilla: [\citenum{JosephsonMorest:1998}])
                                & %delay
Min.\ synaptic delay \\ \midrule
%*****************************************
%*****************************************
TS\ensuremath{\rightarrow}DS                        
                                & %Type 
AMPA and/or NMDA glutamate receptors [\citenum{FerragamoGoldingEtAl:1998a}]  
                                & 
As per TS\ensuremath{\rightarrow}TS.
                                & % RF Pre
                                & % RF Post
Few local axonal collaterals lie in dendritic plane [\citenum{SmithRhode:1989}], but DS dendrites cover 1/3 of VCN
                                & 
Rare  (cat: [\citenum{SmithRhode:1989}])
likely 2 inputs (mice: [\citenum{FerragamoGoldingEtAl:1998a, OertelWuEtAl:1990}]) 
%                                & 
---                                     
                                & 
Min. synaptic delay \\ \midrule
%*****************************************
%*****************************************
TS\ensuremath{\rightarrow}TV                        
                                & %Type
Glutamatergic AMPA receptor  [\citenum{DoucetRossEtAl:1999,FerragamoGoldingEtAl:1998a,ZhangOertel:1993}].
%Small round vesicles present in deep layer of DCN (guinea pig [\citenum{Alibardi:1999}])
                                & 
Similar to ANF AMPA receptors.        
                                & % RF Pre

                                & % RF Post
Strong On-CF with weak off-CF  (See fig 13 [\citenum{OstapoffBensonEtAl:1999}]). 
Mainly in iso-band but poor classification in (rats: [\citenum{DoucetRossEtAl:1999,FriedlandPongstapornEtAl:2003}])         
                                & %Number
3 of 4 neurons had late EPSPs to AN shock (very young mice [\citenum{ZhangOertel:1993}]).
[\citenum{OstapoffMorestEtAl:1999}] 
%                                & %Position
None (rats: [\citenum{RubioJuiz:2004}]).
Mainly dendritic (mice: [\citenum{OertelWuEtAl:1990}],
guinea pigs: [\citenum{PalmerWallaceEtAl:2003,Alibardi:1999}])   
                                & %Delay 
$\sim$2 ms (mice Fig.~10 \textit{cell 1} [\citenum{ZhangOertel:1993}])
% 150--250 ms (VCN Glutamate puffs, mice:  [\citenum{ZhangOertel:1993}]).
\\ \midrule


%*****************************************
DS\ensuremath{\rightarrow}TS                        
                                & % Type 
Glycine GlyR receptors (mice [\citenum{FerragamoGoldingEtAl:1998a}]).
%Flat vesicles (DS) apposed to TS units (cat [\citenum{SmithRhode:1989}])     
Could be mixed Gly/GABA [\citenum{AltschulerJuizEtAl:1993}] 
                                & % Time Constant
2.5 ms (Rise 0.4 ms, rat: [\citenum{AwatramaniTurecekEtAl:2005}],
mice: [\citenum{LimOleskevichEtAl:2003}])

%Decay  5.47 $\pm$0.19 (very young MNTB rat [\citenum{AwatramaniTurecekEtAl:2005}])
%Decay 6--13 ms (Slice prep 30 C degrees; VCN guinea pig [\citenum{HartyManis:1998}]).
%Activation to 1mM Gly 2.0$\pm$1.2 ms (range 0.8 to 4.6 ms), deactivation to 1s Gly $\tau_{\textrm{fast}}$ 15.5 ms and $\tau_{\textrm{fast}}$ 73.4 ms (MNTB mice [\citenum{LeaoOleskevichEtAl:2004}]).

1.6 ms (mice [\citenum{Oertel:1983}])
5.4 ms (mice: [\citenum{OertelWickesberg:1993,WickesbergOertel:1993}])    
Activation 2.0$\pm$1.2 ms Decay 5.3 ms (Gly puffs at 22$^\circ$C (Q$_{10}$ 2.1) in  guinea pig VCN [\citenum{HartyManis:1998}])
                                & % RF Pre
DS axon terminals cover 300 $\mu$m of VCN (mice [\citenum{OertelWuEtAl:1990}]).
AVCN collaterals centred on soma isofreq. as dend, 1 octave above and 2 oct below (gerbil [\citenum{ArnottWallaceEtAl:2004}]) 
SBW=5.1kHz$\pm$4.5 kHz all Ch, CS 4.66$\pm$4.45kHz 88$\pm$19\% suppression, CT 6.28$\pm$ 4.65kHz    96$\pm$5\% suppression [\citenum{RhodeGreenberg:1994a}]
                                & % RF Post

                                & %Number
1 or 2 on soma; many gly and mixed gly/GABA on trunks; see Table 1[\citenum{SmithRhode:1989}] 
more FL vesicles on soma in high CF regions [\citenum{JosephsonMorest:1998}]      

%                                & %Position
%(mice [\citenum{FerragamoGoldingEtAl:1998a}]) 
(cat shared with TV terminals [\citenum{SmithRhode:1989}]: 
Soma 17\% of terminals 21\% TAC,
Prox 23\% of terminals 46\% TAC,
Dist 27\% of terminals 22\% TAC.)


$\sim$70 (high) $\sim$60 (low CF) per soma, 
$\sim$1.7 per axon, FL $\sim$20 (highCF)
$\sim$10 (lowCF) (chinchilla [\citenum{JosephsonMorest:1998}])                

                                & 
2 ms (Shock to VCN, mice Fig.~10 \textit{cell 5} [\citenum{ZhangOertel:1993}])

1.2-3.5 msec shock to AN [\citenum{FerragamoGoldingEtAl:1998a,NeedhamPaolini:2003,Oertel:1983}]
Commissural DS units: 1.52 ms shock to cCN [\citenum{NeedhamPaolini:2006}].
\\ \midrule
%*****************************************
%*****************************************
DS\ensuremath{\rightarrow}DS                        
                                & %type 
Glycinergic                   [\citenum{FerragamoGoldingEtAl:1998a}]?                    
                                & % time constant 
6-13 msec decay time [\citenum{AwatramaniTurecekEtAl:2005,HartyManis:1996,HartyManis:1998,LeaoOleskevichEtAl:2004}]
                                & % RF Pre
Axon collaterals: 1/2 of CN
                                & % RF Post
Dendrites: 1/3 rd cover of CN
                                & % number
None observed but large Gly terminals apposed to DS (cat [\citenum{SmithRhode:1989}]) 
IPSPs seen in slice without DCN [\citenum{FerragamoGoldingEtAl:1998a}]                     
%                                & 
(cat shared with TV terminals [\citenum{SmithRhode:1989}]: 
Soma 28\% of terminals 87\% TAC,
Prox 20\% of terminals 81\% TAC,
Dist 33\% of terminals 22\% TAC.)
                                & 
Min. synaptic delay 
Commissural DS units: 1.52 ms shock to cCN [\citenum{NeedhamPaolini:2006}].
\\ \midrule
%*****************************************
%*****************************************
DS\ensuremath{\rightarrow}TV                        
                                & %Type
Glycinergic [\citenum{DoucetRyugoEtAl:1999,OertelWuEtAl:1990,OstapoffMorestEtAl1999,SpirouDavisEtAl:1999,ZhangOertel:1993}]. 
                                & 
Rise 0.4 ms Decay 2.3 ms [\citenum{AwatramaniTurecekEtAl:2005}]
6-13 msec decay time [\citenum{FerragamoGoldingEtAl:1998a,HartyManis:1996,HartyManis:1998,LeaoOleskevichEtAl:2004}]
                                & 
Lateral sidebands equivalent to DS bandwidth
[\citenum{OstapoffMorestEtAl:1999,SpirouDavisEtAl:1999}] 
in rats strong ventral to iso-frequency band (ie stronger high \ensuremath{\rightarrow}low cf)
[\citenum{DoucetRyugoEtAl:1999,FriedlandPongstapornEtAl:2003}] 
Generally tonotopic arrangement of OnC axons in DCN [\citenum{ArnottWallaceEtAl:2004}]
Notch Noise possibly shows asymmetry in projection [\citenum{ReissYoung:2005}].      
                                & 
Few somatic glycine inputs 
[\citenum{OsenOttersenEtAl:1990,OstapoffMorestEtAl:1999,ZhangOertel:1993}]
$\sim$4 somatic contacts most of which were Glycine (FP) [\citenum{Alibardi:1999}]
33.85 \% total synaptic contact in deep layer (rat [\citenum{RubioJuiz:2004}])
%                                & 
Few somatic glycine inputs [\citenum{OsenOttersenEtAl:1990,OstapoffMorestEtAl:1999,ZhangOertel:1993}] 
few  somatic contacts [\citenum{Alibardi:1999}]                   
Half somatic, half dendritic (rat [\citenum{RubioJuiz:2004}])
                                & 
0.15 msec min EPSP latency to VCN shock , assume IPSP latency similar
[\citenum{ZhangOertel:1993}]\\ \midrule
%*****************************************
%*****************************************
TV\ensuremath{\rightarrow}TS                        
                                & %type 
Glycine [\citenum{OertelWickesberg:1993,OstapoffMorestEtAl:1999,SaintBensonEtAl:1991,WickesbergOertel:1993}]
Gly\slash GABA mixed (Cat [\citenum{OsenOttersenEtAl:1990}])                  
                                & % time constant 
Rise 0.4 ms, Decay 2.5 ms (spontaneous IPSCs in rat MNTB neurons [\citenum{AwatramaniTurecekEtAl:2005}])
Decay 1.6 ms (mice VCN [\citenum{Oertel:1983}])
Decay 5.4 ms [\citenum{OertelWickesberg:1993,WickesbergOertel:1993}]    
Rise 2.0$\pm$1.2 ms (0.8--4.6 ms), Decay 5.3 ms (Gly puffs at 22$^\circ$C (Q$_{10}$ 2.1) in  guinea pig VCN [\citenum{HartyManis:1998}])
                                & %Pre RF
Slightly lateral, lighter on-CF [\citenum{OstapoffMorestEtAl:1999}] 
concentrated on CF [\citenum{ZhangOertel:1993}] 
FL vesicles more dense in high CF [\citenum{JosephsonMorest:1998}]                        
                                & % Post RF
                                & % number
Many [\citenum{OstapoffMorestEtAl:1999}] see
Table 1 [\citenum{SmithRhode:1989}] 
PL $\sim$70 (high) $\sim$60 (low CF) per soma,
$\sim$1.7 per axon, 
FL $\sim$20 (highCF) $\sim$10 (lowCF)  [\citenum{JosephsonMorest:1998}]                        
%                                & %position 
(cat shared with DS terminals [\citenum{SmithRhode:1989}]: 
Soma 17\% of terminals 21\% TAC,
Prox 23\% of terminals 46\% TAC,
Dist 27\% of terminals 22\% TAC.)


Soma and mainly trunk [\citenum{AltschulerJuizEtAl:1993}] 
see Table 1 [\citenum{SmithRhode:1989}] 
more FL vesicles on soma in high CF regions, some hillock contacts (75\% inhib) [\citenum{JosephsonMorest:1998}]
                                & %
0.1--0.3 msec glut or shock VCN [\citenum{ZhangOertel:1993}]
0.6 ms after AN excitation in choppers [\citenum{Wickesberg:1996}]. 
2.5 msec from AN shock to inhibition [\citenum{WickesbergOertel:1993}]. 
\\ \midrule
%*****************************************
%*****************************************
TV\ensuremath{\rightarrow}DS                        
                                & % Type 
Glycine [\citenum{OstapoffMorestEtAl:1999,SaintBensonEtAl:1991}]. 
Mixed  Glycine/GABA [\citenum{OsenOttersenEtAl:1990}]                 
                                & %Time Constant
[\citenum{OstapoffMorestEtAl:1999}]  as above                                  
                                & %pre RF
Slightly lateral, lighter on-CF [\citenum{OstapoffMorestEtAl:1999}]
concentrated on CF [\citenum{ZhangOertel:1993}] 
No inhibitory sidebands evident in RA                                   
                                & %post RF
Dendrites cover 1/3 CN (guinea pig [\citenum{PalmerJiangEtAl:1996}]).
                                & See table 1 [\citenum{SmithRhode:1989}]     
 %                               & See Table 1 [\citenum{SmithRhode:1989}]    
(cat shared with DS terminals [\citenum{SmithRhode:1989}]: 
Soma 28\% of terminals 87\% TAC,
Prox 20\% of terminals 81\% TAC,
Dist 33\% of terminals 22\% TAC.)
                      
                                & 
As above in TV\ensuremath{\rightarrow}TS
\\ \midrule
%Granule\ensuremath{\rightarrow}GLG                     
               %                & NMDA  Glutamatergic receptor                 [\citenum{FerragamoGoldingEtAl:1998a}]                     
               %                & % Time constant
%     [\citenum{GardnerTrussellEtAl:1999}]      
               %                & % Pre
               %                & % post   
               %                & % number
%                   [\citenum{FerragamoGoldingEtAl:1998a}]                     
               %                & % Position                                           
               %                & % Delay 
%\\ \midrule
%*****************************************
%*****************************************
DS\ensuremath{\rightarrow}GLG                                 
                                & 
Glycinergic GlyR receptor                  
                                & % Time constant
                                & 
                                & 
D stellates have collaterals in GCD (mice [\citenum{OertelWuEtAl:1990}])
Gly IPSP observed in 1 of 5 cells [\citenum{FerragamoGoldingEtAl:1998}]    
%                                & 
                                & % Delay 
Short, see Fig 3B in [\citenum{FerragamoGoldingEtAl:1998}]
\\ \midrule
%*****************************************
%*****************************************
GLG\ensuremath{\rightarrow}TS                         
                                & % Type 
GABAergic GABA$_{\textrm{A}}$ receptor  (mice [\citenum{FerragamoGoldingEtAl:1998}], chinchilla [\citenum{JosephsonMorest:1998}])
%Ferragamo et al. 1998 found no GABAergic IPSPs but the cells were still sensitive to bicuculine
                                & % Time constant
 9 ms  (MNTB rat neurons, combination of two decay time constants, fast 5-10 and slow 20-60 ms, rise time 0.7 ms [\citenum{AwatramaniTurecekEtAl:2005}])

                                & % RF Pre
200 $\mu$m (localised in GCD, mice [\citenum{FerragamoGoldingEtAl:1998}])
                                & % RF Post
Distal dendrites [\citenum{FerragamoGoldingEtAl:1998a}]
High CF\ensuremath{\rightarrow}low CF (chinchilla [\citenum{JosephsonMorest:1998}])                        
                                & %Number    
5--15 somatic and dendritic (cat PL vesicles [\citenum{SmithRhode:1989}])     
%                                & % Position
%(cat PL vesicles [\citenum{SmithRhode:1989}]:
%Soma: 47\% of terminals, 21\% TAC. 
%Prox: 34\% of terminals, 46\% TAC. 
%Dist: 33\% of terminals, 22\% TAC.)

& %Delay
Min.\ synaptic delay plus cable delay from distal dendrites.
Somatic GABA terminals delay from superior olive $\sim$1 ms
\\ \midrule
%*****************************************
%*****************************************
GLG\ensuremath{\rightarrow}DS                         
& % Type
{GABAergic} {GABA$_{\textrm{A}}$} receptor [\citenum{EvansZhao:1998,FerragamoGoldingEtAl:1998a,Mugnaini:1985,MugnainiOsenEtAl:1980,SaintMorestEtAl:1989}]
& % Time Constant                                            
As per GLG\ensuremath{\rightarrow}TS
% 9 ms  (MNTB rat neurons, combination of two decay time constants, fast 5-10 and slow 20-60, rise time 0.7 ms [\citenum{AwatramaniTurecekEtAl:2005}])
& % RF Pre
As per GLG\ensuremath{\rightarrow}TS
& % RF Post
Dendrites reach into GCD, within reach of Golgi axons [\citenum{OertelWuEtAl:1990,ArnottWallaceEtAl:2004}]     
Dendrites cover 1/3 CN (effect AN input 2 oct below, 1 oct above) (guinea pig [\citenum{PalmerJiangEtAl:1996}]).
Wideband to wideband on CF [\citenum{EvansZhao:1998}] 
Hyperpolarising effects above high freq edge in mid CF neurons (rat [\citenum{PaoliniClark:1999}]).                    
& % Number 
$\sim$20 weak inputs[\citenum{SaintMorestEtAl:1989}]
Strong effect to {GABA$_{\textrm{A}}$} blocker  [\citenum{FerragamoGoldingEtAl:1998a}]
%& % Position
(cat PL vesicles [\citenum{SmithRhode:1989}]:
Soma: 36\% of terminals, 87\% TAC. 
Prox: 18\% of terminals, 81\% TAC. 
Dist: 41\% of terminals, 22\% TAC.)
& %Delay
Hyperpolarisation occurs 10-15 msec after click (rat [\citenum{PaoliniClark:1999}])
\\ \midrule

OCB$\rightarrow$DS
& Acetylcholine (ACh)
& 
&
& Dendrites cover 1/3 CN (guinea pig [\citenum{PalmerJiangEtAl:1996}]).
& 1--2 strong syn \citep{MuldersPaoliniEtAl:2009,HorvathKrausEtAl:2000,MuldersPaoliniEtAl:2003}
%&
& $\sim$5 ms \citep{MuldersPaoliniEtAl:2009}\\
\end{longtable}





%%% Local Variables: 
%%% mode: latex
%%% mode: tex-fold
%%% TeX-master: "LiteratureReview"
%%% TeX-PDF-mode: nil
%%% End: 
