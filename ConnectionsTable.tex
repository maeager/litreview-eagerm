%% LaTeX2e file 
%% generated by the `filecontents' environment
%% from source `ltxtable' on 
%%
%\begin{table}[tp]
%  \centering
%\begin{center}
%\renewcommand{\tabularxcolumn}[1]{>{\arraybackslash}m{#1}}  %vertically centered column cells
\newcolumntype{Z}{>{\centering\arraybackslash}X}  % centered cell
\newcolumntype{Y}{>{\raggedright\arraybackslash}X} %left aligned
\begin{longtable}{ZYYYYYY}%
%
\caption{Evidence of Connections in the VCN}\label{tab:Connections} \\
\toprule {\textbf{Connection}} 
  & {\textbf{Synapse Type}} & {\textbf{Synaptic Time Constant (ms)}} &             \multicolumn{2}{c}{\textbf{Receptive Field}}             & {\textbf{Number of Synapses}} %& {\textbf{Placement}}     
  & {\textbf{Delay (ms)}}\\
  &                         &                                        & \small{Pre-cell axon coverage} & \small{Post-cell dendrite coverage} &               %                & \small{Somatic and dendritic}
  & \small{Pre-cell activation via shock eliciting PSP} \\ \midrule 
\endfirsthead

\multicolumn{7}{c}{{\bfseries \tablename\ \thetable{} -- continued from previous page}} \\
\midrule {\textbf{Connection}} 
  &  {\textbf{Syn. Type}}   &        {\textbf{Time Constant}}        &     {\textbf{Axon Field}}      &       {\textbf{Dend. Field}}        &  {\textbf{No. Synapses}}   %   & {\textbf{Placement}}     
  & {\textbf{Delay}} \\ \midrule 
\endhead

\midrule \multicolumn{7}{c}{{Continued on next page}} \\ %\midrule
\endfoot
\bottomrule
\endlastfoot

ANF\ensuremath{\rightarrow}TS                       
  & % Synapse Type 
AMPA  (possible NMDA) [\citenum{FerragamoGoldingEtAl:1998a}]. 
%2. Glutamatergic NMDA receptors [\citenum{FerragamoGoldingEtAl:1998a}].
% Glutamate and glutamine antibodies [\citenum{HackneyOsenEtAl:1990}]. 
% Whole cell patch recordings NMDA \ensuremath{\rightarrow} AMPA during development [\citenum{BellinghamLimEtAl:1998}]. 
% GluR3, GluR4 main AMPA subunits electron microscopic study [\citenum{WangWentholdEtAl:1998}].              
  & % Time Constant 
0.36 ms (Rise time 0.15 ms [\citenum{GardnerTrussellEtAl:1999}])
%[\citenum{Oertel:1983}]
  & % RF Pre
70 \um (HSR), 100$\mu$m (LSR)  [\citenum{CantBenson:2003}]
87.2$\pm$41 \um (HSR), 230.5$\pm$73 \um (LSR) (cat [\citenum{RyugoParks:2003}])
(See also [\citenum{OertelWuEtAl:1990,Ryugo:2008,MeltzerRyugo:2006,RyugoParks:2003,Ryugo:1992,BrownBerglundEtAl:1988,RoullierCronin-SchreiberEtAl:1986,FeketeRouillerEtAl:1984}]) 
  & % RF Post
75--100$\mu$m (mice [\citenum{OertelWuEtAl:1990}]) 
% 0.23-0.39 oct (anesthetized guinea pig [\citenum{PalmerJiangEtAl:1996}]) 
% Q\textsubscript{10} 5.52$pm$1.4  (guinea pig [\citenum{JiangPalmerEtAl:1996}]) compared to \Qten of 6.3 in AN;
% CS Q\textsubscript{10} 4,CT Q\textsubscript{10} 2 (low cf), 3.67 (high CF)  ( guinea pig [\citenum{PalmerWallaceEtAl:2003}])
% Q\textsubscript{10} 5.3  [\citenum{RhodeSmith:1986}].
% Q\textsubscript{10} 7.4 (unanesthetised) 5.3 (barbiturate) (cat [\citenum{RhodeKettner:1987}]).                      
  & % Number 
%only 5 or 6 (mice \citenum{FerragamoGoldingEtAl:1998a,CaoOertel:2010})
0--6 per soma depending on subtypes, mostly dendritic contacts%(mice: [\citenum{FerragamoGoldingEtAl:1998a,CaoOertel:2010}])
%Dendrites more coverage of ANF contacts [\citenum{Cant:1981,FayPopper:1994,ReddCahillEtAl:2002,RyugoWrigthEtAl:1993,Ryugo:1992,RyugoParks:2003,FerragamoGoldingEtAl:1998a,SmithRhode:1989,JosephsonMorest:1998}].
% & % Position
%Mainly dendritic, $\sim$30\% soma coverage (Cat [\citenum{Cant:1981,Cant:1982,RyugoWrightEtAl:1993,TolbertMorest:1982a}]). 
% (cat RL vesicles [\citenum{SmithRhode:1989}]
% Soma: 36\% of terminals (10.5\% actual area coverage), 
% Prox: 43\% of terminals (29\% actual area cov.), 
% Dist: 40\% of terminals (12\%  actual area cov.))
(cat: [\citenum{SmithRhode:1989,Cant:1981,FayPopper:1994,ReddCahillEtAl:2002,RyugoWrigthEtAl:1993,Ryugo:1992,RyugoParks:2003}], mice [\citenum{FerragamoGoldingEtAl:1998a,CaoOertel:2010}], chinchilla: [\citenum{JosephsonMorest:1998}])
%Some cells had synapses surrounding the axon initial segment (chinchilla [\citenum{JosephsonMorest:1998}]). 
%Type II ANFs diffuse synapses on distal dendrites (mice [\citenum{BensonBrown:2004}]).
  & % Delay 
0.7 ms (range 0.48-0.92 ms) (mice: [\citenum{FerragamoGoldingEtAl:1998a,Oertel:1983}]). 
%0.5 ms (mice theoretical: [\citenum{Brown:1993,BrownLedwith:1990}]).
0.5 ms (chinchilla: [\citenum{WickesbergOertel:1993}]). 

%  3.6$\pm$0.38 ms [\citenum{RhodeSmith:1986}] 
%  3.6$\pm$1.2 ms (anesthetised cat [\citenum{RhodeKettner:1987}]) 
% 3.5 ms (unanesthetised cat [\citenum{RhodeKettner:1987}]) 
% Acoustic threshold
%  depolarisation min SPL 31.7$\pm$2.9, 
%  AP min SPL 41.8$\pm$3.8  [\citenum{PaoliniClareyEtAl:2004}]
\\ \midrule
%*****************************************
%*****************************************
ANF\ensuremath{\rightarrow}DS                   
  & % Synapse Type 
AMPA [\citenum{FerragamoGoldingEtAl:1998a,WentholdHunterEtAl:1993}].
  & % Time Constant 
0.36 ms (Rise time 0.15 ms [\citenum{GardnerTrussellEtAl:1999}])
%Rise 0.15, Decay 0.36 [\citenum{GardnerTrussellEtAl:1999,Oertel:1983}].  
  & % RF Pre
See above \ANFTS
  & % RF Post
250--350 \um (rat: [\citenum{DoucetRyugo:1997}])
%AVCN collaterals centred on soma isofreq. as dend, 
Acoustic RA: 1 octave above and 2 octaves below (gerbil: [\citenum{ArnottWallaceEtAl:2004}], rat: [\citenum{PaoliniClark:1998}])
% Q\textsubscript{10} 3.56$\pm$1.38 [\citenum{JiangPalmerEtAl:1996}]. 
% Q\textsubscript{10} 3.1  (Rat, at 10kHz dual component, stronger below CF [\citenum{PaoliniClark:1999}]). 
% Q\textsubscript{10} 1.33-2.87 [\citenum{PalmerWallaceEtAl:2003}]. 
% 2 oct below, 1 oct above (guinea pig [\citenum{PalmerJiangEtAl:1996}, rat [\citenum{PaoliniClark:1999}]). 
  & % Number
Many weak inputs [\citenum{FerragamoGoldingEtAl:1998a}],  
%30 somatic inputs (30\% round); see table 1 [\citenum{SmithRhode:1989}].
%                                 
Dense somatic and dendritic coverage (mice: [\citenum{Cant:1981,Cant:1982,RyugoWrightEtAl:1993,FerragamoGoldingEtAl:1998a}], cat [\citenum{SmithRhode:1989}])
%Soma: 36\% of terminals (43\%  of TAC), 
%Prox: 62\% of terminals (59\% of TAC, 73\% actual coverage), 
%Dist: 25.5\% of terminals (7.5\% of TAC). 
  & % Delay
As above, maybe less considering time to peak,
(see latencies in [\citenum{PaoliniClark:1999}], \OnC \FSL$=2.8 \pm0.09$ ms [\citenum{RhodeSmith:1986}]) 
% depolarisation min SPL 45.6$\pm$2.7, AP min SPL 56.6$\pm$2.2 [\citenum{PaoliniClareyEtAl:2004}]
\\ \midrule
%*****************************************
%*****************************************
ANF \ensuremath{\rightarrow}TV 
  & % Type
AMPA  [\citenum{ZhangOertel:1993b}]                          
  & % Time Constant   
0.4 ms (Rise time 0.15 ms [\citenum{GardnerTrussellEtAl:1999}])
  & % RF Pre
$\sim$70$\mu$m (mice: [\citenum{SpirouDavisEtAl:1999}])
  & % RF Post
$\sim$70$\mu$m (mice: [\citenum{ZhangOertel:1993b,SpirouDavisEtAl:1999}])
  & % Number
Mainly dendritic (guinea pig: [\citenum{Alibardi:1999}], cats: [\citenum{Liberman:1993,SpirouDavisEtAl:1999}], rat: [\citenum{RubioJuiz:2004}])
%, guinea pig: $\sim$4 somatic contacts per cell, 10\% AN, rest dendritic [\citenum{Alibardi:1999}] 
%17.7\% (ANF terminals, 32.5\% GLUT) of total synaptic terminals in deep layer (rat [\citenum{RubioJuiz:2004}])
% [\citenum{SpirouDavisEtAl:1999,ZhangOertel:1993b}] ;       
% & % Position  
  & %
1.0--1.5 ms [\citenum{ZhangOertel:1993b,Rhode:1999,SpirouDavisEtAl:1999}]
\\ \midrule
%*****************************************
%*****************************************
ANF\ensuremath{\rightarrow}G                      
  & % Type 
AMPA (LSR), NMDA (type II ANF, granule cells)
[\citenum{Cant:1992,FerragamoGoldingEtAl:1998a,RyugoWrightEtAl:1993,Ryugo:1992,RyugoParks:2003}].
Diffuse release sites [\citenum{HurdHutsonEtAl:1999}]             
  & %Time Constant 
0.4 ms (AMPA)   [\citenum{GardnerTrussellEtAl:1999}]. 
-- (NMDA, diffuse) [\citenum{HurdHutsonEtAl:1999}].    
  & % Pre
LSR 175~$\mu$m (1 mm low CF)[\citenum{Ryugo:2008}], type II ANF less organised in GCD \citenum{WeedmanPongstapornEtAl:1996,RyugoWrigthEtAl:1993}]
  & % Post 
%Wide Dynamic range [\citenum{GhoshalKim:1997}] 
100--250 $\mu$m [\citenum{FerragamoGoldingEtAl:1998a}]                     
  & 
% & % Position
Mainly contacts on thin dendrites (0.5--1 $\mu$m width)  [\citenum{BensonBrown:2004,FerragamoGoldingEtAl:1998}]                
  & % Delay
1.3 [\citenum{FerragamoGoldingEtAl:1998a}], 
(type II up to 10ms, theoretical [\citenum{Brown:1993}])
\\ \midrule
%*****************************************
%*****************************************
TS\ensuremath{\rightarrow}TS                    
  & % Type 
AMPA and/or NMDA  [\citenum{FerragamoGoldingEtAl:1998a}]        
  & % 
Similar to primary endings but not measured.  
NMDA not measured.           
  & 
$\sim$70$\mu$m (cat [\citenum{SmithRhode:1989}]) 
  & % RF
$\sim$70$\mu$m (cat [\citenum{SmithRhode:1989}]) 
  & % Number
Rare (cat: [\citenum{SmithRhode:1989}]) 
1 input (mice: [\citenum{FerragamoGoldingEtAl:1998a}])                     
$\sim$5 (more prevalent at the axon hillock, chinchilla: [\citenum{JosephsonMorest:1998}])
% & %Position
%Somatic (high and low CF), more prevalent at the axon hillock   
%(chinchilla: [\citenum{JosephsonMorest:1998}])
  & %delay
Min.\ synaptic delay \\ \midrule
%*****************************************
%*****************************************
TS\ensuremath{\rightarrow}DS                        
  & %Type 
AMPA and/or NMDA  [\citenum{FerragamoGoldingEtAl:1998a}]  
  & 
As per TS\ensuremath{\rightarrow}TS.
  & % RF Pre
Axonal collaterals lie in dendritic plane of TS cell [\citenum{SmithRhode:1989}]
  & % RF Post
AVCN collaterals centred on soma isofreq. as dend, 1 octave above and 2 oct below (gerbil: [\citenum{ArnottWallaceEtAl:2004}]
  & 
Rare  (cat: [\citenum{SmithRhode:1989}])
likely 2 inputs (mice: [\citenum{FerragamoGoldingEtAl:1998a, OertelWuEtAl:1990}]) 
% & 
%---                                     
  & 
Min.\ synaptic delay \\ \midrule
%*****************************************
%*****************************************
TS\ensuremath{\rightarrow}TV  (Excluding rats [\citenum{RubioJuiz:2004}])                      
  & %Type
AMPA and/or NMDA  [\citenum{DoucetRossEtAl:1999,FerragamoGoldingEtAl:1998a,ZhangOertel:1993b}].
%Small round vesicles present in deep layer of DCN (guinea pig [\citenum{Alibardi:1999}])
  & 
See \ANFTV dynamics of AMPA in TV cells        
  & % RF Pre
Strong On-CF with weak off-CF  (See fig 13 [\citenum{OstapoffMorestEtAl:1999}]). 
Mainly in iso-band but poor classification (rats: [\citenum{DoucetRossEtAl:1999,FriedlandPongstapornEtAl:2003}])         
  & % RF Post
See \TV dendrite field \ANFTV  
  & % Number
late EPSPs present after AN shock (mice: [\citenum{ZhangOertel:1993b}], cat:\citenum{OstapoffMorestEtAl:1999}] 
% & %Position
Mainly dendritic (mice: [\citenum{OertelWuEtAl:1990}],
guinea pigs: [\citenum{PalmerWallaceEtAl:2003,Alibardi:1999}])   
  & %Delay 
$\sim$2 ms (mice Fig.~10 \textit{cell 1} [\citenum{ZhangOertel:1993b}])
% 150--250 ms (VCN Glutamate puffs, mice:  [\citenum{ZhangOertel:1993b}]).
0.15 ms (minimum EPSP latency to VCN shock, mice: [\citenum{ZhangOertel:1993b}])
\\ \midrule


%*****************************************
DS\ensuremath{\rightarrow}TS                        
  & % Type 
GlyR  (mice: [\citenum{FerragamoGoldingEtAl:1998a}]).
%Flat vesicles (DS) apposed to TS units (cat [\citenum{SmithRhode:1989}])     
%(Could be mixed Gly/GABA [\citenum{AltschulerJuizEtAl:1993}]) 
  & % Time Constant
See above for GlyR \tfast
2.5 ms (Rise time 0.4 ms, \AVCN rats,mice: [\citenum{LimOleskevichEtAl:2003}])
 $2.9 \pm 0.3$ ms (with \tslow $12.3 \pm 16.4$ ms, \tfast as fast as 1.5 ms, \MNTB rats: [\citenum{AwatramaniTurecekEtAl:2004}])
% $2.1 \pm 0.1$ ms (25$^\circ$C mIPSCs MNTB rats: [\citenum{AwatramaniTurecekEtAl:2004}])
% $0.8 \pm 0.2$ ms (37$^\circ$C mIPSCs MNTB rats: [\citenum{AwatramaniTurecekEtAl:2004}])
% $1.1 \pm 0.2$ ms (37$^\circ$C evoked IPSCs MNTB rats: [\citenum{AwatramaniTurecekEtAl:2004}])
% $2.28 \pm 0.08$ ms mIPSCs, $5.42 \pm 0.19$ ms evoked IPSCs (25$^\circ$C, rise time $0.42 \pm 0.05$ ms, MNTB mice: [\citenum{LeaoOleskevichEtAl:2004}])
$2.8 \pm 0.4$ ms (Desensitisation, \MNTB rat: [\citenum{AwatramaniTurecekEtAl:2005}]) 

%Decay  5.47 $\pm$0.19 (very young MNTB rat [\citenum{AwatramaniTurecekEtAl:2005}])
%6--13 ms (guinea pig: [\citenum{HartyManis:1998}]).
%Activation to 1mM Gly 2.0$\pm$1.2 ms (range 0.8 to 4.6 ms), deactivation to 1s Gly $\tau_{\textrm{fast}}$ 15.5 ms and $\tau_{\textrm{fast}}$ 73.4 ms (MNTB mice [\citenum{LeaoOleskevichEtAl:2004}]).

% 1.6 ms (mice [\citenum{Oertel:1983}])
% 5.4 ms (mice: [\citenum{OertelWickesberg:1993,WickesbergOertel:1993}])    
% 5.3 ms (Activation 2.0$\pm$1.2 ms, guinea pig: [\citenum{HartyManis:1998}])

  & % RF Pre
250--300 $\mu$m (mice [\citenum{OertelWuEtAl:1990}]).
AVCN collaterals centred on soma isofreq. as dend, 1 octave above and 2 oct below  \CF (gerbil: [\citenum{ArnottWallaceEtAl:2004}]) 
%SBW=5.1kHz$\pm$4.5 kHz all Ch, CS 4.66$\pm$4.45kHz 88$\pm$19\% suppression, CT 6.28$\pm$ 4.65kHz    96$\pm$5\% suppression [\citenum{RhodeGreenberg:1994a}]
  & % RF Post
See \ANFTS for TS dendrite field
  & %Number
1 or 2 on soma,  many proximal dendrites (cat: [\citenum{SmithRhode:1989}]) 
%more FL vesicles on soma in high CF regions (chinchilla:[\citenum{JosephsonMorest:1998}])      
% & %Position
%(mice [\citenum{FerragamoGoldingEtAl:1998a}]) 
([\citenum{SmithRhode:1989}] cat, Flat vesicles shared with TV terminals: 
Soma 17\% of synapses, 21\% \TAC;
Prox 23\% of syn, 46\% \TAC;
Dist 27\% of syn, 22\% \TAC)
% $\sim$70 (high CF) $\sim$60 (low CF) per soma, 
$\sim$1.7 per axon,  $\sim$20 (high \CF) $\sim$10 (low \CF) 
(chinchilla: [\citenum{JosephsonMorest:1998}])                
  & 
0.15 ms (Shock to VCN, mice: [\citenum{ZhangOertel:1993b}])
1.2--3.5 msec (shock to AN, mice [\citenum{FerragamoGoldingEtAl:1998a,NeedhamPaolini:2003,Oertel:1983}])
Commissural DS units: 1.52 ms (shock to cCN, rat: [\citenum{NeedhamPaolini:2006}]), $6.2 \pm 4$ ms
(guinea pigs: [\citenum{BabalianJacommeEtAl:2002}])

\\ \midrule
%*****************************************
%*****************************************
DS\ensuremath{\rightarrow}DS                        
  & %type 
GlyR                   [\citenum{FerragamoGoldingEtAl:1998a}]                    
  & % time constant 
See above \DSTS
  & % RF Pre
See above \DSTS
  & % RF Post
See \ANFDS
  & % number
None observed but large Gly terminals apposed to DS cells (cat [\citenum{SmithRhode:1989}]) 
weak \IPSPs seen in slice without \DCN [\citenum{FerragamoGoldingEtAl:1998a}]                     
% & 
([\citenum{SmithRhode:1989}] cat FL vesicles shared with TV terminals: 
Soma 28\% of synapses, \TAC;
Prox 20\% of syn,% \TAC;
Dist 33\% of syn,% \TAC)
  & 
Min. synaptic delay. 
Commissural DS units: 1.52 ms (shock to cCN, rats: [\citenum{NeedhamPaolini:2006}]).
\\ \midrule
%*****************************************
%*****************************************
DS\ensuremath{\rightarrow}TV                        
  & %Type
GlyR [\citenum{DoucetRyugoEtAl:1999,OertelWuEtAl:1990,OstapoffMorestEtAl:1999,SpirouDavisEtAl:1999,ZhangOertel:1993b}]. 
  & 
See above \DSTS
%Rise 0.4 ms Decay 2.3 ms [\citenum{AwatramaniTurecekEtAl:2005}]
%6-13 msec decay time [\citenum{FerragamoGoldingEtAl:1998a,HartyManis:1996,HartyManis:1998,LeaoOleskevichEtAl:2004}]
  & 
Tonotopic on-\CF (microscopy \& physiology: [\citenum{ArnottWallaceEtAl:2004}], physiology: [\citenum{OstapoffMorestEtAl:1999,SpirouDavisEtAl:1999}]),
% Lateral sidebands equivalent to \OnC bandwidth [\citenum{OstapoffMorestEtAl:1999,SpirouDavisEtAl:1999}], 
Stronger high\ensuremath{\rightarrow}low \CF (microscopy: [\citenum{DoucetRyugoEtAl:1999,FriedlandPongstapornEtAl:2003}], 
physiology: [\citenum{ReissYoung:2005}])      
  & 
%Few somatic glycine inputs [\citenum{OsenOttersenEtAl:1990,OstapoffMorestEtAl:1999,ZhangOertel:1993b}]
% $\sim$4 somatic contacts most of which were Glycine (FP) [\citenum{Alibardi:1999}]\citenum{OstapoffMorestEtAl:1999,OstapoffFengEtAl:1994}
%33.85 \% total synaptic contact in deep layer (rat [\citenum{RubioJuiz:2004}])
% & 
few somatic ($\sim$4), mostly dendritic contacts (cat: [\citenum{OsenOttersenEtAl:1990,OstapoffMorestEtAl:1999}, mice: \citenum{ZhangOertel:1993b}], 
guinea pig: [\citenum{Alibardi:1999}]),                    
Half somatic, half dendritic (rat: [\citenum{RubioJuiz:2004}])
  & 
0.15 ms (minimum EPSP latency to VCN shock, assume IPSP latency similar, mice: [\citenum{ZhangOertel:1993b}])
\\ \midrule
%*****************************************
%*****************************************
TV\ensuremath{\rightarrow}TS                        
  & %type 
GlyR [\citenum{OertelWickesberg:1993,OstapoffMorestEtAl:1999,SaintBensonEtAl:1991,WickesbergOertel:1993}]
% Gly\slash GABA mixed (Cat [\citenum{OsenOttersenEtAl:1990}])                  
  & % time constant 
See above \DSTS
  & %Pre RF
Slightly lateral, lighter on-CF [\citenum{OstapoffMorestEtAl:1999}] 
concentrated on CF [\citenum{ZhangOertel:1993b}] 
FL vesicles more dense in high CF [\citenum{JosephsonMorest:1998}]                        
  & % Post RF
See \ANFTS for \TS dendritic field 
  & % number
Many dendritic, few somatic [\citenum{OstapoffMorestEtAl:1999,AltschulerJuizEtAl:1993}] 
%see Table 1 [\citenum{SmithRhode:1989}] 
% PL $\sim$70 (high) $\sim$60 (low CF) per soma,
% $\sim$1.7 per axon, 

% & %position 
(cat: See shared Gly synapses in \DSTS above  [\citenum{SmithRhode:1989}])
%: 
% Soma 17\% of terminals 21\% TAC,
% Prox 23\% of terminals 46\% TAC,
% Dist 27\% of terminals 22\% TAC.)
%Soma and mainly trunk [\citenum{AltschulerJuizEtAl:1993}] 
%see Table 1 [\citenum{SmithRhode:1989}] 
$\sim$20 (highCF) $\sim$10 (lowCF)                          
more FL vesicles on soma in high CF regions, some hillock contacts (75\% inhib) (chinchilla: [\citenum{JosephsonMorest:1998}])
  & %
0.1--0.3 msec glut or shock VCN [\citenum{ZhangOertel:1993b}]
0.6 ms after AN excitation in choppers [\citenum{Wickesberg:1996}]. 
2.5 msec from AN shock to inhibition [\citenum{WickesbergOertel:1993}]. 
\\ \midrule
%*****************************************
%*****************************************
TV\ensuremath{\rightarrow}DS                        
  & % Type 
Glycine GlyR receptor [\citenum{OstapoffMorestEtAl:1999,SaintBensonEtAl:1991}]. 
%Mixed  Glycine/GABA [\citenum{OsenOttersenEtAl:1990}]                 
  & %Time Constant
See above \TVTS
  & %pre RF
See above \TVTS % Slightly lateral, lighter on-CF (cat: [\citenum{OstapoffMorestEtAl:1999}]) concentrated on CF [\citenum{ZhangOertel:1993b}] 
No inhibitory sidebands evident in RA                                   
  & %post RF
Dendrites cover 1/3 CN (guinea pig [\citenum{PalmerJiangEtAl:1996}]).
  & % number position
%See table 1 [\citenum{SmithRhode:1989}]     
% & See Table 1 [\citenum{SmithRhode:1989}]    

(See \DSDS for Gly vesicles shared with DS [\citenum{SmithRhode:1989}] )
  & 
As above in \TVTS
\\ \midrule
%Granule\ensuremath{\rightarrow}GLG                     
% & NMDA  Glutamatergic receptor                 [\citenum{FerragamoGoldingEtAl:1998a}]                     
% & % Time constant
%     [\citenum{GardnerTrussellEtAl:1999}]      
% & % Pre
% & % post   
% & % number
%                   [\citenum{FerragamoGoldingEtAl:1998a}]                     
% & % Position                                           
% & % Delay 
%\\ \midrule
%*****************************************
%*****************************************
DS\ensuremath{\rightarrow}GLG                                 
  & 
Glycinergic GlyR receptor                  
  & % Time constant
  & % pre cell
Golgi dendrites spread 100 \um from the cell body [\citenum{FerragamoGoldingEtAl:1998}]
  & % post cell
D stellates have axon collaterals in GCD (mice [\citenum{OertelWuEtAl:1990}])
Gly IPSP observed in 1 of 5 cells [\citenum{FerragamoGoldingEtAl:1998}]    
  & 
  & % Delay 
Short, see Fig 3B in [\citenum{FerragamoGoldingEtAl:1998}]
\\ \midrule
%*****************************************
%*****************************************
GLG\ensuremath{\rightarrow}TS                         
  & % Type 
GABA$_{\textrm{A}}$  (mice: [\citenum{FerragamoGoldingEtAl:1998}], chinchilla: [\citenum{JosephsonMorest:1998}])
%Ferragamo et al. 1998 found no GABAergic IPSPs but the cells were still sensitive to bicuculine
  & % Time constant
$7 \pm 2$ ms  (\twd, MNTB rat neurons at 37$^\circ$C, \tfast 5--10 and \tslow 20--60 ms, rise time 0.7 ms [\citenum{AwatramaniTurecekEtAl:2005}]).  
$86 \pm 17$ ms ($\twd$, Desensitisation to high frequency trains [\citenum{AwatramaniTurecekEtAl:2005}])
  & % RF Pre
200 $\mu$m (localised in GCD, mice [\citenum{FerragamoGoldingEtAl:1998}])
  & % RF Post
Distal dendrites [\citenum{FerragamoGoldingEtAl:1998a}]
High CF\ensuremath{\rightarrow}low CF (chinchilla [\citenum{JosephsonMorest:1998}])                        
  & %Number    
%5--15 somatic and dendritic (cat PL vesicles [\citenum{SmithRhode:1989}])     

([\citenum{SmithRhode:1989}] cat PL vesicles:
Soma: 47\% of synapses, 21\% TAC; 
Prox: 34\% of syn, 46\% TAC; 
Dist: 33\% of syn, 22\% TAC)

  & %Delay
Min.\ synaptic delay plus cable delay from distal dendrites.
Somatic GABA terminals delay from superior olive $\sim$1 ms
\\ \midrule
%*****************************************
%*****************************************
GLG\ensuremath{\rightarrow}DS                         
  & % Type
{GABA$_{\textrm{A}}$}  [\citenum{EvansZhao:1998,FerragamoGoldingEtAl:1998a,Mugnaini:1985,MugnainiOsenEtAl:1980,SaintMorestEtAl:1989}]
  & % Time Constant                                            
As per GLG\ensuremath{\rightarrow}TS
% 9 ms  (MNTB rat neurons, combination of two decay time constants, fast 5-10 and slow 20-60, rise time 0.7 ms [\citenum{AwatramaniTurecekEtAl:2005}])
  & % RF Pre
As per GLG\ensuremath{\rightarrow}TS 
  & % RF Post
See \ANFDS for DS receptive field, assuming dendrites reach into GCD proximal to
their extension in the core [\citenum{OertelWuEtAl:1990,ArnottWallaceEtAl:2004}]
%Dendrites cover 1/3 CN (effectively 2 oct below, 1 oct above) (guinea pig [\citenum{PalmerJiangEtAl:1996}]).
%Wideband to wideband on CF [\citenum{EvansZhao:1998}] 
%Hyperpolarising effects above high freq edge in mid CF neurons (rat [\citenum{PaoliniClark:1999}]).                    
  & % Number  Position
$\sim$20 weak inputs[\citenum{SaintMorestEtAl:1989}]
%Strong effect to {GABA$_{\textrm{A}}$} blocker  [\citenum{FerragamoGoldingEtAl:1998a}]
Many distal  ([\citenum{SmithRhode:1989}] cat PL vesicles:
Soma 36\% of synapses, 87\% TAC; 
Prox 18\% of syn, 81\% TAC;
Dist 41\% of syn, 22\% TAC.)
  & %Delay
Hyper-polar-is-a-tion occurs 10-15 msec after click (rat [\citenum{PaoliniClark:1999}])
\\ \midrule

\DSGLG   
  & % Type
GlyR
  & % kinetics
See \DSDS 
  & % pre                                            
D stellates have collaterals in GCD fig 4 \& 5 \citenum{OertelWuEtAl:1990}  
  & % post
See Golgi dendrite spread in \ANFGLG 
  & %strength position
Gly \IPSPs observed in 1 of 5  Golgi cells \citenum{FerragamoGoldingEtAl:1998}    
  & %delay
Short, see Fig 3B in \citenum{FerragamoGoldingEtAl:1998}
\\\hline

OCB $\rightarrow$DS  
  & % Type
Acetylcholine (ACh) 
  & % Kinetics 
---                    
  & % pre
Unknown tonotopicity 
  & % post 
See \ANFDS for DS dendrite field                                           
  & % strength position 
1--2 strong synapses \citenum{MuldersPaoliniEtAl:2009,HorvathKrausEtAl:2000,MuldersPaoliniEtAl:2003}                   

  & %delay 
$\sim$5 ms \citenum{MuldersPaoliniEtAl:2009}\\

OCB\ensuremath{\rightarrow} TS                   
  & %Type
%Excitatory response to Acetylcholine \citenum{FujinoOertel:2001} 
ACh nicotinic and muscarinic  [\citenum{FujinoOertel:2001}]
%Inhibit leaky potassium currents (volt-insensitive)
%\citenum{FujinoOertel:2001} 
  & % Kinetics
---                                
  & % Pre field
Unknown tonotopicity                        
  & % post
See \ANFTS for TS dendrite field                  
  & % strength and position
SS vesicles (perhaps ACh) contact the axon hillock [\citenum{JuizHelfertEtAl:1996a}]
Distal ends in \GCD                        
  & %Delay
$\sim$5 ms [\citep{MuldersPaoliniEtAl:2009}] \\\hline

\end{longtable}





%%% Local Variables: 
%%% mode: latex
%%% mode: tex-fold
%%% TeX-master: "LiteratureReview"
%%% TeX-PDF-mode: nil
%%% End: 
