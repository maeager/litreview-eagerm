

\section{T~stellate cells}


\subsection{Background}
\label{sec:backgrTS}


\begin{itemize}
\item histology staining
\begin{itemize}
\item multipolar (Nissl staining) and stellate (Golgi impregnation) \citep{Osen:1969,BrawerMorestEtAl:1974,Lorente:1981}
\end{itemize}
\item axonal projections
\begin{itemize}
\item \textbf{T} trapezoid body or ventral acoustic stria (T stellate), and \textbf{D} dorsal acoustic stria (D stellate) \citep{OertelWuEtAl:1990}
\item T stellates project locally in VCN and deep layer of DCN -> match Chopper \citep{RhodeOertelEtAl:1983,SmithRhode:1989}
\item D stellate projections wide in VCN,DCN and cCN -> match OnC  \citep{SmithRhode:1989}
\end{itemize}
\item somatic innervation: type 1 (few somatic) type II (many somatic and dendritic) \citep{Cant:1981}
\item anatomical and physiological: \citet{SmithRhode:1989}
\begin{itemize}
\item T stellate $\rightarrow$ type 1 $\rightarrow$ chopping PSTH
\item D stellate $\rightarrow$ type 2 $\rightarrow$ OnC PSTH
\end{itemize}
\item Chopper subdivision:
\begin{itemize}
\item sustained and transient (no evidence for anatomical differences in T\~{}stellate cells)
\end{itemize}
\item dendritic differences \citep{DoucetRyugo:1997,DoucetRyugo:2006}
\begin{itemize}
\item planar (T) and radial (D)
\end{itemize}
\end{itemize}

\subsection{Acoustic Response of T stellate cells}
\label{sec-1_2}



\begin{itemize}
\item regular, tonic response to tones \citep{RhodeOertelEtAl:1983,SmithRhode:1989,BlackburnSachs:1989}
\item ``Chopping'' precise regular timing that degrades throughout stimulus\citep{YoungRobertEtAl:1988,BlackburnSachs:1989}
\begin{itemize}
\item sustained (70\%) $\rightarrow$ constant rate, ISI histogram sharp, CV < 0.3, CV constant
\item transient (30\%) $\rightarrow$ rate decreases, CV starts below 0.3 then varies
\end{itemize}
\item Inhibition
\begin{itemize}
\item Gly, GABA tuned on frequency to reduce peak excitation \citep{CasparyBackoffEtAl:1994}
\item inhibitory side bands mainly D stellate \citep{FerragamoGoldingEtAl:1998a} but periolivary also contribute \citep{AdamsWarr:1976,Adams:1983,ShoreHelfertEtAl:1991,OstapoffBensonEtAl:1997}
\end{itemize}
\end{itemize}
\citep{PalombiCaspary:1992,RhodeSmith:1986,NelkenYoung:1994,PaoliniClareyEtAl:2005,PaoliniClareyEtAl:2004}  

\begin{itemize}
\item sustained firing despite AN adaptation
\begin{itemize}
\item signals the sound intensity consistently, hence precise level information
\item Phasic also do level, but tonic suits encoding of spectrum across population since encoding the peaks and valleys is relatively independent of time after onset of sound \citep{BlackburnSachs:1990,May:2003,MayPrellEtAl:1998,MaySachs:1998}
\item suits encong of envelope of sounds, important for speech (envelops under 50 Hz \citep{ShannonZengEtAl:1995}
\end{itemize}
\item AM coding in choppers encoded over wide range of intensities \citep{RhodeGreenberg:1994,FrisinaSmithEtAl:1990}
\begin{itemize}
\item other work in Am coding by CN neurons  \citep{Moller:1972,Moller:1974a,Moller:1974,MooreCashin:1974,Frisina:1984,PalmerWinterEtAl:1986,KimRhodeEtAl:1986,WinterPalmer:1990a,Palmer:1990,PalmerWinter:1992,FrisinaSmithEtAl:1990a,Frisina:1983,GorodetskaiaBibikov:1985,RhodeGreenberg:1994,ShofnerSheftEtAl:1996,FrisinaKarcichEtAl:1996,DAngeloSterbingEtAl:2003,Aggarwal:2003}
\end{itemize}
\item phasic firing in AN maintained by bushy
\begin{itemize}
\item phasic info important: enhances formant transitions, and provides accurate information about the location of sound sources even in reverberant environments, critical in hearing \cite{DelgutteKiang:1984,DelgutteKiang:1984a,DelgutteKiang:1984b,DelgutteKiang:1984c,DelgutteKiang:1984d,DavoreIhlefeldEtAl:2009}
\end{itemize}
\end{itemize}
\subsection{Mechanisms Making Phasic Input Tonic}
\label{sec-1_3}


   steady depolarising current shows intracellular ability to be tonic \cite{Oertel:1983,OertelWuEtAl:1988} BUT - how does the input remain stable given AN adaptation?


\begin{enumerate}
\item selective processing of HSR and LSR input
\begin{enumerate}
\item feed-forward excitation in TS cells
\begin{itemize}
\item axon collaterals in local isofrequency (most cells in PVCN are TS cells)
\end{itemize}
\item co-activation of phasic inhibition
\begin{itemize}
\item DS inhibition ispi and contralaterally
\begin{itemize}
\item onset inhibition strongest, affecting TS cells after first spike
\item broad tuning sharpens FSL
\end{itemize}
\item TV sharply tuned inhibition (Ferr98)
\begin{itemize}
\item TV response variable and non-monotonic
\item \citep{Rhode:1999}  labelled TV cells phasic in aneasthetised cats
\item unanesthetised cats and gerbils are phasic or tonic  \citep{DingVoigt:1997,ShofnerYoung:1985}
\end{itemize}
\item Others - Glycine from ipsi periolivary region, GABA from both
          periolivary regions \citep{AdamsWarr:1976,ShoreHelfertEtAl:1991,OstapoffBensonEtAl:1997}
\end{itemize}
\item Absense of LT potassium in TS
\begin{itemize}
\item labelled \citep{ManisMarx:1991,BalOertel:2001,FerragamoOertel:2002,CaoShatadalEtAl:2007}
\item unlabelled \citep{RothmanManis:2003,RothmanManis:2003a,RothmanManis:2003b,Rothman:1999}
\end{itemize}
\item Activation of NMDA
\begin{itemize}
\item \citep{CaoOertel:2010} shows TS cells activate large currents through NMDA receptors
\item NMDA longer lasting, reducing phasic nature of input
\end{itemize}
\item Little synaptic depression
\begin{itemize}
\item SD less tahn bushy and octopus \citep{WuOertel:1987,ChandaXu-Friedman:2010,CaoOertel:2010}
\item excitation of TS adapts less than other VCN neurons
\end{itemize}
\end{enumerate}
\end{enumerate}
\subsection{Mechanisms of Tonic Firing Obscure Temporal Features}
\label{sec-1_4}



\begin{itemize}
\item sFSL largest in TS of core VCN units by 1msec -> onset inhibition + longer integration time \citep{GisbergenGrashuisEtAl:1975,GisbergenGrashuisEtAl:1975a,GisbergenGrashuisEtAl:1975b,YoungRobertEtAl:1988,PaoliniClareyEtAl:2004}
\item integration window longest for choppers \citep{McGinleyOertel:2006}
\item inhibition from high CF units alters FSL to tones \citep{Wickesberg:1996}
\item Onset: Volley of Excitation + feedforward excitation + DS inhibition
\item After onset: Phasic excitation + feedforward excitation + NMDA activation + TV inhibition (+ small DS inhibition) + GABA inhibition = stable excitation but loss of temporal features
\end{itemize}
\subsection{Neuromodulatory Effects in T~stellate Cells}
\label{sec-1_5}



\begin{itemize}
\item sensitive to neuromodulatory currents \citep{FujinoOertel:2001}
\begin{itemize}
\item high input resistance $\rightarrow$ amplify small current inputs \citep{FujinoOertel:2001}
\item no LKT in TS,  LKT makes bushy and optopus insensitive to steady currents \citep{OertelFujino:2001,McGinleyOertel:2006}
\item Ih higher in TS \& activated more at lower potentials than in bushy and octopus, so that it is less active at rest
\item high resistance $\rightarrow$ greater voltage changes in small modulating current $\rightarrow$ Ih can be modulated by G-protein coupled receptors, hence making TS more excitable when Ih activated \citep{RodriguesOertel:2006}
\end{itemize}
\end{itemize}

\begin{enumerate}
\item Driving inputs
\end{enumerate}
Proximal dendrites and at the soma:

\begin{itemize}
\item ANF provide glutamatergic excitation for T stellates  \citep{Cant:1981,FerragamoGoldingEtAl:1998a,Alibardi:1998a}
\begin{itemize}
\item only 5 or 6 in mice \citep{FerragamoGoldingEtAl:1998a,CaoOertel:2010}
\end{itemize}
\item Recurrent excitation from other T stellate cells \citep{FerragamoGoldingEtAl:1998a}
\item Glycine from DS cells \citep{FerragamoGoldingEtAl:1998a}
\item Glycine from TV cells \citep{WickesbergOertel:1990,ZhangOertel:1993b}
\item Neuromodulatory
\end{itemize}
     No signs of PSP or PSCs hence distal or G-protein coupled, effects on time-course minimal
     
a. Golgi cells (GABA)

\begin{itemize}
\item no IPSPs or IPSCs but presence of GABAa receptors and response changes to bicuculine \citep{WuOertel:1986,OertelWickesberg:1993,FerragamoGoldingEtAl:1998a}
\item dend filter obscures PSPs
\item Golgi cells are GABAergic and lie within the granule cell domains around the VCN and terminate near the fine distal dendrites of T stellate cells
\end{itemize}
b. Periolivary cells (GABA + GAD - glutamic acid decarboxylase) 

\begin{itemize}
\item observed in PVCN close to TS \citep{AdamsMugnaini:1987}
\item GAD effectively Glycine \citep{GoldingOertel:1997}
\item can also arise from GABAergic neurons in ipsi LNTB and DM Periolivary
\end{itemize}
c. VNTB cells (ACh)

\begin{itemize}
\item collateral branches of OC go to GCD \citep{MellottMottsEtAl:2011,SherriffHenderson:1994,OsenRoth:1969}
\item TS have nicotinic and muscarinic ACh receptors \citep{FujinoOertel:2001}
\item ACh input to TS, together with OC-cochlea, enhances spectral peaks in noise  \citep{FujinoOertel:2001}
\end{itemize}
d. NE and 5HT

\begin{itemize}
\item Raphe nuclei (5HT)
\item Locus coeruleus Peribrachial cells (NE)
\item both terminate in PVCN \citep{KlepperHerbert:1991,Thompson:2003,ThompsonLauder:2005,Thompson:2003a,ThompsonWiechmann:2002,BehrensSchofieldEtAl:2002,ThompsonThompson:2001,ThompsonThompson:2001a,ThompsonMooreEtAl:1995,ThompsonThompsonEtAl:1994}
\item both increase firing in T stellates \citep{OertelWrightEtAl:2010} in presence of glut and gly blockers -> hence act on post synapse (TS cells)
\item both G-protein coupled, both act on either pre or post synaptic cells
\item NE affects probabilty of release at calyx of Held
\item NE increases firing rate of choppers \citep{KosslVater:1989,Ebert:1996}
\item 5HT excites or inhibits choppers \emph{in vivo} \citep{EbertOstwald:1992}
\end{itemize}

\subsection{Major Ascending Output}
\label{sec-1_6}


\begin{itemize}
\item review \citep{DoucetRyugo:2006}
\end{itemize}

TS cell axons exit the CN through the trapezoidal body, cross the midline and ultimately terminate in the cIC \citep{Adams:1979}


\begin{itemize}
\item Collaterals: local, DCN, LSO cVNTB cVNLL \citep{Warr:1969,SmithJorisEtAl:1993,Thompson:1998,DoucetRyugo:2003}
\item Deep DCN (bulk of acoustic input?)
\item in mice TS terminals > ANF \citep{CaoMcGinleyEtAl:2008}
\item on CF \citep{SmithRhode:1989,FriedlandPongstapornEtAl:2003,DoucetRyugo:1997}
\item DCN review \citep{OertelYoung:2004}
\item LSO excitation
\item TS project to LSO \citep{Thompson:1998,DoucetRyugo:2003,ThompsonThompson:1991a}
\item LSO detect interaural intensity differences primarily from ipsi Bushy cells and contra MNTB (inhib)
\item OC feedback
\end{itemize}

a. MOC: cVNTB excitation 

\begin{itemize}
\item involved in efferent feedback loop, ACh-ergic MOC neurons TS synapses in cVNTB \citep{WarrBeck:1996,Warr:1992,Warr:1982,VeneciaLibermanEtAl:2005,ThompsonThompson:1991,SmithJorisEtAl:1993}
\item feedback direct to TS is positive, but efferent MOC-OHC-ANF reduces activation of ANF \citep{WarrenLiberman:1989,WiederholdKiang:1970}
\item other \citep{RobertsonMulders:2000,WinterRobertsonEtAl:1989}
\end{itemize}

b. LOC

\begin{itemize}
\item TS terminate in vicinity of LOC neurons \citep{Warr:1982,ThompsonThompson:1988,ThompsonThompson:1991,DoucetRyugo:2003}
\item feedback through LOC $\rightarrow$ cochlea $\rightarrow$ ANF loop $\rightarrow$ TS affect/regulate response of LOC. hence ANF.
\item LOC balance inputs from both ears \citep{DarrowMaisonEtAl:2006}
\end{itemize}

c. VNLL

\begin{itemize}
\item The functional consequences of these direct and indirect connections with TS cells with the IC are not well understood
\end{itemize}
%\section{TS cells in birds}
\label{sec-1_7}


VCN analog is the \emph{nucleus angularis}
%\section{Summary}
\label{sec-1_8}


{\it As a population, T stellate cells encode the spectrum of sounds. They
receive acoustic input from the auditory nerve fibers. Several
mechanisms contribute to that transformation: Feed-forward excitation
through other T stellate cells, co-activation of excitation and
inhibition, reduction in synaptic depression, and the amplification of
excitatory synaptic current over time through NMDA receptors. They
deliver that information to nuclei that make use of spectral
information.  T stellate cells terminate in the DCN, to olivocochlear
efferent neurons, to the lateral superior olive, to the contralateral
inferior colliculus. These targets use spectral information to
localize sounds, to adjust the sensitivity of the inner ear, and to
recognise and understand sounds. Birds also process sounds through
neurons that resemble T stellate cells in their projections and also
in their cellular properties, attesting to the fundamental importance
that T stellate-like cells have for hearing in vertebrates.}






%%% Local Variables: 
%%% mode: latex
%%% mode: tex-fold
%%% TeX-master: "LiteratureReview"
%%% TeX-PDF-mode: nil
%%% End: 
