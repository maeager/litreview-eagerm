\section{T~Stellate Cells}\label{sec:t-stellate-cells}

\subsection{Morphology}

\begin{itemize}
\item histology staining
\begin{itemize}
\item multipolar (Nissl staining) and stellate (Golgi impregnation) \citep{Osen:1969,BrawerMorestEtAl:1974,Lorente:1981}
\end{itemize}
\item axonal projections
\begin{itemize}
\item \textbf{T} trapezoid body or ventral acoustic stria (T stellate), and \textbf{D} dorsal acoustic stria (D stellate) \citep{OertelWuEtAl:1990}
\item T stellates project locally in VCN and deep layer of DCN -> match Chopper \citep{RhodeOertelEtAl:1983,SmithRhode:1989}
\item D stellate projections wide in VCN,DCN and cCN -> match OnC  \citep{SmithRhode:1989}
\end{itemize}
\item somatic innervation: type 1 (few somatic) type II (many somatic and dendritic) \citep{Cant:1981}
\item anatomical and physiological: \citet{SmithRhode:1989}
\begin{itemize}
\item T stellate $\rightarrow$ type 1 $\rightarrow$ chopping PSTH
\item D stellate $\rightarrow$ type 2 $\rightarrow$ OnC PSTH
\end{itemize}
\item Chopper subdivision:
\begin{itemize}
\item sustained and transient (no evidence for anatomical differences in T\~{}stellate cells)
\end{itemize}
\item dendritic differences \citep{DoucetRyugo:1997,DoucetRyugo:2006}
\begin{itemize}
\item planar (T) and radial (D)
\end{itemize}
\end{itemize}

\subsection{Synaptic Inputs}



\subsubsection{ANF to TS}

%\citep{FerragamoGoldingEtAl:1998a}
% Innervation of T stellate cells by auditory nerve fibers
% Auditory nerve fibers are of two types. Type I fibers are
% large and myelinated and comprise 95% of the total while
% type II fibers are small, unmyelinated and comprise only
% Ç5% of the total (cats: Kiang et al. 1982; mice: Ehret 1979).
% On the basis of extracellular injections of auditory nerve
% fibers in mice, type I auditory nerve fibers have been ob-
% served to terminate on both D and T stellate cells (M. W.
% Garb and D. Oertel, unpublished observations). Most neu-
% rons in the multipolar cell area of the PVCN (probably T
% stellate cells) are contacted heavily at the cell body unlike
% cells that project through the trapezoid body in cats. In cats
% type I fibers innervate all of the large cells, including those
% that correspond to T and D stellate cells (Liberman 1991,
% 1993). The anatomic findings are consistent with what is
% known about responses to activation of auditory nerve fibers
% in vivo and in vitro. The short-latency, sharply timed re-
% sponses to the onset of tones indicate that chopper and onset-
% chopper units receive input from the large, myelinated audi-
% tory nerve fibers (Blackburn and Sachs 1989; Rhode and
% Smith 1986; Smith and Rhode 1989). In slices from mice,
% both D and T stellate cells respond to shocks of the auditory
% nerve with EPSPs (Oertel et al. 1990; Wu and Oertel 1986).
% As thresholds for EPSPs are low and latencies are õ1 ms,
% the input is probably from myelinated auditory nerve fibers.
% Anatomic and electrophysiological evidence indicates that
% few auditory nerve fibers innervate a T stellate cell. The
% orientation of the dendrites of T stellate cells parallel to the
% path of auditory nerve fibers and spanning a small proportion
% of the tonotopic axis indicates that T stellate cell dendrites
% are positioned to receive input from a limited group of fibers.
% The result that the amplitude of responses to shocks of the
% auditory nerve grow in three or four discrete jumps with
% shock strength indicates that the number of fibers innervating
% one T stellate cell in a mouse is small, perhaps as small as
% three or four (Fig. 1). As any of the jumps in amplitude
% could have resulted from the recruitment of more than one
% fiber and as it is possible that inputs might have been cut or
% damaged, this estimate represents a minimum. This conclu-
% sion is in contrast with the results of similar experiments in
% octopus cells, in which such subthreshold jumps cannot be
% detected (Golding et al. 1995). This result also indicates
% that models of choppers, based on what is known in cats,
% that require the integration of many inputs might be oversim-
% plified (Banks and Sachs 1991; Molnar and Pfeiffer 1968;
% Wang and Sachs 1995).
% It is intriguing that the NMDA-receptor–mediated slow
% depolarizations were generated with shock strengths greater
% than those required to produce apparently maximal mono-
% synaptic EPSPs. This finding suggests that different sources
% of glutamatergic input may activate different populations of
% receptors. It raises the possibility that type I auditory nerve
% fibers act primarily through AMPA receptors, as they are
% known to do in other vertebrate cochlear nuclei (Raman et
% al. 1994; Zhang and Trussell 1994) whereas other sources of
% excitation, alone or in combination, are required to activate
% NMDA receptors. It is conceivable that type II auditory
% nerve fibers contribute to the long, slow depolarization.
% Small, unmyelinated fibers would be expected to have higher
% thresholds for shocks than larger, myelinated fibers and their
% responses would be expected to be later.

Auditory nerve fibre synapses on TS cells are glutamatergic AMPA receptors \citep{FerragamoGoldingEtAl:1998a,WentholdHunterEtAl:1993}.
Histological measures of labelled T stellate cells show the presence of glutamate and glutamine antibodies \citep{HackneyOsenEtAl:1990,WentholdHunterEtAl:1993}. 
More advanced measures using electron microscopy reveal AMPA subunits unique to the cochlear nucleus \citep{WangWentholdEtAl:1998}.
Pharmacologic experiments have also comfirmed monosynaptic EPSPs from AN shocks are be blocked by the AMPA antagonist DNQX \citep{FerragamoGoldingEtAl:1998a}.
Glutamatergic NMDA receptors may also be present in young mice synapses \citep{FerragamoGoldingEtAl:1998a}, however whole cell patch recordings show NMDA dominance rescinds to AMPA during development \citep{BellinghamLimEtAl:1998}.
The time course of EPSPs were measured in mice slice preparations at 34 degrees C with a time constant of 0.36 ms, and a fast rise time of 0.15 ms  \citep{GardnerTrussellEtAl:2001,Gardner:2000,GardnerTrussellEtAl:1999}.
%\citep{Oertel:1983}

The estimated receptive field of single ANFs ($\sim$70$\mu$m HSR, 100$\mu$m LSR  \citep{OertelWuEtAl:1990,Ryugo:2008,MeltzerRyugo:2006,RyugoParks:2003,Ryugo:1992,BrownBerglundEtAl:1988,RoullierCronin-SchreiberEtAl:1986,FeketeRouillerEtAl:1984}) closely matches the dendritic width of TS cells perpendicular to the incoming ANF axons  (75-100$\mu$m \citep[Mouse]{OertelWuEtAl:1990}). 
% 0.23-0.39 oct \citep[anesthetized guinea pig][]{PalmerJiangEtAl:1996} 
The physiological receptive field is also similar between ANFs and TS cells (Q$_{10}=5.52\pm1.4$  \citep[guinea pig]{JiangPalmerEtAl:1996} compared to Q$_{10}$ of 6.3 in ANFs) but varies with different TS cell classification subtypes (CS  4, CT  2 (low CF), 3.67 (high CF)  \citep[guinea pig]{PalmerWallaceEtAl:2003}) and anesthetic in study (%
% Q\textsubscript{10} 5.3  \citep[cat][]{RhodeSmith:1986}.
Q$_{10}$ 7.4 (unanesthetised) 5.3 (barbiturate) \citep[cat][]{RhodeKettner:1987}).                      

 % Number 
Small number \citep{Cant:1981,FayPopper:1994,ReddCahillEtAl:2002,RyugoWrigthEtAl:1993,Ryugo:1992,RyugoParks:2003}
Estimate of 5 synapses to reach AP \citep[Mouse][]{FerragamoGoldingEtAl:1998a}. 
Highly variable, 0-6 per soma possibly two subtypes \citep[Chinchilla][]{JosephsonMorest:1998}.

 % Position
Mainly dendritic, $\sim$30\% soma coverage \citep[Cat][]{Cant:1981,Cant:1982,RyugoWrightEtAl:1993,TolbertMorest:1982a}. 
\citep[cat][Soma: 36$\pm$10.5 \%  of 21 (range 6-38) \% area coverage, Prox: 43$\pm$29 \%  of 46 \% area cov., Distal: 40$\pm$12 \%  of 22 \% area cov.]{SmithRhode:1989}
Some cells had synapses surrounding the axon initial segment \citep{JosephsonMorest:1998}. 
Type II ANFs diffuse synapses on distal dendrites \citep{BensonBrown:2004}.
The theoretical conductance delay from the cochlea to the position of TS cells in the VCN, based on the average distance and myelinated axon width, was estimated to be 0.5 ms  \citep{Brown:1993,BrownLedwith:1990}. 
\citet{Oertel:1983} first calculated the delay experimentally using electrical shocks to the auditory nerve root in mouse slice preparations to be 0.7 ms \citep{Oertel:1983}. 
This was later confirmed in more studies (mean 0.7 ms, range 0.48-0.92 ms) \citep[Mice][]{FerragamoGoldingEtAl:1998a}, but may vary with different animals  \citep[0.5 ms in chinchilla][]{WickesbergOertel:1993}. 


\subsubsection{DS to TS}
% Smith and Rhode (1989) provided anatomical evidence that the OCs synapse on the
% type 1 multipolar cells in the cat PVCN. In the rodent,
% Ferragamo et al. (1998) showed that the D stellates, be-
% lieved to be the rodent equivalent of OCs, provide glycin-
% ergic inhibition to the type 1 multipolars. Pressnitzer et al.
% (2001) reported that transient choppers in the AVCN,
% believed to be type 1 multipolars, receive a wide-band
% inhibition that may arise from OCs. OC projections to the
% DCN have also been implicated in the wide-band glycin-
% ergic inhibition demonstrated physiologically in DCN
% principal cells, as well as in the cells designated type II or
% vertical (Caspary et al., 1987; Young et al., 1992; Nelken
% and Young, 1994; Backoff et al., 1997; Joris and Smith,
% 1998; Spirou et al., 1999; Davis and Young, 2000; Ander-
% son and Young, 2004). Our electron microscopy of OC
% terminals in the deep DCN shows that they can synapse
% on cell bodies or dendrites in this region, which is consis-
% tent with a potential influence of the OCs on these cell
% types.



Sources of glycinergic cochlear nuclear inhibition
Glycinergic inhibition is recorded consistently in T stellate
cells spontaneously and in responses to shocks of the audi-
tory nerve as prominent, rapid IPSPs. The latencies of IPSPs
indicate that they are polysynaptic and arise through inter-
neurons that are intrinsic to the slice. All distinct IPSPs in
T stellate cells, as in other cells of the VCN, are blocked
by strychnine, indicating that they are glycinergic (Wu and
Oertel 1986).
An ability to label glycinergic interneurons with antibod-
ies to glycine conjugates allows the population of glycinergic
neurons to be identified (Oertel and Wickesberg 1993).
Three groups of cells account for immunopositive labeling:
in the DCN, tuberculoventral cells (Osen et al. 1990; Saint
Marie et al. 1991; Wenthold et al. 1987; Wickesberg et al.
1994) and cartwheel cells (Osen et al. 1990; Saint Marie et
al. 1991; Wenthold et al. 1987), and in the VCN, multipolar
cells (Schofield and Cant 1996; Wenthold 1987), which
correspond to D stellate cells (Oertel et al. 1990).
Tuberculoventral cells have been shown to provide disyn-
aptic, glycinergic inhibition to T stellate cells in responses
to shocks of the auditory nerve (Wickesberg and Oertel
1990). Although there is no doubt that tuberculoventral cells
contribute to the disynaptic IPSPs, several experimental

\subsubsection{TS to TS}
% Sources of polysynaptic excitation \citep{FerragamoGoldingEtAl:1998a}
% The late EPSPs observed in T stellate cells indicate that
% T stellate cells receive excitatory input from excitatory inter-
% neurons in the slices. In being separated from their natural
% synaptic inputs, isolated axons cannot contribute to polysyn-
% aptic responses. Monosynaptic responses have latencies be-
% tween 0.5 (synaptic delay) and Ç3 ms (2.5-ms conduction
% delay for an unmyelinated fiber of 0.5-mm plus 0.5-ms syn-
% aptic delay). Therefore EPSPs the latencies of which are
% ú3 ms are polysynaptic and must be generated by excitatory
% interneurons. Two other experimental observations confirm
% this conclusion. As cut axons have not been observed to fire
% spontaneously, the presence of spontaneous EPSPs is an
% indication of the existence of excitatory interneurons. Fur-
% thermore, the activation of EPSPs with the application of
% glutamate indicates that the dendrites of excitatory interneu-
% rons are accessible from the bath.
% T stellate cells are excitatory neurons known to terminate
% in the vicinity of T stellate cells. T stellate cells terminate
% locally in the multipolar cell area of the PVCN (Oertel et
% al. 1990; this study). This area is occupied by T stellate
% cells and occasional D stellate and bushy cells, some or all
% of which are therefore presumably their targets. The ultra-
% structure of T stellate cell terminals and functional studies
% of the inputs to the inferior colliculi is consistent with their
% being excitatory (Oliver 1984, 1987; Smith and Rhode
% 1989).
% The present experiments provide functional evidence in
% support of the conclusion that T stellate cells mediate late
% EPSPs. If T stellate cells are excited by other T stellate cells,
% then disynaptic EPSPs that reflect the firing of other stellate
% cells should be observed under similar conditions as stellate
% cell firing. The present experiments reflect the parallel nature
% of T stellate cell firing and late EPSPs under five experimen-
% tal conditions. 1) Stellate cells consistently are brought to
% threshold Ç1 ms after shocks to the auditory nerve. Disynap-
% tic EPSPs with latencies of Ç1.6 ms are observed but in the
% presence of monosynaptic EPSPs and disynaptic IPSPs the
% early disynaptic EPSPs are sometimes difficult to resolve.
% 2) Strong shocks evoke a long, slow depolarization in T
% stellate cells that causes T stellate cells to fire hundreds of
% milliseconds after a strong shock to the auditory nerve.
% Strong shocks also evoke very late EPSPs in T stellate cells.
% 3) APV reduces late firing and late EPSPs in T stellate cells.
% 4) The removal of extracellular Mg 2/ enhances firing as
% well as late EPSPs. 5) Strychnine and bicuculline enhance
% firing as well as late EPSPs in T stellate cells. In summary,
% although the results of the present experiments are consistent
% with the conclusion that T stellate cells excite one another,
% it does not rule out the possibility that other, hitherto un-
% known, cells contribute to the excitation.
% The only other known excitatory neurons that terminate
% in the vicinity of T stellate cells are granule cells. The den-
% drites of T stellate cells end in bushy branches, some of
% which often come near, but never penetrate, the layer of
% superficial granule cells that overlies them. It is conceivable,
% therefore, that granule cells could provide polysynaptic exci-
% tation.
% findings show that they do not mediate the long trains of
% IPSPs. First, tuberculoventral cells do not fire for prolonged
% periods when activated through eighth nerve inputs (Golding
% and Oertel 1997; Zhang and Oertel 1993). Second, long
% trains of IPSPs are preserved in slices in which the DCN
% was removed from the slice (Fig. 7).
% Considerable experimental evidence indicates that D stel-
% late cells are the source of the trains of IPSPs. First, it is the
% only class of glycine-immunopositive neurons in the VCN.
% Furthermore, pharmacological manipulations produce paral-
% lel changes in the firing of D stellate cells and the appearance
% of IPSPs in T stellate cells. 1) Late IPSPs in T stellate cells
% were evoked by strong shocks that lasted for hundreds of
% milliseconds. D stellate cells fire for long periods in re-
% sponses to strong shocks. 2) Both the trains of IPSPs of T
% stellate cells and the late firing of D stellate cells were
% blocked by APV. 3) Both the trains of IPSPs in T stellate
% cells and late firing of D stellate cells were promoted by
% application of GABAA antagonists. The results that D stellate
% cells contact T stellate cells and that they respond to weak
% shocks with single spikes monosynaptically indicate that
% they contribute to the disynaptic IPSP.


%  3.6$\pm$0.38 ms \citep{RhodeSmith:1986} 
%  3.6$\pm$1.2 ms \citep[anesthetised cat][]{RhodeKettner:1987} 
% 3.5 ms \citep[unanesthetised cat][]{RhodeKettner:1987} 
% Acoustic threshold
%  depolarisation min SPL 31.7$\pm$2.9, 
%  AP min SPL 41.8$\pm$3.8  \citep{PaoliniClareyEtAl:2004}




\subsection{Acoustic Response of T stellate cells}

\begin{itemize}
\item regular, tonic response to tones \citep{RhodeOertelEtAl:1983,SmithRhode:1989,BlackburnSachs:1989}
\item ``Chopping'' precise regular timing that degrades throughout stimulus\citep{YoungRobertEtAl:1988,BlackburnSachs:1989}
\begin{itemize}
\item sustained (70\%) $\rightarrow$ constant rate, ISI histogram sharp, CV < 0.3, CV constant
\item transient (30\%) $\rightarrow$ rate decreases, CV starts below 0.3 then varies
\end{itemize}
\item Inhibition
\begin{itemize}
\item Gly, GABA tuned on frequency to reduce peak excitation \citep{CasparyBackoffEtAl:1994}
\item inhibitory side bands mainly D stellate \citep{FerragamoGoldingEtAl:1998a} but periolivary also contribute \citep{AdamsWarr:1976,Adams:1983,ShoreHelfertEtAl:1991,OstapoffBensonEtAl:1997}
\end{itemize}
\end{itemize}
\citep{PalombiCaspary:1992,RhodeSmith:1986,NelkenYoung:1994,PaoliniClareyEtAl:2005,PaoliniClareyEtAl:2004}  

\begin{itemize}
\item sustained firing despite AN adaptation
\begin{itemize}
\item signals the sound intensity consistently, hence precise level information
\item Phasic also do level, but tonic suits encoding of spectrum across population since encoding the peaks and valleys is relatively independent of time after onset of sound \citep{BlackburnSachs:1990,May:2003,MayPrellEtAl:1998,MaySachs:1998}
\item suits encong of envelope of sounds, important for speech (envelops under 50 Hz \citep{ShannonZengEtAl:1995}
\end{itemize}
\item AM coding in choppers encoded over wide range of intensities \citep{RhodeGreenberg:1994,FrisinaSmithEtAl:1990}
\begin{itemize}
\item other work in Am coding by CN neurons  \citep{Moller:1972,Moller:1974a,Moller:1974,MooreCashin:1974,Frisina:1984,PalmerWinterEtAl:1986,KimRhodeEtAl:1986,WinterPalmer:1990a,Palmer:1990,PalmerWinter:1992,FrisinaSmithEtAl:1990a,Frisina:1983,GorodetskaiaBibikov:1985,RhodeGreenberg:1994,ShofnerSheftEtAl:1996,FrisinaKarcichEtAl:1996,DAngeloSterbingEtAl:2003,Aggarwal:2003}
\end{itemize}
\item phasic firing in AN maintained by bushy
\begin{itemize}
\item phasic info important: enhances formant transitions, and provides accurate information about the location of sound sources even in reverberant environments, critical in hearing \cite{DelgutteKiang:1984,DelgutteKiang:1984a,DelgutteKiang:1984b,DelgutteKiang:1984c,DelgutteKiang:1984d,DavoreIhlefeldEtAl:2009}
\end{itemize}
\end{itemize}
\subsection{Mechanisms Making Phasic Input Tonic}
\label{sec-1_3}


steady depolarising current shows intracellular ability to be tonic \cite{Oertel:1983,OertelWuEtAl:1988} BUT - how does the input remain stable given AN adaptation?


\begin{enumerate}
\item selective processing of HSR and LSR input
\begin{enumerate}
\item feed-forward excitation in TS cells
\begin{itemize}
\item axon collaterals in local isofrequency (most cells in PVCN are TS cells)
\end{itemize}
\item co-activation of phasic inhibition
\begin{itemize}
\item DS inhibition ispi and contralaterally
\begin{itemize}
\item onset inhibition strongest, affecting TS cells after first spike
\item broad tuning sharpens FSL
\end{itemize}
\item TV sharply tuned inhibition (Ferr98)
\begin{itemize}
\item TV response variable and non-monotonic
\item \citep{Rhode:1999}  labelled TV cells phasic in aneasthetised cats
\item unanesthetised cats and gerbils are phasic or tonic  \citep{DingVoigt:1997,ShofnerYoung:1985}
\end{itemize}
\item Others - Glycine from ipsi periolivary region, GABA from both
          periolivary regions \citep{AdamsWarr:1976,ShoreHelfertEtAl:1991,OstapoffBensonEtAl:1997}
\end{itemize}
\item Absense of LT potassium in TS
\begin{itemize}
\item labelled \citep{ManisMarx:1991,BalOertel:2001,FerragamoOertel:2002,CaoShatadalEtAl:2007}
\item unlabelled \citep{RothmanManis:2003,RothmanManis:2003a,RothmanManis:2003b,Rothman:1999}
\end{itemize}
\item Activation of NMDA
\begin{itemize}
\item \citep{CaoOertel:2010} shows TS cells activate large currents through NMDA receptors
\item NMDA longer lasting, reducing phasic nature of input
\end{itemize}
\item Little synaptic depression
\begin{itemize}
\item SD less tahn bushy and octopus \citep{WuOertel:1987,ChandaXu-Friedman:2010,CaoOertel:2010}
\item excitation of TS adapts less than other VCN neurons
\end{itemize}
\end{enumerate}
\end{enumerate}

\subsection{Mechanisms of Tonic Firing Obscure Temporal Features}


\begin{itemize}
\item sFSL largest in TS of core VCN units by 1msec -> onset inhibition + longer integration time \citep{GisbergenGrashuisEtAl:1975,GisbergenGrashuisEtAl:1975a,GisbergenGrashuisEtAl:1975b,YoungRobertEtAl:1988,PaoliniClareyEtAl:2004}
\item integration window longest for choppers \citep{McGinleyOertel:2006}
\item inhibition from high CF units alters FSL to tones \citep{Wickesberg:1996}
\item Onset: Volley of Excitation + feedforward excitation + DS inhibition
\item After onset: Phasic excitation + feedforward excitation + NMDA activation + TV inhibition (+ small DS inhibition) + GABA inhibition = stable excitation but loss of temporal features
\end{itemize}

\citep{JorisSmithEtAl:1994}



\subsection{Neuromodulatory Effects in T~stellate Cells}

\begin{itemize}
\item sensitive to neuromodulatory currents \citep{FujinoOertel:2001}
\begin{itemize}
\item high input resistance $\rightarrow$ amplify small current inputs \citep{FujinoOertel:2001}
\item no LKT in TS,  LKT makes bushy and optopus insensitive to steady currents \citep{OertelFujino:2001,McGinleyOertel:2006}
\item Ih higher in TS \& activated more at lower potentials than in bushy and octopus, so that it is less active at rest
\item high resistance $\rightarrow$ greater voltage changes in small modulating current $\rightarrow$ Ih can be modulated by G-protein coupled receptors, hence making TS more excitable when Ih activated \citep{RodriguesOertel:2006}
\end{itemize}
\end{itemize}

\begin{enumerate}
\item Driving inputs
\end{enumerate}
Proximal dendrites and at the soma:

\begin{itemize}
\item ANF provide glutamatergic excitation for T stellates  \citep{Cant:1981,FerragamoGoldingEtAl:1998a,Alibardi:1998a}
\begin{itemize}
\item only 5 or 6 in mice \citep{FerragamoGoldingEtAl:1998a,CaoOertel:2010}
\end{itemize}
\item Recurrent excitation from other T stellate cells \citep{FerragamoGoldingEtAl:1998a}
\item Glycine from DS cells \citep{FerragamoGoldingEtAl:1998a}
\item Glycine from TV cells \citep{WickesbergOertel:1990,ZhangOertel:1993b}
\item Neuromodulatory
\end{itemize}
     No signs of PSP or PSCs hence distal or G-protein coupled, effects on time-course minimal
     
\subsection{GABA$_A$ergic influence}
% Markers of GABAergic neurotransmission in the cochlear
% nucleus reveal the presence of both cell bodies and terminals
% that could be GABAergic. Antibodies to GABA conjugates
% and to glutamate decarboxylase (GAD) generally label neu-
% rons that are functionally GABAergic. Occasionally GAD
% and GABA are associated with neurons that are functionally
% glycinergic; cartwheel cells of the DCN, for example, are
% labeled for GABA and GAD yet seem to be glycinergic
% (Golding and Oertel 1997; Golding et al. 1996). Function-
% ally GABAergic neurons and their terminals are labeled con-
% sistently for GABA and GAD, however, indicating that the
% source of GABAergic input in T stellate cells would be
% expected to be labeled. GABAergic input could arise from
% neurons intrinsic to the cochlear nuclei or from sites external
% to the nucleus, such as the superior olivary nucleus (Saint
% Marie et al. 1989). Only GABAergic neurons in the cochlear
% nuclei can function in polysynaptic circuits in slices as was
% observed in the present study, however, isolated terminals
% of extrinsic sources cannot be activated synaptically.
% Labeling for GAD and GABA is associated strongly with
% regions that contain granule cells, the molecular and fusiform
% cell layers of the DCN and the superficial granule cell do-
% main of the VCN. In cats and guinea pigs, antibodies to
% GABA conjugates and to GAD, a biosynthetic enzyme, have
% been shown to label specific groups of cells and terminals
% (GABA: Kolston et al. 1992; Osen et al. 1990; Wenthold et
% al. 1986; GAD: Adams and Mugnaini 1987; Moore and
% Moore 1987; Mugnaini 1985; Saint Marie et al. 1989). In
% the DCN, the majority of cell bodies and puncta that were
% labeled with antibodies against GABA and GAD lie in the
% superficial and fusiform cell layers (Adams and Mugnaini
% 1987; Kolston et al. 1992; Moore and Moore 1987; Mugnaini
% 1985; Osen et al. 1990; Saint Marie et al. 1989; Wenthold
% et al. 1986). Labeled neurons are cartwheel, stellate, and
% Golgi cells. As none of these neurons make direct or indirect
% connections with the VCN, it is unlikely that cartwheel,
% superficial stellate or Golgi cells of the DCN contribute to
% GABAergic inhibition in T stellate cells of the VCN.
% GABAergic input to T stellate cells of the VCN could
% arise from Golgi cells in the superficial granule cell domain
% either mono- or disynaptically. Labeled cell bodies identified
% as Golgi cells were observed to be associated with the super-
% ficial granule cell layer (Mugnaini 1985). These neurons
% terminate locally in the superficial granule cell layer with
% very dense terminal arbors that abut the underlying large
% cell area (Ferragamo et al. 1997). The dendrites of D stellate
% cells lie just beneath the superficial granule cell domain,
% poised to be contacted by Golgi cells proximally and distally,
% indicating that D stellate cells could mediate GABAergic
% responses. Furthermore, some of the branches of the distal
% dendrites of T stellate cells approach the superficial granule
% cell domain. If Golgi cells contact T stellate cells directly,
% those contacts can only be on distal dendrites. In contrast
% with glycinergic IPSPs, GABAergic IPSPs were not promi-
% nent in T or D stellate cells; IPSPs that remained in the
% presence of strychnine were small and inconspicuous, if
% present. There are four possible reasons for this observation:
% the synaptic currents associated with GABAergic inputs
% were relatively slower and weaker, they were generated rela-
% tively far from the somatic recording site, they were medi-
% ated through an excitatory interneuron, or there were presyn-
% aptic GABAergic receptors present.

a. Golgi cells (GABA)

\begin{itemize}
\item no IPSPs or IPSCs but presence of GABAa receptors and response changes to bicuculine \citep{WuOertel:1986,OertelWickesberg:1993,FerragamoGoldingEtAl:1998a}
\item dend filter obscures PSPs
\item Golgi cells are GABAergic and lie within the granule cell domains around the VCN and terminate near the fine distal dendrites of T stellate cells
\end{itemize}

b. Periolivary cells (GABA + GAD - glutamic acid decarboxylase) 

\begin{itemize}
\item observed in PVCN close to TS \citep{AdamsMugnaini:1987}
\item GAD effectively Glycine \citep{GoldingOertel:1997}
\item can also arise from GABAergic neurons in ipsi LNTB and DM Periolivary
\end{itemize}
c. VNTB cells (ACh)

\begin{itemize}
\item collateral branches of OC go to GCD \citep{MellottMottsEtAl:2011,SherriffHenderson:1994,OsenRoth:1969}
\item TS have nicotinic and muscarinic ACh receptors \citep{FujinoOertel:2001}
\item ACh input to TS, together with OC-cochlea, enhances spectral peaks in noise  \citep{FujinoOertel:2001}
\end{itemize}
d. NE and 5HT

\begin{itemize}
\item Raphe nuclei (5HT)
\item Locus coeruleus Peribrachial cells (NE)
\item both terminate in PVCN \citep{KlepperHerbert:1991,Thompson:2003,ThompsonLauder:2005,Thompson:2003a,ThompsonWiechmann:2002,BehrensSchofieldEtAl:2002,ThompsonThompson:2001,ThompsonThompson:2001a,ThompsonMooreEtAl:1995,ThompsonThompsonEtAl:1994}
\item both increase firing in T stellates \citep{OertelWrightEtAl:2010} in presence of glut and gly blockers -> hence act on post synapse (TS cells)
\item both G-protein coupled, both act on either pre or post synaptic cells
\item NE affects probabilty of release at calyx of Held
\item NE increases firing rate of choppers \citep{KosslVater:1989,Ebert:1996}
\item 5HT excites or inhibits choppers \emph{in vivo} \citep{EbertOstwald:1992}
\end{itemize}

\subsection{Major Ascending Output}
\label{sec-1_6}


\begin{itemize}
\item review \citep{DoucetRyugo:2006}
\end{itemize}

TS cell axons exit the CN through the trapezoidal body, cross the midline and ultimately terminate in the cIC \citep{Adams:1979}


\begin{itemize}
\item Collaterals: local, DCN, LSO cVNTB cVNLL \citep{Warr:1969,SmithJorisEtAl:1993,Thompson:1998,DoucetRyugo:2003}
\item Deep DCN (bulk of acoustic input?)
\item in mice TS terminals > ANF \citep{CaoMcGinleyEtAl:2008}
\item on CF \citep{SmithRhode:1989,FriedlandPongstapornEtAl:2003,DoucetRyugo:1997}
\item DCN review \citep{OertelYoung:2004}
\item LSO excitation
\item TS project to LSO \citep{Thompson:1998,DoucetRyugo:2003,ThompsonThompson:1991a}
\item LSO detect interaural intensity differences primarily from ipsi Bushy cells and contra MNTB (inhib)
\item OC feedback
\end{itemize}

a. MOC: cVNTB excitation 

\begin{itemize}
\item involved in efferent feedback loop, ACh-ergic MOC neurons TS synapses in cVNTB \citep{WarrBeck:1996,Warr:1992,Warr:1982,VeneciaLibermanEtAl:2005,ThompsonThompson:1991,SmithJorisEtAl:1993}
\item feedback direct to TS is positive, but efferent MOC-OHC-ANF reduces activation of ANF \citep{WarrenLiberman:1989,WiederholdKiang:1970}
\item other \citep{RobertsonMulders:2000,WinterRobertsonEtAl:1989}
\end{itemize}

b. LOC

\begin{itemize}
\item TS terminate in vicinity of LOC neurons \citep{Warr:1982,ThompsonThompson:1988,ThompsonThompson:1991,DoucetRyugo:2003}
\item feedback through LOC $\rightarrow$ cochlea $\rightarrow$ ANF loop $\rightarrow$ TS affect/regulate response of LOC. hence ANF.
\item LOC balance inputs from both ears \citep{DarrowMaisonEtAl:2006}
\end{itemize}

c. VNLL

\begin{itemize}
\item The functional consequences of these direct and indirect connections with TS cells with the IC are not well understood
\end{itemize}
%\section{TS cells in birds}
\label{sec-1_7}


VCN analog is the \emph{nucleus angularis}
%\section{Summary}
\label{sec-1_8}


{\it As a population, T stellate cells encode the spectrum of sounds. They
receive acoustic input from the auditory nerve fibers. Several
mechanisms contribute to that transformation: Feed-forward excitation
through other T stellate cells, co-activation of excitation and
inhibition, reduction in synaptic depression, and the amplification of
excitatory synaptic current over time through NMDA receptors. They
deliver that information to nuclei that make use of spectral
information.  T stellate cells terminate in the DCN, to olivocochlear
efferent neurons, to the lateral superior olive, to the contralateral
inferior colliculus. These targets use spectral information to
localize sounds, to adjust the sensitivity of the inner ear, and to
recognise and understand sounds. Birds also process sounds through
neurons that resemble T stellate cells in their projections and also
in their cellular properties, attesting to the fundamental importance
that T stellate-like cells have for hearing in vertebrates.}






%%% Local Variables: 
%%% mode: latex
%%% mode: tex-fold
%%% TeX-master: "LiteratureReview"
%%% TeX-PDF-mode: nil
%%% End: 
