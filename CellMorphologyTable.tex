\begin{longtable}{cXXXX}%
%
\caption{Cell Morphology in the VCN}\label{tab:Connections} \\
\toprule  
Cells & \multicolumn{2}{X}{Cell body} & Dendrites & Axons \\
 & Cross-section (\umsq) & Diameter (\um) &  & \\ \midrule 
\endfirsthead

\multicolumn{5}{c}{{\bfseries \tablename\ \thetable{} -- continued from previous page}} \\
\midrule Cells & X-section Area & Diameter & Dendrites &Axons \\ \midrule 
\endhead

\midrule \multicolumn{5}{c}{{Continued on next page}} \\ %\midrule
\endfoot
\bottomrule
\endlastfoot

TS                                & 
262.3{\textpm}62 \citep[rat][]{DoucetRyugoEtAl:1999} 
249{\textpm}95  \citep[rat][]{DoucetRyugo:1997}
266{\textpm}156 \citep[rat][]{DoucetRyugo:2006}
648 \citep[cat][]{SmithRhode:1989} 
366{\textpm}100  \citep[cat][]{ReddCahillEtAl:2002} 
200 (CS), 240 (CT) \citep[guinea pig][]{PalmerWallaceEtAl:2003} 
192{\textpm}47 (high CF), 253{\textpm}63 (low CF) \citep[chinchilla][]{JosephsonMorest:1998}
& 
16--20\um  \citep[rat][]{DoucetRyugo:1997,DoucetRyugoEtAl:1999,DoucetRyugo:2006} 
14.5{\textpm}2.0 (high CF)  17.6{\textpm}2.7 (low CF) \citep[chinchilla][]{JosephsonMorest:1998} 
& 
2-3 main dend, total dend length $\sim$2000\um \citep[guinea pig][]{PalmerWallaceEtAl:2003} 
75-100\um and 150-300\um parallel,3--4 primary dendrites \citep[cat][]{SmithRhode:1989}
200 \um parallel, 2--5 \um in diam  \citep[chinchilla][]{JosephsonMorest:1998}  
& 
axon width 0.5--1.1 \um \citep{OertelWuEtAl:1990} 
hillock initial segment averaged 2.5{\textpm}0.96 \um, 
proximal diam and up to 13 \um long, distal initial segment 0.7{\textpm}0.2 \um,
myelination began about 20 \um from the soma \citep[chinchilla[][]JosephsonMorest:1998}\\\hline
TV                                & 
& 
18.4{\textpm}3.2   \citep[guinea pig][]{SaintBensonEtAl:1991} 
13.34{\textpm}2.17  \citep[guinea pig][]{Alibardi:1999}
& 
Dend and axonal arbors restricted to isofrequency lamina                      & 
\\\hline
DS                                & 
963  \citep[cat][]{SmithRhode:1989} 
501{\textpm}168  \citep[rat][]{DoucetRyugoEtAl:1999}
466{\textpm}137 \citep[rat][]{DoucetRyugo:1997} 
418{\textpm}140 \citep[rat][]{DoucetRyugo:2006}
571{\textpm}228  \citep[cat][]{ReddCahillEtAl:2002} 
450 \citep[guinea pig][]{PalmerWallaceEtAl:2003}                  
&  
27  \citep[guinea pig][]{ArnottWallaceEtAl:2004} 
22--26  \citep[rat][]{DoucetRyugo:1997} 
20--30  \citep[rat][]{PaoliniClark:1999}                   
& 
250-350 \um span of dend \citep[rat][]{DoucetRyugo:1997} with aspinous dendrites, 
4 of 5 cells had 4 main dendrites, total dend length 6222 to 7351 \um
(mean-6665 \um), dendrites extended widely in all directions,
\~{}70 \um perpendicular to AN,3-6 primary dendrites at right angles to AN        \citep{SmithRhode:1989}
& axon width 0.7-1.2 \um \citep{OertelWuEtAl:1990}\\\hline
Golgi                               & 
&
15 \um \citep[mouse][]{FerragamoGoldingEtAl:1998}                     
& 
Smooth, tapering dendrites, between 50 and 100 \um long, emanated in all directions \citep{FerragamoGoldingEtAl:1998}
Also see \citep{Cant:1993,MugnainiOsenEtAl:1980}         & 
A dense, axonal plexus, limited to the plane of the granule cell domain, extended about 250 \um
from the soma in all directions \citep{FerragamoGoldingEtAl:1998} \\
\end{longtable}



%%% Local Variables: 
%%% mode: latex
%%% mode: tex-fold
%%% TeX-master: "LiteratureReview"
%%% TeX-PDF-mode: nil
%%% End: 
