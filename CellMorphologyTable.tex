\begin{longtable}{cXXXX}%
%
\caption{Cell Morphology in the VCN}\label{tab:Connections} \\
\toprule  
Cells & Cell body xsection area & Soma diameter & Dendrite span &Other \\ \midrule 
\endfirsthead

\multicolumn{5}{c}{{\bfseries \tablename\ \thetable{} -- continued from previous page}} \\
\midrule Cells & Cell body xsection area & Soma diameter & Dendrite span &Other \\ \midrule 
\endhead

\midrule \multicolumn{5}{c}{{Continued on next page}} \\ %\midrule
\endfoot
\bottomrule
\endlastfoot

                               TS                                & soma area 262.3{\textpm}62 planar cell in rat
\citep{DoucetRyugoEtAl:1999} area 249{\textpm}95\umsq \citep{DoucetRyugo:1997}
CS 648 \umsq \citep{SmithRhode:1989} cat 366{\textpm}100\umsq
\citep{ReddCahillEtAl:2002} soma xsectional area 200\umsq (CS), 240 (CT)
\citep{PalmerWallaceEtAl:2003} type I stellate (IC label) chinchilla DAP
(high CF) 192{\textpm}47 $\mu $m2,PV (low CF) 253{\textpm}63 $\mu $m2
                  \citep{JosephsonMorest:1998}                   & 16-20\um diam \citep{DoucetRyugo:1997} type I
stellate (IC label) chinchilla DAP (high CF) 14.5{\textpm}2.0 $\mu $m,
PV(low CF) 17.6{\textpm}2.7 $\mu $m \citep{JosephsonMorest:1998} & 2-3 main
dend, total dend length \~{}2000\um \citep{PalmerWallaceEtAl:2003} 75-100\um
and 150-300\um parallel,3-4 primary dendrites \citep{SmithRhode:1989}
dendrites emanated from soma in one direction (rarely bipolar) up to 200$\mu
$m, 2-5 $\mu $m in diam chinchilla \citep{JosephsonMorest:1998}  & axon width
.5-1.1\um \citep{OertelWuEtAl:1990} hillock initial segment averaged
2.5{\textpm}0.96 $\mu $m proximal diam and up to 13 $\mu $m long, distal
initial segment 0.7{\textpm}0.2 $\mu $m,myelination began about 20 $\mu $m
from the soma \citep{JosephsonMorest:1998}\\\hline
                               TV                                & & 18.4{\textpm}3.2 \um diam guinea pig \citep{SaintBensonEtAl:1991} 13.34{\textpm}2.17 \um (11-16
            range) guinea pig \citep{Alibardi:1999}              & Dend and axonal arbors restricted
                     to isofrequency lamina                      & \\\hline
                               DS                                & 963 \umsq xsection OC
\citep{SmithRhode:1989} 501{\textpm}168 \umsq rat \citep{DoucetRyugoEtAl:1999}
466{\textpm}137\umsq \citep{DoucetRyugo:1997} cat xsection 571{\textpm} 228
\umsq \citep{ReddCahillEtAl:2002} OC xsectional area 450\umsq
                 \citep{PalmerWallaceEtAl:2003}                  & guinea pig 27\um diameter
\citep{ArnottWallaceEtAl:2004} 22-26 \um diam \citep{DoucetRyugo:1997} 20-30
                 \um \citep{PaoliniClark:1999}                   & 
250-350\um span of dend \citep{DoucetRyugo:1997} with aspinous dendrites, 4
of 5 cells had 4 main dendrites, total dend length 6222 to 7351 \um
(mean-6665 \um), dendrites extended widely in all directions,\~{}70 \um
perpendicular to AN,3-6 primary dendrites at right angles to AN
                    \citep{SmithRhode:1989}                      & axon width 0.7-1.2\um
\citep{OertelWuEtAl:1990}\\\hline
                             Golgi                               & 15 $\mu $m in diameter
                   FerragamoGoldingEtAl:1998                     & & Smooth, tapering dendrites, between 50 and 100
$\mu $m long, emanated in all directions \citep{FerragamoGoldingEtAl:1998}
        Also see \citep{Cant:1993,MugnainiOsenEtAl:1980}         & A dense, axonal plexus,
limited to the plane of the granule cell domain, extended about 250 $\mu $m
from the soma in all directions \citep{FerragamoGoldingEtAl:1998} \\

\end{longtable}

