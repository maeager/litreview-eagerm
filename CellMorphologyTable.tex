\begin{longtable}{cXXXX}%
%
\caption{Cell Morphology in the VCN}\label{tab:Connections} \\
\toprule  
Cells & \multicolumn{2}{X}{Cell body} & Dendrites & Axons \\
 & Cross-section (\umsq) & Diameter (\um) &  & \\ \midrule 
\endfirsthead

\multicolumn{5}{c}{{\bfseries \tablename\ \thetable{} -- continued from previous page}} \\
\midrule Cells & X-section Area & Diameter & Dendrites &Axons \\ \midrule 
\endhead

\midrule \multicolumn{5}{c}{{Continued on next page}} \\ %\midrule
\endfoot
\bottomrule
\endlastfoot

TS                                & 
262.3{$\pm$}62 (rat [\citenum{DoucetRyugoEtAl:1999}]) 
249{$\pm$}95  (rat [\citenum{DoucetRyugo:1997}])
266{$\pm$}156 (rat [\citenum{DoucetRyugo:2006}])
648 (cat [\citenum{SmithRhode:1989}]) 
366{$\pm$}100  (cat [\citenum{ReddCahillEtAl:2002}]) 
200 (CS), 240 (CT) (guinea pig [\citenum{PalmerWallaceEtAl:2003}]) 
192{$\pm$}47 (high CF), 253{$\pm$}63 (low CF) (chinchilla [\citenum{JosephsonMorest:1998}])
& 
16--20\um  (rat [\citenum{DoucetRyugo:1997,DoucetRyugoEtAl:1999,DoucetRyugo:2006}]) 
14.5{$\pm$}2.0 (high CF)  17.6{$\pm$}2.7 (low CF) (chinchilla [\citenum{JosephsonMorest:1998}]) 
& 
2-3 main dend, total dend length $\sim$2000\um (guinea pig [\citenum{PalmerWallaceEtAl:2003}]) 
75-100\um and 150-300\um parallel,3--4 primary dendrites (cat [\citenum{SmithRhode:1989}])
200 \um parallel, 2--5 \um in diam  (chinchilla [\citenum{JosephsonMorest:1998}])  
& 
axon width 0.5--1.1 \um [\citenum{OertelWuEtAl:1990}] 
hillock initial segment averaged 2.5{$\pm$}0.96 \um, 
proximal diam and up to 13 \um long, distal initial segment 0.7{$\pm$}0.2 \um,
myelination began about 20 \um from the soma (chinchilla [\citenum{JosephsonMorest:1998}])\\\hline
TV                                & 
n/a & 
18.4{$\pm$}3.2   (guinea pig [\citenum{SaintBensonEtAl:1991}]) 
13.34{$\pm$}2.17  (guinea pig [\citenum{Alibardi:1999}])
& 
Dend and axonal arbors restricted to isofrequency lamina                      & 
\\\hline
DS                                & 
963  (cat [\citenum{SmithRhode:1989}]) 
501{$\pm$}168  (rat [\citenum{DoucetRyugoEtAl:1999}])
466{$\pm$}137 (rat [\citenum{DoucetRyugo:1997}]) 
418{$\pm$}140 (rat [\citenum{DoucetRyugo:2006}])
571{$\pm$}228  (cat [\citenum{ReddCahillEtAl:2002}]) 
450 (guinea pig [\citenum{PalmerWallaceEtAl:2003}])                  
&  
27  (guinea pig [\citenum{ArnottWallaceEtAl:2004}]) 
22--26  (rat [\citenum{DoucetRyugo:1997}]) 
20--30  (rat [\citenum{PaoliniClark:1999}])                   
& 
250-350 \um span of dend (rat [\citenum{DoucetRyugo:1997}]) with aspinous dendrites, 
4 of 5 cells had 4 main dendrites, total dend length 6222 to 7351 \um
(mean-6665 \um), dendrites extended widely in all directions,
%\~{}70 \um perpendicular to AN,
3-6 primary dendrites at right angles to AN        [\citenum{SmithRhode:1989}]
& 
axon width 0.7-1.2 \um [\citenum{OertelWuEtAl:1990}]\\\hline
Golgi                               & 
&
15 \um (mouse [\citenum{FerragamoGoldingEtAl:1998}])                     
& 
Smooth, tapering dendrites, between 50 and 100 \um long, emanated in all directions (mice [\citenum{FerragamoGoldingEtAl:1998}])
Also see [\citenum{Cant:1993,MugnainiOsenEtAl:1980}]         
& 
A dense, axonal plexus, limited to the plane of the granule cell domain, extended about 250 \um
from the soma in all directions [\citenum{FerragamoGoldingEtAl:1998}] \\
\end{longtable}



%%% Local Variables: 
%%% mode: latex
%%% mode: tex-fold
%%% TeX-master: "LiteratureReview"
%%% TeX-PDF-mode: nil
%%% End: 
