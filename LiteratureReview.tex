\documentclass[10pt,a4paper,twoside,openright]{book}

\usepackage{nag}
\usepackage{../hg/manuscript/style/uomthesis}
\input{../hg/manuscript/user-defined}

\usepackage[acronym]{glossaries}
\input{../hg/manuscript/misc/glossary}
\makeglossaries

\usepackage{appendix}
\graphicspath{{../SimpleResponsesChapter/gfx/}{../figures/}{/media/data/Work/cnstellate/}{/media/data/Work/cnstellate/ResponsesNoComp/ModulationTransferFunction/}}
\usepackage{rotating,calc}
\usepackage{booktabs,ltxtable,lscape}

\usepackage[displaymath,floats,graphics,textmath,sections,footnotes,showlabels,psfixbb]{preview}
\ifPreview
\PreviewEnvironment[{*[][]{}}]{tabularx}
\PreviewMacro[{*[][]{}}]{\yellownote}
\fi

\lfoot{\footnotesize\today\ at \thistime}

\pretolerance=150
\tolerance=100
\setlength{\emergencystretch}{3em}

\overfullrule=1mm
% \usepackage[notcite]{showkeys}

\graphicspath{{../../figures/}{./gfx/}{/media/data/Work/thesis/SimpleResponsesChapter/}{/media/data/Work/thesis/SimpleResponsesChapter/gfx/}{/media/data/Work/cnstellate/}{/media/data/Work/cnstellate/golgi/}{/media/data/Work/cnstellate/}{/media/data/Work/cnstellate/TV_Notch/}}

\newcommand{\um}{$\mu$m}
\newcommand{\umsq}{$\mu$m$^2$}

% xelatex font
% \usepackage[T1]{fontenc}
% \usepackage{fontspec}
% \usepackage{graphicx}
% \defaultfontfeatures{Mapping=tex-text}
% \setromanfont{Gentium}
% \setromanfont [BoldFont={Gentium Basic Bold},
%                 ItalicFont={Gentium Basic Italic}]{Gentium Basic}
% \setsansfont{Charis SIL}
% \setmonofont[Scale=0.8]{DejaVu Sans Mono}


\begin{document}
		% TOC, LOF, LOT
		{%
			\singlespacing%
			\tableofcontents%
			\listoffigures%
			% Do not include a list of tables if you have less
			% than 10 tables, as per SGS suggestion.
			\listoftables
                        \printglossaries
		   \clearpage%
		}%
\setcounter{chapter}{0}
%\chapter{Literature Review}
%\medskip{}
%\centerline{\today\\ Draft Version:  \input{.hg/tags.cache}

%************************************************
\chapter{Introduction}\label{Ch1:introduction}
%************************************************
%   This section should give a rough introduction to the themes
%   included in the thesis and how the thesis is structured
%*****************************************

%\PARstart{A}{ }

\glsresetall[main,acronym]
%*****************************************
%*****************************************
%*****************************************
%*****************************************
\section{Literature Review}\label{Ch1:litreview}
%*****************************************

\section{Auditory System}\label{Ch1:AuditorySystem}


\subsection{Peripheral Auditory System}

\todo[inline]{Intro to section  on peripheral AN}


\todo[inline]{Small introduction to the outer and middle ear, and inner}

%\subsubsection{Inner Ear}

% \citep{RyugoParks:2003}
% Upon passing the Schwann-glia border (marking entrance
% into the central nervous system), individual auditory nerve
% fibres penetrate a variable distance into the nucleus, de-
% pending upon fibre CF, and bifurcate into an ascending
% branch and a descending branch. The ascending branch has a
% relatively straight trajectory into the AVCN and terminates
% as a large, axosomatic ending called the endbulb of Held.
% The descending branch likewise has a straight trajectory
% through the PVCN before entering the DCN. Along the
% way, these main branches give rise to short collaterals. The
% collaterals ramify further and exhibit en passant swellings
% and terminal boutons. Fibers of similar CFs disperse to
% form a 3-dimensional sheet running through the nucleus,
% and stacks of these sheets represent the isofrequency con-
% tours of the nucleus (Fig. 9, bottom). The sheets have a
% horizontal orientation within the ventral cochlear nucleus
% but twist caudally to form parasagittal sheets in the DCN.
% These projections underlie the tonotopic organization of
% the resident neurons of the cochlear nucleus [19,160,191].

% \citep{EvansNelson:1973,SpirouYoung:1991,YoungSpirouEtAl:1992,SpirouDavisEtAl:1999,YoungNelkenEtAl:1993,ArleKim:1991a}




% 5. Structure-function correlates
% 5.1. SR and peripheral correlates
% Morphologic specializations have been found in the in-
% nervation pattern of inner hair cells with respect to SR fibre
% groupings. High-SR fibres (>18 spikes/s) have thick periph-
% eral processes that tend to contact the “pillar” side of the
% inner hair cell, whereas low-SR fibres (<18 spikes/s) have
% thin peripheral processes that tend to contact the modiolar
% side of the hair cell [98,111]. Furthermore, there is SR
% segregation within the spiral ganglion. Low-SR neurons
% tend to be distributed on the side of the scala vestibuli,
% whereas high-SR fibres can be found throughout the gan-
% glion [82,100]. These peripheral differences are maintained
% by the pattern of central projections, and embedded within
% the tonotopic organization.
% 5.2. SR and central correlates
% There are morphologic correlates that correspond to
% groupings of fibres with respect to SR. Compared to fibres
% of high SR (>18 spikes/s), fibres of low SR (<18 spikes/s)
% exhibit different innervation characteristics with the IHCs
% [99,111], give rise to greater collateral branching in the
% AVCN [51], emit collaterals that preferentially innervate
% the small cell cap [100,177], and manifest striking special-
% izations in the large axosomatic endings, the endbulbs of
% Held [185] and their synapses [178].
% The typical high-SR fibre traverses the nucleus and gives
% rise to short collaterals that branch a few times before
% terminating (Fig. 10A). There was a suggestion that projec-
% tions of the different SR groups might be segregated along
% a medial-lateral axis within the core of the AVCN [94] but
% single-unit labelling studies do not unambiguously support or
% refute this proposal [51,82,100,208]. There are usually one
% or two terminal endbulbs at the anterior tip of the ascending
% branch, and the remaining terminals appear as en passant
% swellings or terminal boutons. It is presumed that these
% swellings are sites of synaptic interactions with other neu-
% ronal elements in the cochlear nucleus. Approximately 95%
% of all terminal endings were small and round, definable as
% “bouton-like” [163]. The remaining endings were modified
% endbulbs that tended to contact the somata of globular bushy
% cells and large endbulbs of Held that contacted the somata
% of spherical bushy cells. In contrast to birds, low-frequency
% myelinated auditory nerve fibres in mammals give rise
% to endbulbs. Furthermore, the endbulbs of low-frequency
% fibres tend to be the largest of the entire population of
% fibres.

% There is a clear SR-related difference in axonal branch-
% ing and the number of endings. Low-SR fibres give rise to
% greater collateral branching in the AVCN compared to that
% of high-SR fibres [51,100,101,208]. In cats, the ascending
% branch of low-SR fibres give rise to longer collaterals, twice
% as many branches (there are approximately 50 branches per
% low-SR fibre compared to 25 per high-SR fibre), and twice as
% many bouton endings (Fig. 10B). These endings, while more
% numerous, are also smaller compared to those of high-SR
% fibres [163]. The greater total collateral length is illustrated
% by low-SR fibres that have an average of 5 mm of collaterals
% per ascending branch compared to 2.8 mm of collaterals per
% high-SR fibre [51]. The inference from these observations is
% that low-SR fibres contact more neurons distributed over a
% wider region of the cochlear nucleus than do high-SR fibres.
% If the perception of loudness is proportional to the num-
% ber of active neurons [195], then this branching differential
% may provide the substrate. The activation of high-threshold,
% low-SR fibres by loud sounds would not only increase the
% pool of active auditory nerve fibres but also produce a spread
% of activity throughout the AVCN. This recruitment would
% be useful because the discharge rate of high-SR fibres is al-
% ready saturated at moderate sound levels.
% There is no systematic difference in the average number
% of terminals generated by the descending branch with re-
% spect to fibre SR. Low-SR fibres do, however, have a wider
% distribution across the frequency axis in the DCN as com-
% pared to high-SR fibres [171]. The endings lie within the
% deep layers of the DCN, below the pyramidal cell layer, and
% terminate primarily within the neuropil. The average termi-
% nal field width for low-SR fibres is 230.5 ± 73 ␮m, whereas
% that for high-SR fibres is 87.2 ± 41 ␮m. The significance of
% terminal arborization differences between high- and low-SR
% fibres might be involved in details of isofrequency laminae.
% The relatively short and narrow arborization of high-SR,
% low-threshold fibres could occupy the center of the lamina
% and endow those neurons with lower thresholds and sharper
% tuning. In contrast, the longer and broader terminal field of
% low-SR fibres could preferentially innervate the “edges” of
% the lamina. This kind of organization might establish a func-
% tional segregation of units having distinct physiological fea-
% tures within an isofrequency lamina, as has been proposed in
% the inferior colliculus [161] and auditory cortex [184,186].



\subsection{Auditory Nerve Fibres}


% 4.2. Spontaneous discharge rate (SR)
% Unlike birds, SR in the mammalian auditory nerve forms
% a bimodal distribution where 30–40% of the fibres have SR
% <10 spikes/s and 60–70% have SR >30 spikes/s. Threshold
% is correlated to the amount of spontaneous spike activity
% (SR) that occurs in the absence of experimenter-controlled
% stimulation. Low-SR fibres have relatively high thresholds,
% whereas high-SR fibres have low thresholds. Across the au-
% dible frequency range, fibres of similar CFs can vary in SR
% from near 0 to >100 spikes/s. The bimodal SR distribution
% is present across the entire audible frequency range for the
% animal (e.g. cat [50,88,97]; gerbil [183]; guinea pig [208]),
% and implies a general organizational principle for the
% mammalian auditory nerve. Fibers of the different SR
% groupings exhibit distinct physiologic features, especially
% in terms of their contribution to the dynamic range of
% hearing [50,59,179] and representation of speech sounds
% [110,180,212,223]. The collective evidence suggests that
% different SR groupings of auditory nerve fibres serve sep-
% arate roles in acoustic information processing. It might be
% that the high-SR fibres with their low thresholds prefer to
% function in quiet, whereas low-SR fibres with their high
% thresholds operate better in loud and noisy environments.




%%% Local Variables: 
%%% mode: latex
%%% mode: tex-fold
%%% TeX-master: "LiteratureReview"
%%% TeX-PDF-mode: nil
%%% End: 


%*****************************************
\section{Cochlear Nucleus}\label{Ch1:CochelarNucleus}


Recent Reviews of characterised cells and projections
\citep{CantBenson:2003,RyugoParks:2003,SmithMassieEtAl:2005,YoungOertel:2004,OertelWrightEtAl:2010}

%\citep{CantBenson:2003}
% Except for a few differences to be mentioned later, cell types in
% rat and cat appear to be quite similar and are also identifiable
% in a number of other species, including human [6,87,136]
% and other primates [87,141]; chinchilla [138,165]; gerbil
% [145,165]; guinea pig [75,76,133]; kangaroo rat [45,251];
% mole [114]; mouse [239,252,262,264]; porpoise [162];
% rabbit [53,172] and several species of bats [59,208,269].

% Smith and Rhode [220] were able to divide the large mul-
% tipolar neurons in the posterior part of the AVCN and the
% anterior part of the PVCN of the cat into two groups based
% on differences in physiological response properties, synaptic
% organization, the pathway taken by the axons, and the types
% of vesicles contained in their synaptic terminals. Their com-
% prehensive study has provided a framework for a synthesis
% of results from a number of laboratories, all of which are
% compatible with the conclusion that the ventral cochlear nu-
% cleus contains at least two functionally distinct populations
% of multipolar cells.



reviews \citep{BruggeGeisler:1978}

\citet{DoucetRyugo:2006} showed the limited number of VCN multipolar neuron studies that have performed phyiological unit assesments as well as classification of morphology via labelling.

% auditory brain stem. Most of the different neuron popula-
% tions in the magnocellular core are contacted by individual
% auditory nerve fibers. These cells, along with the incoming
% auditory nerve fibers, are arranged in isofrequency contours
% and have identical CFs. They can, however, exhibit different
% thresholds and temporal patterns in their spike discharges.
% That is, when presented with short tone bursts at CF, the
% PSTH can display reliable but distinctly different patterns
% [16,18,147,156]. These PSTHs are named by their shapes,
% such as primary-like, chopper, onset, pauser, or build-up.
% Intracellular recording and staining methods have demon-
% strated that neurons with different firing patterns generally
% manifest different morphological features, including but not
% limited to, dendritic appearance, cell body size and shape,
% and axonal projections (e.g. [53,157,158,164]). Some of
% these differences are attributed, at least in part, to the nature
% of the neuron’s input from the auditory nerve [86,117], but
% intrinsic membrane properties assume equal prominence
% [5,61,107,108,129].


% 6.3. Receptors and transmitters
% The auditory system has demanding functions with re-
% spect to processing sound for localization, identification,
% and communication. Key to these functions is the ability to
% receive and transmit faithfully rapid changes in the acoustic
% signal. The large endbulbs of Held reflect one specialization
% for secure transmission [130,146,168] and the amino acid
% glutamate is the most likely candidate that facilitates rapid
% transmission [57,119,150]. One difficulty in the identifica-
% tion of glutamate as the auditory nerve neurotransmitter is its
% ubiquitous distribution in tissue and the inability to measure
% its release during synaptic activity. Consequently, research
% efforts have been directed towards the study of pharmaco-
% logic agonists and antagonists of glutamate, the analysis of
% glutamate receptor subunits, and/or the role of glutamate
% transporters. As in birds, the bulk of available evidence im-
% plicates glutamate as the neurotransmitter for the auditory
% nerve. Quantitative immunogold staining methods revealed
% significantly greater labeling over primary endings com-
% pared to nonprimary endings (containing flat or pleomorphic
% vesicles) and glia [64]. Furthermore, immunogold staining
% was significantly lower in primary terminals that had been
% depleted of glutamate by potassium-induced depolarization.
% The production and/or utilization of glutamate is mediated
% by aspartate aminotransferase. The soluble cytoplasmic
% fraction of this amino acid has been immunolocalized to
% endings of auditory nerve fibers [2,166]. The presence of
% glutamate and one of its metabolic precursors in the presy-
% naptic endings is complemented by glutamate receptors
% in cochlear nucleus neurons (e.g. [143,144,167,213,217]).
% These receptors include ionotropic receptors that are formed
% by several subunits surrounding a central ion channel, and
% metabotropic receptors that associate with G-proteins and
% tend to mediate long-term responses.
% The ionotropic receptors include AMPA, kainate, delta,
% and NMDA types. AMPA receptors mediate the fast exci-
% tatory transmission and consist of four subunits, GluR1–4
% with flip, flop, and other splice variants. Primary endings
% in the cochlear nucleus oppose AMPA receptors composed
% mainly of GluR3 and GluR4 [143,167,213]. The GluR4 sub-
% units gate rapidly and are specialized for auditory nerve in-
% put [119]. The GluR2 subunits, which exhibit slower AMPA
% kinetics, are associated with parallel fiber inputs in the DCN
% [56]. Primary endings are also associated with delta recep-
% tors but it is not known if delta receptors form functional
% receptors.
% NMDA receptors are characterized by a voltage-dependent
% calcium channel. It is thought that depolarization via the
% AMPA receptors is required for the NMDA receptor to
% open. The calcium-permeability of NMDA is presumed to
% form the basis of the long-term effects underlying learn-
% ing and memory. There are a number of NMDA receptors,
% including NR1 (with eight splice variants), NR2A-D, and
% NR3. Receptors are composed of NR1 plus one or more
% variants of NR2 which determine their physiological prop-
% erties. NR1 has a widespread distribution in the cochlear
% nucleus [15,143]. NR2A and NR2C are found in the su-
% perficial granule cell layer of the ventral cochlear nucleus
% and in medium sized neurons of the deeper layers of the
% DCN. NR2A is in the large neurons of the rostral AVCN
% and NR2B is in pyramidal cells of the DCN. There is
% pharmacologic evidence that reveals an early presence of
% NMDA receptors in the developing auditory system that di-
% minishes shortly after weaning [54,73]. Such data indicate
% that NMDA receptors may play a role in the development
% of the non-NMDA receptors.
% Metabotropic glutamate receptors are thought of as sin-
% gle molecules coupled to G-proteins. The G-proteins are
% linked to intracellular transduction pathways that may un-
% derlie plasticity. There are three types of metabotropic re-
% ceptors based on their pharmacologic properties and second
% messenger cascades [126]. In the cochlear nucleus, immuno-
% cytochemical staining demonstrates labeling of mGluR1␣
% and mGluR5 [143], which are involved in the initiation
% of the phosphoinositol transduction pathway [127]. These
% metabotropic receptors are distributed in the perisynaptic
% membrane that flanks the postsynaptic membrane density
% [128,145].
% Transmitter released from auditory nerve endings must be
% rapidly removed from the synaptic cleft so that the postsy-
% naptic cell is prepared for the next transmission. The removal
% and inactivation of neurotransmitter is accomplished by a
% highly efficient system of uptake and transporter molecules
% that surround glutamatergic synapses [9,10,46]. It has been
% speculated that glial processes and neuronal membranes lin-
% ing the intercellular cisternae between primary endings and
% cochlear nucleus neurons might house the relevant trans-
% porter molecules [154]. Regulation of residual transmitter
% by glial transporters has been suggested as a means to con-
% trol synaptic strength [210]. A preference for transporter
% molecules to be distributed along the membranes lining these
% perisynaptic cisternae could provide important insights into
% synaptic function.



\section{T~Stellate Cells}\label{sec:t-stellate-cells}

 TS cells encode complex features of the stimulus important for the recognition of
 natural sounds and are a major source of excitatory input to the inferior colliculus
 \citep{OertelWrightEtAl:2010}.

 This section gives a brief description of the cell morphology,
 immuno-histochemistry, intrinsic properties, and synaptic contacts of T stellate
 cells. The determination of how theses elements contribute to the acoustic behaviour

 \subsection{Morphology of T~Stellate Cells}\label{sec:morph-t-stell}

 T stellate cells lie in the core region of the VCN, primarily in the posteroventral
 section (PVCN) with some in the posterior part of the anteroventral section (pAVCN)
 \citep{Osen:1969,Lorente:1981,BrawerMorestEtAl:1974,OertelWuEtAl:1990,DoucetRyugo:2006,DoucetRyugo:1997}.

 Histology staining of the cochlear nucleus began almost a century ago
 \citep{Lorente:1933}, and the role of classification and naming distinct cell types
 began. Star-like cell bodies observed with Golgi impregnation were called
 \textit{stellate} cells \citep{Osen:1969}. Nissl staining showed the multiple
 dendritic morphology of TS and DS cells, hence the name \textit{multipolar} was used
 \citep{BrawerMorestEtAl:1974,Lorente:1981}.  Distinction based on somatic
 innervation in multipolar neurons separated them into two types: type 1 (few
 somatic) type II (many somatic and dendritic) \citep{Cant:1981}.  The axonal
 projections of TS cells leave the CN ventrally through the \textbf{t}rapezoid body
 or ventral acoustic stria (hence the T in T stellate cells), while DS cells' axons
 head \textbf{d}orsally toward the DCN and some may exit the CN via the dorsal
 acoustic stria \citep{OertelWuEtAl:1990}.  Further nomenclature based on dendritic
 differences into planar (TS cells) and radial (DS cells) has also been suggested
 \citep{DoucetRyugo:1997,DoucetRyugo:2006}
%distinction between TS and DS cells is made by their axonal projections, dendritic projections, and their immunohistochemistry.


T stellates project locally in VCN and deep layer of DCN -> match Chopper \citep{RhodeOertelEtAl:1983,SmithRhode:1989}
D stellate projections wide in VCN,DCN and cCN -> match OnC  \citep{SmithRhode:1989}



Chopper subdivision:
\begin{itemize}
\item sustained and transient (no evidence for anatomical differences in T stellate cells)
\end{itemize}




  For consistency, the TS cell type modelled in this thesis represent each of the
  various names given to neurons with similar characteristics (T stellate, type 1
  multipolar, planar, and chopping PSTH units) in different animals, with closest
  association with rodents and cats. The DS cell type includes all those previously named as D stellate, type 2
  multipolar, radial, and onset chopping PSTH unit



 \subsection{Intrinsic Mechanisms of T~Stellate Cells}\label{sec:intr-mech-tstellate} %Making Phasic Input Tonic}


Type 1 Current Clamp, single exponential undershoot
      \citep{FengKuwadaEtAl:1994,ManisMarx:1991,WuOertel:1984}         
\citep{FujinoOertel:2001,FerragamoGoldingEtAl:1998a} 



\citep{RothmanManis:2003,RothmanManis:2003a,RothmanManis:2003b,Rothman:1999}


No Low threshold
K \citep{ManisMarx:1991} 
IKA has a role in modulating the rate of repetitive
firing.  
Effect of Inhibition on T stellate cells could be to reset IA \citep{RothmanManis:2003b}                       

Effective somatic membrane time constant 6.5{$\pm$}5.7 msec \citep{ManisMarx:1991}
type I 9.1{$\pm$}4.5 \citep{ManisMarx:1991} 
6.2 to 18.0 msec
\citep{FengKuwadaEtAl:1994} 6.9{$\pm$}3msec, 10-90\% rise time was
         1.05{$\pm$}0.4msec \citep{IsaacsonWalmsley:1995}           

Linear I-V                       \citep{ManisMarx:1991}                     

447{$\pm$}265 Mohm isolated guinea pig stellate cell type 1 current clamp
\citep{ManisMarx:1991} 

44 to 151 M$\Omega $ (mean 89.4 {$\pm$}24.4) mouse slice prep \citep{FerragamoGoldingEtAl:1998a}
stellate 
231{$\pm$}113M$\Omega$, 14.9$\pm$9~pF primary membrane capacitance, room temp rat
\citep{IsaacsonWalmsley:1995} dog \citep{BalBaydasEtAl:2009} 
176{$\pm$}35.9 Mohm membrane time constant 8.8{$\pm$}1.4 (n=21)        

steady depolarising current shows intracellular ability to be tonic
\cite{Oertel:1983,OertelWuEtAl:1988} BUT - how does the input remain stable
given AN adaptation?



\begin{enumerate}
\item selective processing of HSR and LSR input
\begin{enumerate}
\item feed-forward excitation in TS cells
\begin{itemize}
\item axon collaterals in local isofrequency (most cells in PVCN are TS cells)
\end{itemize}
\item co-activation of phasic inhibition
\begin{itemize}
\item DS inhibition ispi and contralaterally
\begin{itemize}
\item onset inhibition strongest, affecting TS cells after first spike
\item broad tuning sharpens FSL
\end{itemize}
\item TV sharply tuned inhibition (Ferr98)
\begin{itemize}
\item TV response variable and non-monotonic
\item \citep{Rhode:1999}  labelled TV cells phasic in anaesthetised cats
\item unanaesthetised cats and gerbils are phasic or tonic  \citep{DingVoigt:1997,ShofnerYoung:1985}
\end{itemize}
\item Others - Glycine from ipsi periolivary region, GABA from both
          periolivary regions \citep{AdamsWarr:1976,ShoreHelfertEtAl:1991,OstapoffBensonEtAl:1997}
\end{itemize}
\item Absense of LT potassium in TS
\begin{itemize}
\item labelled \citep{ManisMarx:1991,BalOertel:2001,FerragamoOertel:2002,CaoShatadalEtAl:2007}
\item unlabelled \citep{RothmanManis:2003,RothmanManis:2003a,RothmanManis:2003b,Rothman:1999}
\end{itemize}
\item Activation of NMDA
\begin{itemize}
\item \citep{CaoOertel:2010} shows TS cells activate large currents through NMDA receptors
\item NMDA longer lasting, reducing phasic nature of input
\end{itemize}
\item Little synaptic depression
\begin{itemize}
\item SD less than bushy and octopus \citep{WuOertel:1987,ChandaXu-Friedman:2010,CaoOertel:2010}
\item excitation of TS adapts less than other VCN neurons
\end{itemize}
\end{enumerate}
\end{enumerate}



\subsection{Synaptic Inputs of T~Stellate Cells}\label{sec:synaptic-inputs-tstellate}


 \subsubsection{Afferent innervation}

% \citep{FerragamoGoldingEtAl:1998a} Innervation of T stellate cells by auditory
% nerve fibers Auditory nerve fibers are of two types. Type I fibers are large
% and myelinated and comprise 95% of the total while
% type II fibers are small, unmyelinated and comprise only
% Ç5% of the total (cats: Kiang et al. 1982; mice: Ehret 1979).
% On the basis of extracellular injections of auditory nerve fibers in mice,
% type I auditory nerve fibers have been ob- served to terminate on both D and T
% stellate cells (M. W.  Garb and D. Oertel, unpublished observations). Most
% neu- rons in the multipolar cell area of the PVCN (probably T stellate cells)
% are contacted heavily at the cell body unlike cells that project through the
% trapezoid body in cats. In cats type I fibers innervate all of the large
% cells, including those that correspond to T and D stellate cells (Liberman
% 1991, 1993). The anatomic findings are consistent with what is known about
% responses to activation of auditory nerve fibers in vivo and in vitro. The
% short-latency, sharply timed re- sponses to the onset of tones indicate that
% chopper and onset- chopper units receive input from the large, myelinated
% audi- tory nerve fibers (Blackburn and Sachs 1989; Rhode and Smith 1986; Smith
% and Rhode 1989). In slices from mice, both D and T stellate cells respond to
% shocks of the auditory nerve with EPSPs (Oertel et al. 1990; Wu and Oertel
% 1986).  As thresholds for EPSPs are low and latencies are õ1 ms, the input is
% probably from myelinated auditory nerve fibers.  Anatomic and
% electrophysiological evidence indicates that few auditory nerve fibers
% innervate a T stellate cell. The orientation of the dendrites of T stellate
% cells parallel to the path of auditory nerve fibers and spanning a small
% proportion of the tonotopic axis indicates that T stellate cell dendrites are
% positioned to receive input from a limited group of fibers.  The result that
% the amplitude of responses to shocks of the auditory nerve grow in three or
% four discrete jumps with shock strength indicates that the number of fibers
% innervating one T stellate cell in a mouse is small, perhaps as small as three
% or four (Fig. 1). As any of the jumps in amplitude could have resulted from
% the recruitment of more than one fiber and as it is possible that inputs might
% have been cut or damaged, this estimate represents a minimum. This conclu-
% sion is in contrast with the results of similar experiments in octopus cells,
% in which such subthreshold jumps cannot be detected (Golding et
% al. 1995). This result also indicates that models of choppers, based on what
% is known in cats, that require the integration of many inputs might be
% oversim- plified (Banks and Sachs 1991; Molnar and Pfeiffer 1968; Wang and
% Sachs 1995).  It is intriguing that the NMDA-receptor–mediated slow
% depolarizations were generated with shock strengths greater than those
% required to produce apparently maximal mono- synaptic EPSPs. This finding
% suggests that different sources of glutamatergic input may activate different
% populations of receptors. It raises the possibility that type I auditory nerve
% fibers act primarily through AMPA receptors, as they are known to do in other
% vertebrate cochlear nuclei (Raman et al. 1994; Zhang and Trussell 1994)
% whereas other sources of excitation, alone or in combination, are required to
% activate NMDA receptors. It is conceivable that type II auditory nerve fibers
% contribute to the long, slow depolarization.  Small, unmyelinated fibers would
% be expected to have higher thresholds for shocks than larger, myelinated
% fibers and their responses would be expected to be later.

 Auditory nerve fibre synapses on TS cells are glutamatergic AMPA receptors
 \citep{FerragamoGoldingEtAl:1998a,WentholdHunterEtAl:1993}.  Histological measures
 of labelled T stellate cells show the presence of glutamate and glutamine antibodies
 \citep{HackneyOsenEtAl:1990,WentholdHunterEtAl:1993}.  More advanced measures using
 electron microscopy reveal AMPA subunits, unique to the cochlear nucleus, apposing
 TS cells \citep{WangWentholdEtAl:1998}.  Pharmacologic experiments have also
 confirmed monosynaptic EPSPs from AN shocks are be blocked by the AMPA antagonist
 DNQX \citep{FerragamoGoldingEtAl:1998a}. 


 Glutamatergic NMDA receptors may also be present at ANF synapses
 \citep[mice][]{FerragamoGoldingEtAl:1998a} and can be activated to produce large
 synaptic currents \citep{CaoOertel:2010}.  Whole cell patch recordings show NMDA
 dominance over AMPA at birth reverses during development, leaving little to no
 observable NMDA EPSCs at the soma in mature rats \citep{BellinghamLimEtAl:1998}.
%\citep{Oertel:1983}
 Five percent of ANFs are unmyelinated type II fibres (cat: \citep{KiangRhoEtAl:1982},mice: \citep{Ehret:1979} and their axons enter the
 outer shell or GCD where they are likely to terminate on distal dendrites of T
 stellate cells using diffuse synapses, possibly mediated by NMDA receptors or
 the lack of glutamate re-uptake
 \citep{BensonBrown:2004,Ryugo:2008,RyugoHaenggeliEtAl:2003,RyugoParks:2003}.
 (cats: Kiang et al. 1982; mice: Ehret 1979)

 The ANF synaptic contacts on the cell body of T stellate cells are relatively small,
 a distinguishing contrast between the densely contacted D stellate cells
 \citep{Cant:1981,Cant:1982,RyugoWrightEtAl:1993,TolbertMorest:1982a,FayPopper:1994,ReddCahillEtAl:2002,RyugoWrigthEtAl:1993,Ryugo:1992,RyugoParks:2003}
 The dendritic ANF input is mostly proximal ($<$100 \um) with the density of contacts
 diminishing toward the distal ends \citep{SmithRhode:1989}.  T stellate cells have
 $\sim$30\% somatic coverage, but less than 40\% of those contacts are from ANFs
 \citep{Cant:1981,Cant:1982,RyugoWrightEtAl:1993,TolbertMorest:1982a,SmithRhode:1989},
 and is highly variable (mean 13 terminals \citep*[36$\pm$10.5\% of somatic terminals in
 cat][]{SmithRhode:1989}, \citep*[0--6 terminals per soma in
 chinchilla][]{JosephsonMorest:1998}).  \citet{FerragamoGoldingEtAl:1998a} estimated
 a small number of independent ANFs (4 to 6) were needed to reach AP in mice T
 stellate cells.  Some cells had ANF synapses surrounding the axon initial segment
 \citep{JosephsonMorest:1998}.


  


% How chopping responses are produced is not completely understood. It has been
% suggested that stellate cells integrate input from large numbers of auditory nerve
% fibers. However, stellate cells in mice have been shown to receive input from only
% a few (four to six) sharply timed auditory nerve fiber inputs (175).  Activation of
% these inputs with trains of shocks produces entrained responses rather than
% chopping (172, 175), raising two questions: How are stellate cells prevented from
% encoding the timing of auditory nerve inputs after the initial action potential in
% response to sound, and how is their steady firing in response to tones produced
% from inputs that have strong onset transients?


 The estimated receptive field of single ANFs in mice and cats ($\sim$70$\mu$m HSR,
 100$\mu$m LSR
 \citep{OertelWuEtAl:1990,Ryugo:2008,MeltzerRyugo:2006,RyugoParks:2003,Ryugo:1992,BrownBerglundEtAl:1988,RoullierCronin-SchreiberEtAl:1986,FeketeRouillerEtAl:1984})
 closely matches the dendritic width of TS cells perpendicular to the incoming ANF
 axons (75-100$\mu$m \citep[Mouse]{OertelWuEtAl:1990}).
% 0.23-0.39 oct \citep[anesthetized guinea pig][]{PalmerJiangEtAl:1996}
 The physiological receptive field is also similar between ANFs and TS cells
 (Q$_{10}=5.3$ \citep[cat][]{RhodeSmith:1986}Q$_{10}=5.52\pm1.4$ compared to
 Q$_{10}=6.3$ in ANFs \citep[guinea pig]{JiangPalmerEtAl:1996}) but varies with
 different TS cell classification subtypes (CS Q$_{10}=4$, CT Q$_{10}=2$ (low CF),
 and Q$_{10}=3.67$ (high CF) \citep[guinea pig]{PalmerWallaceEtAl:2003}) and the type
 of anaesthetic used in the study (Q$_{10}=7.4$ unanesthetised Q$_{10}=5.3$
 barbiturate \citep[cat][]{RhodeKettner:1987}).

 The theoretical conductance delay from the cochlea to the position of TS cells in
 the VCN, based on the average distance and myelinated axon width, was estimated to
 be 0.5 ms \citep{Brown:1993,BrownLedwith:1990}.  Oertel and colleagues first
 calculated the delay experimentally using electrical shocks to the auditory nerve
 root in slice preparations in mice \citep[0.7 ms][]{Oertel:1983} and in chinchilla
 \citep[0.5 ms][]{WickesbergOertel:1993}. This was later confirmed in more studies
 with mean 0.7 ms (range 0.48--0.92 ms) \citep[mice][]{FerragamoGoldingEtAl:1998a}.



 \subsubsection{Glycinergic inhibition from D~stellate cells}

 DS\ensuremath{\rightarrow}TS

% Although well timed inhibition has been hypothesized to play
% an important role in auditory processing (Batra and Fitzpatrick,
% 1997; Brand et al., 2002; Needham and Paolini, 2003), the kinetics
% of glycinergic signaling in mature animals is unknown. Brand et
% al. (2002) demonstrated that the increase in the discharge rate of
% MSO neurons induced by the application of strychnine was de-
% pendent on the interaural time difference (ITD) of the sound
% stimuli (i.e., it shifted the ITD tuning curve along the abscissa).
% Models explaining these results rely on the novel presumption
% that glycinergic inhibition must decay in the microsecond range
% and be evoked at high frequency (Brand et al., 2002). In this
% paper, we demonstrate that glycinergic IPSCs of auditory neu-
% rons may have kinetic properties consistent with this role.

% \citep{DoucetRossEtAl:1999}
% glycine immunostaining
% within the cochlear nucleus in a variety of species, includ-
% ing guinea pigs (Wenthold et al., 1987; Saint Marie et al.,
% 1991; Kolston et al., 1992), rats (Mugnaini, 1985; Peyret et
% al., 1987; Aoki et al., 1988; Gates et al., 1996), mice
% (Wickesberg et al., 1991), cats (Osen et al., 1990), and
% baboons (Moore et al., 1996).


 Evidence of glycine in the cochlear nucleus, through staining or
 immunohostochemistry, has been studied in many species including guinea pigs
 \citep{JuizHelfertEtAl:1996a,HelfertBonneauEtAl:1989,Wenthold:1987,WentholdHuieEtAl:1987,AltschulerBetzEtAl:1986,SaintBensonEtAl:1991,KolstonOsenEtAl:1992,PeyretCampistronEtAl:1987,Alibardi:2003a,MahendrasingamWallamEtAl:2004,MahendrasingamWallamEtAl:2000,BabalianJacommeEtAl:2002},
 rats
 \citep{OsenLopezEtAl:1991,Mugnaini:1985,AokiSembaEtAl:1988,GatesWeedmanEtAl:1996,Alibardi:2003,LimOleskevichEtAl:2003,SrinivasanFriaufEtAl:2004,DoucetRossEtAl:1999},
 mice \citep{WickesbergWhitlonEtAl:1991,LimOleskevichEtAl:2003,YangDoievEtAl:2002},
 cats \citep{OsenOttersenEtAl:1990,SmithRhode:1989}, baboons
 \citep{MooreOsenEtAl:1996}, gerbils \citep{GleichVater:1998}, and bats
 \citep{KemmerVater:2001a}.


Glycine GlyR receptors (mouse [\citenum{FerragamoGoldingEtAl:1998a}]).
Flat vesicles (Glycine) apposed to TS units (cat [\citenum{SmithRhode:1989}])
Could be mixed Gly/GABA [\citenum{AltschulerJuizEtAl:1993}], most likely from periolivary terminals

The fast dynamics of the glycinergic synapse is essential for transmitting
temporal information to higher centres.  Early studies in slice preparations in
the VCN estimated the decay time constant as fast as 1.6 ms
\citenum[mouse][]{Oertel:1983}, but several studies found values toward 5.3 ms
(\citep*[mouse][]{OertelWickesberg:1993,WickesbergOertel:1993}, \citep*[guinea~pig VCN][]{HartyManis:1998}) In more recent developments, spontaneous IPSCs in
MNTB neurons in rats (a close analogue of neurons in the VCN core) provide an
accurate measure or the dynamics of the receptor ($\tau$). The weighted decay
time constant of IPSCs in young rats ($3.9 \mathrm{ms} \pm 0.5$) is a
combination of ($\tau_{\textrm{fast}}$ and $\tau_{\textrm{slow}}$)
\citet{AwatramaniTurecekEtAl:2004} measured the miniature IPSCs in mature rats
and found the fast exponential dominated ($\tau_{\textrm{fast}}$ 2.1 $\pm$ 0.1
msec).  

%Evoked IPSCs had an average $\tau_{\textrm{fast}}$ of 2.9 0.3 msec (96% of the fit) and a $\tau_{\textrm{slow}}$ of 12.3 16.4 msec.

% At physiological temperatures, glycinergic mIPSCs were fast as those measured at
% room temperature ($\tau_{\textrm{fast}}=0.8 \pm 0.2$ msec). The evoked IPSCs
% were also briefer at 37$^\circ$C ($\tau_{\textrm{fast}}=1.0 \pm 0.2$ msec) (Fig. 2
% A). 


The rise time (10\%-90\%) of IPSCs at room temp is faster in AVCN glycinergic mIPSCs
($0.46\pm0.05$ ms) compared to MNTB ($0.60\pm0.03$ ms) \citep{LimOleskevichEtAl:2003}, and the decay time constant equates to 2.5 ms at body temperature.

Rise 0.4 ms, Decay 2.5 ms (spontaneous IPSCs in rat MNTB neurons, [\citenum{AwatramaniTurecekEtAl:2005}]) 
The rise time of glycinergic IPSCs was consistent across rodents also measured 0.46$\pm$0.05 ms spontaneous IPSCs In
AVCN bushy cells in mice \citep{LimOleskevichEtAl:2003}.


%Decay  5.47 $\pm$0.19 (very young MNTB rat [\citenum{AwatramaniTurecekEtAl:2005}])
%Decay 6--13 ms (Slice prep 30 C degrees; VCN guinea pig [\citenum{HartyManis:1998}]).
%Activation to 1mM Gly 2.0$\pm$1.2 ms (range 0.8 to 4.6 ms), deactivation to 1s Gly $\tau_{\textrm{fast}}$ 15.5 ms and $\tau_{\textrm{fast}}$ 73.4 ms (MNTB mouse [\citenum{LeaoOleskevichEtAl:2004}]).

Decay 1.6 ms \citenum[mouse VCN,]{Oertel:1983}
Decay 5.4 ms [\citenum{OertelWickesberg:1993,WickesbergOertel:1993}]
Activation 2.0$\pm$1.2 ms Decay 5.3 ms (Gly puffs at 22$^\circ$C (Q$_{10}$ 2.1) in  guinea pig VCN [\citenum{HartyManis:1998}])

%                                & % RF Pre
DS axon terminals cover 300 $\mu$m of VCN (mouse [\citenum{OertelWuEtAl:1990}]).
AVCN collaterals centred on soma isofreq. as dend, 1 octave above and 2 oct below (gerbil [\citenum{ArnottWallaceEtAl:2004}])
SBW=5.1kHz$\pm$4.5 kHz all Ch, CS 4.66$\pm$4.45kHz 88$\pm$19\% suppression, CT 6.28$\pm$ 4.65kHz    96$\pm$5\% suppression [\citenum{RhodeGreenberg:1994b}]
%                                & % RF Post

%                                & %Position
(mice [\citenum{FerragamoGoldingEtAl:1998a}])
See Table 1 (cat [\citenum{SmithRhode:1989}])
$\sim$70 (high) $\sim$60 (low CF) per soma,
$\sim$1.7 per axon, FL $\sim$20 (highCF)
$\sim$10 (lowCF) (chinchilla [\citenum{JosephsonMorest:1998}])
%                                & %Number
1 or 2 on soma; many gly and mixed gly/GABA on trunks; see Table 1[\citenum{SmithRhode:1989}]
more FL vesicles on soma in high CF regions [\citenum{JosephsonMorest:1998}]
%                                &
1.2-3.5 msec shock to AN [\citenum{FerragamoGoldingEtAl:1998a,NeedhamPaolini:2003,Oertel:1983}]
Commissural DS units: 1.52 ms shock to cCN [\citenum{NeedhamPaolini:2006}].

% Smith and Rhode (1989) provided anatomical evidence that the OCs synapse on the
% type 1 multipolar cells in the cat PVCN. In the rodent,
% Ferragamo et al. (1998) showed that the D stellates, be-
% lieved to be the rodent equivalent of OCs, provide glycin-
% ergic inhibition to the type 1 multipolars. Pressnitzer et al.
% (2001) reported that transient choppers in the AVCN,
% believed to be type 1 multipolars, receive a wide-band
% inhibition that may arise from OCs. OC projections to the
% DCN have also been implicated in the wide-band glycin-
% ergic inhibition demonstrated physiologically in DCN
% principal cells, as well as in the cells designated type II or
% vertical (Caspary et al., 1987; Young et al., 1992; Nelken
% and Young, 1994; Backoff et al., 1997; Joris and Smith,
% 1998; Spirou et al., 1999; Davis and Young, 2000; Ander-
% son and Young, 2004). Our electron microscopy of OC
% terminals in the deep DCN shows that they can synapse
% on cell bodies or dendrites in this region, which is consis-
% tent with a potential influence of the OCs on these cell
% types.



% Sources of glycinergic cochlear nuclear inhibition
% Glycinergic inhibition is recorded consistently in T stellate
% cells spontaneously and in responses to shocks of the audi-
% tory nerve as prominent, rapid IPSPs. The latencies of IPSPs
% indicate that they are polysynaptic and arise through inter-
% neurons that are intrinsic to the slice. All distinct IPSPs in
% T stellate cells, as in other cells of the VCN, are blocked
% by strychnine, indicating that they are glycinergic (Wu and
% Oertel 1986).

% An ability to label glycinergic interneurons with antibod-
% ies to glycine conjugates allows the population of glycinergic
% neurons to be identified (Oertel and Wickesberg 1993).
% Three groups of cells account for immunopositive labeling:
% in the DCN, tuberculoventral cells (Osen et al. 1990; Saint
% Marie et al. 1991; Wenthold et al. 1987; Wickesberg et al.
% 1994) and cartwheel cells (Osen et al. 1990; Saint Marie et
% al. 1991; Wenthold et al. 1987), and in the VCN, multipolar
% cells (Schofield and Cant 1996; Wenthold 1987), which
% correspond to D stellate cells (Oertel et al. 1990).
% Tuberculoventral cells have been shown to provide disyn-
% aptic, glycinergic inhibition to T stellate cells in responses
% to shocks of the auditory nerve (Wickesberg and Oertel
% 1990). Although there is no doubt that tuberculoventral cells
% contribute to the disynaptic IPSPs, several experimental
% findings show that they do not mediate the long trains of
% IPSPs. First, tuberculoventral cells do not fire for prolonged
% periods when activated through eighth nerve inputs (Golding
% and Oertel 1997; Zhang and Oertel 1993). Second, long
% trains of IPSPs are preserved in slices in which the DCN
% was removed from the slice (Fig. 7).

% Considerable experimental evidence indicates that D stel-
% late cells are the source of the trains of IPSPs. First, it is the
% only class of glycine-immunopositive neurons in the VCN.
% Furthermore, pharmacological manipulations produce paral-
% lel changes in the firing of D stellate cells and the appearance
% of IPSPs in T stellate cells. 1) Late IPSPs in T stellate cells
% were evoked by strong shocks that lasted for hundreds of
% milliseconds. D stellate cells fire for long periods in re-
% sponses to strong shocks. 2) Both the trains of IPSPs of T
% stellate cells and the late firing of D stellate cells were
% blocked by APV. 3) Both the trains of IPSPs in T stellate
% cells and late firing of D stellate cells were promoted by
% application of GABAA antagonists. The results that D stellate
% cells contact T stellate cells and that they respond to weak
% shocks with single spikes monosynaptically indicate that
% they contribute to the disynaptic IPSP.


\subsubsection{Glycinergic inhibition from Tuberculoventral cells}



\subsubsection{Recurrent local excitation between T~stellate cells}

 Little synaptic depression
 SD less than bushy and octopus \citep{WuOertel:1987,ChandaXu-Friedman:2010,CaoOertel:2010}
 excitation of TS adapts less than other VCN neurons

% \citep{Oertel:2005} Excitatory
% interconnections between stellate cells produce prolonged excitation that
% presumably
% contributes to the shaping of responses to sound (175).
% How chopping responses are produced is not completely understood. It has
% been suggested that stellate cells integrate input from large numbers of
% auditory
% nerve fibers. However, stellate cells in mice have been shown to receive input
% from only a few (four to six) sharply timed auditory nerve fiber inputs (175).
% Activation of these inputs with trains of shocks produces entrained responses
% rather than chopping (172, 175), raising two questions: How are stellate cells
% prevented from encoding the timing of auditory nerve inputs after the initial
% action potential in response to sound, and how is their steady firing in
% response
% to tones produced from inputs that have strong onset transients?  


% Sources of polysynaptic excitation \citep{FerragamoGoldingEtAl:1998a}
% The late EPSPs observed in T stellate cells indicate that
% T stellate cells receive excitatory input from excitatory inter-
% neurons in the slices. In being separated from their natural
% synaptic inputs, isolated axons cannot contribute to polysyn-
% aptic responses. Monosynaptic responses have latencies be-
% tween 0.5 (synaptic delay) and Ç3 ms (2.5-ms conduction
% delay for an unmyelinated fiber of 0.5-mm plus 0.5-ms syn-
% aptic delay). Therefore EPSPs the latencies of which are
% ú3 ms are polysynaptic and must be generated by excitatory
% interneurons. Two other experimental observations confirm
% this conclusion. As cut axons have not been observed to fire
% spontaneously, the presence of spontaneous EPSPs is an
% indication of the existence of excitatory interneurons. Fur-
% thermore, the activation of EPSPs with the application of
% glutamate indicates that the dendrites of excitatory interneu-
% rons are accessible from the bath.
% T stellate cells are excitatory neurons known to terminate
% in the vicinity of T stellate cells. T stellate cells terminate
% locally in the multipolar cell area of the PVCN (Oertel et
% al. 1990; this study). This area is occupied by T stellate
% cells and occasional D stellate and bushy cells, some or all
% of which are therefore presumably their targets. The ultra-
% structure of T stellate cell terminals and functional studies
% of the inputs to the inferior colliculi is consistent with their
% being excitatory (Oliver 1984, 1987; Smith and Rhode
% 1989).
% The present experiments provide functional evidence in
% support of the conclusion that T stellate cells mediate late
% EPSPs. If T stellate cells are excited by other T stellate cells,
% then disynaptic EPSPs that reflect the firing of other stellate
% cells should be observed under similar conditions as stellate
% cell firing. The present experiments reflect the parallel nature
% of T stellate cell firing and late EPSPs under five experimen-
% tal conditions. 1) Stellate cells consistently are brought to
% threshold Ç1 ms after shocks to the auditory nerve. Disynap-
% tic EPSPs with latencies of Ç1.6 ms are observed but in the
% presence of monosynaptic EPSPs and disynaptic IPSPs the
% early disynaptic EPSPs are sometimes difficult to resolve.
% 2) Strong shocks evoke a long, slow depolarization in T
% stellate cells that causes T stellate cells to fire hundreds of
% milliseconds after a strong shock to the auditory nerve.
% Strong shocks also evoke very late EPSPs in T stellate cells.
% 3) APV reduces late firing and late EPSPs in T stellate cells.
% 4) The removal of extracellular Mg 2/ enhances firing as
% well as late EPSPs. 5) Strychnine and bicuculline enhance
% firing as well as late EPSPs in T stellate cells. In summary,
% although the results of the present experiments are consistent
% with the conclusion that T stellate cells excite one another,
% it does not rule out the possibility that other, hitherto un-
% known, cells contribute to the excitation.
% The only other known excitatory neurons that terminate
% in the vicinity of T stellate cells are granule cells. The den-
% drites of T stellate cells end in bushy branches, some of
% which often come near, but never penetrate, the layer of
% superficial granule cells that overlies them. It is conceivable,
% therefore, that granule cells could provide polysynaptic exci-
% tation.







\subsection{Acoustic Response of T stellate cells}


%  3.6$\pm$0.38 ms \citep{RhodeSmith:1986}
%  3.6$\pm$1.2 ms \citep[anesthetised cat][]{RhodeKettner:1987}
% 3.5 ms \citep[unanesthetised cat][]{RhodeKettner:1987}
% Acoustic threshold
%  depolarisation min SPL 31.7$\pm$2.9,
%  AP min SPL 41.8$\pm$3.8  \citep{PaoliniClareyEtAl:2004}


\begin{itemize}
\item regular, tonic response to tones \citep{RhodeOertelEtAl:1983,SmithRhode:1989,BlackburnSachs:1989}
\item ``Chopping'' precise regular timing that degrades throughout stimulus\citep{YoungRobertEtAl:1988,BlackburnSachs:1989}
\begin{itemize}
\item sustained (70\%) $\rightarrow$ constant rate, ISI histogram sharp, CV < 0.3, CV constant
\item transient (30\%) $\rightarrow$ rate decreases, CV starts below 0.3 then varies
\end{itemize}
\item Inhibition
\begin{itemize}
\item Gly, GABA tuned on frequency to reduce peak excitation \citep{CasparyBackoffEtAl:1994}
\item inhibitory side bands mainly D stellate \citep{FerragamoGoldingEtAl:1998a} but periolivary also contribute \citep{AdamsWarr:1976,Adams:1983,ShoreHelfertEtAl:1991,OstapoffBensonEtAl:1997}
\end{itemize}
\end{itemize}
\citep{PalombiCaspary:1992,RhodeSmith:1986,NelkenYoung:1994,PaoliniClareyEtAl:2005,PaoliniClareyEtAl:2004}

\begin{itemize}
\item sustained firing despite AN adaptation
\begin{itemize}
\item signals the sound intensity consistently, hence precise level information
\item Phasic also do level, but tonic suits encoding of spectrum across population since encoding the peaks and valleys is relatively independent of time after onset of sound \citep{BlackburnSachs:1990,May:2003,MayPrellEtAl:1998,MaySachs:1998}
\item suits encoding of envelope of sounds, important for speech (envelops under 50 Hz \citep{ShannonZengEtAl:1995}
\end{itemize}
\item AM coding in choppers encoded over wide range of intensities \citep{RhodeGreenberg:1994,FrisinaSmithEtAl:1990}
\begin{itemize}
\item other work in Am coding by CN neurons  \citep{Moller:1972,Moller:1974a,Moller:1974,MooreCashin:1974,Frisina:1984,PalmerWinterEtAl:1986,KimRhodeEtAl:1986,WinterPalmer:1990a,Palmer:1990,PalmerWinter:1992,FrisinaSmithEtAl:1990a,Frisina:1983,GorodetskaiaBibikov:1985,RhodeGreenberg:1994,ShofnerSheftEtAl:1996,FrisinaKarcichEtAl:1996,DAngeloSterbingEtAl:2003,Aggarwal:2003}
\end{itemize}
\item phasic firing in AN maintained by bushy
\begin{itemize}
\item phasic info important: enhances formant transitions, and provides accurate information about the location of sound sources even in reverberant environments, critical in hearing \cite{DelgutteKiang:1984,DelgutteKiang:1984a,DelgutteKiang:1984b,DelgutteKiang:1984c,DelgutteKiang:1984d,DavoreIhlefeldEtAl:2009}
\end{itemize}
\end{itemize}


\subsection{Mechanisms of Tonic Firing Obscure Temporal Features}


\begin{itemize}
\item sFSL largest in TS of core VCN units by 1msec -> onset inhibition + longer integration time \citep{GisbergenGrashuisEtAl:1975,GisbergenGrashuisEtAl:1975a,GisbergenGrashuisEtAl:1975b,YoungRobertEtAl:1988,PaoliniClareyEtAl:2004}
  lv\item integration window longest for choppers \citep{McGinleyOertel:2006}
\item inhibition from high CF units alters FSL to tones \citep{Wickesberg:1996}
\item Onset: Volley of Excitation + feedforward excitation + DS inhibition
\item After onset: Phasic excitation + feedforward excitation + NMDA activation + TV inhibition (+ small DS inhibition) + GABA inhibition = stable excitation but loss of temporal features
\end{itemize}

\citep{JorisSmithEtAl:1994}


\subsection{Neuromodulatory Effects in T~stellate Cells}


% Diverse mechanisms are known to modulate glutamate re-
% lease from the calyx (Trussell, 2002). For example, presynaptic of
% glycine and GABAA receptors facilitate release (Turecek and
% Trussell, 2001, 2002), whereas metabotropic GABA (Takahashi et
% al., 1998), glutamate (Takahashi et al., 1996), adenosine (Kimura
% et al., 2003), and noradrenaline receptors (Leao and Von Gers-
% dorff, 2002) suppress release. By themselves, modulatory systems
% in the MNTB may have little impact on the reliability or timing of
% calyceal transmission. However, their effectiveness may become
% apparent in the context of the glycinergic transmission we de-
% scribe here. Thus, presynaptic modulators may act in synergy
% with the postsynaptic glycinergic IPSPs to enhance or diminish
% excitatory transmission.


\begin{itemize}
\item sensitive to neuromodulatory currents \citep{FujinoOertel:2001}
\begin{itemize}
\item high input resistance $\rightarrow$ amplify small current inputs \citep{FujinoOertel:2001}
\item no LKT in TS,  LKT makes bushy and octopus insensitive to steady currents \citep{OertelFujino:2001,McGinleyOertel:2006}
\item Ih higher in TS \& activated more at lower potentials than in bushy and octopus, so that it is less active at rest
\item high resistance $\rightarrow$ greater voltage changes in small modulating current $\rightarrow$ Ih can be modulated by G-protein coupled receptors, hence making TS more excitable when Ih activated \citep{RodriguesOertel:2006}
\end{itemize}
\end{itemize}

\begin{enumerate}
\item Driving inputs
\end{enumerate}
Proximal dendrites and at the soma:

\begin{itemize}
\item ANF provide glutamatergic excitation for T stellates  \citep{Cant:1981,FerragamoGoldingEtAl:1998a,Alibardi:1998a}
\begin{itemize}
\item only 5 or 6 in mice \citep{FerragamoGoldingEtAl:1998a,CaoOertel:2010}
\end{itemize}
\item Recurrent excitation from other T stellate cells \citep{FerragamoGoldingEtAl:1998a}
\item Glycine from DS cells \citep{FerragamoGoldingEtAl:1998a}
\item Glycine from TV cells \citep{WickesbergOertel:1990,ZhangOertel:1993b}
\item Neuromodulatory
\end{itemize}
     No signs of PSP or PSCs hence distal or G-protein coupled, effects on time-course minimal

\subsubsection{GABA$_A$ergic influence}

%GABAergic GABA$_{\textrm{A}}$ receptor  (bicuculine-sensitive VCN T stellate cell, mouse slice preparation [\citenum{FerragamoGoldingEtAl:1998}], chinchilla [\citenum{JosephsonMorest:1998}])
%Ferragamo et al. 1998 found no GABAergic IPSPs but the cells were still sensitive to bicuculine

GABA staining \citep{SaintMorestEtAl:1989}

\GABAa Receptor Antagonist Bicuculline Alters Response Properties
of Posteroventral Cochlear Nucleus Neurons \citep{PalombiCaspary:1992}
GABAergic inhibitory inputs control the
post-onset discharge rate of some PVCN neurons. They may sup-
press tonic activity, resulting in more phasic discharge patterns.

% CasparyPalombiEtAl:2002
%      Caspary, Palombi, Hughes       2002 Hear Res 168, 163--73
%      GABAergic} inputs shape responses to amplitude modulated stimuli in the inferior colliculus

% BackoffShadduckEtAl:1999
%      Backoff, Shadduck Palombi, ... 1999 Hear Res 134, 77--88
%      Gamma-aminobutyric acidergic and glycinergic inputs shape coding of amplitude modulation in the chinchilla cochlear nucleus

% BackoffPalombiEtAl:1997
%      Backoff, Palombi, Caspary      1997 Hear Res 110, 155--63
%      Glycinergic and {GABAergic} inputs affect short-term suppression in the cochlear nucleus

% CasparyHelfertEtAl:1997
%      Caspary, Helfert, Palombi      1997 in: Acoustic Signal Processing in the Central Auditory System
%      The role of {GABA} in shaping frequency response properties in the chinchilla inferior colliculus

% PalombiCaspary:1996
%      Palombi, Caspary               1996 Hear Res 100, 41--58
%      Physiology of the young adult {Fischer} 344 rat inferior colliculus: responses to contralateral monaural stimuli

% CasparyBackoffEtAl:1994
%      Caspary, Backoff, Finlayson... 1994 J Neurophysiol 72, 2124--33
%      Inhibitory inputs modulate discharge rate within frequency receptive fields of anteroventral cochlear nucleus neurons

% PalombiBackoffEtAl:1994
%      Palombi, Backoff, Caspary      1994 Hear Res 75, 175--83
%      Paired tone facilitation in dorsal cochlear nucleus neurons: a short-term potentiation model testable in vivo

% CasparyPalombi:1993
%      Caspary, Palombi               1993 Assoc Res Otolaryngol Abstr 109, 
%      GABA} inputs control discharge rate within the excitatory response area of chinchilla inferior colliculus neurons

% CasparyPalombiEtAl:1993
%      Caspary, Palombi, Backoff, ... 1993 in: The Mammalian Cochlear Nuclei: Organisation and Function
%      GABA} and glycine inputs control discharge rate within the excitatory response area of primary-like and phase-locked {AVCN} neurons

% Markers of GABAergic neurotransmission in the cochlear
% nucleus reveal the presence of both cell bodies and terminals
% that could be GABAergic. Antibodies to GABA conjugates
% and to glutamate decarboxylase (GAD) generally label neu-
% rons that are functionally GABAergic. Occasionally GAD
% and GABA are associated with neurons that are functionally
% glycinergic; cartwheel cells of the DCN, for example, are
% labeled for GABA and GAD yet seem to be glycinergic
% (Golding and Oertel 1997; Golding et al. 1996). Function-
% ally GABAergic neurons and their terminals are labeled con-
% sistently for GABA and GAD, however, indicating that the
% source of GABAergic input in T stellate cells would be
% expected to be labeled. GABAergic input could arise from
% neurons intrinsic to the cochlear nuclei or from sites external
% to the nucleus, such as the superior olivary nucleus (Saint
% Marie et al. 1989). Only GABAergic neurons in the cochlear
% nuclei can function in polysynaptic circuits in slices as was
% observed in the present study, however, isolated terminals
% of extrinsic sources cannot be activated synaptically.
% Labeling for GAD and GABA is associated strongly with
% regions that contain granule cells, the molecular and fusiform
% cell layers of the DCN and the superficial granule cell do-
% main of the VCN. In cats and guinea pigs, antibodies to
% GABA conjugates and to GAD, a biosynthetic enzyme, have
% been shown to label specific groups of cells and terminals
% (GABA: Kolston et al. 1992; Osen et al. 1990; Wenthold et
% al. 1986; GAD: Adams and Mugnaini 1987; Moore and
% Moore 1987; Mugnaini 1985; Saint Marie et al. 1989). In
% the DCN, the majority of cell bodies and puncta that were
% labeled with antibodies against GABA and GAD lie in the
% superficial and fusiform cell layers (Adams and Mugnaini
% 1987; Kolston et al. 1992; Moore and Moore 1987; Mugnaini
% 1985; Osen et al. 1990; Saint Marie et al. 1989; Wenthold
% et al. 1986). Labeled neurons are cartwheel, stellate, and
% Golgi cells. As none of these neurons make direct or indirect
% connections with the VCN, it is unlikely that cartwheel,
% superficial stellate or Golgi cells of the DCN contribute to
% GABAergic inhibition in T stellate cells of the VCN.
% GABAergic input to T stellate cells of the VCN could
% arise from Golgi cells in the superficial granule cell domain
% either mono- or disynaptically. Labeled cell bodies identified
% as Golgi cells were observed to be associated with the super-
% ficial granule cell layer (Mugnaini 1985). These neurons
% terminate locally in the superficial granule cell layer with
% very dense terminal arbors that abut the underlying large
% cell area (Ferragamo et al. 1997). The dendrites of D stellate
% cells lie just beneath the superficial granule cell domain,
% poised to be contacted by Golgi cells proximally and distally,
% indicating that D stellate cells could mediate GABAergic
% responses. Furthermore, some of the branches of the distal
% dendrites of T stellate cells approach the superficial granule
% cell domain. If Golgi cells contact T stellate cells directly,
% those contacts can only be on distal dendrites. In contrast
% with glycinergic IPSPs, GABAergic IPSPs were not promi-
% nent in T or D stellate cells; IPSPs that remained in the
% presence of strychnine were small and inconspicuous, if
% present. There are four possible reasons for this observation:
% the synaptic currents associated with GABAergic inputs
% were relatively slower and weaker, they were generated rela-
% tively far from the somatic recording site, they were medi-
% ated through an excitatory interneuron, or there were presyn-
% aptic GABAergic receptors present.


% \citep{AwatramaniTurecekEtAl:2005}
% To ascertain if GABAergic transmission persisted
% in still older animals (P17–P22 rats), we positioned the stim-
% ulating electrode after the slices were bathed in 500 nM
% strychnine. Under these conditions, small, slow IPSCs (56 Ϯ
% 19 pA, wd = 24 Ϯ 4 ms; n ϭ 4; data not shown) could be
% evoked, indicating that weak GABAergic inputs persist in
% more mature MNTB

a. Golgi cells (GABA)

\begin{itemize}
\item no IPSPs or IPSCs but presence of GABAa receptors and response changes to bicuculine \citep{WuOertel:1986,OertelWickesberg:1993,FerragamoGoldingEtAl:1998a}
\item dend filter obscures PSPs
\item Golgi cells are GABAergic and lie within the granule cell domains around the VCN and terminate near the fine distal dendrites of T stellate cells
\end{itemize}

b. Periolivary cells (GABA + GAD - glutamic acid decarboxylase)

\begin{itemize}
\item observed in PVCN close to TS \citep{AdamsMugnaini:1987}
\item GAD effectively Glycine \citep{GoldingOertel:1997}
\item can also arise from GABAergic neurons in ipsi LNTB and DM Periolivary
\end{itemize}

c. VNTB cells (ACh)

\begin{itemize}
\item collateral branches of OC go to GCD \citep{MellottMottsEtAl:2011,SherriffHenderson:1994,OsenRoth:1969}
\item TS have nicotinic and muscarinic ACh receptors \citep{FujinoOertel:2001}
\item ACh input to TS, together with OC-cochlea, enhances spectral peaks in noise  \citep{FujinoOertel:2001}
\end{itemize}

The olivocochlear bundle, whose terminals contain high concentrations of AChE, sends collaterals to the CN with most terminals in the GCD \citep{MellottMottsEtAl:2011,SherriffHenderson:1994,OsenRoth:1969}. %(Schuknecht,Churchill \& Doran1959, Shute \&Lewis 1965, Rasmussen 1967, Osen \& Roth 1969).
%The AChE-positive terminals of this fiber bundle appearto be limited in their distribution to the molecular granule and cell layers, where they aggregate into glomeruli(Osen \& Roth 1969).

In rats, onset choppers are monosynaptically excited by shocks to the OCB \citep{MuldersPaoliniEtAl:2003,MuldersWinterEtAl:2002,MuldersPaoliniEtAl:2009}.



d. NE and 5HT

\begin{itemize}
\item Raphe nuclei (5HT)
\item Locus coeruleus Peribrachial cells (NE)
\item both terminate in PVCN \citep{KlepperHerbert:1991,Thompson:2003,ThompsonLauder:2005,Thompson:2003a,ThompsonWiechmann:2002,BehrensSchofieldEtAl:2002,ThompsonThompson:2001,ThompsonThompson:2001a,ThompsonMooreEtAl:1995,ThompsonThompsonEtAl:1994}
\item both increase firing in T stellates \citep{OertelWrightEtAl:2010} in presence of glut and gly blockers -> hence act on post synapse (TS cells)
\item both G-protein coupled, both act on either pre or post synaptic cells
\item NE affects probability of release at calyx of Held
\item NE increases firing rate of choppers \citep{KosslVater:1989,Ebert:1996}
\item 5HT excites or inhibits choppers \emph{in vivo} \citep{EbertOstwald:1992}
\end{itemize}




\subsection{Major Ascending Output}
\label{sec-1_6}


\begin{itemize}
\item review \citep{DoucetRyugo:2006}
\end{itemize}

TS cell axons exit the CN through the trapezoidal body, cross the midline and ultimately terminate in the cIC \citep{Adams:1979}

\begin{itemize}
\item Collaterals: local, DCN, LSO cVNTB cVNLL \citep{Warr:1969,SmithJorisEtAl:1993,Thompson:1998,DoucetRyugo:2003}
\item Deep DCN (bulk of acoustic input?)
\item in mice TS terminals > ANF \citep{CaoMcGinleyEtAl:2008}
\item on CF \citep{SmithRhode:1989,FriedlandPongstapornEtAl:2003,DoucetRyugo:1997}
\item DCN review \citep{OertelYoung:2004}
\item LSO excitation
\item TS project to LSO \citep{Thompson:1998,DoucetRyugo:2003,ThompsonThompson:1991a}
\item LSO detect interaural intensity differences primarily from ipsi Bushy cells and contra MNTB (inhib)
\item OC feedback
\end{itemize}

a. MOC: cVNTB excitation

\begin{itemize}
\item involved in efferent feedback loop, ACh-ergic MOC neurons TS synapses in cVNTB \citep{WarrBeck:1996,Warr:1992,Warr:1982,VeneciaLibermanEtAl:2005,ThompsonThompson:1991,SmithJorisEtAl:1993}
\item feedback direct to TS is positive, but efferent MOC-OHC-ANF reduces activation of ANF \citep{WarrenLiberman:1989,WiederholdKiang:1970}
\item other \citep{RobertsonMulders:2000,WinterRobertsonEtAl:1989}
\end{itemize}


b. LOC

\begin{itemize}
\item TS terminate in vicinity of LOC neurons \citep{Warr:1982,ThompsonThompson:1988,ThompsonThompson:1991,DoucetRyugo:2003}
\item feedback through LOC $\rightarrow$ cochlea $\rightarrow$ ANF loop $\rightarrow$ TS affect/regulate response of LOC. hence ANF.
\item LOC balance inputs from both ears \citep{DarrowMaisonEtAl:2006}
\end{itemize}

c. VNLL

\begin{itemize}
\item The functional consequences of these direct and indirect connections with TS cells with the IC are not well understood
\end{itemize}

% \citep{CantBenson:2003}
% Type I multipolar cells of the ventral cochlear
% nucleus encode complex features of the stimulus important
% for the recognition of natural sounds and are a major
% source of excitatory input to the inferior colliculus
% One group of multipolar cells (to be referred to here as
% type I multipolar cells) described by Smith and Rhode [220]
% appear to be equivalent to the T-stellate cells described in
% mouse [151], the planar cells described in rat [55], and per-
% haps also to the type I stellate cells described in the an-
% terior part of the AVCN of cat [36]. They probably also
% include many of the neurons studied by Feng and coworkers
% [58,165]. The dendrites of type I multipolar cells are ori-
% ented such that they would be expected to receive auditory
% nerve inputs from a restricted frequency range [54,55,151].
% Compared to other cell types in the VCN, these neurons re-
% ceive relatively sparse synaptic inputs to their cell bodies
% [12,102,220,237]. Input from auditory nerve fibers and from
% abundant terminals containing glycine, GABA, or both are
% distributed on the dendritic surface [103,113,200,256]. The
% few terminals that contact the somatic surface usually appear
% to be inhibitory, but some cells receive apparently excita-
% tory contacts on the axonal hillock and initial segment [102].
% Most multipolar cells in the anterior AVCN also receive few
% somatic inputs (type I stellate cells, [36,102]). These cells
% are on average smaller than those in the posterior AVCN
% and anterior PVCN [44], and there may be some subtle dif-
% ferences in their synaptic organization [102]. Whether the
% multipolar cells in the anterior AVCN share all of the prop-
% erties ascribed to the large type I multipolar cells described
% by Smith and Rhode [220] remains to be determined.
% The type I multipolar cells are narrowly tuned and respond
% to tone bursts with regular trains of action potentials, a re-
% sponse referred to as a “chopper” pattern (e.g. [168,220]).
% Neurons that exhibit chopper responses can differ substan-
% tially in their dendritic morphology ([58,179,194]; cf. [30])
% which suggests that a further subdivision of this class of
% neurons may be possible. In mouse, the equivalent cells
% (T-stellate cells) appear to integrate input from the auditory
% nerve with that from other multipolar cells of both types
% [61]. In general, the response properties of chopper units
% suggest that they play an important role in encoding com-
% plex acoustic stimuli, perhaps including speech sounds (e.g.
% [26,131,180]).
% The projection pattern of type I multipolar cells is illus-
% trated in Fig. 2F. The axons leave the cochlear nucleus via
% the trapezoid body [55,151,220,245], where they make up
% the ventral thin fiber component [31,215,245,248]. Possibly
% because they are thinner than the axons of the other cell
% types, there have been few reports of successful intra-axonal
% injections of these fibers so it is not entirely clear whether
% the different projections arise from the same or different
% populations. Multipolar cells are a major source of input
% from the cochlear nucleus to the contralateral inferior col-
% liculus [2,12,24,33,37,102,154,156,191,205]. It seems likely
% that most, if not all, type I multipolar cells participate in
% this projection [102]. The projection arises from neurons
% throughout the VCN, including all but the most anterior part
% of the AVCN and the octopus cell area in the PVCN. The
% same neurons that project to the inferior colliculus also send
% collateral branches to the DCN ([4]; also, [55,61,167,217]).
% In both targets, the synaptic terminals contain round synap-
% tic vesicles, compatible with an excitatory effect (IC: [154];
% DCN: [220]). The projections from the cochlear nucleus
% have been shown to directly contact neurons in the infe-
% rior colliculus that project to the medial geniculate nucleus
% [156]. A smaller projection to the ipsilateral inferior col-
% liculus also arises from multipolar cells in the VCN (e.g.
% [2,154]). The axons that make up this projection travel in
% the lateral trapezoid body tract [245,248].
% Multipolar cells in the VCN give rise to projections to
% the dorsomedial periolivary nucleus in cat [215] or supe-
% rior paraolivary nucleus in rat and guinea pig [64,201], to
% the ventral nucleus of the trapezoid body [64,215] and to
% the ventral nucleus of the lateral lemniscus [64,91,206,215].
% The cells that give rise to these projections are probably the
% type I multipolar cells [218]. Although it has not been estab-
% lished definitely, it seems likely that these projections arise
% from the same cells that project to the inferior colliculus.
% Multipolar cells of unknown type project to the ipsilateral
% lateral superior olivary nucleus and the lateral periolivary
% region in cats [41,233,248]. In addition to their projection
% to the DCN, the type I multipolar cells give rise to exten-
% sive collateral branches within the VCN [4,61,151,220,238].
% These appear to play an important role in shaping late re-
% sponses of cells in the VCN to auditory nerve stimulation
% (e.g. [61]).

























%\section{TS cells in birds}
\label{sec-1_7}

















VCN analog is the \emph{nucleus angularis}
%\section{Summary}
\label{sec-1_8}


{\it As a population, T stellate cells encode the spectrum of sounds. They
receive acoustic input from the auditory nerve fibers. Several
mechanisms contribute to that transformation: Feed-forward excitation
through other T stellate cells, co-activation of excitation and
inhibition, reduction in synaptic depression, and the amplification of
excitatory synaptic current over time through NMDA receptors. They
deliver that information to nuclei that make use of spectral
information.  T stellate cells terminate in the DCN, to olivocochlear
efferent neurons, to the lateral superior olive, to the contralateral
inferior colliculus. These targets use spectral information to
localize sounds, to adjust the sensitivity of the inner ear, and to
recognise and understand sounds. Birds also process sounds through
neurons that resemble T stellate cells in their projections and also
in their cellular properties, attesting to the fundamental importance
that T stellate-like cells have for hearing in vertebrates.}




%%% Local Variables:
%%% mode: latex
%%% mode: tex-fold
%%% TeX-master: "LiteratureReview"
%%% TeX-PDF-mode: nil
%%% End:



%\section{D~stellate cells}

%\subsubsection{Background}




% The glycine-positive, presumptive commissural neu-
% rones differed clearly from most of the surrounding cells
% by exhibiting only very few perisomatic and peridendritic
% (single-glycine- and double-labelled) puncta (Figs. 4C,
% 5 C). In fact the coverage of these cells appeared to be
% even less than that of the octopus cells, although the
% individual puncta appeared larger. Isolated dendritic
% segments were distinguishable because of their glycine-
% immunoreactivity.
% \citep{KolstonOsenEtAl:1992}


% \citep{PaoliniClark:1999} Long-lasting hyperpolarization was also seen in
% neurons during the presentation of frequency tones just above the high-frequency
% edge of their response area. With increasing click intensity both the
% depolarization and hyperpolarization increased in amplitude
 

% Evidence for inhibition on\OnC neurons can also be seen with
% click stimuli. Hyperpolarization was observed in this investi-
% gation in the presence or absence of action potentials evoked
% by a single click stimulus. Click stimulation would excite not
% only a large proportion of excitatory projections to\OnC neurons
% but would also evoke polysynaptic inhibitory and excitatory
% drive from intrinsic or extrinsic sources. The resultant hyper-
% polarization may be due to interplay between such excitatory
% and inhibitory influences. On presentation of successive click
% stimuli, hyperpolarization was not evident, although action
% potential generation decreased, suggesting the presence of in-
% hibitory influences.
% However, in past in vivo intracellular investigations, inhibi-
% tion was not consistently seen. Smith and Rhode (1989) rarely
% observed dips in the level of depolarization or distinct hyper-
% polarization below the cell’s resting potential that might indi-
% cate inhibition, whereas Feng et al. (1994) did observe hyper-
% polarizing influences in a Ct neuron akin to those observed in
% this investigation. This inconsistency may in part be related to
% the frequency of presentation as hyperpolarizing responses
% below resting potential were only seen in this investigation
% when tones were presented at or above the high-frequency edge
% of the neurons’ response area. The use of a nonbarbiturate
% anesthetic may also account for the presence of hyperpolariza-
% tion seen in this investigation. The ability of barbiturates to
% influence inhibitory mechanisms was well documented by
% Evans and Nelson (1973) in the cochlear nucleus. In this
% investigation, we used urethan, which when compared with
% barbiturates appears to have only minor effects on synaptic
% transmission (Crawford 1970; Maggi and Meli 1986).




% Further evidence for the involvement of inhibitory mecha-
% nisms in shaping the\OnC response is provided by Ferragamo et
% al. (1998). They demonstrated with an in vitro preparation that
% D-stellate cells receive both glycinergic and GABAergic inhi-
% bition. D-stellate cells responded to auditory nerve stimulation
% with a fast followed by a slow depolarization. Although there
% was an absence of hyperpolarization, application of picrotoxin,
% a GABAA receptor blocker, to the cochlear nucleus slice en-
% hanced the firing of D-stellate cells. Ferragamo et al. (1998)
% propose that this GABAergic inhibition may originate from
% Golgi cells in the superficial granule cell domain. This re-
% sponse to auditory nerve stimulation is also seen in a stellate
% cell response type to intracochlear electrical stimulation re-
% corded in vivo (Paolini and Clark 1998). This response type
% also showed little visible evidence of inhibition, although a




% spike could not be evoked on the second excitatory postsyn-
% aptic potential response. Results obtained in this investigation
% and those of previous investigations (Ferragamo et al. 1998;
% Paolini and Clark 1998) suggest that polysynaptic inhibitory
% drive may play an important role in regulating the influence of
% excitation on these\OnC neurons.
% In addition to GABAergic inhibition, glycinergic inhibitory
% input may also result from tuberculoventral cells in the dorsal
% cochlear nucleus (DCN) (Saint Marie et al. 1991; Wickesberg
% and Oertel 1990). Our recent study has shown that the DCN is
% an important contributor to on-frequency inhibition of VCN
% activity (Paolini et al. 1998). Suppression of DCN by applica-
% tion of muscimol, a GABA agonist, resulted in a decrease in
% response threshold for units in the VCN (Paolini et al. 1998).
% Extrinsic inhibitory input may also originate from the superior
% olivary complex, which has been shown to be an important
% source of inhibitory input to the cochlear nucleus (Ostapoff et
% al. 1997; Saint Marie et al. 1993).
% Inhibitory projections may contribute to the\OnC response
% through shunting inhibition, as proposed by Smith and Rhode
% (1989). They suggest that the output of the cell may be reduced
% when the membrane potential is close to the reversal potential
% of chloride ions. At this potential, the amplitude of any depo-
% larization will be substantially reduced when the chloride chan-
% nel is activated, leading to increased conductance. Shunting
% inhibition may account for the drop in the level of depolariza-
% tion to sustained levels seen in this investigation. The extent of
% depolarization seen during the sustained component changes
% with frequency of presentation. In response to tones presented
% 30 – 40 dB above threshold at the low-frequency edge of the
% neurons’ response area, the amplitude of the sustained depo-
% larization approached that of the initial component. This in-
% crease in the level of depolarization observed during the sus-
% tained component may result from a decrease of inhibitory
% drive or of shunting inhibition.
% Alternatively, the increase in the level of sustained depolar-
% ization seen on presentation of tones on the low-frequency side
% of CF may also be explained by an increase in auditory nerve
% fibre convergence. As the frequency is decreased and at high
% intensities more auditory nerve fibres are recruited as low-
% frequency tails are commonly seen in the tuning characteristics
% of these fibres. However, an increase in the level of depolar-
% ization and spike activity was not seen during the initial com-
% ponent of the response when tones were presented at these
% frequencies. This suggests that, although more auditory nerve
% fibres are excited by low-frequency stimuli at high intensities,
% it is unlikely that they all contribute to the summation of inputs
% to\OnC neurons.




%\citep{SmithMassieEtAl:2005}

% Onset choppers. 
The combined evidence from several
 different studies (Cant and Gaston, 1982; Wenthold, 1987;
 Kolston et al., 1992; Shore et al., 1992; Schofield and Cant,
 1996b; Alibardi, 1998; Needham and Paolini, 2003;
 Palmer et al., 2003; Arnott et al., 2004) suggests that the
 large VCN multipolar cells whose response to short tones
 have been labeled onset-chopper (OnC; Rhode et al., 1983a;
 Smith and Rhode, 1989) closely resemble a population of
 large glycinergic multipolar cells that project to the opposite CN. Single-cell labelling studies in the cat (Smith and
 Rhode, 1989) and guinea pig (Palmer et al., 2003; Arnott
 et al., 2004) have reported that the axons of these cells
 provide collateral innervation to both the ipsilateral dorsal and ventral cochlear nuclei before heading dorsally out
 of the CN. 

% The only direct evidence that the glycinergic
%  multipolar cells projecting to the opposite cochlear nucleus
%  and the cells with\OnC response features are the same
%  population is one juxtacellularly labeled cell in the guinea
%  pig (Arnott et al., 2004) and one intra-axonally labeled cell
%  in the cat that we described in a preliminary report \citep{SmithMassieEtAl:2005}




Type II multipolar cells of the ventral cochlear
 nucleus are the source of widespread inhibitory inputs
 to the ipsilateral dorsal cochlear nucleus and the
 contralateral dorsal and ventral cochlear nuclei \citep{CantBenson:2003}.
 A second, less plentiful, group of large multipolar cells
 described by Smith and Rhode in the cat (\citealt{SmithRhode:1989}; referred
 to here as type II multipolar cells) probably corresponds
 to the D-stellate cells in mouse \citep{Oertel:1983} and to the radiate
 cells in rat \citep{DoucetRyugoEtAl:1999} and also probably to the commissural cells
 described in a number of species.% [13,42,113,204,213,254].
 These neurons can be distinguished from the type I multipolar cells (TS cells) in a number of ways. On average, they are
 larger, and their dendrites radiate across the isofrequency
 laminae of the VCN. %[54,61,151,169]. 
Both the dendrites
 and the somatic surface are covered with synaptic terminals, many of which appear to arise from the auditory nerve.% [13,220]. (The so-called type II stellate cells in the anterior
% AVCN [36] are also characterized by extensive somatic in-
% nervation, but as noted above, it has not been established
% whether the cells in the anterior AVCN should be consid-
% ered to be a different population from the multipolar cells in
% the rest of the AVCN and in the PVCN.) 
%Few of the synap-
% tic terminals on the type II multipolar cells are glyciner-
% gic [113], but many of the cells themselves are glycinergic
% [23,54,113,254]. Type II multipolar cells give rise to termi-
% nal dendritic tufts that characteristically extend into the gran-
% ule cell layer bordering the VCN ([55,151,179,204,220];
% cf. [142]).
% The type II multipolar cells are broadly tuned and corre-
% spond to a group of units characterized physiologically as
% “onset chopper” (OnC ) units, which show a well-timed on-
% set response followed by a few regularly timed spikes that
% are not sustained throughout the stimulus as are those of
% “chopper” units [168,169,220]. 


\subsubsection{TODO Cellular Mechanisms of D~stellate Cells}

DS cells have the largest cell body in the \VCN  (27 \um~guinea pig: \citealt{ArnottWallaceEtAl:2004},
22--26 \um rat: \citealt{DoucetRyugo:1997}, 20--30 \um~rat: \citealt{PaoliniClark:1999}). 

% The multipolar and cross-sectional area  (cat:
% \citealt{ReddCahillEtAl:2002,SmithRhode:1989}, rat:
% \citealt{DoucetRyugoEtAl:1999DoucetRyugo:1997,DoucetRyugo:2006}, guinea pig:
% \citealt{PalmerWallaceEtAl:2003}).


% Cross-sectional area
% 963          (cat [\citenum{SmithRhode:1989}]) 
% 501{$\pm$}168  (rat [\citenum{DoucetRyugoEtAl:1999}])
% 466{$\pm$}137 (rat [\citenum{DoucetRyugo:1997}]) 
% 418{$\pm$}140 (rat [\citenum{DoucetRyugo:2006}])
% 571{$\pm$}228  (cat [\citenum{ReddCahillEtAl:2002}]) 
% 450           (guinea pig [\citenum{PalmerWallaceEtAl:2003}])                  


Four to 5 apinous dendrites eminate from D stellate cells in rats and span 250--350 \um in all directions \citep{DoucetRyugo:1997}.
%]) with aspinous dendrites, 
%4 of 5 cells had 4 main dendrites, total dend length 6222 to 7351 \um
%(mean-6665 \um), 
% Dendrites extended widely in all directions. ,
%$\sim$70 \um perpendicular to AN 3-6 primary dendrites at right angles to AN        [\citenum{SmithRhode:1989}]
%& 
%axon width 0.7-1.2 \um [\citenum{OertelWuEtAl:1990}]\\\hline

Intracellular recordings in DS cells shows mixed behaviour between phasic and
regular firing, and generally have low membrane resistance.  DS cell action
potentials have double exponentials with undershoot
\citep{PaoliniClark:1999,WuOertel:1984}.  Membrane properties of \OnC cell have
not been adequately characterised, but the information that is available (D
stellate in mice \citealt{OertelWuEtAl:1990}) that suggests the low-threshold
potassium channel that is important in extending the phase-locking range of
bushy cells \citep{ManisMarx:1991,Oertel:1983} is not present in \OnC neurons
\citep{WhiteYoungEtAl:1994}.  The intermediate property was classified in mice
as Type I-i have high thresholds mediated by small \IKLT
\citep{RothmanManis:2003b}.
%&     Fast      & Linear \citep{PaoliniClark:1999} & 
Small amounts of \IKLT contribute to the low membrane resistance (40 M$\Omega$ in mice \citealt{OertelWuEtAl:1990},  $96.2 \pm 27.8$ M$\Omega$ mice: \citealt{FerragamoGoldingEtAl:1998a}.



% \subsubsection{Synaptophysiology of D~stellate Cells}

% \citep{FerragamoGoldingEtAl:1998a}
% % D stellate cells received glycinergic inhibition. Figs. 13A
% % and 14 show that 1 mM strychnine eliminated the occasional
% % IPSPs but that glycinergic inhibition in D stellate cells did
% % not prominently affect synaptic responses.
% % To test whether NMDA receptors mediate the long-lasting
% % firing of D stellate cells in responses to shocks, 100 mM
% % APV was applied to the bath. Figure 13A shows that APV
% % reversibly eliminated the long-lasting depolarization. The
% % remaining early excitation was blocked reversibly by DNQX
% % (n Å 2/2). To determine whether NMDA receptors were
% % intrinsic to the recorded cell or on excitatory interneurons,
% % the voltage dependence of the late response was examined
% % (Fig. 13B). The long, late depolarization that caused the D
% % stellate cell to fire for Ç200 ms was shortened to Ç100 ms
% % when the cell was hyperpolarized, this is consistent with
% % synaptic excitation mediated by NMDA receptors that were
% % intrinsic to the D stellate cell.
% % GABAergic inhibition plays a prominent role in the synap-
% % tic responses of D stellate cells. Even in the absence of
% % visible IPSPs, picrotoxin, a blocker of GABAA receptors,
% % enhanced the firing of the cell (Fig. 14; n Å 1/1). Both
% % the frequency and duration of firing were augmented in the
% % presence of picrotoxin.


\subsubsection{Acoustic Response of D~stellate  Cells}



The \OnC units are widely scattered throughout the VCN %[168,169]
matching the distri- bution of the commissural cells
\citep{PaoliniClareyEtAl:2005,NeedhamPaolini:2007}. %[113,204,254].
Compatible with their wide-reaching dendritic arbor, the type II multi- polar
cells appear to integrate inputs over a wide frequency range
\citep{PalmerWallaceEtAl:2003}.  The known targets of the type II multipolar
cells are illustrated in Figure~\ref{fig:microcircuit}. The axons leave the
cochlear nucleus via the intermediate acoustic stria. %[3,61,113,204,220].
The distribution of multipolar neurons in the VCN whose axons leave via the IAS
is very similar to that of the commissural cells %(compare [3,42])
, and the only known targets of these axons are the contralateral dorsal and
ventral cochlear nuclei.% [13,113,204].
The axons also give rise to extensive collateral branches within the parent VCN
and send widespread terminations to the ipsilateral DCN
(References). %[4,55,61,151,164,220].
In both the ipsilateral and contralateral cochlear nuclei, the axonal terminals
contain pleomorphic or flattened synaptic vesicles, consistent with an
inhibitory effect on the target cells. %[13,220].

The type II multipolar cells do not appear to project to the inferior colliculus, and their role would be expected to form a major source of wideband inhibition to the VCN and DCN. % [54]. [12,102]. 
%It is not known if they give rise to any of the branches making up other projections attributed to multipolar cells.








%\subsubsection{Periodicity and Temporal Coding in the Ventral Cochlear Nucleus}


%\subsubsection{Commissural and dorsal output of D~stellate cells}




\subsubsection{Neuromodulatory Effects on D~stellate Cells}


% \citep{DoucetRyugo:1997}
% Depending on somatic location, dendritic processes were
% extended into the nearest and most proximal region of the
% surrounding granule cell domain. Caudally-distributed
% cells projected their dendrites into the lamina between the
% PVCN and the DCN, laterally located cells extended into
% the superficial layer of granule cells along the lateral edge
% of the AVCN, dorsally-located cells reached the subpedun-
% cular region in the dorsomedial part of VCN, and medially
% positioned cells contacted the medial sheet of granule cells
% or sometimes even intermingled with fibres associated
% with the vestibular nerve root.



\begin{itemize}
\item sensitive to neuromodulatory currents \citep{FujinoOertel:2001}
% \begin{itemize}
% \item high input resistance $\rightarrow$ amplify small current inputs \citep{FujinoOertel:2001}
% \item no LKT in TS,  LKT makes bushy and octopus insensitive to steady currents \citep{OertelFujino:2001,McGinleyOertel:2006}
% \item Ih higher in TS \& activated more at lower potentials than in bushy and octopus, so that it is less active at rest
% \item high resistance $\rightarrow$ greater voltage changes in small modulating current $\rightarrow$ Ih can be modulated by G-protein coupled receptors, hence making TS more excitable when Ih activated \citep{RodriguesOertel:2006}
% \end{itemize}
\end{itemize}


\begin{itemize}
\item ANF provide glutamatergic excitation for D stellates  \citep{Cant:1981,FerragamoGoldingEtAl:1998a,Alibardi:1998a}
\item Excitation from other T stellate cells \citep{FerragamoGoldingEtAl:1998a}
\item Glycine from other DS cells \citep{FerragamoGoldingEtAl:1998a}
\item Glycine from TV cells \citep{WickesbergOertel:1990,ZhangOertel:1993b}
\end{itemize}

     
\subsubsection{Golgi cells (GABA)}

Small, single-GABA-labelled terminals (presumably Golgi cells) apposed to
dendrites of \DS cells (glycine-positive) in the granule cell domain have been
confirmed histologically in different animals (cat: \citealt{SmithRhode:1989},
guinea pig: \citealt{KolstonOsenEtAl:1992}).


\begin{itemize}
\item no IPSPs or IPSCs but presence of GABAa receptors and response changes to bicuculine \citep{WuOertel:1986,OertelWickesberg:1993,FerragamoGoldingEtAl:1998a}
\item dend filter obscures PSPs
\item Golgi cells are GABAergic and lie within the granule cell domains around the VCN and terminate near the fine distal dendrites of T stellate cells
\end{itemize}

b. Periolivary cells (GABA + GAD - glutamic acid decarboxylase) 

\begin{itemize}
\item observed in PVCN close to TS \citep{AdamsMugnaini:1987}
\item GAD effectively Glycine \citep{GoldingOertel:1997}
\item can also arise from GABAergic neurons in ipsi LNTB and DM Periolivary
\end{itemize}

\subsubsection{VNTB cells (ACh)}

\begin{itemize}
\item collateral branches of OCB go to GCD \citep{MellottMottsEtAl:2011,SherriffHenderson:1994,OsenRoth:1969}
\item DS have nicotinic and muscarinic ACh receptors \citep{FujinoOertel:2001}
\item ACh input to DS, together with OC-cochlea, enhances spectral peaks in noise  \citep{FujinoOertel:2001}
\end{itemize}

\subsubsection{NE and 5HT}

\begin{itemize}
\item Raphe nuclei (5HT)
\item Locus coeruleus Peribrachial cells (NE)
\item both terminate in PVCN \citep{KlepperHerbert:1991,Thompson:2003,ThompsonLauder:2005,Thompson:2003a,ThompsonWiechmann:2002,BehrensSchofieldEtAl:2002,ThompsonThompson:2001,ThompsonThompson:2001a,ThompsonMooreEtAl:1995,ThompsonThompsonEtAl:1994}
\item both increase firing in T stellates \citep{OertelWrightEtAl:2011} in presence of glut and gly blockers -> hence act on post synapse (TS cells)
\item both G-protein coupled, both act on either pre or post synaptic cells
\item NE affects probability of release at calyx of Held
\item NE increases firing rate of choppers \citep{KosslVater:1989,Ebert:1996}
\item 5HT excites or inhibits choppers \emph{in vivo} \citep{EbertOstwald:1992}
\end{itemize}





% \citep{JamalZhangEtAl:2011,
% LuRubioEtAl:2008,
% WuJen:2006,
% AwatramaniTurecekEtAl:2005,
% Ben-Ari:2005,
% HuaWangEtAl:2005,
% IrfanZhangEtAl:2005,
% JenWu:2005,
% LuBurgerEtAl:2005,
% MartyLlano:2005,
% Rubio:2005,
% MahendrasingamWallamEtAl:2004,
% RubioJuiz:2004,
% Alibardi:2003,
% CasparyPalombiEtAl:2002,
% CamposCaboEtAl:2001,
% LiaoVanEtAl:2000,
% LimAlvarezEtAl:2000,
% MahendrasingamWallamEtAl:2000,
% MarianowskiLiaoEtAl:2000,
% PirkerSchwarzerEtAl:2000,
% WangCasparyEtAl:2000,
% KemmerVater:1997,
% OstapoffBensonEtAl:1997,
% JuizHelfertEtAl:1996,
% LeReesEtAl:1996,
% EbertOstwald:1995,
% EbertOstwald:1995a,
% GleichBielenbergEtAl:1995,
% SunejaBensonEtAl:1995a,
% WinerLarueEtAl:1995,
% JuizAlbinEtAl:1994,
% VareckaWuEtAl:1994,
% CasparyPalombi:1993,
% CasparyPalombiEtAl:1993,
% PotashnerBensonEtAl:1993,
% KolstonOsenEtAl:1992,
% PalombiCaspary:1992,
% OsenLopezEtAl:1991,
% OsenOttersenEtAl:1990,
% CarrFujitaEtAl:1989,
% JuizHelfertEtAl:1989,
% SaintMorestEtAl:1989,
% OberdorferParakkalEtAl:1988,
% FexAltschulerEtAl:1986,
% MelanderHokfeltEtAl:1986,
% PeyretGeffardEtAl:1986,
% WentholdZempelEtAl:1986,
% Mugnaini:1985,
% ThompsonCortezEtAl:1985,
% CasparyHaveyEtAl:1979,
% Wenthold:1979,
% Davies:1975,
% FisherDavies:1976}


% ChandaXu-Friedman:2010
%      Chanda, Xu-Friedman            2010 J Neurophysiol 104, 2063--74
%      Neuromodulation by GABA converts a relay into a coincidence detector.

% FukuiBurgerEtAl:2010
%      Fukui, Burger, Ohmori, Rubel   2010 J Neurosci 30, 12075--83
%      GABAergic inhibition sharpens the frequency tuning and enhances phase locking in chicken nucleus magnocellularis neurons.

% IrieOhmori:2008
%      Irie, Ohmori                   2008 Biochem Biophys Res Commun 367, 503--08
%      Presynaptic {GABAB} receptors modulate synaptic facilitation and depression at distinct synapses in fusiform cells of mouse dorsal cochlear nucleus

% LuRubioEtAl:2008
%      Lu, Rubio, Trussell            2008 Neuron 57, 524--35
%      Glycinergic {Transmission} {Shaped} by the {Corelease} of {GABA} in a {Mammalian} {Auditory} {Synapse

% KuleszaKadnerEtAl:2007
%      Kulesza, Kadner, Berrebi       2007 J Neurophysiol 97, 1610--20
%      Distinct {Roles} for {Glycine} and {GABA} in {Shaping} the {Response} {Properties} of {Neurons} in the {Superior} {Paraolivary} {Nucleus} of the {Rat

% Lu:2007
%      Lu                             2007 J Neurophysiol 97, 1018--29
%      Endogenous {mGluR} {Activity} {Suppresses} {GABAergic} {Transmission} in {Avian} {Cochlear} {Nucleus} {Magnocellularis} {Neurons

% WuJen:2006
%      Wu, Jen                        2006 Hear Res 215, 56--66
%      The role of {GABAergic} inhibition in shaping duration selectivity of bat inferior collicular neurons determined with temporally patterned sound trains

% AwatramaniTurecekEtAl:2005
%      Awatramani, Turecek, Trussell  2005 J Neurophysiol 93, 819--28
%      Staggered {Development} of {GABAergic} and {Glycinergic} {Transmission} in the {MNTB

% Ben-Ari:2005
%      Ben-Ari                        2005 Trends Neurosci 28, 277
%      The multiple facets of {GABA

% BurgerPfeifferEtAl:2005
%      Burger, Pfeiffer, Westrum, ... 2005 J Comp Neurol 489, 11--22
%      Expression of {GABA(B)} receptor in the avian auditory brainstem: {Ontogeny,} afferent deprivation, and ultrastructure

% Gutierrez:2005
%      Gutierrez                      2005 Trends Neurosci 28, 297--303
%      The dual {glutamatergic-GABAergic} phenotype of hippocampal granule cells

% HestrinGalarreta:2005
%      Hestrin, Galarreta             2005 Trends Neurosci 28, 304--09
%      Electrical synapses define networks of neocortical {GABAergic} neurons

% HuaWangEtAl:2005
%      Hua, Wang, Xiao                2005 Lin Chuang Er Bi Yan Hou Ke Za Zhi 19, 315--7
%      [Distribution of gama-aminobutyric acid {(GABA)ergic} neurons in rats cochlear nuclei after unilateral cochlea ablation]

% IrfanZhangEtAl:2005
%      Irfan, Zhang, Wu               2005 Hear Res 203, 159--71
%      Synaptic transmission mediated by ionotropic glutamate, glycine and {GABA} receptors in the rat's ventral nucleus of the lateral lemniscus

% JenWu:2005
%      Jen, Wu                        2005 Hear Res 202, 222--34
%      The role of {GABAergic} inhibition in shaping the response size and duration selectivity of bat inferior collicular neurons to sound pulses in rapid sequences

% LlinasUrbanoEtAl:2005
%      Llinas, Urbano, Leznik, Ram... 2005 Trends Neurosci 28, 325--33
%      Rhythmic and dysrhythmic thalamocortical dynamics: {GABA} systems and the edge effect

% LuBurgerEtAl:2005
%      Lu, Burger, Rubel              2005 J Neurophysiol 93, 1429--38
%      GABA(B)} receptor activation modulates {GABA(A)} receptor-mediated inhibition in chicken nucleus magnocellularis neurons

% MartyLlano:2005
%      Marty, Llano                   2005 Trends Neurosci 28, 284--89
%      Excitatory effects of {GABA} in established brain networks

% MerchanAguilarEtAl:2005
%      Merchan, Aguilar, Lopez-Pov... 2005 Neurosci 136, 907--25
%      The inferior colliculus of the rat: {Quantitative} immunocytochemical study of {GABA} and glycine

% OnoYanagawaEtAl:2005
%      Ono, Yanagawa, Koyano          2005 Neurosci Res 51, 475--92
%      GABAergic} neurons in inferior colliculus of the {GAD67-GFP} knock-in mouse: {Electrophysiological} and morphological properties

% RepresaBen-Ari:2005
%      Represa, Ben-Ari               2005 Trends Neurosci 28, 278--83
%      Trophic actions of {GABA} on neuronal development

% Rubio:2005
%      Rubio                          2005 J Comp Neurol 485, 266--66
%      Differential distribution of synaptic endings containing glutamate, glycine, and {GABA} in the rat dorsal cochlear nucleus (vol 477, pg 253, 2004)

% LujanShigemotoEtAl:2004
%      Lujan, Shigemoto, Kulik, Juiz  2004 J Comp Neurol 475, 36--46
%      Localization of the {GABAB} receptor 1a/b subunit relative to glutamatergic synapses in the dorsal cochlear nucleus of the rat

% MahendrasingamWallamEtAl:2004
%      Mahendrasingam, Wallam, Pol... 2004 Eur J Neurosci 19, 993--1004
%      An immunogold investigation of the distribution of {GABA} and glycine in nerve terminals on the somata of spherical bushy cells in the anteroventral cochlear nucleus of guinea pig

% RubioJuiz:2004
%      Rubio, Juiz                    2004 J Comp Neurol 477, 253--72
%      Differential distribution of synaptic endings containing glutamate, glycine, and {GABA} in the rat dorsal cochlear nucleus

% SivaramakrishnanSterbing-DAngeloEtAl:2004
%      Sivaramakrishnan, Sterbing-... 2004 J Neurosci 24, 5031--43
%      GABA(A)} synapses shape neuronal responses to sound intensity in the inferior colliculus

% ZhangSunejaEtAl:2004
%      Zhang, Suneja, Potashner       2004 J Neurosci Res 75, 361--70
%      Protein kinase {A} and calcium/calmodulin-dependent protein kinase {II} regulate glycine and {GABA} release in auditory brain stem nuclei




% WisdenCopeEtAl:2002
%      Wisden, Cope, Klausberger, ... 2002 Neuropharmacology 43, 530--49
%      Ectopic expression of the {GABAA} receptor [alpha]6 subunit in hippocampal pyramidal neurons produces extrasynaptic receptors and an increased tonic inhibition


















Roles in the IC
% ZhangKelly:2003
%      Zhang, Kelly                   2003 J Neurophysiol 90, 477--90
%      Glutamatergic and {GABAergic} {Regulation} of {Neural} {Responses} in {Inferior} {Colliculus} to {Amplitude-Modulated} {Sounds
% CasparyPalombiEtAl:2002
%      Caspary, Palombi, Hughes       2002 Hear Res 168, 163--73
%      GABAergic} inputs shape responses to amplitude modulated stimuli in the inferior colliculus

% CasparyHelfertEtAl:1997
%      Caspary, Helfert, Palombi      1997 in: Acoustic Signal Processing in the Central Auditory System
%      The role of {GABA} in shaping frequency response properties in the chinchilla inferior colliculus

% Pollak:1997
%      Pollak                         1997 Ann Otol Rhinol Laryn 168, 44--54
%      Roles of {GABAergic} inhibition for the binaural processing of multiple sound sources in the inferior colliculus















Birds
% LuTrussell:2001
%      Lu, Trussell                   2001 J Physiol 535, 125--31
%      Mixed excitatory and inhibitory {GABA-mediated} transmission in chick cochlear nucleus

% BartheldCodeEtAl:1989
%      von Bartheld, Code, Rubel      1989 J Comp Neurol 287, 470--83
%      GABAergic} neurons in brainstem auditory nuclei of the chick: distribution, morphology, and connectivity

% CarrFujitaEtAl:1989
%      Carr, Fujita, Konishi          1989 J Comp Neurol 286, 190--207
%      Distribution of {GABAergic} neurons and terminals in the auditory system of the barn owl



RNA expression
% JamalZhangEtAl:2011
%      Jamal, Zhang, Finlayson, Po... 2011 Neuroscience , 
%      The level and distribution of the GABA(B)R2 receptor subunit in the rat's central auditory system.

% CamposCaboEtAl:2001
%      Campos, de Cabo, Wisden, Ju... 2001 Neurosci 102, 625--38
%      Expression of {GABA(A)} receptor subunits in rat brainstem auditory pathways: cochlear nuclei, superior olivary complex and nucleus of the lateral lemniscus

% LiaoVanEtAl:2000
%      Liao, Van Den Abbeele, Herm... 2000 Hear Res 150, 12--26
%      Expression of {NMDA,} {AMPA} and {GABA(A)} receptor subunit {mRNAs} in the rat auditory brainstem. {II.} {Influence} of intracochlear electrical stimulation

% MarianowskiLiaoEtAl:2000
%      Marianowski, Liao, Van Den ... 2000 Hear Res 150, 1--11
%      Expression of {NMDA,} {AMPA} and {GABA(A)} receptor subunit {mRNAs} in the rat auditory brainstem. {I.} {Influence} of early auditory deprivation

% PirkerSchwarzerEtAl:2000
%      Pirker, Schwarzer, Wieselth... 2000 Neurosci 101, 815--50
%      GABAA} receptors: immunocytochemical distribution of 13 subunits in the adult rat brain

% HysonSadler:1997
%      Hyson, Sadler                  1997 J Mol Neurosci 8, 193--205
%      Differences in expression of {GABAA} receptor subunits, but not benzodiazepine binding, in the chick brainstem auditory system


% JuizAlbinEtAl:1994
%      Juiz, Albin, Helfert, Altsc... 1994 Brain Res 639, 193--201
%      Distribution of {GABAA} and {GABAB} binding sites in the cochlear nucleus of the guinea pig


% VareckaWuEtAl:1994
%      Varecka, Wu, Rotter, Frostholm 1994 J Comp Neurol 339, 341--52
%      GABAA/benzodiazepine} receptor alpha 6 subunit {mRNA} in granule cells of the cerebellar cortex and cochlear nuclei: expression in developing and mutant mice



Pharmacological/Physiological effects
% LimAlvarezEtAl:2000
%      Lim, Alvarez, Walmsley         2000 J Physiol 525 Pt 2, 447--59
%      GABA} mediates presynaptic inhibition at glycinergic synapses in a rat auditory brainstem nucleus

% WangCasparyEtAl:2000
%      Wang, Caspary, Salvi           2000 Neuroreport 11, 1137--40
%      GABA-A} antagonist causes dramatic expansion of tuning in primary auditory cortex


% BrenowitzDavidEtAl:1998
%      Brenowitz, David, Trussell     1998 Neuron 20, 135--41
%      Enhancement of synaptic efficacy by presynaptic {GABA(B)} receptors


% BackoffPalombiEtAl:1997
%      Backoff, Palombi, Caspary      1997 Hear Res 110, 155--63
%      Glycinergic and {GABAergic} inputs affect short-term suppression in the cochlear nucleus

% ShoreBledsoe:1997
%      Shore, Bledsoe                 1997 Assoc Res Otolaryngol Abstr , 
%      Effects of {GABA} and {VNTB} stimulation on forward masking functions in ventral cochlear nucleus

% GoldingOertel:1996
%      Golding, Oertel                1996 J Neurosci 16, 2208--19
%      Context-dependent synaptic action of glycinergic and {GABAergic} inputs in the dorsal cochlear nucleus


% LeReesEtAl:1996
%      Le Beau, Rees, Malmierca       1996 J Neurophysiol 75, 902--19
%      Contribution of {GABA-} and glycine-mediated inhibition to the monaural temporal response properties of neurons in the inferior colliculus

% EbertOstwald:1995
%      Ebert, Ostwald                 1995 Hear Res 91, 160--6
%      GABA} alters the discharge pattern of chopper neurons in the rat ventral cochlear nucleus

% EbertOstwald:1995a
%      Ebert, Ostwald                 1995 Exp Brain Res 104, 310--22
%      GABA} can improve acoustic contrast in the rat ventral cochlear nucleus

% RazaMilbrandtEtAl:1994
%      Raza, Milbrandt, Arneric, C... 1994 Hear Res 77, 221--30
%      Age-related changes in brainstem auditory neurotransmitters: measures of {GABA} and acetylcholine function

% YangPollak:1994
%      Yang, Pollak                   1994 J Neurophysiol 71, 2014--24
%      GABA} and glycine have different effects on monaural response properties in the dorsal nucleus of the lateral lemniscus of the mustache bat

% CasparyPalombi:1993
%      Caspary, Palombi               1993 Assoc Res Otolaryngol Abstr 109, 
%      GABA} inputs control discharge rate within the excitatory response area of chinchilla inferior colliculus neurons

% CasparyPalombiEtAl:1993
%      Caspary, Palombi, Backoff, ... 1993 in: The Mammalian Cochlear Nuclei: Organisation and Function
%      GABA} and glycine inputs control discharge rate within the excitatory response area of primary-like and phase-locked {AVCN} neurons

% OtisDeEtAl:1993
%      Otis, De Koninck, Mody         1993 J Physiol (Lond) 463, 391--407
%      Characterization of synaptically elicited {GABAB} responses using patch- clamp recordings in rat hippocampal slices

% PalombiCaspary:1992
%      Palombi, Caspary               1992 J Neurophysiol 67, 738--46
%      GABAA} receptor antagonist bicuculline alters response properties of posteroventral cochlear nucleus neurons


% WalshMcGeeEtAl:1990
%      Walsh, McGee, Fitzakerley      1990 J Neurophysiol 64, 961--77
%      GABA} actions within the caudal cochlear nucleus of developing kittens

% CasparyHaveyEtAl:1979
%      Caspary, Havey, Faingold       1979 Brain Res 172, 179--85
%      Effects of microiontophoretically applied glycine and {GABA} on neuronal response patterns in the cochlear nuclei

% FaingoldGehlbachEtAl:1989
%      Faingold, Gehlbach, Caspary    1989 Brain Res 500, 
%      On the role of {GABA} as an inhibitory neurotransmitter in inferior colliculus neurons: iontophoretic studies

\citep{RamanZhangEtAl:1994}
\citep{FrisinaWaltonEtAl:1993,FrisinaSmithEtAl:1990}

\citep{JosephsonMorest:2003,BellinghamLimEtAl:1998,BilakMorest:1998,HunterPetraliaEtAl:1993}

Reviews of AMPA subtype in the cochlear nucleus and auditory brainstem nuclei
\citep{GardnerTrussellEtAl:2001,Parks:2000}.

BellinghamLimEtAl:1998
     Bellingham, Lim, Walmsley      1998 J Physiol (Lond) 511, 861--69
     Developmental changes in {EPSC} quantal size and quantal content at a central glutamatergic synapse in the rat

BilakMorest:1998
     Bilak, Morest                  1998 Synapse 28, 251--70
     Differential expression of the metabotropic glutamate receptor {mGluR1alpha} by neurons and axons in the cochlear nucleus: in situ hybridization and immunohistochemistry

OtisWuEtAl:1996
     Otis, Wu, Trussell             1996 J Neurosci 16, 1634--44
     Delayed clearance of transmitter and the role of glutamate transporters at synapses with multiple release sites


Ultrastructural Analysis

The cytological composition and ultrastructure of the
DCN are relatively well known. Apart from large neurons
that project ouside the nucleus (mainly pyramidal cells),
other smaller interneurons, called cartwheel, Golgi,
32 Glycinergic and GABAergic neurons, L. Alibardi
stellate, unipolar brush cells, commissural and vertical
(or tuberculo-ventral) neurons, have been character-
ized (Kane, 1974; Mugnaini et al. 1980, 1997; Fiori
 Mugnaini, 1981; Kane et al. 1981; Wouterlood 
Mugnaini, 1984; Mugnaini, 1985; Berrebi  Mugnaini,
1991; Floris et al. 1994; Ryugo  Willard, 1985; Alibardi,
1999a,b, 2000a,b; Kemmer  Vater, 2001)



% Alibardi:2003
%      Alibardi                       2003 J Anat 203, 31--56
%      Ultrastructural distribution of glycinergic and {{GABAergic}} neurons and axon terminals in the rat dorsal cochlear nucleus, with emphasis on granule cell areas


% KemmerVater:2001a
%      Kemmer, Vater                  2001 Anat Embryol 203, 429--47
%      Functional organization of the dorsal cochlear nucleus of the horseshoe bat {(Rhinolophus} rouxi) studied by {GABA} and glycine immunocytochemistry and electron microscopy

% AcsadyKamondiEtAl:1998
%      Acsady, Kamondi, Sik, Freun... 1998 J Neurosci 18, 3386--403
%      GABAergic} cells are the major postsynaptic targets of mossy fibers in the rat hippocampus


% JuizCamposEtAl:1996
%      Juiz, Campos, Helfert, Alts... 1996 J Hirnforsch 37, 51--6
%      Silver intensification of immunocolloidal gold on ultrathin plastic sections applied to the study of the neuronal distribution of {GABA} and glycine

% JuizHelfertEtAl:1996
%      Juiz, Helfert, Bonneau, Cam... 1996 J Hirnforsch 37, 561--74
%      Distribution of glycine and {GABA} immunoreactivities in the cochlear nucleus: quantitative patterns of putative inhibitory inputs on three cell types

% JuizAlbinEtAl:1994
%      Juiz, Albin, Helfert, Altsc... 1994 Brain Res 639, 193--201
%      Distribution of {GABAA} and {GABAB} binding sites in the cochlear nucleus of the guinea pig

% PotashnerBensonEtAl:1993
%      Potashner, Benson, Ostapoff... 1993 in: The Mammalian Cochlear Nuclei: Organisation and Function
%      Glycine and {GABA:} transmitter candidates of projections descending to the cochlear nucleus

% KolstonOsenEtAl:1992
%      Kolston, Osen, Hackney, Ott... 1992 Anat Embryol 186, 443--65
%      An atlas of glycine- and {GABA-like} immunoreactivity and colocalization in the cochlear nuclear complex of the guinea pig



% OberdorferParakkalEtAl:1988
%      Oberdorfer, Parakkal, Altsc... 1988 Hear Res 33, 229--38
%      Ultrastructural localization of {GABA-immunoreactive} terminals in the anteroventral cochlear nucleus of the guinea pig

% Mugnaini:1985
%      Mugnaini                       1985 J Comp Neurol 235, 61--81
%      GABA} neurons in the superficial layers of the rat dorsal cochlear nucleus: light and electron microscopic immunocytochemistry.

MugnainiOsenEtAl:1980

% SaintMorestEtAl:1989
%      Saint Marie, Morest, Brandon   1989 Hear Res 42, 97--112
%      The form and distribution of {GABAergic} synapses on the principal cell types of the ventral cochlear nucleus of the cat
% SunejaPotashnerEtAl:1998
%      Suneja, Potashner, Benson      1998 Exp Neurol 151, 273--88
%      Plastic changes in glycine and {GABA} release and uptake in adult brain stem auditory nuclei after unilateral middle ear ossicle removal and cochlear ablation



Histology and Microscopy, immuno-reactive labeling
% KemmerVater:2001a
%      Kemmer, Vater                  2001 Anat Embryol 203, 429--47
%      Functional organization of the dorsal cochlear nucleus of the horseshoe bat {(Rhinolophus} rouxi) studied by {GABA} and glycine immunocytochemistry and electron microscopy

% RiquelmeSaldanaEtAl:2001
%      Riquelme, Saldana, Osen, Ot... 2001 J Comp Neurol 432, 409--24
%      Colocalization of {GABA} and glycine in the ventral nucleus of the lateral lemniscus in rat: an in situ hybridization and semiquantitative immunocytochemical study


% ChaudhryReimerEtAl:1998
%      Chaudhry, Reimer, Bellocchi... 1998 J Neurosci 18, 9733--50
%      The vesicular {GABA} transporter, {VGAT,} localizes to synaptic vesicles in sets of glycinergic as well as {GABAergic} neurons

% Gil-LoyzagaBartolomeEtAl:1998
%      Gil-Loyzaga, Bartolome, Ibanez 1998 Histol Histopathol 13, 415--24
%      Synaptophysin immunoreactivity in the cat cochlear nuclei

% GleichVater:1998
%      Gleich, Vater                  1998 Cell Tissue Res 293, 207--25
%      Postnatal development of {GABA-} and glycine-like immunoreactivity in the cochlear nucleus of the {Mongolian} gerbil {(Meriones} unguiculatus)

% LachicaKatoEtAl:1998
%      Lachica, Kato, Lippe, Rubel    1998 J Neurobiol 37, 321--37
%      Glutamatergic and {GABAergic} agonists increase {[Ca2+]i} in avian cochlear nucleus neurons


% YangWuEtAl:1998
%      Yang, Wu, Fang                 1998 Zhonghua Yi Xue Za Zhi 78, 30--2
%      [Changes of {GABA} immunoreactivity in aged rat cochlear nucleus]


% KemmerVater:1997
%      Kemmer, Vater                  1997 Cell Tissue Res 287, 487--506
%      The distribution of {GABA} and glycine immunostaining in the cochlear nucleus of the mustached bat {(Pteronotus} parnellii)

% OstapoffBensonEtAl:1997
%      Ostapoff, Benson, Saint Marie  1997 J Comp Neurol 381, 500--12
%      GABA}- and glycine-immunoreactive projections from the superior olivary complex to the cochlear nucleus in guinea pig


% JuizCamposEtAl:1996
%      Juiz, Campos, Helfert, Alts... 1996 J Hirnforsch 37, 51--6
%      Silver intensification of immunocolloidal gold on ultrathin plastic sections applied to the study of the neuronal distribution of {GABA} and glycine

% JuizHelfertEtAl:1996
%      Juiz, Helfert, Bonneau, Cam... 1996 J Hirnforsch 37, 561--74
%      Distribution of glycine and {GABA} immunoreactivities in the cochlear nucleus: quantitative patterns of putative inhibitory inputs on three cell types


% GleichBielenbergEtAl:1995
%      Gleich, Bielenberg, Strutz     1995 Neuroreport 7, 29--32
%      Sound induced expression of {c-Fos} in {GABA} positive neurones of the gerbil cochlear nucleus

% SunejaBensonEtAl:1995a
%      Suneja, Benson, Gross, Pota... 1995 J Neurochem 64, 147--60
%      Uptake and release of {D-aspartate,} {GABA,} and glycine in guinea pig brainstem auditory nuclei


% WinerLarueEtAl:1995
%      Winer, Larue, Pollak           1995 J Comp Neurol 355, 317--53
%      GABA} and glycine in the central auditory system of the mustache bat: structural substrates for inhibitory neuronal organization


% PotashnerBensonEtAl:1993
%      Potashner, Benson, Ostapoff... 1993 in: The Mammalian Cochlear Nuclei: Organisation and Function
%      Glycine and {GABA:} transmitter candidates of projections descending to the cochlear nucleus

% KolstonOsenEtAl:1992
%      Kolston, Osen, Hackney, Ott... 1992 Anat Embryol 186, 443--65
%      An atlas of glycine- and {GABA-like} immunoreactivity and colocalization in the cochlear nuclear complex of the guinea pig


% VaterKosslEtAl:1992
%      Vater, Kossl, Horn             1992 J Comp Neurol 325, 183--206
%      GAD-} and {GABA-immunoreactivity} in the ascending auditory pathway of horseshoe and mustached bats

% OsenLopezEtAl:1991
%      Osen, Lopez, Slyngstad, Ott... 1991 J Neurocytol 20, 17--25
%      GABA-like} and glycine-like immunoreactivities of the cochlear root nucleus in rat

% OsenOttersenEtAl:1990
%      Osen, Ottersen, Storm-Mathisen 1990 in: Glycine Neurotransmission
%      Colocalization of glycine-like and {GABA-like} immunoreactivities: a semiquantitative study of individual neurons in the dorsal cochlear nucleus of cat.

% OstapoffMorestEtAl:1990
%      Ostapoff, Morest, Potashner    1990 J Chem Neuroanat 3, 285--95
%      Uptake and retrograde transport of {[3H]GABA} from the cochlear nucleus to the superior olive in the guinea pig


% SaintMorestEtAl:1989
%      Saint Marie, Morest, Brandon   1989 Hear Res 42, 97--112
%      The form and distribution of {GABAergic} synapses on the principal cell types of the ventral cochlear nucleus of the cat


% JuizHelfertEtAl:1989
%      Juiz, Helfert, Wenthold, De... 1989 Brain Res 504, 173--79
%      Immunocytochemical localization of the {GABAA/benzodiazepine} receptor in the guinea pig cochlear nucleus: evidence for receptor localization heterogeneity

% FexAltschulerEtAl:1986
%      Fex, Altschuler, Kachar, We... 1986 Brain Res 366, 106--17
%      GABA} visualized by immunocytochemistry in the guinea pig cochlea in axons and endings of efferent neurons

% MelanderHokfeltEtAl:1986
%      Melander, Hokfelt, Rokaeus,... 1986 J Neurosci 6, 3640--54
%      Coexistence of galanin-like immunoreactivity with catecholamines, 5- hydroxytryptamine, {GABA} and neuropeptides in the rat {CNS

% OberdorferParakkalEtAl:1988
%      Oberdorfer, Parakkal, Altsc... 1988 Hear Res 33, 229--38
%      Ultrastructural localization of {GABA-immunoreactive} terminals in the anteroventral cochlear nucleus of the guinea pig


% PeyretGeffardEtAl:1986
%      Peyret, Geffard, Aran          1986 Hear Res 23, 115--21
%      GABA} immunoreactivity in the primary nuclei of the auditory central nervous system

% WentholdZempelEtAl:1986
%      Wenthold, Zempel, Parakkal,... 1986 Brain Res 380, 7--18
%      Immunocytochemical localization of {GABA} in the cochlear nucleus of the guinea pig


% Mugnaini:1985
%      Mugnaini                       1985 J Comp Neurol 235, 61--81
%      GABA} neurons in the superficial layers of the rat dorsal cochlear nucleus: light and electron microscopic immunocytochemistry.


% ThompsonCortezEtAl:1985
%      Thompson, Cortez, Lam          1985 Brain Res 339, 119--22
%      Localization of {GABA} immunoreactivity in the auditory brainstem of guinea pigs


% Wenthold:1979
%      Wenthold                       1979 Brain Res 162, 338--43
%      Release of endogenous glutamic acid, aspartic acid and {GABA} from cochlear nucleus slices

% FisherDavies:1976
%      Fisher, Davies                 1976 J Neurochem 27, 1145--55
%      GABA} and its related enzymes in the lower auditory system of the guinea pig

% Davies:1975
%      Davies                         1975 Brain Res 83, 27--33
%      The distribution of {GABA} transaminase-containing neurones in the cat cochlear nucleus


%\section{Tuberculoventral cells}

%\subsection{Background}

\subsubsection{Morphology and Cellular Mechanisms of Tuberculoventral cells}

Tuberculoventral neurons in the deep layer of the DCN provide a delayed, frequency-specific
glycinergic inhibition to TS and DS cells in the VCN
\citep{ZhangOertel:1993,WickesbergOertel:1988}.  

Planar multipolar or vertical cells are the most populated in the deep layers of the \DCN
and correspond to neurons with a Type II \EIRA \citep{Rhode:1999} and are immunolabelled with glycine.

Not all vertical cells send axona collaterals to the \VCN \citep{Rhode:1999} and not all Type II units can be antidromically activated by shocks to the VCN

lateral tuberculoventral tract


The dendrites of TV cells are
aligned with ANFs and indicating narrow frequency tuning. TV cells have low
spontaneous rates and variable PSTHs; “pauser,” “chopper,” or “onset/sustained”
have been recorded \citep{ShofnerYoung:1985,SpirouDavisEtAl:1999}. They have
little or no response to wide band noise and firing rates to CF tones that are
non-monotonic functions of intensity.

Anterograde labelling in the DCN suggests glycinergic tuberculoventral cells
project tonotopically to the VCN not just on-CF, but also to the low and high
frequency side bands in the AVCN
\citep{OstapoffFengEtAl:1994,MunirathinamOstapoffEtAl:2004}.  Ultra-structural
labelling of synapses in the rat DCN suggest TV cells are inhibited by DS cells
and from sources in the DCN but excitatory inputs were not found from TS cells
\citep{RubioJuiz:2004}.  Intracellular responses from labeled TV cells in the mouse
show clear excitatory input from TS cells and diffuse inhibitory input from DS
cells \citep{ZhangOertel:1993}.


Tuberculoventral cells have a classic type I regular-spiking response to current clamp 
\citep{ZhangOertel:1993}. , with double exponential action potentials 

% Double
%               undershoots suggest ILT , Regular spiking               &               &     \citep{ZhangOertel:1993}     & 
% 100M{\textpm}20, but then state 85{\textpm}10 in table 1
%                       \citep{ZhangOertel:1993}                        & 
% \citep{EvansNelson:1973,WickesbergOertel:1990,WickesbergOertel:1993,WickesbergOertel:1988,WickesbergWhitlonEtAl:1991,Wickesberg:1996,YoungBrownell:1976,YoungVoigt:1981,ZhangOertel:1993}\\\hline


% \subsubsection{Synaptophysiology of Tuberculoventral cells}

% \subsubsubsection{Afferent input}

% In rats, the sole excitatory afferent input to Tuberculoventral cells comes from
% the auditory nerve \citep{RubioJuiz:2004}.


% Glutamate AMPA receptors [\citenum{ZhangOertel:1993}]                          
% The IPSC dynamics of the AMPA synapse in TV cells is significantly slower in TV cells (time constant 0.4 ms, rise time 0.15 ms) than the VCN stellate and DCN fusiform AMPA synapses (0.36 ms)  \citep{GardnerTrussellEtAl:1999}.


% $\sim$70$\mu$m (mice [\citenum{SpirouDavisEtAl:1999}])
%                               % RF Post
% $\sim$70$\mu$m (mice [\citenum{ZhangOertel:1993,SpirouDavisEtAl:1999}])
%                                 % Number
% $\sim$4 somatic contacts per cell, 10\% AN (guinea pig [\citenum{Alibardi:1999}]), 
% 17.7\% (ANF terminals, 32.5\% GLUT) of total synaptic terminals in deep layer (rat [\citenum{RubioJuiz:2004}])
% % [\citenum{SpirouDavisEtAl:1999,ZhangOertel:1993}] ;       
% %                                & % Position  
% Mainly dendritic {$<{}100\mu$}m [\citenum{Alibardi:1999,Liberman:1993,SpirouDavisEtAl:1999}]
%   %
% 1.0--1.5 ms [\citenum{ZhangOertel:1993,Rhode:1999,SpirouDavisEtAl:1999}]


% \citep{RubioJuiz:2004}
% In the DL 517 synaptic profiles were sampled. Synap-
% tic endings immunoreactive for glutamate were ϳ30% (n ϭ
% 168) of the total computed in this layer (Fig. 7B). The four
% subtypes of GLU-synaptic endings were observed and the
% GLU1 (n ϭ 81) was 3-fold more abundant than GLU4 (n ϭ
% 51), GLU2 (n ϭ 16), and GLU3 (n ϭ 20). Synaptic profiles
% immunoreactive for glycine and GABA represented 17% (n ϭ
% 88) of the total (Fig. 7B), and the GLY/GABA1 subtype (n ϭ
% 69) was three times more abundant than GLY/GABA2 (n ϭ
% 19). Synaptic profiles immunoreactive for glycine were the
% most heavily represented in this layer (48% of the total
% 7B), and the GLY2 subtype (n ϭ 175) was almost 2.5-fold
% more abundant than GLY1 (n ϭ 71). GABA-synaptic end-
% ings were less numerous (n ϭ 15; Fig. 7B).


% Synaptic endings on vertical cells (VCs). VCs are
% the most abundant inhibitory interneuron in the DL of the
% DCN. Cell bodies were located in the DL and had an approx-
% imate size of 15 ␮m in diameter. The nucleus was indented
% and centered located in the cell body and was surrounded by
% rough endoplasmic reticulum. Usually, a dendritic tree was
% observed attached to the cell body towards the superficial or
% the deepest part of the DL. Dendrites analyzed had an ap-
% proximate diameter of 1.2–2.5 ␮m. Cell bodies and dendrites
% were observed only immunoreactive for glycine (data not
% shown). These inhibitory interneurons received six of the
% nine subtypes of synaptic endings described above (Fig. 8).
% GLU1 was the only subtype found immunoreactive for glu-
% tamate on vertical cells and was preferentially on dendrites.
% Both types of synaptic endings immunoreactive for glycine
% and GABA (GLY/GABA1 and GLY/GABA2) were observed
% and both preferentially distributed on the cell body. How-
% ever, GLY/GABA1 was almost 3-fold more abundant than
% GLY/GABA2. The two types of synaptic endings immunore-
% active for glycine were observed distributed on vertical cells.
% GLY1 was observed preferentially on the cell body. The
% GLY2 was almost 2-fold more abundant and was found both
% on the cell body and on dendrites. GABA endings, although
% rare, were observed on both cell bodies and dendrites.


% The observation of inhibitory syn-
% apses on DCN interneurons suggests the presence of dis-
% inhibitory circuits. Consistent with this idea, inhibitory
% postsynaptic potentials can be recorded from VCs and can
% be blocked with both strychnine and bicuculline (Davis
% and Young, 2000). Indirect evidence of inhibition onto VCs
% also comes from their lack of spontaneous activity despite
% the presence of short latency excitation following auditory
% nerve stimulation (Zhang and Oertel, 1994). The blockade
% of glycine and or GABA does not increase spontaneous
% rates in the cell (Davis and Young, 2000).

% source of inhibitory afferents onto VCs is unknown and
% may come from intrinsic and/or extrinsic sources (Fig. 11).
% It has been suggested that glycinergic D-stellate cells from
% the ventral cochlear nucleus contact VCs (Davis and
% Young, 2000). Connections among VCs as well as those
% between CWCs and VCs may also be present (Manis et al.,
% 1994; Smith and Rhode, 1999). Consistent with this idea,
% GLY/GABA terminals were observed on both VC and
% CWCs.

% \subsubsubsection{DS to TV}

% OnC projections to the DCN have also been implicated in the wide-band glycin-
% ergic inhibition demonstrated physiologically in DCN principal cells, as well as
% in the cells designated type II or vertical (Caspary et al., 1987; Young et al.,
% 1992; Nelken and Young, 1994; Backoff et al., 1997; Joris and Smith, 1998;
% Spirou et al., 1999; Davis and Young, 2000; Ander- son and Young, 2004). Our
% electron microscopy of OnC terminals in the deep DCN shows that they can synapse
% on cell bodies or dendrites in this region, which is consis- tent with a
% potential influence of the OnCs on these cell types.

% \citep{DoucetRossEtAl:1999,DoucetRyugo:1997,FriedlandPongstapornEtAl:2003,DoucetRyugo:2006}
% Retrogradely labelled radiate neurons exhibited intense glycine immunoreactivity

% \subsubsubsection{TS to TV}
% %                       TS\ensuremath{\rightarrow}TV                        & AMPA
% % Glutamate
% % \citep{DoucetRossEtAl:1999,FerragamoGoldingEtAl:1998a,ZhangOertel:1993} No
% % TS terminals on TV cells in rats \citep{RubioJuiz:2004} In deep layer guinea
% % pig \citep{PalmerWallaceEtAl:2003} small round vesicles
% % \citep{Alibardi:1999} ;3 of 4 neurons had late EPSPs to AN shock, very young
% %                       mice \citep{ZhangOertel:1993}                       &       Similar to AN AMPA receptors.        & Strong
% % On-CF with weak off-CF See fig 13 \citep{OstapoffBensonEtAl:1999} ; mainly
% % in iso-band but poor classification in rats
% %         \citep{DoucetRossEtAl:1999,FriedlandPongstapornEtAl:2003}         & 
% % \citep{OstapoffBensonEtAl:1999} 3 of 4 shocks elicited late EPSPs
% %                         \citep{ZhangOertel:1993}                          & Mainly to fusiform cell layer, possibly on ends
% % of TV cell dendrites \citep{OertelWuEtAl:1990} CS and CT collaterals in deep
% % DCN possibly somatic \citep{PalmerWallaceEtAl:2003} somatic small round
% %   vesicles presumably from T stellates guinea pig \citep{Alibardi:1999}   & 0.15
% % sec min EPSP latency to VCN Glutamate puffs, main excitation at 0.3 sec Fig
% % 11a, AN shock produces late EPSPs about 3 msec \citep{ZhangOertel:1993}



% TS\ensuremath{\rightarrow}TV                        

% Glutamatergic AMPA receptor  [\citenum{DoucetRossEtAl:1999,FerragamoGoldingEtAl:1998a,ZhangOertel:1993}].
% %Small round vesicles present in deep layer of DCN (guinea pig [\citenum{Alibardi:1999}])

% Similar to ANF AMPA receptors.        
%         % RF Pre

%  % RF Post
% Strong On-CF with weak off-CF  (See fig 13 [\citenum{OstapoffBensonEtAl:1999}]). 
% Mainly in iso-band but poor classification in (rats: [\citenum{DoucetRossEtAl:1999,FriedlandPongstapornEtAl:2003}])         
%  %Number
% 3 of 4 neurons had late EPSPs to AN shock (very young mice [\citenum{ZhangOertel:1993}]).
% [\citenum{OstapoffMorestEtAl:1999}] 
% %                                & %Position
% No TS terminals on TV cells in rats (rats: [\citenum{RubioJuiz:2004}]).
% Mainly to fusiform cell layer, possibly on ends of TV cell dendrites (mice [\citenum{OertelWuEtAl:1990}]). 
% CS and CT collaterals in deep DCN possibly somatic (guinea pigs [\citenum{PalmerWallaceEtAl:2003}]).
% Somatic small round vesicles presumably from T stellates  (guinea pigs [\citenum{Alibardi:1999}])   
%  %Delay 
% 2 ms (Shock to VCN, AN shock produces late EPSPs about 3 msec, mice Fig.~10 \textit{cell 1} [\citenum{ZhangOertel:1993}])
% 150--250 ms (VCN Glutamate puffs, mice:  [\citenum{ZhangOertel:1993}]).


\subsubsection{Acoustic Response of Tuberculoventral cells}


% Responses of tuberculoventral neurons to sound
 Recordings \textit{in vivo} indicate that tuberculoventral cells probably have type II characteristics
 and respond with “onset” or “chopper”
 temporal response patterns \citep{ZhangOertel:1993b}. Units with type II responses are sharply tuned, they
 have thresholds - 10 dB higher than other units with which
 they are intermingled, and they do not respond to broad-
 band noise \citep{SpirouDavisEtAl:1999,YoungBrownell:1976,Young:1980,SachsYoung:1980,YoungVoigt:1982,ShofnerYoung:1985,VoigtYoung:1990,YoungSpirouEtAl:1992,Rhode:1999}. Young and his colleagues have
 shown that most neurons in the deep DCN respond to
 sound with either of two major types of response maps, type
 II or type IV 
\citep{EvansNelson:1973,ShofnerYoung:1985,VoigtYoung:1980,VoigtYoung:1990,Young:1980,YoungBrownell:1976}. 

%Type III units may represent a sepa-
% rate population of cells, although their distinction
% from type II units is not absolutely clear (Young and Voigt
% 1982). Units with type IV responses can be subdivided into
% those that do and those that do not have well-defined
% borders of a central inhibitory
% area (Spirou and Young 1991). The question how response maps and anatomic
% types are correlated has not been completely resolved.
 Many neurons in the DCN that could be driven electrically
 from the VCN had response maps of type II \citep{Young:1980}.
% Of 29 type II units tested, 6, or - l/5, could be driven antidro-
% mically from the VCN; of 7 type II units tested 1, or l/7,
% could be driven from the dorsal acoustic stria. 
Therefore
 some, but not necessarily all, type II units are tuberculoven-
 tral cells. Type II units have variable temporal firing pat-
 terns, “pauser,” “chopper,” or “onset/sustained,”
 and have firing rates that are nonmonotonic
 functions of intensity \citep{ShofnerYoung:1985}. Rhode and his colleagues,
 too, find variable temporal firing patterns in the deep layers
 of the dorsal cochlear nucleus. They suggest that “onset/
 graded” responses come from type II units (Rhode and
 Greenberg 1992). Onset / sustained \citep{ShofnerYoung:1985} and onset / graded peristimulus
 time histograms (Rhode and Greenberg 1992) are similar in that these units
 have a high probability of firing at the beginning of a tone at
 the best frequency and that the probability of firing drops
 off after the beginning of the tone.



\subsubsection{TODO Notch detection in DCN and echo suppression in VCN}



%\subsubsection{Output of Tuberculoventral cells}

%\subsubsection{Neuromodulatory effects in Tuberculoventral cells}





\section{Golgi cells}
    
\subsection{Cellular Mechanisms of Golgi Cells}

\subsection{Synaptophysiology of Golgi Cells}

\subsubsection{Auditory Nerve Fibre Input}



    
\subsection{Acoustic Response of Golgi cells}
    
\subsection{GABA in the Ventral Cochlear Nucleus}
    
\subsection{Neuromodulatory effects of Golgi cells}





%*****************************************
\section{Summary of T~Stellate Microciruit Connections}


Proposals for connectivity around T stellate cells microcircuit are hotly
contested in the literature.

The first network models in the cochlear nucleus revolved around the DCN
\citep{DavisVoigt:1991,ArleKim:1990,ArleKim:1991a,Arle:1992} Selective
processing of different ANF inputs using some form of inhibition was the first
step toward including interneurons in a T stellate cell model
\citep{LaiWinslowEtAl:1994,LaiWinslowEtAl:1994a}.

Recurrent excitation between TS cells is thought to be present in mice
\citep{FerragamoGoldingEtAl:1998a} and has been investigated in two modelling
studies \citep{BahmerLangner:2006,WiegrebeMeddis:2004}, but neither study
represents a realistic implementation of the stellate microcircuit.
\citet{BahmerLangner:2006} used excitatory onset units to regulate the recurrent
T stellate cells, unfortunately the only excitatory onset units in the cochelar
nucleus are octopus cells, which do not have axonal collaterals in the \VCN\@.
Recurrent networks in the cortex prefer inhibition for synchronisation
\citep{LyttonSejnowski:1991,BushSejnowski:1996}.


\begin{landscape}
%\setcitestyle{numbers}
{\small\LTXtable{220mm}{ConnectionsTable}}
%\setcitestyle{authoryear}
\end{landscape}

% Proposed neuronal connections
% The present considerations have provided evidence for the
% connections that are summarized in Fig. 15. We propose that
% T stellate cells receive excitatory, glutamatergic input from
% a small number of type I auditory nerve fibers (monosynaptic
% EPSPs) as well as through collaterals of other T stellate
% cells (late EPSPs) (Oertel et al. 1990). The topographic
% arrangement of tuberculoventral cells indicates that roughly
% the same group of auditory nerve fibers innervates tuberculo-
% ventral cells which, in turn, provide delayed, glycinergic
% inhibition (Wickesberg and Oertel 1988, 1990). D stellate
% cells contribute to the disynaptic IPSP and at high shock
% strengths can provide trains of late IPSPs to T stellate cells.
% D Stellate cells are driven by type I auditory nerve fibers
% (Oertel et al. 1990; this study), and they receive GABAergic
% inhibition, of which Golgi cells are a likely source (Mugnaini
% 1985). Golgi cells lie in the granule cell domain, away from
% the terminals of type I auditory nerve fibers. The finding
% that they are activated by shocks to the auditory nerve more
% slowly than that to T or D stellate cells in the vicinity sug-
% gests that they are activated by type II auditory nerve fibers
% (Benson et al. 1996; Ferragamo et al. 1997).


\section{Summary of Cell Morphology}

\begin{landscape}
%\afterpage
{\small\LTXtable{210mm}{CellMorphologyTable}}
\end{landscape}



%*****************************************
\section{Neural Modelling in the Cochlear Nucleus}

\subsection{Neuron Models with Chopping Charateristics}

\subsection{Microcircuits and Networks in the Cochlear Nucleus}

\begin{landscape}
{\small\LTXtable{210mm}{ModellingCNTable}}
\end{landscape}
%*****************************************
%*****************************************
%*****************************************
%*****************************************
%*****************************************
\setcitestyle{numbers}
\bibliographystyle{plainnat}%abbrvnat}%bmc_article} % Style BST file
\bibliography{../hg/manuscript/bib/MyBib}
\newpage
\listoftodos
\end{document}

%%% Local Variables:
%%% mode: latex
%%% mode: tex-fold
%%% TeX-PDF-mode: nil
%%% TeX-master t
%%% End:
