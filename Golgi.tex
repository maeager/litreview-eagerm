
%\section{Golgi cells}

\citep{Ryugo:2008}
    
%% from Mugnaini



% This paper describes the fine structure of granule cells and granule-associated interneurons (termed Golgi cells) in the cochlear nuclei of cat, rat and mouse. 
% Granule cells and Golgi cells are present in defined regions of ventral and dorsal cochlear nuclei collectively termed "cochlear granule cell domain'. The granule cells are small neurons with two or three short dendrites that give rise to a few branches with terminal expansions. These participate in glomerular synaptic arrays similar to those of the cerebellar cortex. In the glomeruli the dendrites form short type 1 synapses with a large, centrally-located mossy bouton containing round synaptic vesicles and type 2 synapses with peripherally located, smaller boutons containing pleomorphic vesicles. The granule cell axons is thin and beaded and, on its way to the molecular layer of the DCN, takes a straight course, which in ventral nucleus is parallel to the pial surface. Neurons of the second category resemble cerebellar Golgi cells and occur everywhere interspersed among the granule cells. They are usually larger than the granule cells and give rise to dendrites which may branch close to and curve around the cell body. The dendrites contain numerous mitochondria and are laden with thin appendages, giving them a hairy appearance. 
% Both the cell body and the stem dendrites participate in glomerular synaptic arrays. 
% Golgi cell glomeruli are distinguishable from the granule cell glomeruli by unique features of the dendritic profiles and by longer, type 1 synaptic junctions with the central mossy bouton. 
% The Golgi cell axon forms a beaded plexus close to the parent cell body. The synaptic vesicle population of the mossy boutons suggests that they are a heterogeneous group and may have multiple origins. 
% Apparently, each of the various classes participates in both granule and Golgi cell glomeruli. 
% The smaller peripheral boutons with pleomorphic vesicles in the two types of glomeruli may represent Golgi cell axons which make synaptic contacts with both granule and Golgi cells. The Golgi cell axons which make synaptic contacts with both granule and Golgi cells. The Golgi cell dendrites, on the other hand, are also contacted by small boutons en passant with round synaptic vesicles, which may represent granule cell axons. A tentative scheme of the circuitry in the cochlear granule cell domain is presented. The similarity with the cerebellar granule cell layer is striking.


\subsubsection{TODO Cellular Mechanisms of Golgi Cells}

Golgi cells lie in 
The granule cells are small neurons with soma diameter less than 10 \um \citep{MugnainiOsenEtAl:1980}.




%\subsection{Synaptophysiology of Golgi Cells}

%\subsubsection{Auditory Nerve Fibre Input}

Golgi cells                                

 15 \um (mice: \citep{FerragamoGoldingEtAl:1998})                     
 
Smooth, tapering dendrites, between 50 and 100 \um long, emanated in all directions (mice: \citep{FerragamoGoldingEtAl:1998})
Also see \citep{Cant:1993,MugnainiOsenEtAl:1980} 

A dense, axonal plexus, limited to the plane of the granule cell domain, extended about 250 \um
from the soma in all directions \citep{FerragamoGoldingEtAl:1998}


Intracellular recordings of Golgi cells, in one study in mice, have shown a classic type I current response  \citep{FerragamoGoldingEtAl:1998}.
Golgi cells are classic repetitively-firing neurons due to their type I~current clamp response \citep{FerragamoGoldingEtAl:1998}.  
This suggests Golgi cells are simple integrators.  

Their response to auditory nerve shocks were delayed by approximately 0.7~ms relative to the core \VCN~units \citep{FerragamoGoldingEtAl:1998}.

Regular spiking with overshooting action potentials and double exponential undershoot

% Inward rectifying FerragamoGoldingEtAl:1998     130 Mohm     
% FerragamoGoldingEtAl:1998 


    
\subsubsection{TODO Acoustic Response of Golgi cells}

The physiological response of Golgi cells has not been extensively studied.
Extracellular recordings from labelled Golgi cells is not available in the literature; however, the \GCD~(or marginal shell of the \VCN~in cats) has been studied by one group \citet{GhoshalKim:1997} without direct labelling of recorded units.
Any extracellular spikes recorded in the \GCD~are most likely from Golgi cells since granule cell somata are less than $10 \mu{m}$ and their narrow axons are unlikely to elicit electrical activity in the electrodes.
The majority of recorded units showed a monotonic increase in firing rate with increasing sound intensity \citep[Figure~\ref{fig:GolgiKimFig2}][]{GhoshalKim:1997}.


Their monotonic responses to tones and noise over a wide dynamic range provides regulation of activity in granule cells.
The contribution of a delayed, negative feedback onto \VCN~units is analogous to automatic gain control provides strong evidence for regulation of activity in granule cells. The general assumption of the functional role of Golgi cells is to regulate granule cells but they may also provide automatic gain control to the principal VCN~units, primarily D and T stellate cells \citep{FerragamoGoldingEtAl:1998a}.

    
%\subsection{GABA in the Ventral Cochlear Nucleus}
    
%\subsection{Neuromodulatory effects of Golgi cells}

